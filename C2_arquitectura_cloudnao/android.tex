\subsection{Aplicación para Android CloudNAO}
\label{\detokenize{introduction:welcome-to-cloudnao-android-s-documentation}}
Esta aplicación móvil es el componente dentro de la arquitectura CloudNAO
que permite interactuar con el robot sin necesidad de instalar software en el
robot. Se conecta con el robot usando el SDK para Android de NAOqi, hace peticiones
a la API RESTful de CloudNAO, y además utiliza la autenticación
y la base de datos en tiempo real de Firebase utilizando su SDK para Android.

Es una aplicación moderna que no se limita a ser un simple control remoto, y
añade funcionalidades como la escritura de logs generados por el robot en la nube,
y el potente análisis de imágenes simplemente haciendo peticiones a la API
RESTful de CloudNAO.

En las siguientes secciones se describe con mayor detalle cada elemento de la
aplicación, ya sea para la parte que ve el usuario final como los componentes
internos para que desarrolladores puedan mantenerla o añadir nuevas
funcionalidades.

\subsubsection{Descripción de la aplicación}
\paragraph{Requisitos}
\label{\detokenize{users_docs:requisitos}}
La aplicación fue probada con éxito sobre dispositivos con arquitecturas ARM,
y con una versión menor o igual a Android 5.1.1 (API 22). Los problemas con
nuevas plataformas son que el SDK provisto por Aldebaran, no es compatible con
arquitecturas x86, ni con nuevas versiones de Android. Esta incompatibilidad
se debe a la forma en que se compiló el archivo
\sphinxstyleemphasis{java-naoqi-sdk-2.1.4-android.jar}.


\paragraph{Fucionalidades}
\label{\detokenize{users_docs:fucionalidades}}
La aplicación ofrece algunas características que más que nada sirven para
mostrar casos de uso para algunos recursos dentro de la API RESTful
de CloudNAO, y para utilizar productos de Firebase como la autenticación
o la base de datos en tiempo real.

Cabe señalar que al usar a Firebase como BaaS, la aplicación móvil
comparte información con la aplicación web de CloudNAO. Por ejemplo, los
usuarios que se registren en una u otra, pueden iniciar sesión en cualquiera
sin problemas; similar es el caso de los robots que registren y la información
de cada uno.

En general las siguientes son las funcionalidades ofrecidas por las aplicación.


\subparagraph{Autenticación de usuarios.}
\label{\detokenize{users_docs:autenticacion-de-usuarios}}
Para poder ocupar la aplicación los usuarios deben iniciar sesión con un correo
y contraseña. La aplicación cuenta con la opción de registro, si el usuario
es nuevo.


\subparagraph{Los usuarios pueden añadir nuevos robots.}
\label{\detokenize{users_docs:los-usuarios-pueden-anadir-nuevos-robots}}
Para llevar un control de los robots que tiene un usuario se le permite añadir
los robots que sean necesarios.


\subparagraph{Selección de robots previamente guardados.}
\label{\detokenize{users_docs:seleccion-de-robots-previamente-guardados}}
El usuario cuenta con una lista que le muestra los robots disponibles
que previamente creo.


\subparagraph{Conexión con un robot seleccionado.}
\label{\detokenize{users_docs:conexion-con-un-robot-seleccionado}}
Al elegir un robot entre los que el usuario posee, simplemente escribe la
dirección IP del robot, y se crea una conexión eon éste dentro de la misma
red. Con esto, es posible acceder a todas las características disponibles en la
aplicación.


\subparagraph{Control remoto para el robot.}
\label{\detokenize{users_docs:control-remoto-para-el-robot}}
Una primera necesidad que se encontró y por la que surgió todo el proyecto
fue crear un control remoto para el robot NAO. Es muy básico pero admite
comandos para hacer al robot caminar sobre sus tres ejes y cambiar entre dos
posturas.


\subparagraph{Imagen en vivo de la cámara del robot.}
\label{\detokenize{users_docs:imagen-en-vivo-de-la-camara-del-robot}}
Esta característica está integrada con el control remoto, para poder ver lo
que ve el robot a través de su cámara.


\subparagraph{Guarda valores de ALMemory en la nube.}
\label{\detokenize{users_docs:guarda-valores-de-almemory-en-la-nube}}
Otra función dentro del control remoto es la de la opción de guardar ciertos
valores de la memoria del robot en la nube, para consultarlos o descargarlos
después en la aplicación web. El conjunto de valores que por ahora
están disponibles para su almacenamiento son los siguientes:
rightUSSensorValue, leftUSSensorValue, rightFootTotalWeight, RightBumperPressed, leftFootTotalWeight, ChestButtonPressed, RearTactilTouched, leftFootContact, LeftBumperPressed, footContact, FrontTactilTouched, BatteryChargeChanged, PostureChanged, rightFootContact, MiddleTactilTouched, GyrometerX, GyrometerY, AccelerometerX, AccelerometerY, AccelerometerZ, TorsoAngleX, TorsoAngleY.


\subparagraph{Detección de rostros en una fotografía enviada por el robot.}
\label{\detokenize{users_docs:deteccion-de-rostros-en-una-fotografia-enviada-por-el-robot}}
La detección  de rostros es un problema levemente resuelto por el equipo
de Aldebaran; existe un módulo dentro de NAOqi para seguir rostros o detectarlos,
así como reconocer algunos antes guardados. Sin embargo, los modelos de visión
que utilizan no son lo suficientemente potentes, ya que deben ejecutarse dentro
del robot. Esta funcionalidad dentro de la aplicación, muestra la facilidad con
la que se puede hacer procesamiento de imágenes con solo una petición HTTP.
El usuario simplemente toma una fotografía con la cámara del robot, añade
la etiqueta de la persona que se encuentra en esa imagen y esto se envía
a la API de CloudNAO que a la vez se conecta con la de Kairos. Si todo fue
exitoso un nuevo rostro se conserva para su futuro reconocimiento.


\subparagraph{Reconocimiento de rostros previamente almacenados.}
\label{\detokenize{users_docs:reconocimiento-de-rostros-previamente-almacenados}}
Es la continuación de la característica mencionada anteriormente, reconoce
a sujetos antes guardados a través de una foto de sus rostros. Cuando
la aplicación ha detectado los rostros y reconocido, brinda la posibilidad
de que el robot ejecute una animación y diga un breve saludo a cada persona
encontrada.


\subparagraph{Detección de objectos de entre 80 categorías en una imagen de la cámara del robot.}
\label{\detokenize{users_docs:deteccion-de-objectos-de-entre-80-categorias-en-una-imagen-de-la-camara-del-robot}}
Funciona de manera similar a la detección de rostros. Envía una imagen capturada
con la cámara del robot a la API RESTful de CloudNAO, que ejecuta un módulo
con la API de detección de objectos de Tensorflow y se procesa la respuesta para
que en la pantalla se dibujen unos
recuadros delimitadores sobre los objectos detectados así como una etiqueta
con el nombre que le corresponde. El robot simplemente
dice la cantidad de objetos que ve de acuerdo a las categorías válidas.


\subparagraph{Reconocimiento óptico de caracteres y traducción de texto encontrados en una imagen enviada por el robot.}
\label{\detokenize{users_docs:reconocimiento-optico-de-caracteres-y-traduccion-de-texto-encontrados-en-una-imagen-enviada-por-el-robot}}
La aplicación solicita una imagen del robot, esa imagen es parte de la petición
a la API RESTful de CloudNAO, igual que en las dos funcionalidades anteriores.
Se buscan caracteres dentro de la imagen, para su posterior traducción.
El resultado se muestra en dos partes, una es el texto original y el otro
es el texto traducido al español. El robot tiene la opción de convertir la
traducción en un discurso oral.


% \subsubsection{Demostración de la aplicación funcionando}
% \label{\detokenize{users_docs:demostracion-de-la-aplicacion-funcionando}}
% En esta sección se describe de manera pictórica el flujo de la aplicación.
% Desde que el usuario inicia sesión, selecciona un robot y se conecta a él,
% y recorre cada una de las opciones del menú.

% Cada parte va acompañada de una captura de pantalla de la aplicación y
% una fotografía del robot si esta es necesaria.


% \paragraph{Inicio de sesión}
% \label{\detokenize{users_docs:inicio-de-sesion}}

% \paragraph{Selección de Robot}
% \label{\detokenize{users_docs:seleccion-de-robot}}

% \paragraph{Recorrido por el menú}
% \label{\detokenize{users_docs:recorrido-por-el-menu}}

% \paragraph{Control remoto}
% \label{\detokenize{users_docs:control-remoto}}

% \paragraph{Reconocimiento facial}
% \label{\detokenize{users_docs:reconocimiento-facial}}

% \paragraph{Reconocimiento óptico de caracteres y traducción}
% \label{\detokenize{users_docs:reconocimiento-optico-de-caracteres-y-traduccion}}

% \paragraph{Detección de objetos}
% \label{\detokenize{users_docs:deteccion-de-objetos}}

% \paragraph{Cierra la conexión con el robot}
% \label{\detokenize{users_docs:cierra-la-conexion-con-el-robot}}

% \paragraph{Cierra la sesión del usuario}
% \label{\detokenize{users_docs:cierra-la-sesion-del-usuario}}

\subsubsection{Guía para desarrolladores}
\label{\detokenize{dev_docs::doc}}\label{\detokenize{dev_docs:guia-para-desarrolladores}}

Para desarrollar la aplicación fueron tres los elementos importantes:
\begin{itemize}
\item {} 
Un entorno de desarrollo integrado (\textbf{Android Studio}).

\item {} 
El robot humanoide NAO (no simulado).

\item {} 
El SDK para Android de NAOqi.

\end{itemize}

Además de los elementos anteriores se necesitan una computadora y acceso a una red inalámbrica
en cada uno de los dispositivos.

En las siguientes subsecciones se describen las herramientas ocupadas durante 
el desarrollo de la aplicación:
Android Studio, Firebase Realtime
Database, Firebase UI para la autenticación, y las bibliotecas Butterknife
y Volley para el manejo de la IU y para las peticiones a la API REST,
respectivamente. Después de la breve descripción de los elementos
mencionados se listan las actividades principales (subclases de \texttt{Activity}) dentro de la aplicación móvil.


\paragraph{Android Studio}
\label{\detokenize{dev_docs:android-studio}}
Android Studio es el entorno de desarrollo integrado (IDE por sus siglas en
inglés) oficial para desarrollar sobre la plataforma Android, está basado en
el popular IntelliJ IDEA.

La instalación de Android Studio incluye:
\begin{itemize}
\item {} 
El Android SDK

\item {} 
Herramientas del Android SDK y herramientas de plataforma. Intrumentos para el debugging y el testing de tus aplicaciones.

\item {} 
Una imagen del sistema para el emulador de Android. Permite crear y probar aplicaciones sobre dispositivos virutales diferentes.

\end{itemize}

La versión de Android Studio utilizada en este proyecto fue la 2.3.3. Probablemente
existan problemas de compatibilidad con nuevas versiones.


\paragraph{Estructura de un proyecto}
\label{\detokenize{dev_docs:estructura-de-un-proyecto}}
Cada proyecto en Android Studio contiene uno o más módulos con archivos de código fuente y archivos de recursos. Entre los tipos de módulos se incluyen los siguientes:
\begin{itemize}
\item {} 
módulos de apps para Android

\item {} 
módulos de bibliotecas


\end{itemize}

De manera predeterminada, Android Studio muestra los archivos de tu proyecto en la vista de proyectos de Android, como se muestra en la figura 1. Esta vista se organiza en módulos para proporcionar un rápido acceso a los archivos de origen clave de tu proyecto.

Cada módulo de la aplicación contiene las siguientes carpetas:
\begin{itemize}
\item {} 
\sphinxstylestrong{manifests}: contiene el archivo \sphinxcode{AndroidManifest.xml}.

\item {} 
\sphinxstylestrong{java}: contiene los archivos de código fuente de Java.

\item {} 
\sphinxstylestrong{res}: Contiene todos los recursos, como diseños XML, cadenas de IU e imágenes de mapa de bits.

\end{itemize}

%
%
%\paragraph{Sistema de compilación de Gradle}
%\label{\detokenize{dev_docs:sistema-de-compilacion-de-gradle}}
%Android Studio usa Gradle como la base del sistema de compilación, con más capacidades específicas de Android a través del complemento de Android para Gradle. Este sistema de compilación se ejecuta en una herramienta integrada desde el menú de Android Studio, y lo hace independientemente de la línea de comandos. Puedes usar las funciones del sistema de compilación para lo siguiente:
%\begin{itemize}
%\item {} 
%personalizar, configurar y extender el proceso de compilación;
%
%\item {} 
%crear múltiples APK para tu aplicación, con diferentes funciones utilizando el mismo proyecto y los mismos módulos;
%
%\item {} 
%volver a usar códigos y recursos entre conjuntos de archivos de origen.
%
%\end{itemize}
%
%Recurriendo a la flexibilidad de Gradle, puedes lograr todo esto sin modificar los archivos de origen de tu app. Los archivos de compilación de Android Studio se denominan build.gradle. Son archivos de texto sin formato que usan la sintaxis Groovy para configurar la compilación con elementos proporcionados por el complemento de Android para Gradle. Cada proyecto tiene un archivo de compilación de nivel superior para todo el proyecto y archivos de compilación de nivel de módulo independientes para cada módulo. Cuando importas un proyecto existente, Android Studio genera automáticamente los archivos de compilación necesarios.
%

\paragraph{Ejemplo de una aplicación}
\label{\detokenize{dev_docs:una-primera-aplicacion}}
En esa sección se muestra como crear una aplicación para Android muy simple
y básica. El objetivo es exponer algunos conceptos que son requeridos al
desarrollar una aplicación. Cabe recordar que no se pretende que este sea un
curso intensivo, no sólo para esta sección sino para todo el documento escrito;
simplemente se espera que quien lea esto conozca las herramientas y pueda ir
de la nada a construir algo. Esto no es documentación oficial por lo que todo
lo descrito, puede cambiar con las constantes actualizaciones de los entornos
de desarrollo.

La aplicación está basada en la primera aplicación descrita en el libro Android
Programming The Big Nerd Ranch Guide (Bill Phillips, Chris Stewart, Kristin Marsicano).
En nuestro caso se llamará Data Structures Quiz. Esta aplicación prueba el
conocimientos del usuario sobre estructuras de datos. El usuario presiona
alguna de las opciones, para responder la pregunta que aparece e inmediatamente
se refleja el resultado de si fue o no correcta la elección.

La aplicación cosiste de una \sphinxstyleemphasis{actividad} y un \sphinxstyleemphasis{layout}.

Una \sphinxstyleemphasis{actividad} es una instancia de \sphinxcode{Activity} una clase del Android SDK.
Una actividad es la encargada de manejar la interacción del usuario con una
pantalla de información.

Se escriben subclases de \sphinxcode{Activity} para implementar la funcionalidad que la
aplicación requiera. Una aplicación puede tener una o muchas subclases.

Data Structures Quiz es una aplicación muy simple, sólo necesitará una sola
subclase de \sphinxcode{Activity} llamada \sphinxcode{QuizActivity}. \sphinxcode{QuizActivity} manejará la
interfaz de usuario.

Un \sphinxstyleemphasis{layout} define un conjunto de objetos en la IU y sus posiciones en la
pantalla. Un layout está compuesto por definiciones escritas en un archivo
XML. Cada definición es usada para crear un objeto que aparece en la pantalla,
como un botón o texto.

Data Structures Quiz incluye un archivo del layout llamado \sphinxcode{activity\_main.xml}.


\paragraph{Creación del proyecto}
\label{\detokenize{dev_docs:creacion-del-proyecto}}
El primer paso es crear un un proyecto en Android Studio, El proyecto contiene
los archivos para construir una aplicación.

Para crear un nuevo proyecto, simplemente se sigue el guía que provee Android
Studio, basta con dar siguiente a todo por ahora.

Las pantallas que se muestran en la figura ilustran cada página o ventana
que se crea al seguir las instrucciones. En resumen, se define el nombre
de la aplicación, se especifica que dispositivos podrán ejecutar la aplicación,
y se elige una actividad que servirá como punto de partida.


\paragraph{Diseño de la IU}
\label{\detokenize{dev_docs:diseno-de-la-iu}}
El archivo \sphinxcode{activity\_main.xml}, debe contener los siguientes elementos:
\begin{itemize}
\item {} 
Un \sphinxcode{LinearLayout} orientado verticalmente.

\item {} 
Un \sphinxcode{TextView}.

\item {} 
Un \sphinxcode{LinearLayout} orientado horizontalmente

\item {} 
Dos botones (\sphinxcode{Button}).

\end{itemize}

Modificando el archivo debe quedar como sigue:

\fvset{hllines={, ,}}%
\begin{sphinxVerbatim}[commandchars=\\\{\}]
\PYG{c+cp}{\PYGZlt{}?xml version=\PYGZdq{}1.0\PYGZdq{} encoding=\PYGZdq{}utf\PYGZhy{}8\PYGZdq{}?\PYGZgt{}}
\PYG{n+nt}{\PYGZlt{}LinearLayout} \PYG{n+na}{xmlns:android=}\PYG{l+s}{\PYGZdq{}http://schemas.android.com/apk/res/android\PYGZdq{}}
  \PYG{n+na}{xmlns:tools=}\PYG{l+s}{\PYGZdq{}http://schemas.android.com/tools\PYGZdq{}}
  \PYG{n+na}{android:layout\PYGZus{}width=}\PYG{l+s}{\PYGZdq{}match\PYGZus{}parent\PYGZdq{}}
  \PYG{n+na}{android:layout\PYGZus{}height=}\PYG{l+s}{\PYGZdq{}match\PYGZus{}parent\PYGZdq{}}
  \PYG{n+na}{tools:context=}\PYG{l+s}{\PYGZdq{}com.example.ivan.datastructuresquiz.MainActivity\PYGZdq{}}
  \PYG{n+na}{android:orientation=}\PYG{l+s}{\PYGZdq{}vertical\PYGZdq{}}\PYG{n+nt}{\PYGZgt{}}

  \PYG{n+nt}{\PYGZlt{}TextView}
      \PYG{n+na}{android:id=}\PYG{l+s}{\PYGZdq{}@+id/textView\PYGZdq{}}
      \PYG{n+na}{android:layout\PYGZus{}width=}\PYG{l+s}{\PYGZdq{}wrap\PYGZus{}content\PYGZdq{}}
      \PYG{n+na}{android:layout\PYGZus{}height=}\PYG{l+s}{\PYGZdq{}wrap\PYGZus{}content\PYGZdq{}}
      \PYG{n+na}{android:padding=}\PYG{l+s}{\PYGZdq{}24dp\PYGZdq{}}
      \PYG{n+na}{android:text=}\PYG{l+s}{\PYGZdq{}La búsqueda sobre un árbolo binario de búsqueda, ¿qué complejidad tiene? (Caso promedio)\PYGZdq{}} \PYG{n+nt}{/\PYGZgt{}}

  \PYG{n+nt}{\PYGZlt{}LinearLayout}
      \PYG{n+na}{android:layout\PYGZus{}width=}\PYG{l+s}{\PYGZdq{}wrap\PYGZus{}content\PYGZdq{}}
      \PYG{n+na}{android:layout\PYGZus{}height=}\PYG{l+s}{\PYGZdq{}wrap\PYGZus{}content\PYGZdq{}}
      \PYG{n+na}{android:orientation=}\PYG{l+s}{\PYGZdq{}horizontal\PYGZdq{}}
      \PYG{n+na}{android:layout\PYGZus{}gravity=}\PYG{l+s}{\PYGZdq{}center\PYGZus{}horizontal\PYGZdq{}}\PYG{n+nt}{\PYGZgt{}}

      \PYG{n+nt}{\PYGZlt{}Button}
          \PYG{n+na}{android:id=}\PYG{l+s}{\PYGZdq{}@+id/AButton\PYGZdq{}}
          \PYG{n+na}{android:layout\PYGZus{}width=}\PYG{l+s}{\PYGZdq{}wrap\PYGZus{}content\PYGZdq{}}
          \PYG{n+na}{android:layout\PYGZus{}height=}\PYG{l+s}{\PYGZdq{}wrap\PYGZus{}content\PYGZdq{}}
          \PYG{n+na}{android:text=}\PYG{l+s}{\PYGZdq{}O(log(n))\PYGZdq{}}
          \PYG{n+nt}{/\PYGZgt{}}

      \PYG{n+nt}{\PYGZlt{}Button}
          \PYG{n+na}{android:id=}\PYG{l+s}{\PYGZdq{}@+id/BButton\PYGZdq{}}
          \PYG{n+na}{android:layout\PYGZus{}width=}\PYG{l+s}{\PYGZdq{}wrap\PYGZus{}content\PYGZdq{}}
          \PYG{n+na}{android:layout\PYGZus{}height=}\PYG{l+s}{\PYGZdq{}wrap\PYGZus{}content\PYGZdq{}}
          \PYG{n+na}{android:text=}\PYG{l+s}{\PYGZdq{}O(n)\PYGZdq{}} \PYG{n+nt}{/\PYGZgt{}}
  \PYG{n+nt}{\PYGZlt{}/LinearLayout\PYGZgt{}}
\PYG{n+nt}{\PYGZlt{}/LinearLayout\PYGZgt{}}
\end{sphinxVerbatim}

Comparando el XML con la figura :: cada widget tiene un elemento XML, y el
nombre del elemento es el tipo de widget.

Cada elemento tiene un conjunto de atributos XML. Cada atrbuto es una instrucción
sobre cómo debe configurarse el widget.


\subparagraph{La jerarquía del view.}
\label{\detokenize{dev_docs:la-jerarquia-del-view}}
Los widgets existen un una jerarquía de objetos \sphinxstylestrong{View} llamada la jerarquía
view. La figura :: muestra la jerarquía que corresponde a nuestro archivo
\sphinxcode{activity\_main.xml}.

La raíz de esta jerarquía es un \sphinxcode{LinearLayout}. Como elemento raíz debe especificar
el namespace XML de recursos de Android en \sphinxcode{http//schemas.android.com/apk/res/android}.

\sphinxcode{LinearLayout} hereda de una subclase de \sphinxcode{View} llamada ViewGroup. Un \sphinxcode{ViewGroup} es un
widget que contiene un arreglo con otros widgets. Se utiliza \sphinxcode{LinearLayout} cuando
se desea que un conjunto de widgets estén acomodados sobre una sola columna
o fila. Otras subclases \sphinxcode{ViewGroup} son \sphinxcode{FrameLayout}, \sphinxcode{TabletLayout} y \sphinxcode{RelativeLayout}.

Cuando un widget está contenido en un \sphinxcode{ViewGroup}, se dice ser su hijo. La raíz
\sphinxcode{LinearLayout} tiene dos hijos: un \sphinxcode{TextView} y otro \sphinxcode{LinearLayout}. Éste último
tiene dos botones hijos.


\subparagraph{De diseños XML a objetos View}
\label{\detokenize{dev_docs:de-disenos-xml-a-objetos-view}}
Cuando se creó el proyecto, se generó una subclase de \sphinxcode{Activity} llamada
\sphinxcode{MainActivity}. El archivo de esta clase está en el directorio \sphinxstyleemphasis{app/java} del
proyecto. La carpeta \sphinxstyleemphasis{java} es donde se encuentra el código de java de cada
proyecto.

Mostremos el resultado final de cómo debe lucir la clase \sphinxcode{MainActivity},
y después expliquemos cada parte.

\fvset{hllines={, ,}}%
\begin{sphinxVerbatim}[commandchars=\\\{\}]
\PYG{k+kd}{public} \PYG{k+kd}{class} \PYG{n+nc}{MainActivity} \PYG{k+kd}{extends} \PYG{n}{AppCompatActivity} \PYG{o}{\PYGZob{}}
  \PYG{c+cm}{/**}
\PYG{c+cm}{   * Añade atributos a la clase}
\PYG{c+cm}{   */}
  \PYG{k+kd}{private} \PYG{n}{Button} \PYG{n}{mAOptionButton}\PYG{o}{;}
  \PYG{k+kd}{private} \PYG{n}{Button} \PYG{n}{mBOptionButton}\PYG{o}{;}

  \PYG{n+nd}{@Override}
  \PYG{k+kd}{protected} \PYG{k+kt}{void} \PYG{n+nf}{onCreate}\PYG{o}{(}\PYG{n}{Bundle} \PYG{n}{savedInstanceState}\PYG{o}{)} \PYG{o}{\PYGZob{}}
      \PYG{k+kd}{super}\PYG{o}{.}\PYG{n+na}{onCreate}\PYG{o}{(}\PYG{n}{savedInstanceState}\PYG{o}{)}\PYG{o}{;}
      \PYG{n}{setContentView}\PYG{o}{(}\PYG{n}{R}\PYG{o}{.}\PYG{n+na}{layout}\PYG{o}{.}\PYG{n+na}{activity\PYGZus{}main}\PYG{o}{)}\PYG{o}{;}

      \PYG{c+cm}{/**}
\PYG{c+cm}{       * Obtiene las referencias a los widgets}
\PYG{c+cm}{       */}
      \PYG{n}{mAOptionButton} \PYG{o}{=} \PYG{o}{(}\PYG{n}{Button}\PYG{o}{)} \PYG{n}{findViewById}\PYG{o}{(}\PYG{n}{R}\PYG{o}{.}\PYG{n+na}{id}\PYG{o}{.}\PYG{n+na}{AButton}\PYG{o}{)}\PYG{o}{;}
      \PYG{n}{mBOptionButton} \PYG{o}{=} \PYG{o}{(}\PYG{n}{Button}\PYG{o}{)} \PYG{n}{findViewById}\PYG{o}{(}\PYG{n}{R}\PYG{o}{.}\PYG{n+na}{id}\PYG{o}{.}\PYG{n+na}{BButton}\PYG{o}{)}\PYG{o}{;}

      \PYG{c+cm}{/**}
\PYG{c+cm}{       * Configura los escuchadores}
\PYG{c+cm}{       */}

      \PYG{n}{mAOptionButton}\PYG{o}{.}\PYG{n+na}{setOnClickListener}\PYG{o}{(}\PYG{k}{new} \PYG{n}{View}\PYG{o}{.}\PYG{n+na}{OnClickListener}\PYG{o}{(}\PYG{o}{)} \PYG{o}{\PYGZob{}}
          \PYG{n+nd}{@Override}
          \PYG{k+kd}{public} \PYG{k+kt}{void} \PYG{n+nf}{onClick}\PYG{o}{(}\PYG{n}{View} \PYG{n}{view}\PYG{o}{)} \PYG{o}{\PYGZob{}}
              \PYG{c+cm}{/**}
\PYG{c+cm}{               * Muestra un Toast con el resultado para el usuario}
\PYG{c+cm}{               */}
              \PYG{n}{Toast}\PYG{o}{.}\PYG{n+na}{makeText}\PYG{o}{(}\PYG{n}{MainActivity}\PYG{o}{.}\PYG{n+na}{this}\PYG{o}{,} \PYG{l+s}{\PYGZdq{}Correct\PYGZdq{}}\PYG{o}{,} \PYG{n}{Toast}\PYG{o}{.}\PYG{n+na}{LENGTH\PYGZus{}SHORT}\PYG{o}{)}\PYG{o}{.}\PYG{n+na}{show}\PYG{o}{(}\PYG{o}{)}\PYG{o}{;}
          \PYG{o}{\PYGZcb{}}
      \PYG{o}{\PYGZcb{}}\PYG{o}{)}\PYG{o}{;}
      \PYG{n}{mBOptionButton}\PYG{o}{.}\PYG{n+na}{setOnClickListener}\PYG{o}{(}\PYG{k}{new} \PYG{n}{View}\PYG{o}{.}\PYG{n+na}{OnClickListener}\PYG{o}{(}\PYG{o}{)} \PYG{o}{\PYGZob{}}
          \PYG{n+nd}{@Override}
          \PYG{k+kd}{public} \PYG{k+kt}{void} \PYG{n+nf}{onClick}\PYG{o}{(}\PYG{n}{View} \PYG{n}{view}\PYG{o}{)} \PYG{o}{\PYGZob{}}
              \PYG{n}{Toast}\PYG{o}{.}\PYG{n+na}{makeText}\PYG{o}{(}\PYG{n}{MainActivity}\PYG{o}{.}\PYG{n+na}{this}\PYG{o}{,} \PYG{l+s}{\PYGZdq{}Incorrect\PYGZdq{}}\PYG{o}{,} \PYG{n}{Toast}\PYG{o}{.}\PYG{n+na}{LENGTH\PYGZus{}SHORT}\PYG{o}{)}\PYG{o}{.}\PYG{n+na}{show}\PYG{o}{(}\PYG{o}{)}\PYG{o}{;}
          \PYG{o}{\PYGZcb{}}
      \PYG{o}{\PYGZcb{}}\PYG{o}{)}\PYG{o}{;}
  \PYG{o}{\PYGZcb{}}
\PYG{o}{\PYGZcb{}}
\end{sphinxVerbatim}

El método \sphinxcode{onCreateBundle()} es llamado cuando una instancia de las subclase
de \sphinxcode{Activity} es creada. Cuando una actividad es creada, necesita de una IU para
manejar. Para que esta actividad tenga su propia IU, se llama el método
\sphinxcode{setContentView}. Este método \sphinxstyleemphasis{infla} un layout y lo pone en la pantalla.
Cuando un layout es inflado, cada unos de sus widgets es instanciado como
lo definen sus atributos. Se especifica que layout inflar pasando el identificador
del recurso.

Cada layout es un recurso. Un recurso es una pieza de la aplicación que no es
código, cosas como imágenes, archivos XML, etc. Estos recursos están en
el directorio \sphinxstyleemphasis{app/res}. Para acceder a un recurso desde el código se hace
usando su identificador (\sphinxstyleemphasis{resource ID}).

Ahora, se definen dos variables miembro dentro de la clase, son de tipo
\sphinxcode{Button}, ya que se conectarán con los botones definidos en el layout.

En una actividad se pueden obtener las referencias a un widget inflado llamando
al método \sphinxcode{findViewById}. Este método recibe como parámetro el id del recurso
y retorna un objeto View. En nuestro caso se asignan a las variables
\sphinxcode{mAOptionButton} y \sphinxcode{mBOptionButton} los objetos View, a partir del id
de los recursos de los botones.

Finalmente se crean unos escuchadores para cada botón. Las aplicaciones en
Android son conducidas por eventos, por lo que cuando una aplicación espera
por un evento en específico se dice que está «escuchando» por ese evento. La
sintaxis es extraña pero simplemente se crea implementa una clase anónima
y se sobreescriben sus métodos para manejar las respuestas a eventos. En nuestra
aplicación los escuchadores de los botones muestran un mensaje en la pantalla
cuando el usuario presiona alguno de los botones.


\paragraph{SDK para Android de NAOqi}
\label{\detokenize{dev_docs:sdk-para-android-de-naoqi}}
Para ejecutar módulos de manera remota sobre los robots de NAOqi es necesario
contar con un SDK para la plataforma y el lenguaje con el que se desee
desarrollar. La plataforma en cuestión es Android, y Aldebaran brinda un SDK
de Java específicamente para este sistema. Como ya se mencionó en la sección
sobre \sphinxstylestrong{NAOqi}, el SDK de Java utiliza el \sphinxstylestrong{framework qi}.

El SDK provisto  no ha sido actualizado y en el archivo \sphinxstyleemphasis{.jar} no es compatible
con nuevas versiones de Android a partir de la 5.

El SDK se añade al proyecto como si fuera cualquier bibliotecas desarrollada
por un tercero. Cambiando la vista del proyecto que viene por defecto, \sphinxstyleemphasis{Android},
a \sphinxstyleemphasis{Project}, se puede copiar el archivo \sphinxcode{javanaoqi-sdk-2.1.4-android.jar}
al directorio en la ubicación \sphinxstyleemphasis{app/libs}. Después se configura el archivo
del gradle de la aplicación y del proyecto.


\paragraph{Configuración de Gradle}
\label{\detokenize{dev_docs:configuracion-de-gradle}}
El sistema de construcción de Android Studio está basado en Gradle, y el plugin
de Android para Gradle adiciona varias características que están
contruidas específicamente para aplicaciones de Android.\\

\begin{remark}
Gradle es un sistema de automatización de construcción de código abierto
enfocado a la flexibilidad y desempeño.
\end{remark}

La última versión del plugin de Gradle compatible con el SDK que brinda
Aldebaran es la 1.3.1. Por lo tanto después de sumar a nuestro directorio
\sphinxstyleemphasis{app/libs} se configura lo siguiente en el archivo \sphinxcode{buldgradle} del
proyecto:

\fvset{hllines={, ,}}%
\begin{sphinxVerbatim}[commandchars=\\\{\}]
\PYG{n+nx}{buildscript} \PYG{p}{\PYGZob{}}
  \PYG{n+nx}{dependencies} \PYG{p}{\PYGZob{}}
    \PYG{n+nx}{classpath} \PYG{l+s+s1}{\PYGZsq{}com.android.tools.build:gradle:1.3.1\PYGZsq{}}
  \PYG{p}{\PYGZcb{}}
\PYG{p}{\PYGZcb{}}
\end{sphinxVerbatim}

Por último sólo se agrega a las dependencias dentro del archivo \sphinxcode{buildgradle}
de la aplicación \sphinxcode{compilefiles('libs/java-naoqi-sdk-2.1.4-android.jar')}, que
compila el SDK de NAOqi.


\subsubsection{Firebase para Android}
\label{\detokenize{dev_docs:firebase-para-android}}

% \paragraph{Añadir Firebase a un proyecto de Android}
% \label{\detokenize{dev_docs:anadir-firebase-a-un-proyecto-de-android}}

Los servicios de Firebase utilizados son Firebase Realtime Database
y Authentication. Este último a través de la biblioteca Firebase UI.

Antes de poder utilizar Firebase en una aplicación se debe agregar Firebase a ésta. Después de crear un proyecto en la consola 
de Firebase se siguen los pasos que ésta nos muestra
para añadir Firebase a nuestras aplicaciones ya sea web 
o móviles, en nuestro caso para dispositivos Android.
La consola genera un archivo de configuración
que se añade a la aplicación y luego
se agrega el SDK para poder utilizar las bibliotecas de Firebase en el proyecto.

Cada biblioteca de Firebase que se desee utilizar se añade a las dependencias.
Por ejemplo, sumando a las dependencias de Gradle  \sphinxcode{com.google.firebase:firebase-database:11.6.2}
se tiene acceso a a la base de datos en tiempo real que provee Firebase.


\paragraph{Firebase Realtime Database}
\label{\detokenize{dev_docs:firebase-realtime-database}}
Para comenzar a usar la Firebase Realtime Database en la aplicación,
se debe sumar a las dependencias del archivo \sphinxcode{buildgradle}, de la aplicación,
\sphinxcode{compile'com.google.firebase:firebase-database:11.8.0'}.

%Los datos de Firebase se escriben en una referencia de \sphinxcode{FirebaseDatabase};
%para recuperarlos, se debe adjuntar un agente de escucha o escuchador asíncrono a la
%referencia. El agente de escucha se activa una vez para el estado inicial de los
%datos y otra vez cuando los datos cambian.

%
%\subparagraph{Obtener una DatabaseReference}
%\label{\detokenize{dev_docs:obtener-una-databasereference}}

Para leer y escribir en la base de datos, necesitas una instancia de
\texttt{DatabaseReference}:

\fvset{hllines={, ,}}%
\begin{sphinxVerbatim}[commandchars=\\\{\}]
\PYG{k+kd}{private} \PYG{n}{DatabaseReference} \PYG{n}{mDatabase}\PYG{o}{;}
\PYG{n}{mDatabase} \PYG{o}{=} \PYG{n}{FirebaseDatabase}\PYG{o}{.}\PYG{n+na}{getInstance}\PYG{o}{(}\PYG{o}{)}\PYG{o}{.}\PYG{n+na}{getReference}\PYG{o}{(}\PYG{o}{)}\PYG{o}{;}
\end{sphinxVerbatim}


\subparagraph{Leer y escribir datos.}
\label{\detokenize{dev_docs:leer-y-escribir-datos}}
Para ejecutar una escritura básica se usa \sphinxcode{setValue()} para guardar datos en
una referencia que se especifique.

Los tipos de datos que se pueden almacenar son los siguientes:
\begin{itemize}
\item {}
\texttt{String}

\item {}
\texttt{Long}

\item {}
\texttt{Double}

\item {}
\texttt{Boolean}

\item {}
\texttt{Map\textless{}String, Object\textgreater{}}

\item {}
\texttt{List\textless{}Object\textgreater{}}

\end{itemize}

También se puede pasar un objeto personalizado de Java con la restricción
de que la definición de la clase debe tener un constructor predeterminado
que no recibe parámetros y tiene métodos públicos para obtener a los atributos
del objeto.

El siguiente fragmento muestra como escribir un nuevo usuario en la base de
datos.

\fvset{hllines={, ,}}%
\begin{sphinxVerbatim}[commandchars=\\\{\}]
\PYG{k+kd}{public} \PYG{k+kt}{void} \PYG{n+nf}{saveUserToDatabase}\PYG{o}{(}\PYG{n}{String} \PYG{n}{uid}\PYG{o}{,} \PYG{n}{String} \PYG{n}{email}\PYG{o}{)} \PYG{o}{\PYGZob{}}
  \PYG{n}{mDatabase}\PYG{o}{.}\PYG{n+na}{child}\PYG{o}{(}\PYG{l+s}{\PYGZdq{}users/\PYGZdq{}} \PYG{o}{+} \PYG{n}{uid}\PYG{o}{)}\PYG{o}{.}\PYG{n+na}{setValue}\PYG{o}{(}\PYG{n}{email}\PYG{o}{)}\PYG{o}{;}
\PYG{o}{\PYGZcb{}}
\end{sphinxVerbatim}


\subparagraph{Detección de eventos en valores.}
\label{\detokenize{dev_docs:detecteccion-eventos-de-valores}}
Para leer datos en una ruta de acceso y escuchar actualizaciones, usa el
método \sphinxcode{addValueEventListener()} o el método \sphinxcode{addListenerForSingleValueEvent()}
para agregar un \sphinxcode{ValueEventListener} a un objetos \sphinxcode{DatabaseReference}.

Un escuchador \sphinxcode{ValueEventListener} de una referencia de la base de datos
tiene una método callback \sphinxcode{onDataChange()} que obtiene una captura estática
del contiene en la ruta determinada.

El ejemplo siguiente demuestra como un aplicación que muestra mensajes
alamacenados en la base de datos:

\fvset{hllines={, ,}}%
\begin{sphinxVerbatim}[commandchars=\\\{\}]
\PYG{n}{mMessagesReference}\PYG{o}{.}\PYG{n+na}{addValueEventListener}\PYG{o}{(}\PYG{k}{new} \PYG{n}{ValueEventListener}\PYG{o}{(}\PYG{o}{)}\PYG{o}{\PYGZob{}}
  \PYG{n+nd}{@Override}
  \PYG{k+kd}{public} \PYG{k+kt}{void} \PYG{n+nf}{onDataChange}\PYG{o}{(}\PYG{n}{DataSnapshot} \PYG{n}{snapshot}\PYG{o}{)} \PYG{o}{\PYGZob{}}
    \PYG{n}{Message} \PYG{n}{msg} \PYG{o}{=} \PYG{n}{snapshot}\PYG{o}{.}\PYG{n+na}{getValue}\PYG{o}{(}\PYG{n}{Message}\PYG{o}{.}\PYG{n+na}{class}\PYG{o}{)}\PYG{o}{;}
    \PYG{c+c1}{//...}
  \PYG{o}{\PYGZcb{}}
\PYG{o}{\PYGZcb{}}\PYG{o}{)}
\end{sphinxVerbatim}

El escuchador recibe un objeto \sphinxcode{DataSnapshot} que contiene los datos de la
ubicación especifica al momentos del evento.

La diferencia entre \sphinxcode{addValueEventListener()} y
\sphinxcode{addListenerForSingleValue\\Event()} es que el primero se suscribe a cierta
ubicación para escuchar cambios, y el segundo lee los datos de una ubicación
una sola vez.


\subparagraph{Actualización o borrado de datos.}
\label{\detokenize{dev_docs:actulizacion-o-borrado-de-datos}}
Para actualizar se llama al método \sphinxcode{update\\Children()} y para eliminar datos
en una referencia se usa \sphinxcode{removeValue} o se pasa como parámetro un valor
\sphinxcode{null} a \sphinxcode{setValue()}.

También existe el método \sphinxcode{push()} para añadir una entrada con un identificador
único sobre una referencia.


\paragraph{Firebase UI Authentication}
\label{\detokenize{dev_docs:firebase-ui-authentication}}
%FirebaseUI es una biblioteca de código abierto que ofrece referencias a la IU
%de manera simple y personalizable, sobre los SDK de Firebase, para eliminar
%código repetitivo y promover mejores prácticas (en cuanto a la experiencia
%de usuario y seguridad) para la autenticación.

Es una API para manejar el flujo del inicio de sesión con una
dirección de correo electrónico, números de teléfono, y a través de
proveedores como Google Sign-In, y Facebook Login. Está construido sobre
Firebase Authentication.

%FirebaseUI se integra con \sphinxstylestrong{Smart Lock} para guardar y recuperar credenciales,
%habilita el inicio de sesión con un solo click en una aplicación cuando el
%usuario regresa a esta. Maneja casos de uso complicados como la recuperación
%de la cuenta y el enlace con la cuenta que son inseguros y difíciles de implementar
%usando la API base de Firebase Authentication.


\subparagraph{Configuración.}
\label{\detokenize{dev_docs:configuracion}}
Como prerrequisito, la aplicación debe estar configurada para utilizar Firebase.
Después, se agrega a la biblioteca de FirebaseUI auth en las dependencias de
\sphinxcode{buildgradle} de la aplicación.

\fvset{hllines={, ,}}%
\begin{sphinxVerbatim}[commandchars=\\\{\}]
\PYG{n+nx}{dependencies} \PYG{p}{\PYGZob{}}
  \PYG{n+nx}{compile} \PYG{l+s+s1}{\PYGZsq{}com.firebaseui:firebase\PYGZhy{}ui\PYGZhy{}auth:3.0.0\PYGZsq{}}
\PYG{p}{\PYGZcb{}}
\end{sphinxVerbatim}


\subparagraph{Uso de FirebaseUI para autenticación.}
\label{\detokenize{dev_docs:uso-de-firebaseui-para-autenticacion}}
Antes de llamar al flujo de autenticación de FirebaseUI, la aplicación debe
checar que un usuario ya esté registrado de una sesión previa. Este caso
es cuando el usuario inicio sesión y salió de la aplicación para luego regresar
a esta.

\fvset{hllines={, ,}}%
\begin{sphinxVerbatim}[commandchars=\\\{\}]
\PYG{n}{FirebaseAuth} \PYG{n}{auth} \PYG{o}{=} \PYG{n}{FirebaseAuth}\PYG{o}{.}\PYG{n+na}{getInstance}\PYG{o}{(}\PYG{o}{)}\PYG{o}{;}
\PYG{k}{if} \PYG{o}{(}\PYG{n}{auth}\PYG{o}{.}\PYG{n+na}{getCurrentUser}\PYG{o}{(}\PYG{o}{)} \PYG{o}{!}\PYG{o}{=} \PYG{k+kc}{null}\PYG{o}{)} \PYG{o}{\PYGZob{}}
  \PYG{c+c1}{// el usuario tiene abierta la sesión}
\PYG{o}{\PYGZcb{}} \PYG{k}{else} \PYG{o}{\PYGZob{}}
  \PYG{c+c1}{// el usuario no ha iniciado sesión}
\PYG{o}{\PYGZcb{}}
\end{sphinxVerbatim}

El punto de entrada para el flujo de la autenticación es la clase
\sphinxcode{com.firebase.ui.auth.AuthUI}.


\subparagraph{Inicio de sesión.}
\label{\detokenize{dev_docs:inicio-de-sesion}}
Si un usuario no ha iniciado sesión, entonces el proceso de inicio de sesión
puede empezar creando un intent sign-in usando \sphinxcode{AuthUISignInIntent\\Builder}.
Se puede recuperar una instancia del contructor llamando
\sphinxcode{createSignIn\\IntentBuilder()} en la instancia retomada de AuthUI.

El constructor provee las siguientes opciones de personalización para el flujo
de la autenticación:
\begin{itemize}
\item {} 
El conjunto de provedores de autenticación puede especificarse (Google, Twitter, Facebook)

\item {} 
Un URL con los términos de servicio para la aplicación.

\item {} 
Un tema personalizado puede ser especificado para el flujo, el cual se aplica a todas las actividades en el flujo para que haya consistencia de colores y tipografía.

\end{itemize}

Si no se requiere personalizar , y solo se necesita el correo electrónico para
autenticarse, el flujo del inicio de sesón comienza como sigue:

\fvset{hllines={, ,}}%
\begin{sphinxVerbatim}[commandchars=\\\{\}]
\PYG{c+c1}{// Un valor para el código de petición arbitrario}
\PYG{k+kd}{private} \PYG{k+kd}{static} \PYG{k+kd}{final} \PYG{k+kt}{int} \PYG{n}{RC\PYGZus{}SIGN\PYGZus{}IN} \PYG{o}{=} \PYG{l+m+mi}{123}\PYG{o}{;}

\PYG{c+c1}{// ...}

\PYG{n}{startActivityForResult}\PYG{o}{(}
  \PYG{c+c1}{// obtiene una instancia de AuthUI basado en la aplicación por defecto}
  \PYG{n}{AuthUI}\PYG{o}{.}\PYG{n+na}{getInstance}\PYG{o}{(}\PYG{o}{)}\PYG{o}{.}\PYG{n+na}{createSignInIntentBuilder}\PYG{o}{(}\PYG{o}{)}\PYG{o}{.}\PYG{n+na}{build}\PYG{o}{(}\PYG{o}{)}\PYG{o}{,}
  \PYG{n}{RC\PYGZus{}SIGN\PYGZus{}IN}\PYG{o}{)}\PYG{o}{;}
\end{sphinxVerbatim}

El segundo parámetro de \sphinxcode{startActivityForResult()} (\sphinxcode{RC\_SIGN\_IN})  es un
código de petición para identificar la petición cuando el resultado retorna
a la aplicación en el métodos \sphinxcode{onActivityResult()}.

El flujo de la autenticación suministra varios códigos de respuesta, entre
los más comunes están:
\begin{itemize}
\item {} 
\sphinxcode{Activity.RESULT\_OK}, si es el usuario inició sesión.

\item {} 
\sphinxcode{Activity.RESULT\_CANCELLED}, si el usuario canceló manualmente el inicio de sesión.

\item {} 
\sphinxcode{ErrorCodes.NO\_NETWORK}, si no hay conexión a una red.

\item {} 
\sphinxcode{ErrorCodes.UNKNOWN\_ERROR},  todos los otros errores.

\end{itemize}

Siguiendo con el ejemplo del inicio de sesión con el correo electrónico,
el método \sphinxcode{onActivityResult()} queda como sigue:

\fvset{hllines={, ,}}%
\begin{sphinxVerbatim}[commandchars=\\\{\}]
\PYG{k+kd}{protected} \PYG{k+kt}{void} \PYG{n+nf}{onActivityResult}\PYG{o}{(}\PYG{k+kt}{int} \PYG{n}{requestCode}\PYG{o}{,} \PYG{k+kt}{int} \PYG{n}{resultCode}\PYG{o}{,} \PYG{n}{Intent} \PYG{n}{data}\PYG{o}{)} \PYG{o}{\PYGZob{}}
    \PYG{k+kd}{super}\PYG{o}{.}\PYG{n+na}{onActivityResult}\PYG{o}{(}\PYG{n}{requestCode}\PYG{o}{,} \PYG{n}{resultCode}\PYG{o}{,} \PYG{n}{data}\PYG{o}{)}\PYG{o}{;}
    \PYG{c+c1}{// RC\PYGZus{}SIGN\PYGZus{}IN es el segundo parámetro que se pasó a startActivityForResult}
    \PYG{k}{if} \PYG{o}{(}\PYG{n}{requestCode} \PYG{o}{=}\PYG{o}{=} \PYG{n}{RC\PYGZus{}SIGN\PYGZus{}IN}\PYG{o}{)} \PYG{o}{\PYGZob{}}
        \PYG{n}{IdpResponse} \PYG{n}{response} \PYG{o}{=} \PYG{n}{IdpResponse}\PYG{o}{.}\PYG{n+na}{fromResultIntent}\PYG{o}{(}\PYG{n}{data}\PYG{o}{)}\PYG{o}{;}

        \PYG{c+c1}{// Si el usuario inició sesión exitosamente}
        \PYG{k}{if} \PYG{o}{(}\PYG{n}{resultCode} \PYG{o}{=}\PYG{o}{=} \PYG{n}{RESULT\PYGZus{}OK}\PYG{o}{)} \PYG{o}{\PYGZob{}}
            \PYG{n}{startActivity}\PYG{o}{(}\PYG{n}{SignedInActivity}\PYG{o}{.}\PYG{n+na}{createIntent}\PYG{o}{(}\PYG{k}{this}\PYG{o}{,} \PYG{n}{response}\PYG{o}{)}\PYG{o}{)}\PYG{o}{;}
            \PYG{n}{finish}\PYG{o}{(}\PYG{o}{)}\PYG{o}{;}
        \PYG{o}{\PYGZcb{}} \PYG{k}{else} \PYG{o}{\PYGZob{}}
            \PYG{c+c1}{// El inicio de sesión falló}
            \PYG{k}{if} \PYG{o}{(}\PYG{n}{response} \PYG{o}{=}\PYG{o}{=} \PYG{k+kc}{null}\PYG{o}{)} \PYG{o}{\PYGZob{}}
                \PYG{c+c1}{// El usuario presión el botón de regresar}
                \PYG{n}{showSnackbar}\PYG{o}{(}\PYG{n}{R}\PYG{o}{.}\PYG{n+na}{string}\PYG{o}{.}\PYG{n+na}{sign\PYGZus{}in\PYGZus{}cancelled}\PYG{o}{)}\PYG{o}{;}
                \PYG{k}{return}\PYG{o}{;}
            \PYG{o}{\PYGZcb{}}

            \PYG{k}{if} \PYG{o}{(}\PYG{n}{response}\PYG{o}{.}\PYG{n+na}{getError}\PYG{o}{(}\PYG{o}{)}\PYG{o}{.}\PYG{n+na}{getErrorCode}\PYG{o}{(}\PYG{o}{)} \PYG{o}{=}\PYG{o}{=} \PYG{n}{ErrorCodes}\PYG{o}{.}\PYG{n+na}{NO\PYGZus{}NETWORK}\PYG{o}{)} \PYG{o}{\PYGZob{}}
                \PYG{n}{showSnackbar}\PYG{o}{(}\PYG{n}{R}\PYG{o}{.}\PYG{n+na}{string}\PYG{o}{.}\PYG{n+na}{no\PYGZus{}internet\PYGZus{}connection}\PYG{o}{)}\PYG{o}{;}
                \PYG{k}{return}\PYG{o}{;}
            \PYG{o}{\PYGZcb{}}
            \PYG{n}{showSnackbar}\PYG{o}{(}\PYG{n}{R}\PYG{o}{.}\PYG{n+na}{string}\PYG{o}{.}\PYG{n+na}{unknown\PYGZus{}error}\PYG{o}{)}\PYG{o}{;}
            \PYG{n}{Log}\PYG{o}{.}\PYG{n+na}{e}\PYG{o}{(}\PYG{n}{TAG}\PYG{o}{,} \PYG{l+s}{\PYGZdq{}Sign\PYGZhy{}in error: \PYGZdq{}}\PYG{o}{,} \PYG{n}{response}\PYG{o}{.}\PYG{n+na}{getError}\PYG{o}{(}\PYG{o}{)}\PYG{o}{)}\PYG{o}{;}
        \PYG{o}{\PYGZcb{}}
    \PYG{o}{\PYGZcb{}}
\PYG{o}{\PYGZcb{}}
\end{sphinxVerbatim}


\subparagraph{Cierre de sesión.}
\label{\detokenize{dev_docs:cierre-de-sesion}}
AuthUI brinda un método \sphinxcode{signOut} simple que encapsula todo el proceso
que conlleva un cierre de sesión. El método retornar una objeto \sphinxcode{Task}
que se marca completamente una vez que todo el proceso de cierre de sesión
se ha completado.

\fvset{hllines={, ,}}%
\begin{sphinxVerbatim}[commandchars=\\\{\}]
\PYG{k+kd}{public} \PYG{k+kt}{void} \PYG{n+nf}{signOut}\PYG{o}{(}\PYG{o}{)}\PYG{o}{\PYGZob{}}
    \PYG{n}{AuthUI}\PYG{o}{.}\PYG{n+na}{getInstance}\PYG{o}{(}\PYG{o}{)}\PYG{o}{.}\PYG{n+na}{signOut}\PYG{o}{(}\PYG{k}{this}\PYG{o}{)}
            \PYG{o}{.}\PYG{n+na}{addOnCompleteListener}\PYG{o}{(}\PYG{k}{new} \PYG{n}{OnCompleteListener}\PYG{o}{\PYGZlt{}}\PYG{n}{Void}\PYG{o}{\PYGZgt{}}\PYG{o}{(}\PYG{o}{)} \PYG{o}{\PYGZob{}}
                \PYG{n+nd}{@Override}
                \PYG{k+kd}{public} \PYG{k+kt}{void} \PYG{n+nf}{onComplete}\PYG{o}{(}\PYG{n+nd}{@NonNull} \PYG{n}{Task}\PYG{o}{\PYGZlt{}}\PYG{n}{Void}\PYG{o}{\PYGZgt{}} \PYG{n}{task}\PYG{o}{)} \PYG{o}{\PYGZob{}}
                    \PYG{k}{if} \PYG{o}{(}\PYG{n}{task}\PYG{o}{.}\PYG{n+na}{isSuccessful}\PYG{o}{(}\PYG{o}{)}\PYG{o}{)} \PYG{o}{\PYGZob{}}
                        \PYG{n}{startActivity}\PYG{o}{(}\PYG{n}{MyActivity}\PYG{o}{.}\PYG{n+na}{createIntent}\PYG{o}{(}\PYG{n}{SignInActivity}\PYG{o}{.}\PYG{n+na}{this}\PYG{o}{)}\PYG{o}{)}\PYG{o}{;}
                        \PYG{n}{finish}\PYG{o}{(}\PYG{o}{)}\PYG{o}{;}
                    \PYG{o}{\PYGZcb{}} \PYG{k}{else} \PYG{o}{\PYGZob{}}
                        \PYG{c+c1}{// Falló el cierre de sesión}
                    \PYG{o}{\PYGZcb{}}
                \PYG{o}{\PYGZcb{}}
            \PYG{o}{\PYGZcb{}}\PYG{o}{)}\PYG{o}{;}
\PYG{o}{\PYGZcb{}}
\end{sphinxVerbatim}


\paragraph{ButterKnife}
\label{\detokenize{dev_docs:butterknife}}
ButterKnife es una biblioteca de viewbinding, esto quiere decir que genera
objetos view a partir de recursos, pero lo que la hace especial es que evita
el código repetitivo, como llamar a la función \sphinxcode{findViewById}. ButterKnife
ayuda a enlazar atributos, métodos y views.


\subparagraph{Configuración.}
\label{\detokenize{dev_docs:id1}}
Para comenzar a utilizar butterknife en un proyecto. Solo se agrega a Las
dependencias del \sphinxcode{buildgradle} lo siguiente:

\fvset{hllines={, ,}}%
\begin{sphinxVerbatim}[commandchars=\\\{\}]
\PYG{n+nx}{compile} \PYG{l+s+s1}{\PYGZsq{}com.jakewharton:butterknife:8.8.1\PYGZsq{}}
\PYG{n+nx}{provided} \PYG{l+s+s1}{\PYGZsq{}com.jakewharton:butterknife\PYGZhy{}compiler:8.8.1\PYGZsq{}}
\end{sphinxVerbatim}


\subparagraph{Uso.}
\label{\detokenize{dev_docs:uso}}
Para entender mejor las características que tiene ButterKnife supongamos que
tenemos una actividad con un botón que reaccione a clicks. Sin usar ButterKnife
la actividad queda como sigue:

\fvset{hllines={, ,}}%
\begin{sphinxVerbatim}[commandchars=\\\{\}]
\PYG{k+kd}{public} \PYG{k+kd}{class} \PYG{n+nc}{MainActivity} \PYG{k+kd}{extends} \PYG{n}{AppCompatActivity} \PYG{o}{\PYGZob{}}
  \PYG{k+kd}{private} \PYG{n}{TextView} \PYG{n}{mTextView}\PYG{o}{;}
  \PYG{k+kd}{private} \PYG{n}{Button} \PYG{n}{mButton}\PYG{o}{;}

  \PYG{n+nd}{@Override}
  \PYG{k+kd}{protected} \PYG{k+kt}{void} \PYG{n+nf}{onCreate}\PYG{o}{(}\PYG{n}{Bundle} \PYG{n}{savedInstanceState}\PYG{o}{)} \PYG{o}{\PYGZob{}}
      \PYG{k+kd}{super}\PYG{o}{.}\PYG{n+na}{onCreate}\PYG{o}{(}\PYG{n}{savedInstanceState}\PYG{o}{)}\PYG{o}{;}
      \PYG{n}{setContentView}\PYG{o}{(}\PYG{n}{R}\PYG{o}{.}\PYG{n+na}{layout}\PYG{o}{.}\PYG{n+na}{activity\PYGZus{}main}\PYG{o}{)}\PYG{o}{;}

      \PYG{n}{mButton} \PYG{o}{=} \PYG{o}{(}\PYG{n}{Button}\PYG{o}{)} \PYG{n}{findViewById}\PYG{o}{(}\PYG{n}{R}\PYG{o}{.}\PYG{n+na}{id}\PYG{o}{.}\PYG{n+na}{AButton}\PYG{o}{)}\PYG{o}{;}
      \PYG{n}{mTextView} \PYG{o}{=} \PYG{o}{(}\PYG{n}{TextView}\PYG{o}{)} \PYG{n}{findViewById}\PYG{o}{(}\PYG{n}{R}\PYG{o}{.}\PYG{n+na}{id}\PYG{o}{.}\PYG{n+na}{texView}\PYG{o}{)}\PYG{o}{;}

      \PYG{n}{mAOptionButton}\PYG{o}{.}\PYG{n+na}{setOnClickListener}\PYG{o}{(}\PYG{k}{new} \PYG{n}{View}\PYG{o}{.}\PYG{n+na}{OnClickListener}\PYG{o}{(}\PYG{o}{)} \PYG{o}{\PYGZob{}}
          \PYG{n+nd}{@Override}
          \PYG{k+kd}{public} \PYG{k+kt}{void} \PYG{n+nf}{onClick}\PYG{o}{(}\PYG{n}{View} \PYG{n}{view}\PYG{o}{)} \PYG{o}{\PYGZob{}}
           \PYG{c+c1}{// Haz algo}
          \PYG{o}{\PYGZcb{}}
      \PYG{o}{\PYGZcb{}}\PYG{o}{)}\PYG{o}{;}
    \PYG{o}{\PYGZcb{}}
  \PYG{o}{\PYGZcb{}}
\end{sphinxVerbatim}

Utilizando ButterKnife la actividad cambiaría a los siguiente:

\fvset{hllines={, ,}}%
\begin{sphinxVerbatim}[commandchars=\\\{\}]
\PYG{k+kd}{public} \PYG{k+kd}{class} \PYG{n+nc}{MainActivity} \PYG{k+kd}{extends} \PYG{n}{AppCompatActivity} \PYG{o}{\PYGZob{}}
  \PYG{n+nd}{@BindView}\PYG{o}{(}\PYG{n}{R}\PYG{o}{.}\PYG{n+na}{id}\PYG{o}{.}\PYG{n+na}{textView}\PYG{o}{)}
  \PYG{n}{TextView} \PYG{n}{mTextView}\PYG{o}{;}

  \PYG{n+nd}{@Override}
  \PYG{k+kd}{protected} \PYG{k+kt}{void} \PYG{n+nf}{onCreate}\PYG{o}{(}\PYG{n}{Bundle} \PYG{n}{savedInstanceState}\PYG{o}{)} \PYG{o}{\PYGZob{}}
      \PYG{k+kd}{super}\PYG{o}{.}\PYG{n+na}{onCreate}\PYG{o}{(}\PYG{n}{savedInstanceState}\PYG{o}{)}\PYG{o}{;}
      \PYG{n}{setContentView}\PYG{o}{(}\PYG{n}{R}\PYG{o}{.}\PYG{n+na}{layout}\PYG{o}{.}\PYG{n+na}{activity\PYGZus{}main}\PYG{o}{)}\PYG{o}{;}
      \PYG{c+c1}{// Enlaza butterknife a la actividad}
      \PYG{n}{ButterKnife}\PYG{o}{.}\PYG{n+na}{bind}\PYG{o}{(}\PYG{k}{this}\PYG{o}{)}\PYG{o}{;}
  \PYG{o}{\PYGZcb{}}

  \PYG{n+nd}{@OnClick}\PYG{o}{(}\PYG{n}{R}\PYG{o}{.}\PYG{n+na}{id}\PYG{o}{.}\PYG{n+na}{AButton}\PYG{o}{)}
  \PYG{k+kd}{public} \PYG{k+kt}{void} \PYG{n+nf}{OnClickMyButton}\PYG{o}{(}\PYG{o}{)} \PYG{o}{\PYGZob{}}
    \PYG{c+c1}{// Haz algo}
  \PYG{o}{\PYGZcb{}}

\PYG{o}{\PYGZcb{}}
\end{sphinxVerbatim}

Al comparar se nota como evita duplicar código, y el código de los escuchadores
es mucho más simple y parece explicarse por sí mismo.


\paragraph{Volley}
\label{\detokenize{dev_docs:volley}}
Volley es una biblioteca HTTP que facilita el acceso a la red en aplicaciones
de Android. Algunos de sus beneficios son los siguientes:
\begin{itemize}
\item {} 
Programación automática de peticiones a través de la red.

\item {} 
Múltiples conexiones concurrentes.

\item {} 
Soporte para priorizar peticiones.

\item {} 
Una API para cancelar peticiones.

\item {} 
Personalizable, por ejemplo, se puede añadir la opción de reintentar un petición.

\end{itemize}


\subparagraph{Configuración.}
\label{\detokenize{dev_docs:id2}}
La manera más sencilla de añadir Volley a un proyecto es a través de la dependencia
en el \sphinxcode{buildgradle} de la aplicación.

\fvset{hllines={, ,}}%
\begin{sphinxVerbatim}[commandchars=\\\{\}]
\PYG{n+nx}{dependencies} \PYG{p}{\PYGZob{}}
\PYG{p}{...}
\PYG{n+nx}{compile} \PYG{l+s+s1}{\PYGZsq{}com.android.volley:volley:1.1.0\PYGZsq{}}
\PYG{p}{\PYGZcb{}}
\end{sphinxVerbatim}


\subparagraph{Enviando una petición simple.}
\label{\detokenize{dev_docs:enviando-una-peticion-simple}}
A un alto nivel, se utiliza Volley creando un objeto \sphinxcode{RequestQueue} y
pasándole objetos \sphinxcode{Request}. \sphinxcode{RequestQueue} administra hilos trabajadores
para ejecutar operaciones en red, leyendo y escribiendo desde el caché, y
parseando respuestas. Las peticiones hacen el parseo de respuestas en crudo
y Volley se ocupa de remitir la respuesta parseada de regreso al hilo principal
para su entrega.

Volley provee una método \sphinxcode{VolleynewRequestQueue} que prepara una
\sphinxcode{RequestQueue}, usando valores por defecto, e inicia la cola. El
siguiente ejemplo presenta cómo se crea una nueva cola de peticiones,
así como una simple petición \sphinxcode{GET} a \sphinxcode{www.google.com}, recibiendo  la respuesta
en una cadena y mostrándola en un \sphinxcode{TextView}.

\fvset{hllines={, ,}}%
\begin{sphinxVerbatim}[commandchars=\\\{\}]
\PYG{n+nd}{@BindView}\PYG{o}{(}\PYG{n}{R}\PYG{o}{.}\PYG{n+na}{id}\PYG{o}{.}\PYG{n+na}{text}\PYG{o}{)}
\PYG{n}{TextView} \PYG{n}{mTextView}\PYG{o}{;}

\PYG{c+c1}{// Instancia a la cola de peticones}
\PYG{n}{RequestQueue} \PYG{n}{queue} \PYG{o}{=} \PYG{n}{Volley}\PYG{o}{.}\PYG{n+na}{newRequestQueue}\PYG{o}{(}\PYG{k}{this}\PYG{o}{)}\PYG{o}{;}
\PYG{n}{String} \PYG{n}{url} \PYG{o}{=}\PYG{l+s}{\PYGZdq{}http://www.google.com\PYGZdq{}}\PYG{o}{;}

\PYG{c+c1}{// Solicita una respuesta en formato de cadena haciendo un GET al URL}
\PYG{n}{StringRequest} \PYG{n}{stringRequest} \PYG{o}{=} \PYG{k}{new} \PYG{n}{StringRequest}\PYG{o}{(}\PYG{n}{Request}\PYG{o}{.}\PYG{n+na}{Method}\PYG{o}{.}\PYG{n+na}{GET}\PYG{o}{,} \PYG{n}{url}\PYG{o}{,}
          \PYG{k}{new} \PYG{n}{Response}\PYG{o}{.}\PYG{n+na}{Listener}\PYG{o}{\PYGZlt{}}\PYG{n}{String}\PYG{o}{\PYGZgt{}}\PYG{o}{(}\PYG{o}{)} \PYG{o}{\PYGZob{}}
  \PYG{n+nd}{@Override}
  \PYG{k+kd}{public} \PYG{k+kt}{void} \PYG{n+nf}{onResponse}\PYG{o}{(}\PYG{n}{String} \PYG{n}{response}\PYG{o}{)} \PYG{o}{\PYGZob{}}
      \PYG{n}{mTextView}\PYG{o}{.}\PYG{n+na}{setText}\PYG{o}{(}\PYG{l+s}{\PYGZdq{}Response is: \PYGZdq{}}\PYG{o}{+} \PYG{n}{response}\PYG{o}{.}\PYG{n+na}{substring}\PYG{o}{(}\PYG{l+m+mi}{0}\PYG{o}{,}\PYG{l+m+mi}{500}\PYG{o}{)}\PYG{o}{)}\PYG{o}{;}
  \PYG{o}{\PYGZcb{}}
\PYG{o}{\PYGZcb{}}\PYG{o}{,} \PYG{k}{new} \PYG{n}{Response}\PYG{o}{.}\PYG{n+na}{ErrorListener}\PYG{o}{(}\PYG{o}{)} \PYG{o}{\PYGZob{}}
  \PYG{n+nd}{@Override}
  \PYG{k+kd}{public} \PYG{k+kt}{void} \PYG{n+nf}{onErrorResponse}\PYG{o}{(}\PYG{n}{VolleyError} \PYG{n}{error}\PYG{o}{)} \PYG{o}{\PYGZob{}}
      \PYG{n}{mTextView}\PYG{o}{.}\PYG{n+na}{setText}\PYG{o}{(}\PYG{l+s}{\PYGZdq{}That didn\PYGZsq{}t work!\PYGZdq{}}\PYG{o}{)}\PYG{o}{;}
  \PYG{o}{\PYGZcb{}}
\PYG{o}{\PYGZcb{}}\PYG{o}{)}\PYG{o}{;}
\PYG{c+c1}{// Añade la petición a la cola RequestQueue}
\PYG{n}{queue}\PYG{o}{.}\PYG{n+na}{add}\PYG{o}{(}\PYG{n}{stringRequest}\PYG{o}{)}\PYG{o}{;}
\end{sphinxVerbatim}

Volley siempre entrega respuestas parseadas sobre el hilo principal. Ejecutarlo
sobre el hilo principal es conveniente para poblar elementos de la interfaz de
usuario con los datos recibidos.


% \subsection{referencias}
% \label{\detokenize{dev_docs:referencias}}
% \sphinxurl{https://developer.android.com/studio/intro/index.html?hl=es-419}
% \sphinxurl{https://medium.com/@petehouston/compile-local-jar-files-with-gradle-a078e5c7a520}
% \sphinxurl{https://github.com/firebase/FirebaseUI-Android/blob/master/auth/README.md}
% \sphinxurl{http://jakewharton.github.io/butterknife/}


\subsubsection{Descripción de las clases principales}
\label{\detokenize{dev_docs:documentacion-para-desarrolladores}}

La aplicación está compuesta como cualquier otra, por una
carpeta \textit{manifests}, \textit{res} y \textit{java}.
La carpeta \textit{java} incluye los paquetes \texttt{com.lar.cloudnao.utilities} y \texttt{com.lar.cloudnao}.


\paragraph{Paquete \texttt{utilities}}
d
Es un paquete compuesto por clases e interfaces que tienen como
función realizar tareas asíncronas para manejar imágenes del robot,
enviar peticiones a la API REST, manejar la IU, etcétera.

\codedocumentation{\texttt{Interface }\sphinxcode{OnRobotNameClickListener}}

Una interfaz con un método abstracto que sirve como callback
cuando un usuario da click sobre alguno de los robots obtenidos desde Firebase y mostrados en un \texttt{RecyclerView}.

\codedocumentation{\texttt{Interface }\sphinxcode{VisionRequestCompleted}}

Interfaz con un método abstracto que se llama dentro de una tarea asíncrona para procesar un objeto que se obtiene de la respuesta
a la solicitud del recurso \texttt{vision} de la API REST de CloudNAO.
Hereda métodos de \texttt{AsyncTask}.

\codedocumentation{\texttt{Class }\sphinxcode{BatteryStatusTask}}

Esta clase obtiene el nivel de batería actual en el robot a través de la lectura del valor en 
\texttt{ALMemory}. Utiliza la clase \texttt{AsyncTask}.
Muestra una \texttt{ImageView} que representa el estado de la batería y un \texttt{TextView} con el porcentaje de batería restante.

\codedocumentation{\texttt{Class }\sphinxcode{BoundingBoxesUtils}}

Una clase estática con un método para dibujar cuadros delimitadores 
sobre un \texttt{ImageView}. Para dibujar el cuadro delimitador 
recibe un arreglo con los valores necesarios para generar las 
coordenadas que se usan para dibujar.

\codedocumentation{\texttt{Class }\sphinxcode{ImageTask}}

Clase pública con métodos asíncronos para obtener una imagen desde la cámara del robot de manera remota.  Obtiene un objeto que contiene la imagen, con el espacio de color por defecto del robot NAO (YUV) en un arreglo de bytes, convierte ese arreglo en una imagen jpeg, para después cargar ese objeto jpeg en un bitmap que sirve para llenar el \texttt{ImageView} que muestra la imagen enviada por el robot. Es una subclase
de \texttt{AsyncTask}.

\codedocumentation{\texttt{Class }\sphinxcode{ItemsListAdapter}}

Esta clase es el \texttt{Adapter} para el \texttt{ListView} cuyos elementos 
están compuestos por un \texttt{CheckBox} y un \texttt{TextView}. Se 
utiliza al mostrar la lista de datos que se quieren guardar en 
Firebase desde \texttt{ALMemory}.

\codedocumentation{\texttt{Class }\sphinxcode{MyPagerAdapter}}

Subclase de \texttt{PagerAdapter} para el \texttt{ViewPager} del menú
principal de la aplicación, por ahora tiene un tamaño máximo de 4. 

\codedocumentation{\texttt{Class }\sphinxcode{ProxySubscriberItem}}

Clase pública para modelar los objetos de la lista con los datos de \texttt{ALMemory} que se guardarán en Firebase. \texttt{ItemsListAdapter} contiene un objeto \texttt{List} de
instancias de \texttt{ProxySu	bscriberItem}. Los
atributos son un  valor booleano \texttt{mChecked} y una cadena \texttt{mProxyName} para saber si un ítem está marcado y el nombre
de la llave en \texttt{ALMemory}, respectivamente.

\codedocumentation{\texttt{Class }\sphinxcode{RecyclerViewAdapter}}

La clase pública del \texttt{Adapter} para el \texttt{RecyclerView} que mostrará la lista de robots obtenidos desde Firebase y que le pertenecen al
usuario que inció sesión en la aplicación.

\codedocumentation{\texttt{Class }\sphinxcode{RecyclerViewHolders}}

La clase que modela a los \texttt{ViewHolders} que representan cada robot obtenido de Firebase.
Aquí se utiliza el método abstracto \texttt{onClickRobotName} como callback para ejecutar una acción al dar click sobre un elemento del \texttt{RecyclerView}.

\codedocumentation{\texttt{Class }\sphinxcode{RobotFromFirebase}}

La clase que modela los objetos de la lista de robots obtenidos desde Firebase, cada objeto contiene un identificador único y el nombre que el usuario asignó.

\codedocumentation{\texttt{Class }\sphinxcode{RobotLogs}}

Clase con métodos asíncronos para almacenar valores de \texttt{ALMemory} en Firebase, obtiene valores del robot y los almacena en una ubicación en Firebase con un \texttt{push()}, por lo que se genera un historial de logs.

\codedocumentation{\texttt{Class }\sphinxcode{ViewHolder}}

Clase que modelos los \texttt{ViewHolders} de la lista de datos que se desean guardar en Firebase, están compuestos por un \texttt{CheckBox} y un \texttt{TextView} que describe a cada elemento.

\codedocumentation{\texttt{Class }\sphinxcode{VisionListElement}}

Clase pública que modela los objetos dentro de una lista que muestra resultados de las actividades de reconocimiento de rostros y objetos. Cada instancia tiene un identificador, que puede ser el nombre del sujeto reconocido , la confianza y las coordenadas de que delimitan al elemento encontrado (un cuadro que encierra un rostro, o un rectángulo que acota un objeto).

\codedocumentation{\texttt{Class }\sphinxcode{VisionRESTRequestsTask}}

La clase con métodos asíncronos para hacer peticiones a la API REST 
de CloudNAO. Recibe la imagen desde un \texttt{ImageView}, del que obtiene un 
bitmap y genera una cadena en base 64 que representa esa imagen. 
Después define la estructura y valores del JSON que se envía en una 
petición usando la biblioteca Volley. A partir de la respuesta 
obtenida ejecuta la función callback \texttt{onVisionRequestCompleted} 
para mostrar los resultados en el contexto de donde se solicitó.



\codedocumentation{\texttt{Class }\sphinxcode{VolleyResponseUtils}}

Clase estática para manejar la repuesta de \texttt{Volley}, que puede ser un error o puede ser la estructura que se espera.

\codedocumentation{\texttt{Enum ModelObjectViewPager}}

Un enumerado de Java que representa todas las posibles páginas del ViewPager. Se utiliza en el ménu de la aplicación. En este enumerado se almacenan el identificador del \texttt{Layout} de cada página y un título que le corresponde.


\paragraph{Actividades de la aplicación}

\codedocumentation{\texttt{Class FaceRecognitionActivity}}

La clase de la actividad que realiza el reconocimiento de rostros. Implementa la interfaz \texttt{VisionRequestCompleted} para llamar a su método abstracto desde la clase \texttt{VisionRESTRequestsTask}.


\codedocumentation{\texttt{Class MenuActivity}}

La actividad que muestra el menú principal donde se elige entre otras actividades para interactuar con el robot. Muestra un \texttt{ViewPager} para navegar y elegir la actividad deseada.


\codedocumentation{\texttt{Class ObjectDetectionActivity}}

La actividad encargada de la funcionalidad de la detección de objetos, implementa la interfaz \texttt{VisionRequestCompleted} para ejecutar su método \texttt{VisionRequestCompleted.onVisionRequestComple\\ted(Object)} como callback en \texttt{VisionRESTRequestsTask}.


\codedocumentation{\texttt{Class OCRTranslationActivity}}


Clase de la actividad que se encarga del procesamiento de una imagen 
para detección de texto y su traducción. Implenta la interfaz 
\texttt{VisionRequestCompleted}.

\codedocumentation{\texttt{Class RemoteControllerActivity}}

La actividad para el control remoto del robot. Además de poder 
controlar algunos movimientos como el caminado o el cambio de postura, 
se puede visualizar una imagen en vivo de la cámara del robot, así 
como permite enviar valores de \texttt{ALMemory} a una base de datos en la nube 
a través de Firebase.

\codedocumentation{\texttt{Class Robot}}

No es una actividad pero es la clase pública principal para ejecutar módulos del framework de NAOqi en la aplicación. Es una interfaz para conectarse al robot y ejecutar algunos módulos de NAOqi. Usa el patrón de diseño Singleton para que solo exista una instancia del objeto y funcione como una variable global.


\codedocumentation{\texttt{Class SelectRobotActivity}}

Actividad que muestra los robots registrados por el usuario así como la opción para añadir nuevos robots, o conectarse a uno de la lista. Implementa el método abstracto \texttt{OnRobotNameClickListener.onClickRobotName(String, String)}.

%
%\begin{itemize}
%\item {} 
%\sphinxcode{SignInActivity}, la actividad encargada de todo el flujo para el inicio de sesión. Utiliza FirebaseUI.
%
%\item {} 
%\sphinxcode{SelectRobotActivity}, la actividad que muestra los robots guardados por el usuario que inició sesión. Además aquí se pueden añadir nuevos. Escribe y lee de Firebase Realtime Database. Aquí se crea la instancia única del robot con el que se desea conectar, se crea la sesión y los proxies a los módulos de NAOqi.
%
%\item {} 
%\sphinxcode{MenuActivity}, esta actividad es un menú para elegir entre las actividades restantes. Es un ViewPager, que reacciona al click del usuario.
%
%\item {} 
%\sphinxcode{RemoteControllerActivity}, la actividad que corresponde al control remoto. Utiliza módulos de NAOqi, y la base de datos de Firebase para escribir los logs.
%
%\item {} 
%\sphinxcode{FaceRecognitionActivity}, la actividad que detecta rostros en una imagen. Utiliza Volley para conectarse a la API RESTful de CloudNAO.
%
%\item {} 
%\sphinxcode{OCRTranslationActivity}, esta actividad reconoce texto en una imagen y lo traduce. Utiliza Volley para conectarse a la API RESTful de CloudNAO.
%
%\item {} 
%\sphinxcode{ObjectDetectionActivity}, la actividad que reconoce objetos en una imagen. Utiliza Volley para conectarse a la API RESTful de CloudNAO.
%
%\item {} 
%\sphinxcode{Robot}, es la clase que define al objeto robot. Los métodos para ejecutar módulos de NAOqi, y obtener imágenes de la cámara del robot, valores de la memoria, etc.
%
%\end{itemize}


% \textbf{SignInActivity}
% \label{\detokenize{dev_docs:signinactivity}}\index{SignInActivity (Java class)}

% \begin{fulllineitems}
% \phantomsection\label{\detokenize{dev_docs:com.lar.cloudnao.SignInActivity}}\pysigline{public class \sphinxbfcode{SignInActivity} extends AppCompatActivity}
% La actividad para iniciar sesión en la aplicación, utiliza la biblioteca Firebae UI, para facilitar el flujo del inicio de sesión.

% \end{fulllineitems}



% \textbf{Métodos}
% \label{\detokenize{dev_docs:methods}}

% \textbf{OnClickSignInButton}
% \label{\detokenize{dev_docs:onclicksigninbutton}}\index{OnClickSignInButton() (Java method)}

% \begin{fulllineitems}
% \phantomsection\label{\detokenize{dev_docs:com.lar.cloudnao.SignInActivity.OnClickSignInButton()}}\pysiglinewithargsret{public void \sphinxbfcode{OnClickSignInButton}}{}{}
% Agrega un escuchador al botón signInButton para crear un intent sign-in. Este intent es provisto por la biblioteca de Firebase AuthUI. Esto construye una actividad sign-in, y pasa un entero RC\_SIGN\_IN, que es una variable donde se guarda la respuesta de la actividad sign-in. Cuando esa actividad finaliza su trabajo enviará de regreso onActivityResult que contiene el código de petición y otra información. Esa información es un intent, que podemos usar para descifrar que pasó.

% \end{fulllineitems}



% \textbf{createIntent}
% \label{\detokenize{dev_docs:createintent}}\index{createIntent(Context) (Java method)}

% \begin{fulllineitems}
% \phantomsection\label{\detokenize{dev_docs:com.lar.cloudnao.SignInActivity.createIntent(Context)}}\pysiglinewithargsret{public static Intent \sphinxbfcode{createIntent}}{Context\sphinxstyleemphasis{ context}}{}
% Este método se utiliza por la actividad sign-in en caso de que no se tenga un usuario válido.

% \end{fulllineitems}



% \textbf{onActivityResult}
% \label{\detokenize{dev_docs:onactivityresult}}\index{onActivityResult(int, int, Intent) (Java method)}

% \begin{fulllineitems}
% \phantomsection\label{\detokenize{dev_docs:com.lar.cloudnao.SignInActivity.onActivityResult(int, int, Intent)}}\pysiglinewithargsret{protected void \sphinxbfcode{onActivityResult}}{int\sphinxstyleemphasis{ requestCode}, int\sphinxstyleemphasis{ resultCode}, Intent\sphinxstyleemphasis{ data}}{}
% Verifica y maneja el resultado enviado por la actividad sign-in

% \end{fulllineitems}



% \textbf{onCreate}
% \label{\detokenize{dev_docs:oncreate}}\index{onCreate(Bundle) (Java method)}

% \begin{fulllineitems}
% \phantomsection\label{\detokenize{dev_docs:com.lar.cloudnao.SignInActivity.onCreate(Bundle)}}\pysiglinewithargsret{protected void \sphinxbfcode{onCreate}}{Bundle\sphinxstyleemphasis{ savedInstanceState}}{}
% Es el método que se llama al crear la actividad, si el usuario ya ha iniciado sesión ejecuta la actividad de la clase SelectRobotActivity y finaliza la actual.

% \end{fulllineitems}



% \textbf{SelectRobotActivity}
% \label{\detokenize{dev_docs:selectrobotactivity}}\index{SelectRobotActivity (Java class)}

% \begin{fulllineitems}
% \phantomsection\label{\detokenize{dev_docs:com.lar.cloudnao.SelectRobotActivity}}\pysigline{public class \sphinxbfcode{SelectRobotActivity} extends AppCompatActivity implements OnRobotNameClickListener}
% Actividad que muestra los robots registrados por el usuario así como la opción para añadir nuevos robots, o conectarse a uno de la lista. Implementa el método abstracto \sphinxcode{OnRobotNameClickListeneronClickRobotName(String,String)}.

% \end{fulllineitems}



% \subsubsection{Atributos}
% \label{\detokenize{dev_docs:fields}}

% \paragraph{MyRobot}
% \label{\detokenize{dev_docs:myrobot}}\index{MyRobot (Java field)}

% \begin{fulllineitems}
% \phantomsection\label{\detokenize{dev_docs:com.lar.cloudnao.SelectRobotActivity.MyRobot}}\pysigline{public {\hyperref[\detokenize{dev_docs:com.lar.cloudnao.Robot}]{\sphinxcrossref{Robot}}} \sphinxbfcode{MyRobot}}
% Llama a las instancia global de {\hyperref[\detokenize{dev_docs:com.lar.cloudnao.Robot}]{\sphinxcrossref{\sphinxcode{Robot}}}}

% \end{fulllineitems}



% \paragraph{mNewRobotEditText}
% \label{\detokenize{dev_docs:mnewrobotedittext}}\index{mNewRobotEditText (Java field)}

% \begin{fulllineitems}
% \phantomsection\label{\detokenize{dev_docs:com.lar.cloudnao.SelectRobotActivity.mNewRobotEditText}}\pysigline{ EditText \sphinxbfcode{mNewRobotEditText}}
% Campo para añadir el nombre de un nuevo robot.

% \end{fulllineitems}



% \paragraph{robotListFromFirebaseRecyclerView}
% \label{\detokenize{dev_docs:robotlistfromfirebaserecyclerview}}\index{robotListFromFirebaseRecyclerView (Java field)}

% \begin{fulllineitems}
% \phantomsection\label{\detokenize{dev_docs:com.lar.cloudnao.SelectRobotActivity.robotListFromFirebaseRecyclerView}}\pysigline{ RecyclerView \sphinxbfcode{robotListFromFirebaseRecyclerView}}
% La lista de los robots del usuario, se llena desde Firebase.

% \end{fulllineitems}



% \subsubsection{Métodos}
% \label{\detokenize{dev_docs:id3}}

% \paragraph{addNewRobot}
% \label{\detokenize{dev_docs:addnewrobot}}\index{addNewRobot() (Java method)}

% \begin{fulllineitems}
% \phantomsection\label{\detokenize{dev_docs:com.lar.cloudnao.SelectRobotActivity.addNewRobot()}}\pysiglinewithargsret{public void \sphinxbfcode{addNewRobot}}{}{}
% Escuchador para el botón que permite a los usuarios añadir un nuevo robot.

% \end{fulllineitems}



% \paragraph{onClickRobotName}
% \label{\detokenize{dev_docs:onclickrobotname}}\index{onClickRobotName(String, String) (Java method)}

% \begin{fulllineitems}
% \phantomsection\label{\detokenize{dev_docs:com.lar.cloudnao.SelectRobotActivity.onClickRobotName(String, String)}}\pysiglinewithargsret{public void \sphinxbfcode{onClickRobotName}}{\sphinxhref{http://docs.oracle.com/javase/8/docs/api/java/lang/String.html}{String}\sphinxstyleemphasis{ robotName}, \sphinxhref{http://docs.oracle.com/javase/8/docs/api/java/lang/String.html}{String}\sphinxstyleemphasis{ robotUID}}{}
% Define al método abstracto \sphinxcode{OnRobotNameClickListeneronClickRobotName(String,String)} para servir como callback cuando un usuario da clcik en un robot de la lista. Muestra un cuadro de diálogo con TextView para agregar la dirección IP del robot y ejecutar \sphinxcode{ConnectionTask}.

% \end{fulllineitems}



% \paragraph{onCreate}
% \label{\detokenize{dev_docs:id4}}\index{onCreate(Bundle) (Java method)}

% \begin{fulllineitems}
% \phantomsection\label{\detokenize{dev_docs:com.lar.cloudnao.SelectRobotActivity.onCreate(Bundle)}}\pysiglinewithargsret{protected void \sphinxbfcode{onCreate}}{Bundle\sphinxstyleemphasis{ savedInstanceState}}{}
% Inicia la actividad, verifica que el usuario haya iniciado sesión. Si el usuario es nuevo, entonces se añade a la ubicación /users en la base de datos de Firebase. Luego, agrega un escuchador a la ubicación de la base de datos que contiene a los robots del usuario. Llama al método \sphinxcode{getRobotFirebaseDataSnapshot()} para actualizar {\hyperref[\detokenize{dev_docs:com.lar.cloudnao.SelectRobotActivity.robotListFromFirebaseRecyclerView}]{\sphinxcrossref{\sphinxcode{robotListFromFirebaseRecyclerView}}}}.

% \end{fulllineitems}



% \paragraph{onCreateOptionsMenu}
% \label{\detokenize{dev_docs:oncreateoptionsmenu}}\index{onCreateOptionsMenu(Menu) (Java method)}

% \begin{fulllineitems}
% \phantomsection\label{\detokenize{dev_docs:com.lar.cloudnao.SelectRobotActivity.onCreateOptionsMenu(Menu)}}\pysiglinewithargsret{public boolean \sphinxbfcode{onCreateOptionsMenu}}{Menu\sphinxstyleemphasis{ menu}}{}
% Crea un menú de opciones para que por ahora contiene un elemento, cerrar la sesión del usuario.

% \end{fulllineitems}



% \paragraph{onOptionsItemSelected}
% \label{\detokenize{dev_docs:onoptionsitemselected}}\index{onOptionsItemSelected(MenuItem) (Java method)}

% \begin{fulllineitems}
% \phantomsection\label{\detokenize{dev_docs:com.lar.cloudnao.SelectRobotActivity.onOptionsItemSelected(MenuItem)}}\pysiglinewithargsret{public boolean \sphinxbfcode{onOptionsItemSelected}}{MenuItem\sphinxstyleemphasis{ item}}{}
% Añade un escuchador para cuando se selecciona un elemento en el menú de opciones. Por ahora solo cerrar sesión. Llama al método {\hyperref[\detokenize{dev_docs:com.lar.cloudnao.SelectRobotActivity.signOut()}]{\sphinxcrossref{\sphinxcode{signOut()}}}}.

% \end{fulllineitems}



% \paragraph{signOut}
% \label{\detokenize{dev_docs:signout}}\index{signOut() (Java method)}

% \begin{fulllineitems}
% \phantomsection\label{\detokenize{dev_docs:com.lar.cloudnao.SelectRobotActivity.signOut()}}\pysiglinewithargsret{public void \sphinxbfcode{signOut}}{}{}
% Cierra la sesión através de la biblioteca AuthUI

% \end{fulllineitems}



% \textbf{MenuActivity}
% \label{\detokenize{dev_docs:menuactivity}}\index{MenuActivity (Java class)}

% \begin{fulllineitems}
% \phantomsection\label{\detokenize{dev_docs:com.lar.cloudnao.MenuActivity}}\pysigline{public class \sphinxbfcode{MenuActivity} extends AppCompatActivity}
% La actividad que muestra el menú donde se elige entre otras actividades para interactuar con el robot. Muestra un ViewPager para navegar y elegir la actividad deseada.

% \end{fulllineitems}



% \subsubsection{Atributos}
% \label{\detokenize{dev_docs:id5}}

% \paragraph{MyRobot}
% \label{\detokenize{dev_docs:id6}}\index{MyRobot (Java field)}

% \begin{fulllineitems}
% \phantomsection\label{\detokenize{dev_docs:com.lar.cloudnao.MenuActivity.MyRobot}}\pysigline{public {\hyperref[\detokenize{dev_docs:com.lar.cloudnao.Robot}]{\sphinxcrossref{Robot}}} \sphinxbfcode{MyRobot}}
% La instancia del singleton de la clase {\hyperref[\detokenize{dev_docs:com.lar.cloudnao.Robot}]{\sphinxcrossref{\sphinxcode{Robot}}}}

% \end{fulllineitems}



% \paragraph{mCurrentPageSelected}
% \label{\detokenize{dev_docs:mcurrentpageselected}}\index{mCurrentPageSelected (Java field)}

% \begin{fulllineitems}
% \phantomsection\label{\detokenize{dev_docs:com.lar.cloudnao.MenuActivity.mCurrentPageSelected}}\pysigline{public int \sphinxbfcode{mCurrentPageSelected}}
% Un número que identifica la página que se muestra actualmente.

% \end{fulllineitems}



% \subsubsection{Métodos}
% \label{\detokenize{dev_docs:id7}}

% \paragraph{onClickMenuButton}
% \label{\detokenize{dev_docs:onclickmenubutton}}\index{onClickMenuButton() (Java method)}

% \begin{fulllineitems}
% \phantomsection\label{\detokenize{dev_docs:com.lar.cloudnao.MenuActivity.onClickMenuButton()}}\pysiglinewithargsret{public void \sphinxbfcode{onClickMenuButton}}{}{}
% Define el escuchador para el botón de selección de actividad dentro del menú de actividades.

% \end{fulllineitems}



% \paragraph{onCreate}
% \label{\detokenize{dev_docs:id8}}\index{onCreate(Bundle) (Java method)}

% \begin{fulllineitems}
% \phantomsection\label{\detokenize{dev_docs:com.lar.cloudnao.MenuActivity.onCreate(Bundle)}}\pysiglinewithargsret{protected void \sphinxbfcode{onCreate}}{Bundle\sphinxstyleemphasis{ savedInstanceState}}{}
% Inicializa la actividad, el ViewPager y el Adapter para escuchar cuando se cambia de página.

% \end{fulllineitems}



% \paragraph{onCreateOptionsMenu}
% \label{\detokenize{dev_docs:id9}}\index{onCreateOptionsMenu(Menu) (Java method)}

% \begin{fulllineitems}
% \phantomsection\label{\detokenize{dev_docs:com.lar.cloudnao.MenuActivity.onCreateOptionsMenu(Menu)}}\pysiglinewithargsret{public boolean \sphinxbfcode{onCreateOptionsMenu}}{Menu\sphinxstyleemphasis{ menu}}{}
% Crea un menú de opciones, en la barra superior y contar con una opción para desconectarse del robot (cerrando la sesión de NAOqi)

% \end{fulllineitems}



% \paragraph{onOptionsItemSelected}
% \label{\detokenize{dev_docs:id10}}\index{onOptionsItemSelected(MenuItem) (Java method)}

% \begin{fulllineitems}
% \phantomsection\label{\detokenize{dev_docs:com.lar.cloudnao.MenuActivity.onOptionsItemSelected(MenuItem)}}\pysiglinewithargsret{public boolean \sphinxbfcode{onOptionsItemSelected}}{MenuItem\sphinxstyleemphasis{ item}}{}
% Ejecuta la función \sphinxcode{closeRobotConnection()} si se selecciona la opción de desconectarse del robot desde el menú de opciones.

% \end{fulllineitems}



% \paragraph{startOCRActivity}
% \label{\detokenize{dev_docs:startocractivity}}\index{startOCRActivity() (Java method)}

% \begin{fulllineitems}
% \phantomsection\label{\detokenize{dev_docs:com.lar.cloudnao.MenuActivity.startOCRActivity()}}\pysiglinewithargsret{public void \sphinxbfcode{startOCRActivity}}{}{}
% Inicia la actividad {\hyperref[\detokenize{dev_docs:com.lar.cloudnao.OCRTranslationActivity}]{\sphinxcrossref{\sphinxcode{OCRTranslationActivity}}}}

% \end{fulllineitems}



% \paragraph{startObjectDetectionActivity}
% \label{\detokenize{dev_docs:startobjectdetectionactivity}}\index{startObjectDetectionActivity() (Java method)}

% \begin{fulllineitems}
% \phantomsection\label{\detokenize{dev_docs:com.lar.cloudnao.MenuActivity.startObjectDetectionActivity()}}\pysiglinewithargsret{public void \sphinxbfcode{startObjectDetectionActivity}}{}{}
% Inicia la actividad {\hyperref[\detokenize{dev_docs:com.lar.cloudnao.ObjectDetectionActivity}]{\sphinxcrossref{\sphinxcode{ObjectDetectionActivity}}}}

% \end{fulllineitems}



% \textbf{RemoteControllerActivity}
% \label{\detokenize{dev_docs:remotecontrolleractivity}}\index{RemoteControllerActivity (Java class)}

% \begin{fulllineitems}
% \phantomsection\label{\detokenize{dev_docs:com.lar.cloudnao.RemoteControllerActivity}}\pysigline{public class \sphinxbfcode{RemoteControllerActivity} extends AppCompatActivity}
% La actividad para el control remoto del robot. Además de poder controlar algunos movimientos como el caminado o el cambio de postura, se puede visualizar una imagen en vivo de la cámara del robot, así como permite enviar valores de ALMemory a una base de datos en la nube a través de Firebase.

% \end{fulllineitems}



% \subsubsection{Atributos}
% \label{\detokenize{dev_docs:id11}}

% \paragraph{MyRobot}
% \label{\detokenize{dev_docs:id12}}\index{MyRobot (Java field)}

% \begin{fulllineitems}
% \phantomsection\label{\detokenize{dev_docs:com.lar.cloudnao.RemoteControllerActivity.MyRobot}}\pysigline{public {\hyperref[\detokenize{dev_docs:com.lar.cloudnao.Robot}]{\sphinxcrossref{Robot}}} \sphinxbfcode{MyRobot}}
% Llama la instacia global de {\hyperref[\detokenize{dev_docs:com.lar.cloudnao.Robot}]{\sphinxcrossref{\sphinxcode{Robot}}}}

% \end{fulllineitems}



% \paragraph{mBatteryImageView}
% \label{\detokenize{dev_docs:mbatteryimageview}}\index{mBatteryImageView (Java field)}

% \begin{fulllineitems}
% \phantomsection\label{\detokenize{dev_docs:com.lar.cloudnao.RemoteControllerActivity.mBatteryImageView}}\pysigline{ ImageView \sphinxbfcode{mBatteryImageView}}
% Una imagen con el estado de la batería.

% \end{fulllineitems}



% \paragraph{mBatteryTextView}
% \label{\detokenize{dev_docs:mbatterytextview}}\index{mBatteryTextView (Java field)}

% \begin{fulllineitems}
% \phantomsection\label{\detokenize{dev_docs:com.lar.cloudnao.RemoteControllerActivity.mBatteryTextView}}\pysigline{ TextView \sphinxbfcode{mBatteryTextView}}
% El nivel actual de batería del robot.

% \end{fulllineitems}



% \paragraph{mBatteryWaitTimer}
% \label{\detokenize{dev_docs:mbatterywaittimer}}\index{mBatteryWaitTimer (Java field)}

% \begin{fulllineitems}
% \phantomsection\label{\detokenize{dev_docs:com.lar.cloudnao.RemoteControllerActivity.mBatteryWaitTimer}}\pysigline{ \sphinxhref{http://docs.oracle.com/javase/8/docs/api/java/util/Timer.html}{Timer} \sphinxbfcode{mBatteryWaitTimer}}
% El Timer encargado de repetir {\hyperref[\detokenize{dev_docs:com.lar.cloudnao.RemoteControllerActivity.mTimerBatteryStatusTask}]{\sphinxcrossref{\sphinxcode{mTimerBatteryStatusTask}}}}

% \end{fulllineitems}



% \paragraph{mCameraWaitTimer}
% \label{\detokenize{dev_docs:mcamerawaittimer}}\index{mCameraWaitTimer (Java field)}

% \begin{fulllineitems}
% \phantomsection\label{\detokenize{dev_docs:com.lar.cloudnao.RemoteControllerActivity.mCameraWaitTimer}}\pysigline{ \sphinxhref{http://docs.oracle.com/javase/8/docs/api/java/util/Timer.html}{Timer} \sphinxbfcode{mCameraWaitTimer}}
% Un Timer encargado de la ejecución de {\hyperref[\detokenize{dev_docs:com.lar.cloudnao.RemoteControllerActivity.mTimerTaskCamera}]{\sphinxcrossref{\sphinxcode{mTimerTaskCamera}}}}.

% \end{fulllineitems}



% \paragraph{mJoystick}
% \label{\detokenize{dev_docs:mjoystick}}\index{mJoystick (Java field)}

% \begin{fulllineitems}
% \phantomsection\label{\detokenize{dev_docs:com.lar.cloudnao.RemoteControllerActivity.mJoystick}}\pysigline{public Joystick \sphinxbfcode{mJoystick}}
% Instancia del objeto Joystick importado de la biblioteca Bugstick.

% \end{fulllineitems}



% \paragraph{mLiveImageFromRobot}
% \label{\detokenize{dev_docs:mliveimagefromrobot}}\index{mLiveImageFromRobot (Java field)}

% \begin{fulllineitems}
% \phantomsection\label{\detokenize{dev_docs:com.lar.cloudnao.RemoteControllerActivity.mLiveImageFromRobot}}\pysigline{ ImageView \sphinxbfcode{mLiveImageFromRobot}}
% Imagen en vivo del robot.

% \end{fulllineitems}



% \paragraph{mLogsWaitTimer}
% \label{\detokenize{dev_docs:mlogswaittimer}}\index{mLogsWaitTimer (Java field)}

% \begin{fulllineitems}
% \phantomsection\label{\detokenize{dev_docs:com.lar.cloudnao.RemoteControllerActivity.mLogsWaitTimer}}\pysigline{ \sphinxhref{http://docs.oracle.com/javase/8/docs/api/java/util/Timer.html}{Timer} \sphinxbfcode{mLogsWaitTimer}}
% El Timer encargado de repetir la {\hyperref[\detokenize{dev_docs:com.lar.cloudnao.RemoteControllerActivity.mRobotLogsTask}]{\sphinxcrossref{\sphinxcode{mRobotLogsTask}}}} cada cierto tiempo.

% \end{fulllineitems}



% \paragraph{mRobotLogsTask}
% \label{\detokenize{dev_docs:mrobotlogstask}}\index{mRobotLogsTask (Java field)}

% \begin{fulllineitems}
% \phantomsection\label{\detokenize{dev_docs:com.lar.cloudnao.RemoteControllerActivity.mRobotLogsTask}}\pysigline{ \sphinxhref{http://docs.oracle.com/javase/8/docs/api/java/util/TimerTask.html}{TimerTask} \sphinxbfcode{mRobotLogsTask}}
% Define una tarea para obtener datos de ALMemory y escribirlos en Firebase, con \sphinxcode{RobotLogs} y pueda ser repetida por un Timer.

% \end{fulllineitems}



% \paragraph{mTimerBatteryStatusTask}
% \label{\detokenize{dev_docs:mtimerbatterystatustask}}\index{mTimerBatteryStatusTask (Java field)}

% \begin{fulllineitems}
% \phantomsection\label{\detokenize{dev_docs:com.lar.cloudnao.RemoteControllerActivity.mTimerBatteryStatusTask}}\pysigline{ \sphinxhref{http://docs.oracle.com/javase/8/docs/api/java/util/TimerTask.html}{TimerTask} \sphinxbfcode{mTimerBatteryStatusTask}}
% Estable una area para obtener el nivel de batería del robot, usando \sphinxcode{BatteryStatusTask}, y depués usarla en un Timer para su ejecución.

% \end{fulllineitems}



% \paragraph{mTimerTaskCamera}
% \label{\detokenize{dev_docs:mtimertaskcamera}}\index{mTimerTaskCamera (Java field)}

% \begin{fulllineitems}
% \phantomsection\label{\detokenize{dev_docs:com.lar.cloudnao.RemoteControllerActivity.mTimerTaskCamera}}\pysigline{ \sphinxhref{http://docs.oracle.com/javase/8/docs/api/java/util/TimerTask.html}{TimerTask} \sphinxbfcode{mTimerTaskCamera}}
% Define la tarea para obtener una imagen del robot, con \sphinxcode{ImageTask}, para que pueda ser repetida por un Timer.

% \end{fulllineitems}



% \paragraph{proxiesListView}
% \label{\detokenize{dev_docs:proxieslistview}}\index{proxiesListView (Java field)}

% \begin{fulllineitems}
% \phantomsection\label{\detokenize{dev_docs:com.lar.cloudnao.RemoteControllerActivity.proxiesListView}}\pysigline{ ListView \sphinxbfcode{proxiesListView}}
% Lista de valores que se pueden seleccionar para guardarse en Firebase.

% \end{fulllineitems}



% \paragraph{robotPostureButton}
% \label{\detokenize{dev_docs:robotposturebutton}}\index{robotPostureButton (Java field)}

% \begin{fulllineitems}
% \phantomsection\label{\detokenize{dev_docs:com.lar.cloudnao.RemoteControllerActivity.robotPostureButton}}\pysigline{ Button \sphinxbfcode{robotPostureButton}}
% Cambia la postura del robot de Init a Crouch o viceversa.

% \end{fulllineitems}



% \subsubsection{Métodos}
% \label{\detokenize{dev_docs:id13}}

% \paragraph{OnClickSaveLogs}
% \label{\detokenize{dev_docs:onclicksavelogs}}\index{OnClickSaveLogs() (Java method)}

% \begin{fulllineitems}
% \phantomsection\label{\detokenize{dev_docs:com.lar.cloudnao.RemoteControllerActivity.OnClickSaveLogs()}}\pysiglinewithargsret{public void \sphinxbfcode{OnClickSaveLogs}}{}{}
% Maneja el botón para mostrar la información que se guarda en Firebase o para detener el envío.

% \end{fulllineitems}



% \paragraph{OnClickSelectedProxies}
% \label{\detokenize{dev_docs:onclickselectedproxies}}\index{OnClickSelectedProxies() (Java method)}

% \begin{fulllineitems}
% \phantomsection\label{\detokenize{dev_docs:com.lar.cloudnao.RemoteControllerActivity.OnClickSelectedProxies()}}\pysiglinewithargsret{public void \sphinxbfcode{OnClickSelectedProxies}}{}{}
% Inicia el envío de valores de la memoria a Firebase, de acuerdo a la información que se desea enviar.

% \end{fulllineitems}



% \paragraph{hideProxiesListView}
% \label{\detokenize{dev_docs:hideproxieslistview}}\index{hideProxiesListView() (Java method)}

% \begin{fulllineitems}
% \phantomsection\label{\detokenize{dev_docs:com.lar.cloudnao.RemoteControllerActivity.hideProxiesListView()}}\pysiglinewithargsret{public void \sphinxbfcode{hideProxiesListView}}{}{}
% Oculta el Layout con la lista de valores de la memoria del robot y muestra a {\hyperref[\detokenize{dev_docs:com.lar.cloudnao.RemoteControllerActivity.mLiveImageFromRobot}]{\sphinxcrossref{\sphinxcode{mLiveImageFromRobot}}}} y al layout con los botones para controlar al robot.

% \end{fulllineitems}



% \paragraph{initProxiesListView}
% \label{\detokenize{dev_docs:initproxieslistview}}\index{initProxiesListView() (Java method)}

% \begin{fulllineitems}
% \phantomsection\label{\detokenize{dev_docs:com.lar.cloudnao.RemoteControllerActivity.initProxiesListView()}}\pysiglinewithargsret{public void \sphinxbfcode{initProxiesListView}}{}{}
% Inicializa los objetos que servirán para llenar el ItemListAdapter que muestra una lista con la opción para seleccionar un valor de la memoria del robot que se desea enviar a los logs.

% \end{fulllineitems}



% \paragraph{onBackPressed}
% \label{\detokenize{dev_docs:onbackpressed}}\index{onBackPressed() (Java method)}

% \begin{fulllineitems}
% \phantomsection\label{\detokenize{dev_docs:com.lar.cloudnao.RemoteControllerActivity.onBackPressed()}}\pysiglinewithargsret{public void \sphinxbfcode{onBackPressed}}{}{}
% Detiene todos los Timers cuando se presiona el botón de Regresar en el teléfono.

% \end{fulllineitems}



% \paragraph{onClickCancelLogsSelection}
% \label{\detokenize{dev_docs:onclickcancellogsselection}}\index{onClickCancelLogsSelection() (Java method)}

% \begin{fulllineitems}
% \phantomsection\label{\detokenize{dev_docs:com.lar.cloudnao.RemoteControllerActivity.onClickCancelLogsSelection()}}\pysiglinewithargsret{public void \sphinxbfcode{onClickCancelLogsSelection}}{}{}
% Oculta el Layout con la lista de valores de la memoria que se desea registrar. Utiliza el método {\hyperref[\detokenize{dev_docs:com.lar.cloudnao.RemoteControllerActivity.hideProxiesListView()}]{\sphinxcrossref{\sphinxcode{hideProxiesListView()}}}}

% \end{fulllineitems}



% \paragraph{onClickChangeRobotPosture}
% \label{\detokenize{dev_docs:onclickchangerobotposture}}\index{onClickChangeRobotPosture() (Java method)}

% \begin{fulllineitems}
% \phantomsection\label{\detokenize{dev_docs:com.lar.cloudnao.RemoteControllerActivity.onClickChangeRobotPosture()}}\pysiglinewithargsret{public void \sphinxbfcode{onClickChangeRobotPosture}}{}{}
% El listener para el botón que al presionar el robot cambia entre la posición Init y Crouch.

% \end{fulllineitems}



% \paragraph{onClickLiveCamera}
% \label{\detokenize{dev_docs:onclicklivecamera}}\index{onClickLiveCamera() (Java method)}

% \begin{fulllineitems}
% \phantomsection\label{\detokenize{dev_docs:com.lar.cloudnao.RemoteControllerActivity.onClickLiveCamera()}}\pysiglinewithargsret{public void \sphinxbfcode{onClickLiveCamera}}{}{}
% Listener del botón que activa la cámara en vivo. Utiliza los métodos \sphinxcode{pauseCameraTimer()} y \sphinxcode{resumeCameraTimer()} para detener o activar los timers que se encargan de procesar los datos de la cámara enviados por el robot.

% \end{fulllineitems}



% \paragraph{onCreate}
% \label{\detokenize{dev_docs:id14}}\index{onCreate(Bundle) (Java method)}

% \begin{fulllineitems}
% \phantomsection\label{\detokenize{dev_docs:com.lar.cloudnao.RemoteControllerActivity.onCreate(Bundle)}}\pysiglinewithargsret{protected void \sphinxbfcode{onCreate}}{Bundle\sphinxstyleemphasis{ savedInstanceState}}{}
% Inicia la actividad, verifica que el usuario haya iniciado sesión, e inicializa el Adapter para {\hyperref[\detokenize{dev_docs:com.lar.cloudnao.RemoteControllerActivity.proxiesListView}]{\sphinxcrossref{\sphinxcode{proxiesListView}}}}. Finalmente llama al método \sphinxcode{startBatteryTimer()} para mostrar el nivel y estado de la batería del robot.

% \end{fulllineitems}



% \paragraph{onTouchedLeftRotation}
% \label{\detokenize{dev_docs:ontouchedleftrotation}}\index{onTouchedLeftRotation(MotionEvent) (Java method)}

% \begin{fulllineitems}
% \phantomsection\label{\detokenize{dev_docs:com.lar.cloudnao.RemoteControllerActivity.onTouchedLeftRotation(MotionEvent)}}\pysiglinewithargsret{public boolean \sphinxbfcode{onTouchedLeftRotation}}{MotionEvent\sphinxstyleemphasis{ event}}{}
% El listener del botón para girar al robot sobre su eje \sphinxstylestrong{z} a la izquierda. Mientras el usuario presione el botón, el robot girará, al soltar se detiene.
% \begin{quote}\begin{description}
% \item[{Parámetros}] \leavevmode\begin{itemize}
% \item {} 
% \sphinxstyleliteralstrong{event} \textendash{} el evento se recibe el botón.

% \end{itemize}

% \end{description}\end{quote}

% \end{fulllineitems}



% \paragraph{onTouchedRightRotation}
% \label{\detokenize{dev_docs:ontouchedrightrotation}}\index{onTouchedRightRotation(MotionEvent) (Java method)}

% \begin{fulllineitems}
% \phantomsection\label{\detokenize{dev_docs:com.lar.cloudnao.RemoteControllerActivity.onTouchedRightRotation(MotionEvent)}}\pysiglinewithargsret{public boolean \sphinxbfcode{onTouchedRightRotation}}{MotionEvent\sphinxstyleemphasis{ event}}{}
% El listener del botón para girar al robot sobre su eje \sphinxstylestrong{z} a la derecha. Mientras el usuario presione el botón, el robot girará, al soltar se detiene.
% \begin{quote}\begin{description}
% \item[{Parámetros}] \leavevmode\begin{itemize}
% \item {} 
% \sphinxstyleliteralstrong{event} \textendash{} el evento se recibe el botón.

% \end{itemize}

% \end{description}\end{quote}

% \end{fulllineitems}



% \paragraph{setMyJoystickListener}
% \label{\detokenize{dev_docs:setmyjoysticklistener}}\index{setMyJoystickListener() (Java method)}

% \begin{fulllineitems}
% \phantomsection\label{\detokenize{dev_docs:com.lar.cloudnao.RemoteControllerActivity.setMyJoystickListener()}}\pysiglinewithargsret{public void \sphinxbfcode{setMyJoystickListener}}{}{}
% El listener para {\hyperref[\detokenize{dev_docs:com.lar.cloudnao.RemoteControllerActivity.mJoystick}]{\sphinxcrossref{\sphinxcode{mJoystick}}}}, reacciona cuando se mueve el widget o se suelta. Cuando se arrastra el joystick se cambia las velocidad sobre los ejes con la función {\hyperref[\detokenize{dev_docs:com.lar.cloudnao.RemoteControllerActivity.setMyJoystickListener()}]{\sphinxcrossref{\sphinxcode{setMyJoystickListener()}}}} y luego se pasan como parámetros esas velocidades para hacer caminar al robot.

% \end{fulllineitems}



% \paragraph{setVelocityOnAxis}
% \label{\detokenize{dev_docs:setvelocityonaxis}}\index{setVelocityOnAxis(int, float) (Java method)}

% \begin{fulllineitems}
% \phantomsection\label{\detokenize{dev_docs:com.lar.cloudnao.RemoteControllerActivity.setVelocityOnAxis(int, float)}}\pysiglinewithargsret{public void \sphinxbfcode{setVelocityOnAxis}}{int\sphinxstyleemphasis{ degrees}, float\sphinxstyleemphasis{ offset}}{}
% Cambia la velocidad con la que camina sobre los ejes \sphinxstylestrong{x}, y \sphinxstylestrong{y} a partir de los parámetros recibidos en el método {\hyperref[\detokenize{dev_docs:com.lar.cloudnao.RemoteControllerActivity.setMyJoystickListener()}]{\sphinxcrossref{\sphinxcode{setMyJoystickListener()}}}} del objeto {\hyperref[\detokenize{dev_docs:com.lar.cloudnao.RemoteControllerActivity.mJoystick}]{\sphinxcrossref{\sphinxcode{mJoystick}}}}.
% \begin{quote}\begin{description}
% \item[{Parámetros}] \leavevmode\begin{itemize}
% \item {} 
% \sphinxstyleliteralstrong{degrees} \textendash{} los grados que se movió el Joystick.

% \item {} 
% \sphinxstyleliteralstrong{offset} \textendash{} el radio desde el centro del círculo hasta donde se detectó el click.

% \end{itemize}

% \end{description}\end{quote}

% \end{fulllineitems}



% \paragraph{showProxiesListView}
% \label{\detokenize{dev_docs:showproxieslistview}}\index{showProxiesListView() (Java method)}

% \begin{fulllineitems}
% \phantomsection\label{\detokenize{dev_docs:com.lar.cloudnao.RemoteControllerActivity.showProxiesListView()}}\pysiglinewithargsret{public void \sphinxbfcode{showProxiesListView}}{}{}
% Oculta el layout con la lista de valores de la memoria del robot que se envían a Firebase y se muestran {\hyperref[\detokenize{dev_docs:com.lar.cloudnao.RemoteControllerActivity.mLiveImageFromRobot}]{\sphinxcrossref{\sphinxcode{mLiveImageFromRobot}}}} y los botones para controlar remotamente al robot.

% \end{fulllineitems}



% \textbf{FaceRecognitionActivity}
% \label{\detokenize{dev_docs:facerecognitionactivity}}\index{FaceRecognitionActivity (Java class)}

% \begin{fulllineitems}
% \phantomsection\label{\detokenize{dev_docs:com.lar.cloudnao.FaceRecognitionActivity}}\pysigline{public class \sphinxbfcode{FaceRecognitionActivity} extends AppCompatActivity implements VisionRequestCompleted}
% La clase de la actividad que realiza el reconocimiento de rostros. Implementa la interfaz \sphinxcode{VisionRequestCompleted} para llamar a su método abstracto desde la clase \sphinxcode{VisionRESTRequestsTask}

% \end{fulllineitems}



% \subsubsection{Atributos}
% \label{\detokenize{dev_docs:id15}}

% \paragraph{MyRobot}
% \label{\detokenize{dev_docs:id16}}\index{MyRobot (Java field)}

% \begin{fulllineitems}
% \phantomsection\label{\detokenize{dev_docs:com.lar.cloudnao.FaceRecognitionActivity.MyRobot}}\pysigline{ {\hyperref[\detokenize{dev_docs:com.lar.cloudnao.Robot}]{\sphinxcrossref{Robot}}} \sphinxbfcode{MyRobot}}
% Obtiene la instancia de {\hyperref[\detokenize{dev_docs:com.lar.cloudnao.Robot}]{\sphinxcrossref{\sphinxcode{Robot}}}}.

% \end{fulllineitems}



% \paragraph{mRobotImage}
% \label{\detokenize{dev_docs:mrobotimage}}\index{mRobotImage (Java field)}

% \begin{fulllineitems}
% \phantomsection\label{\detokenize{dev_docs:com.lar.cloudnao.FaceRecognitionActivity.mRobotImage}}\pysigline{ ImageView \sphinxbfcode{mRobotImage}}
% Muestra las imagen obtenida del robot.

% \end{fulllineitems}



% \paragraph{subjectRecognitionLV}
% \label{\detokenize{dev_docs:subjectrecognitionlv}}\index{subjectRecognitionLV (Java field)}

% \begin{fulllineitems}
% \phantomsection\label{\detokenize{dev_docs:com.lar.cloudnao.FaceRecognitionActivity.subjectRecognitionLV}}\pysigline{ ListView \sphinxbfcode{subjectRecognitionLV}}
% Muestra los sujetos encontrados en la imagen.

% \end{fulllineitems}



% \subsubsection{Métodos}
% \label{\detokenize{dev_docs:id17}}

% \paragraph{getImageFromRobot}
% \label{\detokenize{dev_docs:getimagefromrobot}}\index{getImageFromRobot() (Java method)}

% \begin{fulllineitems}
% \phantomsection\label{\detokenize{dev_docs:com.lar.cloudnao.FaceRecognitionActivity.getImageFromRobot()}}\pysiglinewithargsret{public void \sphinxbfcode{getImageFromRobot}}{}{}
% Un escuchador del botón para obtener una imagen remota del robot, usando la clase ImageTask usa la clase \sphinxcode{ImageTask}

% \end{fulllineitems}



% \paragraph{onClickEnrollFace}
% \label{\detokenize{dev_docs:onclickenrollface}}\index{onClickEnrollFace() (Java method)}

% \begin{fulllineitems}
% \phantomsection\label{\detokenize{dev_docs:com.lar.cloudnao.FaceRecognitionActivity.onClickEnrollFace()}}\pysiglinewithargsret{public void \sphinxbfcode{onClickEnrollFace}}{}{}
% El escuchador para el botón que permite añadir un nuevo rostro a la galería de Kairos. Corre la clase \sphinxcode{VisionRESTRequestsTask}

% \end{fulllineitems}



% \paragraph{onClickFaceRecognition}
% \label{\detokenize{dev_docs:onclickfacerecognition}}\index{onClickFaceRecognition() (Java method)}

% \begin{fulllineitems}
% \phantomsection\label{\detokenize{dev_docs:com.lar.cloudnao.FaceRecognitionActivity.onClickFaceRecognition()}}\pysiglinewithargsret{public void \sphinxbfcode{onClickFaceRecognition}}{}{}
% Un escuchador para el botón de reconocimiento de rostros previamente almacenados. Usa la clase \sphinxcode{VisionRESTRequestsTask}

% \end{fulllineitems}



% \paragraph{onCreate}
% \label{\detokenize{dev_docs:id18}}\index{onCreate(Bundle) (Java method)}

% \begin{fulllineitems}
% \phantomsection\label{\detokenize{dev_docs:com.lar.cloudnao.FaceRecognitionActivity.onCreate(Bundle)}}\pysiglinewithargsret{protected void \sphinxbfcode{onCreate}}{Bundle\sphinxstyleemphasis{ savedInstanceState}}{}
% El método que inicia la actividad. Inicia Butterknife para poder utilizar sus métodos. Primero verifica si el usuario ha iniciado sesión, para después inicializar el Adapter y los escuchadores de la lista de sujetos.

% \end{fulllineitems}



% \paragraph{onVisionRequestCompleted}
% \label{\detokenize{dev_docs:onvisionrequestcompleted}}\index{onVisionRequestCompleted(Object) (Java method)}

% \begin{fulllineitems}
% \phantomsection\label{\detokenize{dev_docs:com.lar.cloudnao.FaceRecognitionActivity.onVisionRequestCompleted(Object)}}\pysiglinewithargsret{public void \sphinxbfcode{onVisionRequestCompleted}}{\sphinxhref{http://docs.oracle.com/javase/8/docs/api/java/lang/Object.html}{Object}\sphinxstyleemphasis{ response}}{}
% El callback que maneja la respuesta de la petición a la API RESTful de CloudNAO.

% \end{fulllineitems}



% \paragraph{processFaceEnroll}
% \label{\detokenize{dev_docs:processfaceenroll}}\index{processFaceEnroll(JSONObject) (Java method)}

% \begin{fulllineitems}
% \phantomsection\label{\detokenize{dev_docs:com.lar.cloudnao.FaceRecognitionActivity.processFaceEnroll(JSONObject)}}\pysiglinewithargsret{public void \sphinxbfcode{processFaceEnroll}}{JSONObject\sphinxstyleemphasis{ faceEnroll}}{}
% Método para procesar el nuevo rostro añadido. Agrega al nuevo sujeto a 
% \sphinxcode{mCandidatesFound}, notifica al Adapter para que actualice 
% {\hyperref[\detokenize{dev_docs:com.lar.cloudnao.FaceRecognitionActivity.subjectRecognitionLV}]{\sphinxcrossref{\sphinxcode{subjectRecognitionLV}}}} y finalmente llama 
% \sphinxcode{com.lar.cloudnao.utilities.BoundingBoxesUtils.drawBoxesFromArray(Context,ImageView,ArrayList)} pasando como parámetros \sphinxcode{mCandidatesFound} y 
% {\hyperref[\detokenize{dev_docs:com.lar.cloudnao.OCRTranslationActivity.mRobotImage}]{\sphinxcrossref{\sphinxcode{mRobotImage}}}} para dibujar sobre el nuevo rostro encontrado.

% \end{fulllineitems}



% \paragraph{processFaceRecognitionJSONArray}
% \label{\detokenize{dev_docs:processfacerecognitionjsonarray}}\index{processFaceRecognitionJSONArray(JSONArray) (Java method)}

% \begin{fulllineitems}
% \phantomsection\label{\detokenize{dev_docs:com.lar.cloudnao.FaceRecognitionActivity.processFaceRecognitionJSONArray(JSONArray)}}\pysiglinewithargsret{public void \sphinxbfcode{processFaceRecognitionJSONArray}}{JSONArray\sphinxstyleemphasis{ faces}}{}
% Método para procesar el arreglo de rostros encontrados en la imagen. Llena el arreglo \sphinxcode{mCandidatesFound}, notifica al Adapter para que actualice {\hyperref[\detokenize{dev_docs:com.lar.cloudnao.FaceRecognitionActivity.subjectRecognitionLV}]{\sphinxcrossref{\sphinxcode{subjectRecognitionLV}}}} y finalmente llama \sphinxcode{comlar.cloudnao.utilities.BoundingBoxesUtils} pasando como parámetros \sphinxcode{mCandidatesFound} y {\hyperref[\detokenize{dev_docs:com.lar.cloudnao.OCRTranslationActivity.mRobotImage}]{\sphinxcrossref{\sphinxcode{mRobotImage}}}} para dibujar los cuadros delimitadores sobre los rostros encontrados.

% \end{fulllineitems}



% \paragraph{robotSayHi}
% \label{\detokenize{dev_docs:robotsayhi}}\index{robotSayHi(String) (Java method)}

% \begin{fulllineitems}
% \phantomsection\label{\detokenize{dev_docs:com.lar.cloudnao.FaceRecognitionActivity.robotSayHi(String)}}\pysiglinewithargsret{public void \sphinxbfcode{robotSayHi}}{\sphinxhref{http://docs.oracle.com/javase/8/docs/api/java/lang/String.html}{String}\sphinxstyleemphasis{ subjectName}}{}
% Un método auxiliar para que el robot salude al dar click a un sujeto dentro de {\hyperref[\detokenize{dev_docs:com.lar.cloudnao.FaceRecognitionActivity.subjectRecognitionLV}]{\sphinxcrossref{\sphinxcode{subjectRecognitionLV}}}}

% \end{fulllineitems}



% \textbf{OCRTranslationActivity}
% \label{\detokenize{dev_docs:ocrtranslationactivity}}\index{OCRTranslationActivity (Java class)}

% \begin{fulllineitems}
% \phantomsection\label{\detokenize{dev_docs:com.lar.cloudnao.OCRTranslationActivity}}\pysigline{public class \sphinxbfcode{OCRTranslationActivity} extends AppCompatActivity implements VisionRequestCompleted}
% Clase de la actividad que se encarga del procesamiento de una imagen para detección de texto y su traducción. Implenta la interfaz \sphinxcode{VisionRequestCompleted}.

% \end{fulllineitems}



% \subsubsection{Atributos}
% \label{\detokenize{dev_docs:id19}}

% \paragraph{MyRobot}
% \label{\detokenize{dev_docs:id20}}\index{MyRobot (Java field)}

% \begin{fulllineitems}
% \phantomsection\label{\detokenize{dev_docs:com.lar.cloudnao.OCRTranslationActivity.MyRobot}}\pysigline{ {\hyperref[\detokenize{dev_docs:com.lar.cloudnao.Robot}]{\sphinxcrossref{Robot}}} \sphinxbfcode{MyRobot}}
% Recibe la instancia global de {\hyperref[\detokenize{dev_docs:com.lar.cloudnao.Robot}]{\sphinxcrossref{\sphinxcode{Robot}}}}

% \end{fulllineitems}



% \paragraph{mRobotImage}
% \label{\detokenize{dev_docs:id21}}\index{mRobotImage (Java field)}

% \begin{fulllineitems}
% \phantomsection\label{\detokenize{dev_docs:com.lar.cloudnao.OCRTranslationActivity.mRobotImage}}\pysigline{ ImageView \sphinxbfcode{mRobotImage}}
% La imagen obtenida desde el robot.

% \end{fulllineitems}



% \paragraph{mSourceTextTV}
% \label{\detokenize{dev_docs:msourcetexttv}}\index{mSourceTextTV (Java field)}

% \begin{fulllineitems}
% \phantomsection\label{\detokenize{dev_docs:com.lar.cloudnao.OCRTranslationActivity.mSourceTextTV}}\pysigline{ TextView \sphinxbfcode{mSourceTextTV}}
% Muestra el texto original

% \end{fulllineitems}



% \paragraph{mTextTranslated}
% \label{\detokenize{dev_docs:mtexttranslated}}\index{mTextTranslated (Java field)}

% \begin{fulllineitems}
% \phantomsection\label{\detokenize{dev_docs:com.lar.cloudnao.OCRTranslationActivity.mTextTranslated}}\pysigline{ \sphinxhref{http://docs.oracle.com/javase/8/docs/api/java/lang/String.html}{String} \sphinxbfcode{mTextTranslated}}
% El texto traducido que el robot expresará oralmente.

% \end{fulllineitems}



% \paragraph{mTextTranslatedTV}
% \label{\detokenize{dev_docs:mtexttranslatedtv}}\index{mTextTranslatedTV (Java field)}

% \begin{fulllineitems}
% \phantomsection\label{\detokenize{dev_docs:com.lar.cloudnao.OCRTranslationActivity.mTextTranslatedTV}}\pysigline{ TextView \sphinxbfcode{mTextTranslatedTV}}
% Muestra el texto traducido

% \end{fulllineitems}



% \subsubsection{Métodos}
% \label{\detokenize{dev_docs:id22}}

% \paragraph{getImageFromRobot}
% \label{\detokenize{dev_docs:id23}}\index{getImageFromRobot() (Java method)}

% \begin{fulllineitems}
% \phantomsection\label{\detokenize{dev_docs:com.lar.cloudnao.OCRTranslationActivity.getImageFromRobot()}}\pysiglinewithargsret{public void \sphinxbfcode{getImageFromRobot}}{}{}
% Obtiene una image del robot usando \sphinxcode{ImageTask}

% \end{fulllineitems}



% \paragraph{onClickOCRButton}
% \label{\detokenize{dev_docs:onclickocrbutton}}\index{onClickOCRButton() (Java method)}

% \begin{fulllineitems}
% \phantomsection\label{\detokenize{dev_docs:com.lar.cloudnao.OCRTranslationActivity.onClickOCRButton()}}\pysiglinewithargsret{public void \sphinxbfcode{onClickOCRButton}}{}{}
% Un escuchador para el botón de detección de caracteres y traducción. Hace una petción a la API a través de \sphinxcode{VisionRESTRequestsTask}.

% \end{fulllineitems}



% \paragraph{onClickTranslatedText}
% \label{\detokenize{dev_docs:onclicktranslatedtext}}\index{onClickTranslatedText() (Java method)}

% \begin{fulllineitems}
% \phantomsection\label{\detokenize{dev_docs:com.lar.cloudnao.OCRTranslationActivity.onClickTranslatedText()}}\pysiglinewithargsret{public void \sphinxbfcode{onClickTranslatedText}}{}{}
% El agente escucha que llama al método {\hyperref[\detokenize{dev_docs:com.lar.cloudnao.Robot.say(String)}]{\sphinxcrossref{\sphinxcode{Robotsay(String)}}}} cuando el usuario hace click sobre el TextView que contiene el texto traducido.

% \end{fulllineitems}



% \paragraph{onCreate}
% \label{\detokenize{dev_docs:id24}}\index{onCreate(Bundle) (Java method)}

% \begin{fulllineitems}
% \phantomsection\label{\detokenize{dev_docs:com.lar.cloudnao.OCRTranslationActivity.onCreate(Bundle)}}\pysiglinewithargsret{protected void \sphinxbfcode{onCreate}}{Bundle\sphinxstyleemphasis{ savedInstanceState}}{}
% Inicia la actividad y verifica que el usuario haya iniciado sesión.

% \end{fulllineitems}



% \paragraph{onVisionRequestCompleted}
% \label{\detokenize{dev_docs:id25}}\index{onVisionRequestCompleted(Object) (Java method)}

% \begin{fulllineitems}
% \phantomsection\label{\detokenize{dev_docs:com.lar.cloudnao.OCRTranslationActivity.onVisionRequestCompleted(Object)}}\pysiglinewithargsret{public void \sphinxbfcode{onVisionRequestCompleted}}{\sphinxhref{http://docs.oracle.com/javase/8/docs/api/java/lang/Object.html}{Object}\sphinxstyleemphasis{ response}}{}
% Define el método \sphinxcode{VisionRequestCompleted.onVisionRequestCompleted(Object)} 
% para procesar la respuesta recibida, puede ser un error o un JSON con el texto original con la traducción. Si esto último fue el caso asigna a los TextView 
% {\hyperref[\detokenize{dev_docs:com.lar.cloudnao.OCRTranslationActivity.mSourceTextTV}]
% {\sphinxcrossref{\sphinxcode{mSourceTextTV}}}} y 
% {\hyperref[\detokenize{dev_docs:com.lar.cloudnao.OCRTranslationActivity.mTextTranslatedTV}]{\sphinxcrossref{\sphinxcode{mTextTranslatedTV}}}} el texto original y el 
% traducido, respectivamente.
% \begin{quote}\begin{description}
% \item[{Parámetros}] \leavevmode\begin{itemize}
% \item {} 
% \sphinxstyleliteralstrong{response} \textendash{} El objeto con la respuesta devuelta por la API REST de CloudNAO

% \end{itemize}

% \end{description}\end{quote}

% \end{fulllineitems}



% \textbf{ObjectDetectionActivity}
% \label{\detokenize{dev_docs:objectdetectionactivity}}\index{ObjectDetectionActivity (Java class)}

% \begin{fulllineitems}
% \phantomsection\label{\detokenize{dev_docs:com.lar.cloudnao.ObjectDetectionActivity}}\pysigline{public class \sphinxbfcode{ObjectDetectionActivity} extends AppCompatActivity implements VisionRequestCompleted}
% La actividad encargada de la funcionalidad de la detección de objetos, implementa la interfaz \sphinxcode{VisionRequestCompleted} para ejecutar su método \sphinxcode{VisionRequestCompletedonVisionRequestCompleted(Object)} como callback en \sphinxcode{VisionRESTRequestsTask}

% \end{fulllineitems}



% \subsubsection{Atributos}
% \label{\detokenize{dev_docs:id26}}

% \paragraph{MyRobot}
% \label{\detokenize{dev_docs:id27}}\index{MyRobot (Java field)}

% \begin{fulllineitems}
% \phantomsection\label{\detokenize{dev_docs:com.lar.cloudnao.ObjectDetectionActivity.MyRobot}}\pysigline{ {\hyperref[\detokenize{dev_docs:com.lar.cloudnao.Robot}]{\sphinxcrossref{Robot}}} \sphinxbfcode{MyRobot}}
% Obtiene la instancia del singleton {\hyperref[\detokenize{dev_docs:com.lar.cloudnao.Robot}]{\sphinxcrossref{\sphinxcode{Robot}}}}

% \end{fulllineitems}



% \paragraph{foundObjectsLV}
% \label{\detokenize{dev_docs:foundobjectslv}}\index{foundObjectsLV (Java field)}

% \begin{fulllineitems}
% \phantomsection\label{\detokenize{dev_docs:com.lar.cloudnao.ObjectDetectionActivity.foundObjectsLV}}\pysigline{ ListView \sphinxbfcode{foundObjectsLV}}
% Muesta los objetos detectados en la imagen. El nombre del objeto y la confianza.

% \end{fulllineitems}



% \paragraph{mRobotImage}
% \label{\detokenize{dev_docs:id28}}\index{mRobotImage (Java field)}

% \begin{fulllineitems}
% \phantomsection\label{\detokenize{dev_docs:com.lar.cloudnao.ObjectDetectionActivity.mRobotImage}}\pysigline{ ImageView \sphinxbfcode{mRobotImage}}
% La imagen obtenida desde el robot.

% \end{fulllineitems}



% \subsubsection{Métodos}
% \label{\detokenize{dev_docs:id29}}

% \paragraph{foundObjectsToSpeech}
% \label{\detokenize{dev_docs:foundobjectstospeech}}\index{foundObjectsToSpeech() (Java method)}

% \begin{fulllineitems}
% \phantomsection\label{\detokenize{dev_docs:com.lar.cloudnao.ObjectDetectionActivity.foundObjectsToSpeech()}}\pysiglinewithargsret{public void \sphinxbfcode{foundObjectsToSpeech}}{}{}
% Método para mapear el arreglo \sphinxcode{mFoundObjects} a una cadena que pueda decir el robot.

% \end{fulllineitems}



% \paragraph{getImageFromRobot}
% \label{\detokenize{dev_docs:id30}}\index{getImageFromRobot() (Java method)}

% \begin{fulllineitems}
% \phantomsection\label{\detokenize{dev_docs:com.lar.cloudnao.ObjectDetectionActivity.getImageFromRobot()}}\pysiglinewithargsret{public void \sphinxbfcode{getImageFromRobot}}{}{}
% Un escuchador del botón para obtener una imagen remota del robot, usando la clase ImageTask usa la clase \sphinxcode{ImageTask}

% \end{fulllineitems}



% \paragraph{onClickObjectDetection}
% \label{\detokenize{dev_docs:onclickobjectdetection}}\index{onClickObjectDetection() (Java method)}

% \begin{fulllineitems}
% \phantomsection\label{\detokenize{dev_docs:com.lar.cloudnao.ObjectDetectionActivity.onClickObjectDetection()}}\pysiglinewithargsret{public void \sphinxbfcode{onClickObjectDetection}}{}{}
% Un escuchador para el botón de detección de objetos. Usa la clase \sphinxcode{VisionRESTRequestsTask}

% \end{fulllineitems}



% \paragraph{onCreate}
% \label{\detokenize{dev_docs:id31}}\index{onCreate(Bundle) (Java method)}

% \begin{fulllineitems}
% \phantomsection\label{\detokenize{dev_docs:com.lar.cloudnao.ObjectDetectionActivity.onCreate(Bundle)}}\pysiglinewithargsret{protected void \sphinxbfcode{onCreate}}{Bundle\sphinxstyleemphasis{ savedInstanceState}}{}
% Crea la actividad, verifica que el usuario haya iniciado sesión y configura el Adapter del ListView.

% \end{fulllineitems}



% \paragraph{onVisionRequestCompleted}
% \label{\detokenize{dev_docs:id32}}\index{onVisionRequestCompleted(Object) (Java method)}

% \begin{fulllineitems}
% \phantomsection\label{\detokenize{dev_docs:com.lar.cloudnao.ObjectDetectionActivity.onVisionRequestCompleted(Object)}}\pysiglinewithargsret{public void \sphinxbfcode{onVisionRequestCompleted}}{\sphinxhref{http://docs.oracle.com/javase/8/docs/api/java/lang/Object.html}{Object}\sphinxstyleemphasis{ response}}{}
% El callback que maneja la respuesta de la petición a la API RESTful de CloudNAO.

% \end{fulllineitems}



% \paragraph{processObjectDetectionJSONArray}
% \label{\detokenize{dev_docs:processobjectdetectionjsonarray}}\index{processObjectDetectionJSONArray(JSONArray) (Java method)}

% \begin{fulllineitems}
% \phantomsection\label{\detokenize{dev_docs:com.lar.cloudnao.ObjectDetectionActivity.processObjectDetectionJSONArray(JSONArray)}}\pysiglinewithargsret{public void \sphinxbfcode{processObjectDetectionJSONArray}}{JSONArray\sphinxstyleemphasis{ objectsArray}}{}
% Método para procesar el arreglo de objetos encontrados en la imagen. Llena el arreglo 
% \sphinxcode{mFoundObjects}, notifica al Adapter para que actualice 
% {\hyperref[\detokenize{dev_docs:com.lar.cloudnao.ObjectDetectionActivity.foundObjectsLV}]{\sphinxcrossref{\sphinxcode{foundObjectsLV}}}} y llama 
% \sphinxcode{com.lar.cloudnao.utilities.BoundingBoxesUtils.drawBoxesFromArray(Context,ImageView,ArrayList)} pasando como parámetros \sphinxcode{mFoundObjects} y 
% {\hyperref[\detokenize{dev_docs:com.lar.cloudnao.OCRTranslationActivity.mRobotImage}]{\sphinxcrossref{\sphinxcode{mRobotImage}}}} para dibujar los cuadros delimitadores sobre los objetos encontrados. Finalmente el robot dice qué objetos encontró y cuantos de cada categoría.

% \end{fulllineitems}



% \textbf{Robot}
% \label{\detokenize{dev_docs:robot}}\index{Robot (Java class)}

% \begin{fulllineitems}
% \phantomsection\label{\detokenize{dev_docs:com.lar.cloudnao.Robot}}\pysigline{public class \sphinxbfcode{Robot}}
% La clase pública principal para ejecutar módulos del framework de NAOqi en la aplicación. Es una interfaz para conectarse al robot y ejecutar algunos módulos de NAOqi. Usa el patrón de diseño Singleton para que solo exista una instancia del objeto y funcione como una variable global.

% \end{fulllineitems}



% \subsubsection{Atributos}
% \label{\detokenize{dev_docs:id33}}

% \paragraph{isConnected}
% \label{\detokenize{dev_docs:isconnected}}\index{isConnected (Java field)}

% \begin{fulllineitems}
% \phantomsection\label{\detokenize{dev_docs:com.lar.cloudnao.Robot.isConnected}}\pysigline{public boolean \sphinxbfcode{isConnected}}
% Una bandera para verificar la conexión de la aplicación con el robot.

% \end{fulllineitems}



% \paragraph{mSonarSubscriberId}
% \label{\detokenize{dev_docs:msonarsubscriberid}}\index{mSonarSubscriberId (Java field)}

% \begin{fulllineitems}
% \phantomsection\label{\detokenize{dev_docs:com.lar.cloudnao.Robot.mSonarSubscriberId}}\pysigline{public final \sphinxhref{http://docs.oracle.com/javase/8/docs/api/java/lang/String.html}{String} \sphinxbfcode{mSonarSubscriberId}}
% El identificador del suscriptor al sonar.

% \end{fulllineitems}



% \subsubsection{Constructors}
% \label{\detokenize{dev_docs:constructors}}

% \paragraph{Robot}
% \label{\detokenize{dev_docs:id34}}\index{Robot(Context) (Java constructor)}

% \begin{fulllineitems}
% \phantomsection\label{\detokenize{dev_docs:com.lar.cloudnao.Robot.Robot(Context)}}\pysiglinewithargsret{public \sphinxbfcode{Robot}}{Context\sphinxstyleemphasis{ context}}{}
% El constructor de la clase, simplemente, recibe el contexto en el que se encuentra la aplicación.

% \end{fulllineitems}



% \subsubsection{Métodos}
% \label{\detokenize{dev_docs:id35}}

% \paragraph{createSessionAndProxies}
% \label{\detokenize{dev_docs:createsessionandproxies}}\index{createSessionAndProxies(String, String) (Java method)}

% \begin{fulllineitems}
% \phantomsection\label{\detokenize{dev_docs:com.lar.cloudnao.Robot.createSessionAndProxies(String, String)}}\pysiglinewithargsret{public void \sphinxbfcode{createSessionAndProxies}}{\sphinxhref{http://docs.oracle.com/javase/8/docs/api/java/lang/String.html}{String}\sphinxstyleemphasis{ robotIPAddress}, \sphinxhref{http://docs.oracle.com/javase/8/docs/api/java/lang/String.html}{String}\sphinxstyleemphasis{ robotName}}{}
% Crea la sesión que permite la conexión con el robot, e inicializa los proxies, ALVideoDevice, ALMotion, ALTextToSpeech, ALMemory y ALSonar.
% \begin{quote}\begin{description}
% \item[{Parámetros}] \leavevmode\begin{itemize}
% \item {} 
% \sphinxstyleliteralstrong{robotIPAddress} \textendash{} La dirección IP del robot.

% \item {} 
% \sphinxstyleliteralstrong{robotName} \textendash{} El identificador único del robot para guardar sus datos en Firebase.

% \end{itemize}

% \end{description}\end{quote}

% \end{fulllineitems}



% \paragraph{get}
% \label{\detokenize{dev_docs:get}}\index{get(Context) (Java method)}

% \begin{fulllineitems}
% \phantomsection\label{\detokenize{dev_docs:com.lar.cloudnao.Robot.get(Context)}}\pysiglinewithargsret{public static {\hyperref[\detokenize{dev_docs:com.lar.cloudnao.Robot}]{\sphinxcrossref{Robot}}} \sphinxbfcode{get}}{Context\sphinxstyleemphasis{ context}}{}
% Una método para hacer que solo exista una instancia del objeto Robot.

% \end{fulllineitems}



% \paragraph{getBatteryLevel}
% \label{\detokenize{dev_docs:getbatterylevel}}\index{getBatteryLevel() (Java method)}

% \begin{fulllineitems}
% \phantomsection\label{\detokenize{dev_docs:com.lar.cloudnao.Robot.getBatteryLevel()}}\pysiglinewithargsret{public int \sphinxbfcode{getBatteryLevel}}{}{}
% Obtiene y regresa el nivel de batería del robot.

% \end{fulllineitems}



% \paragraph{getImageFromRobot}
% \label{\detokenize{dev_docs:id36}}\index{getImageFromRobot() (Java method)}

% \begin{fulllineitems}
% \phantomsection\label{\detokenize{dev_docs:com.lar.cloudnao.Robot.getImageFromRobot()}}\pysiglinewithargsret{public \sphinxhref{http://docs.oracle.com/javase/8/docs/api/java/util/List.html}{List}\textless{}\sphinxhref{http://docs.oracle.com/javase/8/docs/api/java/lang/Object.html}{Object}\textgreater{} \sphinxbfcode{getImageFromRobot}}{}{}
% Obtiene la imagen del robot con el método getImageRemot de ALVideoDevice y retorna el objeto que contiene la imagen y otros atributos.

% \end{fulllineitems}



% \paragraph{getRobotName}
% \label{\detokenize{dev_docs:getrobotname}}\index{getRobotName() (Java method)}

% \begin{fulllineitems}
% \phantomsection\label{\detokenize{dev_docs:com.lar.cloudnao.Robot.getRobotName()}}\pysiglinewithargsret{public \sphinxhref{http://docs.oracle.com/javase/8/docs/api/java/lang/String.html}{String} \sphinxbfcode{getRobotName}}{}{}
% Devuelve el nombre del robot. Es el identificador único generado por Firebase.

% \end{fulllineitems}



% \paragraph{getSelectedValuesFromMemory}
% \label{\detokenize{dev_docs:getselectedvaluesfrommemory}}\index{getSelectedValuesFromMemory(List) (Java method)}

% \begin{fulllineitems}
% \phantomsection\label{\detokenize{dev_docs:com.lar.cloudnao.Robot.getSelectedValuesFromMemory(List)}}\pysiglinewithargsret{public \sphinxhref{http://docs.oracle.com/javase/8/docs/api/java/util/HashMap.html}{HashMap}\textless{}\sphinxhref{http://docs.oracle.com/javase/8/docs/api/java/lang/String.html}{String}, \sphinxhref{http://docs.oracle.com/javase/8/docs/api/java/lang/Object.html}{Object}\textgreater{} \sphinxbfcode{getSelectedValuesFromMemory}}{\sphinxhref{http://docs.oracle.com/javase/8/docs/api/java/util/List.html}{List}\textless{}\sphinxhref{http://docs.oracle.com/javase/8/docs/api/java/lang/String.html}{String}\textgreater{}\sphinxstyleemphasis{ SelectedKeys}}{}
% Obtiene valores de ALMemory solicitados en la actividad del control remoto, y retorna un mapa con las llaves y valores desde ALMemory.

% \end{fulllineitems}



% \paragraph{isConnectedToCharger}
% \label{\detokenize{dev_docs:isconnectedtocharger}}\index{isConnectedToCharger() (Java method)}

% \begin{fulllineitems}
% \phantomsection\label{\detokenize{dev_docs:com.lar.cloudnao.Robot.isConnectedToCharger()}}\pysiglinewithargsret{public boolean \sphinxbfcode{isConnectedToCharger}}{}{}
% Verifica si el robot está conectado a la corriente.

% \end{fulllineitems}



% \paragraph{moveRobot}
% \label{\detokenize{dev_docs:moverobot}}\index{moveRobot(float, float) (Java method)}

% \begin{fulllineitems}
% \phantomsection\label{\detokenize{dev_docs:com.lar.cloudnao.Robot.moveRobot(float, float)}}\pysiglinewithargsret{public void \sphinxbfcode{moveRobot}}{float\sphinxstyleemphasis{ x}, float\sphinxstyleemphasis{ y}}{}
% Hace que el robot camine a lo largo de sus ejes X y Y

% \end{fulllineitems}



% \paragraph{moveRobot}
% \label{\detokenize{dev_docs:id37}}\index{moveRobot(float) (Java method)}

% \begin{fulllineitems}
% \phantomsection\label{\detokenize{dev_docs:com.lar.cloudnao.Robot.moveRobot(float)}}\pysiglinewithargsret{public void \sphinxbfcode{moveRobot}}{float\sphinxstyleemphasis{ theta}}{}
% Hace que el robot camine sobre su eje Z, gira a la derecha o izquierda dependiendo del valor del parámetro theta.

% \end{fulllineitems}



% \paragraph{moveRobot}
% \label{\detokenize{dev_docs:id38}}\index{moveRobot(float, float, float) (Java method)}

% \begin{fulllineitems}
% \phantomsection\label{\detokenize{dev_docs:com.lar.cloudnao.Robot.moveRobot(float, float, float)}}\pysiglinewithargsret{public void \sphinxbfcode{moveRobot}}{float\sphinxstyleemphasis{ x}, float\sphinxstyleemphasis{ y}, float\sphinxstyleemphasis{ theta}}{}
% Mueve al robot sobre sus tres ejes.

% \end{fulllineitems}



% \paragraph{onDisconnected}
% \label{\detokenize{dev_docs:ondisconnected}}\index{onDisconnected(Activity) (Java method)}

% \begin{fulllineitems}
% \phantomsection\label{\detokenize{dev_docs:com.lar.cloudnao.Robot.onDisconnected(Activity)}}\pysiglinewithargsret{public void \sphinxbfcode{onDisconnected}}{Activity\sphinxstyleemphasis{ currentActivity}}{}
% Cierra la conexión con el robot.

% \end{fulllineitems}



% \paragraph{rest}
% \label{\detokenize{dev_docs:rest}}\index{rest() (Java method)}

% \begin{fulllineitems}
% \phantomsection\label{\detokenize{dev_docs:com.lar.cloudnao.Robot.rest()}}\pysiglinewithargsret{public void \sphinxbfcode{rest}}{}{}
% Cambia de postura a Crouch

% \end{fulllineitems}



% \paragraph{say}
% \label{\detokenize{dev_docs:say}}\index{say(String) (Java method)}

% \begin{fulllineitems}
% \phantomsection\label{\detokenize{dev_docs:com.lar.cloudnao.Robot.say(String)}}\pysiglinewithargsret{public void \sphinxbfcode{say}}{\sphinxhref{http://docs.oracle.com/javase/8/docs/api/java/lang/String.html}{String}\sphinxstyleemphasis{ textToSpeech}}{}
% El robot dice el texto pasado como argumento.

% \end{fulllineitems}



% \paragraph{sayAnnotatedAnimatedSpeech}
% \label{\detokenize{dev_docs:sayannotatedanimatedspeech}}\index{sayAnnotatedAnimatedSpeech(String) (Java method)}

% \begin{fulllineitems}
% \phantomsection\label{\detokenize{dev_docs:com.lar.cloudnao.Robot.sayAnnotatedAnimatedSpeech(String)}}\pysiglinewithargsret{public void \sphinxbfcode{sayAnnotatedAnimatedSpeech}}{\sphinxhref{http://docs.oracle.com/javase/8/docs/api/java/lang/String.html}{String}\sphinxstyleemphasis{ textToSay}}{}
% Ejecuta un saludo a través de ALTextToSpeech

% \end{fulllineitems}



% \paragraph{stopMovement}
% \label{\detokenize{dev_docs:stopmovement}}\index{stopMovement() (Java method)}

% \begin{fulllineitems}
% \phantomsection\label{\detokenize{dev_docs:com.lar.cloudnao.Robot.stopMovement()}}\pysiglinewithargsret{public void \sphinxbfcode{stopMovement}}{}{}
% Detiene los movimientos del robot

% \end{fulllineitems}



% \textbf{subscribeToCamera}
% \label{\detokenize{dev_docs:subscribetocamera}}\index{subscribeToCamera(int) (Java method)}

% \begin{fulllineitems}
% \phantomsection\label{\detokenize{dev_docs:com.lar.cloudnao.Robot.subscribeToCamera(int)}}\pysiglinewithargsret{public void \sphinxbfcode{subscribeToCamera}}{int\sphinxstyleemphasis{ resolution}}{}
% Se suscribe a la cámara para poder recibir imágenes de manera remota.
% \begin{quote}\begin{description}
% \item[{Parámetros}] \leavevmode\begin{itemize}
% \item {} 
% \sphinxstyleliteralstrong{resolution} \textendash{} un entero válido para la resolución de la cámara del robot.

% \end{itemize}

% \end{description}\end{quote}

% \end{fulllineitems}



% \textbf{subscribeToCamera}
% \label{\detokenize{dev_docs:id39}}\index{subscribeToCamera() (Java method)}

% \begin{fulllineitems}
% \phantomsection\label{\detokenize{dev_docs:com.lar.cloudnao.Robot.subscribeToCamera()}}\pysiglinewithargsret{public void \sphinxbfcode{subscribeToCamera}}{}{}
% Se suscribe a la cámara para recibir imágenes de manera remota, con la resolución por defecto 160 px * 120 px.

% \end{fulllineitems}



% \textbf{subscribeToSonar}
% \label{\detokenize{dev_docs:subscribetosonar}}\index{subscribeToSonar() (Java method)}

% \begin{fulllineitems}
% \phantomsection\label{\detokenize{dev_docs:com.lar.cloudnao.Robot.subscribeToSonar()}}\pysiglinewithargsret{public void \sphinxbfcode{subscribeToSonar}}{}{}
% Se suscribe al sonar para poder obtener valores de ALMemory

% \end{fulllineitems}



% \textbf{unsubscribeToCamera}
% \label{\detokenize{dev_docs:unsubscribetocamera}}\index{unsubscribeToCamera() (Java method)}

% \begin{fulllineitems}
% \phantomsection\label{\detokenize{dev_docs:com.lar.cloudnao.Robot.unsubscribeToCamera()}}\pysiglinewithargsret{public void \sphinxbfcode{unsubscribeToCamera}}{}{}
% Se da de baja el suscriptor de la cámara.

% \end{fulllineitems}



% \textbf{unsubsribeToSonar}
% \label{\detokenize{dev_docs:unsubsribetosonar}}\index{unsubsribeToSonar() (Java method)}

% \begin{fulllineitems}
% \phantomsection\label{\detokenize{dev_docs:com.lar.cloudnao.Robot.unsubsribeToSonar()}}\pysiglinewithargsret{public void \sphinxbfcode{unsubsribeToSonar}}{}{}
% Se da de baja el suscriptor del sonar.

% \end{fulllineitems}



% \textbf{wakeUp}
% \label{\detokenize{dev_docs:wakeup}}\index{wakeUp() (Java method)}

% \begin{fulllineitems}
% \phantomsection\label{\detokenize{dev_docs:com.lar.cloudnao.Robot.wakeUp()}}\pysiglinewithargsret{public void \sphinxbfcode{wakeUp}}{}{}
% Cambia al robot a la postura Init.

% \end{fulllineitems}

