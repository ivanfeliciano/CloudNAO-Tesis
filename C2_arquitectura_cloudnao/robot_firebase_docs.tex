\subsection{Firebase y el robot NAO}\label{\detokenize{firebase-nao-robot}}
El robot NAO dentro de la arquitectura CloudNAO es sobre quien se aplican las
otras tecnologías desarrolladas. La aplicación móvil, utilizando el SDK para Android de NAOqi,
controla al robot, recibe imágenes para sus procesamiento en la nube, almacena
logs para sus posteriores análisis usando la Firebase, también en la nube, y muchas
funcionalidades más. Sin embargo, en esta sección se describe el robot como componente
que no sólo recibe órdenes y envía datos sin ejecutar un programa localmente.

El cómputo en la nube como ya se habló en en secciones pasadas, incluye también
que el robot aunque no realice todo el procesamiento que conlleva el ejecutar
un modelo de aprendizaje profundo para analizar imágenes, procese la salida
de un algoritmo para realizar cierta tarea que desee.

En esta sección se describe cómo conectar al robot a la aplicación web utilizando
como intermediario a Firebase. También se describe como conectarse a la API RESTful
de CloudNAO para que el robot realice algunas actividades con los resultados
que adquiere de ésta.

Para conectar remotamente a la aplicación web con el robot NAO, es necesario
un intermediario, ya que se quiere que la conexión no sea dentro de una misma red
y la respuesta sea inmediata, Ese intermediario es Firebase, que sirve como
backend para conectar estas plataformas.

El robot es una plataforma con recursos muy limitados, no hay un SDK
específicamente hecho para el robot. Por lo anterior, se utilza la API REST
de Firebase Realtime Database para envíar y recibir información a través de
internet. Sin embargo, a pesar de ser una API REST, fue necesaria la creación
de un módulo escrito en Python que sirviera como una interfaz de alto nivel,
donde hacer una petición HTTP fuera igual de sencillo que llamar una función en
los SDK disponibles.


\paragraph{Módulo auxiliar para la API REST}
\label{\detokenize{nao_firebase:modulo-auxiliar-para-la-api-rest}}
Para utilizar la API REST de manera más simple se desarrolló una biblioteca
que encapsulara los métodos de lectura, escritura y el streaming de datos.

Es un módulo de Python, compuesto por dos partes principales; la primera parte
es aquella que maneja las peticiones de escritura y lectura usando los métodos
HTTP antes mencionados, la segunda parte es la encargada de recibir las
notificaciones de parte del servidor. Para la primera el uso de la biblioteca
\sphinxcode{\sphinxupquote{requests}} es suficiente. Para la segunda, se emplean las bibliotecas
\sphinxcode{\sphinxupquote{threading}}, para que en un hilo se escuchen los cambios, \sphinxcode{\sphinxupquote{sseclient}},
un cliente del protocolo \sphinxstylestrong{SSE}, y \sphinxcode{\sphinxupquote{socket}}, el hilo del clente \sphinxstylestrong{SSE}.

A continuación se describe cada componente del módulo \sphinxcode{\sphinxupquote{firebase}}.

\phantomsection\label{\detokenize{nao_firebase:module-firebase}}\index{firebase (módulo)}\index{ClosableSSEClient (clase en firebase)}

\begin{fulllineitems}
\phantomsection\label{\detokenize{nao_firebase:firebase.ClosableSSEClient}}\pysiglinewithargsret{\sphinxbfcode{\sphinxupquote{class }}\sphinxcode{\sphinxupquote{firebase.}}\sphinxbfcode{\sphinxupquote{ClosableSSEClient}}}{\emph{*args}, \emph{**kwargs}}{}
Una clase pública desarrollada por el equipo de Firebase.
Añade una funcionalidad al módulo \sphinxcode{\sphinxupquote{SSEClient}} para
abandonar correctamete el streaming.
\index{close() (método de firebase.ClosableSSEClient)}

\begin{fulllineitems}
\phantomsection\label{\detokenize{nao_firebase:firebase.ClosableSSEClient.close}}\pysiglinewithargsret{\sphinxbfcode{\sphinxupquote{close}}}{}{}
Añade la funcionalidad necesaria para salir del streaming de manera correcta
Se busca el socket dentro del módulo cliente de SSE y se cierra para lanzar una
excepción y abandonar el streaming.

\end{fulllineitems}


\end{fulllineitems}

\index{FirebaseDatabase (clase en firebase)}

\begin{fulllineitems}
\phantomsection\label{\detokenize{nao_firebase:firebase.FirebaseDatabase}}\pysiglinewithargsret{\sphinxbfcode{\sphinxupquote{class }}\sphinxcode{\sphinxupquote{firebase.}}\sphinxbfcode{\sphinxupquote{FirebaseDatabase}}}{\emph{url}}{}
Una clase para usar utilizar la API REST de Firebase con métodos de alto nivel.
Sirve para crear algo similar a las \sphinxstyleemphasis{Referencias} denifidas por Firebase.
Un objeto de esta clase representa una ubicación específica en la base de datos.
Con los métodos se realizan operaciones de escritura y lectura sobre esa ubicación.

El constructor de la clase se encarga de definir la ubicación de la base de datos
sobre la que se harán las operaciones.
Necesita el diccionario \sphinxcode{\sphinxupquote{config}} que contiene el valor del parámetro de autorización.
\index{add\_event\_listener() (método de firebase.FirebaseDatabase)}

\begin{fulllineitems}
\phantomsection\label{\detokenize{nao_firebase:firebase.FirebaseDatabase.add_event_listener}}\pysiglinewithargsret{\sphinxbfcode{\sphinxupquote{add\_event\_listener}}}{\emph{handler\_function}}{}~\index{add\_event\_listener() (método de firebase.FirebaseDatabase)}

\begin{fulllineitems}
\pysiglinewithargsret{\sphinxbfcode{\sphinxupquote{add\_event\_listener}}}{\emph{handler\_function}}{}
\end{fulllineitems}


Este método crea un hilo para escuchar los cambios en la ubicación actual
de la base de datos, y manejar esos cambios en una función.
\begin{quote}\begin{description}
\item[{Parámetros}] \leavevmode
\sphinxstyleliteralstrong{\sphinxupquote{handler\_function}} \textendash{} La función que maneja los eventos enviados por firebase.

\end{description}\end{quote}

\end{fulllineitems}

\index{child() (método de firebase.FirebaseDatabase)}

\begin{fulllineitems}
\phantomsection\label{\detokenize{nao_firebase:firebase.FirebaseDatabase.child}}\pysiglinewithargsret{\sphinxbfcode{\sphinxupquote{child}}}{\emph{path}}{}~
Se encarga de crear una nueva instancia con una ubicación relativa a la actual.
\begin{quote}\begin{description}
\item[{Parámetros}] \leavevmode
\sphinxstyleliteralstrong{\sphinxupquote{path}} \textendash{} La ruta que se añade a la ubicación actual.

\item[{Devuelve}] \leavevmode
Un nuevo objeto de \sphinxcode{\sphinxupquote{FirebaseDatabase}}

\end{description}\end{quote}

\end{fulllineitems}

\index{get() (método de firebase.FirebaseDatabase)}

\begin{fulllineitems}
\phantomsection\label{\detokenize{nao_firebase:firebase.FirebaseDatabase.get}}\pysiglinewithargsret{\sphinxbfcode{\sphinxupquote{get}}}{}{}~
Este método obtiene los datos asociados con la ubicación del objeto a través
del método \sphinxcode{\sphinxupquote{GET}} del protocolo HTTP.

\end{fulllineitems}

\index{push() (método de firebase.FirebaseDatabase)}

\begin{fulllineitems}
\phantomsection\label{\detokenize{nao_firebase:firebase.FirebaseDatabase.push}}\pysiglinewithargsret{\sphinxbfcode{\sphinxupquote{push}}}{\emph{data}}{}~
Genera una nueva ubicación hija de la actual. Esta ubicación tiene un
llave aleatoria única y se escriben los datos en esta.
Por ejemplo, al hacer un \sphinxcode{\sphinxupquote{push(\{"name" : "Rick Deckard"\})}} a \sphinxcode{\sphinxupquote{/users/}}
se produce \sphinxcode{\sphinxupquote{/users/-L43aJQFQNzAIXry8\_6g/name/}} con un valor \sphinxcode{\sphinxupquote{Rick Deckard}}.
Simplemente hace un \sphinxcode{\sphinxupquote{POST}} con los datos a enviar  a la URL del objeto.
\begin{quote}\begin{description}
\item[{Parámetros}] \leavevmode
\sphinxstyleliteralstrong{\sphinxupquote{data}} (\sphinxstyleliteralemphasis{\sphinxupquote{Cualquier tipo de datos válido en un JSON.}}) \textendash{} Los datos que se envían en la petición.

\end{description}\end{quote}

\end{fulllineitems}

\index{remove() (método de firebase.FirebaseDatabase)}

\begin{fulllineitems}
\phantomsection\label{\detokenize{nao_firebase:firebase.FirebaseDatabase.remove}}\pysiglinewithargsret{\sphinxbfcode{\sphinxupquote{remove}}}{}{}~
Elimina todos los datos en la ubicación actual.

\end{fulllineitems}

\index{remove\_event\_listener() (método de firebase.FirebaseDatabase)}

\begin{fulllineitems}
\phantomsection\label{\detokenize{nao_firebase:firebase.FirebaseDatabase.remove_event_listener}}\pysiglinewithargsret{\sphinxbfcode{\sphinxupquote{remove\_event\_listener}}}{}{}~\index{remove\_event\_listener() (método de firebase.FirebaseDatabase)}

\begin{fulllineitems}
\pysiglinewithargsret{\sphinxbfcode{\sphinxupquote{remove\_event\_listener}}}{}{}
\end{fulllineitems}


Abandona el streaming de Firebase y une al hilo con el hilo principal.

\end{fulllineitems}

\index{set() (método de firebase.FirebaseDatabase)}

\begin{fulllineitems}
\phantomsection\label{\detokenize{nao_firebase:firebase.FirebaseDatabase.set}}\pysiglinewithargsret{\sphinxbfcode{\sphinxupquote{set}}}{\emph{data}}{}~
Escribe datos en la ubicación actual. Esto sobreescribe cualquier información
que contenga.
Hace un \sphinxcode{\sphinxupquote{PUT}} sobre la ubicación actual.
:param data: Los datos que se envían en la petición.
:type data: Cualquier tipo de datos válido en un JSON.

\end{fulllineitems}

\index{update() (método de firebase.FirebaseDatabase)}

\begin{fulllineitems}
\phantomsection\label{\detokenize{nao_firebase:firebase.FirebaseDatabase.update}}\pysiglinewithargsret{\sphinxbfcode{\sphinxupquote{update}}}{\emph{data}}{}~
Escribe multiples valores en la ubicación actual de la base de datos. A diferencia
de {\hyperref[\detokenize{nao_firebase:firebase.FirebaseDatabase.set}]{\sphinxcrossref{\sphinxcode{\sphinxupquote{set()}}}}} solo actualiza los valores que se desean de acuerdo a la ubicación
actual. Se hace un \sphinxcode{\sphinxupquote{PATCH}} al URL.
\begin{quote}\begin{description}
\item[{Parámetros}] \leavevmode
\sphinxstyleliteralstrong{\sphinxupquote{data}} (\sphinxstyleliteralemphasis{\sphinxupquote{Cualquier tipo de datos válido en un JSON.}}) \textendash{} Los datos que se envían en la petición.

\end{description}\end{quote}

\end{fulllineitems}


\end{fulllineitems}

\index{RemoteThread (clase en firebase)}

\begin{fulllineitems}
\phantomsection\label{\detokenize{nao_firebase:firebase.RemoteThread}}\pysiglinewithargsret{\sphinxbfcode{\sphinxupquote{class }}\sphinxcode{\sphinxupquote{firebase.}}\sphinxbfcode{\sphinxupquote{RemoteThread}}}{\emph{url}, \emph{handler}}{}
Una clase creada por el equipo de Firebase para que un hilo
se encargue de los eventos enviados por Firebase, con una
funcionalidad que añadí para que una función callback maneje
los datos enviados por Firebase.
\index{close() (método de firebase.RemoteThread)}

\begin{fulllineitems}
\phantomsection\label{\detokenize{nao_firebase:firebase.RemoteThread.close}}\pysiglinewithargsret{\sphinxbfcode{\sphinxupquote{close}}}{}{}~
Deja el streaming, usando {\hyperref[\detokenize{nao_firebase:firebase.RemoteThread.close}]{\sphinxcrossref{\sphinxcode{\sphinxupquote{close()}}}}} del cliente SSE.

\end{fulllineitems}

\index{run() (método de firebase.RemoteThread)}

\begin{fulllineitems}
\phantomsection\label{\detokenize{nao_firebase:firebase.RemoteThread.run}}\pysiglinewithargsret{\sphinxbfcode{\sphinxupquote{run}}}{}{}~
Método para recibir el streaming de Firebase hasta que se cierre la
conexión.

\end{fulllineitems}


\end{fulllineitems}

\index{check\_response() (en el módulo firebase)}

\begin{fulllineitems}
\phantomsection\label{\detokenize{nao_firebase:firebase.check_response}}\pysiglinewithargsret{\sphinxcode{\sphinxupquote{firebase.}}\sphinxbfcode{\sphinxupquote{check\_response}}}{\emph{response}}{}
Función para verficar el código de estado de la respuesta enviada
por firebase.
\begin{quote}\begin{description}
\item[{Parámetros}] \leavevmode
\sphinxstyleliteralstrong{\sphinxupquote{response}} \textendash{} El objeto con la respuesta enviada por Firebase.

\item[{Devuelve}] \leavevmode
Un diccionario con una respuesta de la petición.

\item[{Type}] \leavevmode
dict

\end{description}\end{quote}

\end{fulllineitems}



\paragraph{Módulo para utilizar la API de NAOqi}
\label{\detokenize{nao_firebase:modulo-para-utiizar-modulos-de-naoqi}}
Este módulo es el que se ocupa de integrar la API de NAOqi en la aplicación
que se corre sobre el robot. Aquí se crean los proxies a los módulos de NAOqi,
y los métodos que se utilizan para obtener datos de la memoria del robot,
realizar un movimiento, hablar, entre otros.

\phantomsection\label{\detokenize{nao_firebase:module-nao_robot}}\index{nao\_robot (módulo)}\index{Robot (clase en nao\_robot)}

\begin{fulllineitems}
\phantomsection\label{\detokenize{nao_firebase:nao_robot.Robot}}\pysiglinewithargsret{\sphinxbfcode{\sphinxupquote{class }}\sphinxcode{\sphinxupquote{nao\_robot.}}\sphinxbfcode{\sphinxupquote{Robot}}}{\emph{ip\_address}, \emph{port}}{}
Una clase para encapsular todos los métodos del robot
que van a recibir o enviar información a Firebase.

Los atributos de clase y sus valores son los siguientes:

\fvset{hllines={, ,}}%
\begin{sphinxVerbatim}[commandchars=\\\{\}]
\PYG{n}{cam\PYGZus{}idx} \PYG{o}{=} \PYG{l+m+mi}{0}
\PYG{n}{resolution} \PYG{o}{=} \PYG{l+m+mi}{0}
\PYG{n}{color\PYGZus{}space} \PYG{o}{=} \PYG{l+m+mi}{11}
\PYG{n}{fps} \PYG{o}{=} \PYG{l+m+mi}{10}
\PYG{n}{camera\PYGZus{}subscriber} \PYG{o}{=} \PYG{l+s+s1}{\PYGZsq{}}\PYG{l+s+s1}{cameraSubscriber0001}\PYG{l+s+s1}{\PYGZsq{}}
\PYG{n}{sonar\PYGZus{}subscriber} \PYG{o}{=} \PYG{l+s+s1}{\PYGZsq{}}\PYG{l+s+s1}{sonarSubscriber0001}\PYG{l+s+s1}{\PYGZsq{}}
\PYG{n}{move\PYGZus{}config} \PYG{o}{=} \PYG{p}{[}
    \PYG{p}{[}\PYG{l+s+s1}{\PYGZsq{}}\PYG{l+s+s1}{MaxStepFrequency}\PYG{l+s+s1}{\PYGZsq{}}\PYG{p}{,} \PYG{l+m+mf}{0.5}\PYG{p}{]}\PYG{p}{,}
    \PYG{p}{[}\PYG{l+s+s1}{\PYGZsq{}}\PYG{l+s+s1}{MaxStepTheta}\PYG{l+s+s1}{\PYGZsq{}}\PYG{p}{,} \PYG{l+m+mf}{0.349}\PYG{p}{]}\PYG{p}{,}
    \PYG{p}{[}\PYG{l+s+s1}{\PYGZsq{}}\PYG{l+s+s1}{MaxStepX}\PYG{l+s+s1}{\PYGZsq{}}\PYG{p}{,} \PYG{l+m+mf}{0.04}\PYG{p}{]}\PYG{p}{,}
    \PYG{p}{[}\PYG{l+s+s1}{\PYGZsq{}}\PYG{l+s+s1}{MaxStepY}\PYG{l+s+s1}{\PYGZsq{}}\PYG{p}{,} \PYG{l+m+mf}{0.14}\PYG{p}{]}\PYG{p}{,}
    \PYG{p}{[}\PYG{l+s+s1}{\PYGZsq{}}\PYG{l+s+s1}{StepHeight}\PYG{l+s+s1}{\PYGZsq{}}\PYG{p}{,} \PYG{l+m+mf}{0.02}\PYG{p}{]}
\PYG{p}{]}
\end{sphinxVerbatim}

El constructor de la clase crea los proxies a los módulos de NAOqi.
Recibe la la dirección ip y el puerto del robot.
\index{change\_posture() (método de nao\_robot.Robot)}

\begin{fulllineitems}
\phantomsection\label{\detokenize{nao_firebase:nao_robot.Robot.change_posture}}\pysiglinewithargsret{\sphinxbfcode{\sphinxupquote{change\_posture}}}{\emph{posture}}{}
Cambia entre dos posturas del robot, \sphinxstylestrong{Stand} y \sphinxstylestrong{Crouch}.
\begin{quote}\begin{description}
\item[{Parámetros}] \leavevmode
\sphinxstyleliteralstrong{\sphinxupquote{posture}} \textendash{} El nombre de la postura del robot que se desea obtener

\end{description}\end{quote}

\end{fulllineitems}

\index{get\_battery\_level() (método de nao\_robot.Robot)}

\begin{fulllineitems}
\phantomsection\label{\detokenize{nao_firebase:nao_robot.Robot.get_battery_level}}\pysiglinewithargsret{\sphinxbfcode{\sphinxupquote{get\_battery\_level}}}{}{}
Solicita el valor del nivel de la batería desde la memoria del robot
y lo retorna
\begin{quote}\begin{description}
\item[{Devuelve}] \leavevmode
Un número con el nivel actual de la batería

\item[{Type}] \leavevmode
int

\end{description}\end{quote}

\end{fulllineitems}

\index{get\_image\_from\_robot() (método de nao\_robot.Robot)}

\begin{fulllineitems}
\phantomsection\label{\detokenize{nao_firebase:nao_robot.Robot.get_image_from_robot}}\pysiglinewithargsret{\sphinxbfcode{\sphinxupquote{get\_image\_from\_robot}}}{}{}
Obtiene la imagen del robot codificada en base64. Primero con
\sphinxcode{\sphinxupquote{getImageRemote()}} recibe el contenedor de la imagen, y luego,
el arreglo de bytes de la imagen en el contenedor, se guarda en una
estructura de \sphinxcode{\sphinxupquote{PIL}} para que pueda convertirse en una cadena.
\begin{quote}\begin{description}
\item[{Devuelve}] \leavevmode
La imagen codificada en base64

\item[{Type}] \leavevmode
str

\end{description}\end{quote}

\end{fulllineitems}

\index{get\_values\_from\_memory() (método de nao\_robot.Robot)}

\begin{fulllineitems}
\phantomsection\label{\detokenize{nao_firebase:nao_robot.Robot.get_values_from_memory}}\pysiglinewithargsret{\sphinxbfcode{\sphinxupquote{get\_values\_from\_memory}}}{}{}
Obtiene valores de la memoria del robot y retorna un diccionario
con sus llaves y valores.
\begin{quote}\begin{description}
\item[{Devuelve}] \leavevmode
Un diccionario con los valores de \sphinxcode{\sphinxupquote{ALMemory}}

\item[{Type}] \leavevmode
dict

\end{description}\end{quote}

\end{fulllineitems}

\index{move\_joint() (método de nao\_robot.Robot)}

\begin{fulllineitems}
\phantomsection\label{\detokenize{nao_firebase:nao_robot.Robot.move_joint}}\pysiglinewithargsret{\sphinxbfcode{\sphinxupquote{move\_joint}}}{\emph{joint\_name}, \emph{angle}}{}
Mueve una articulación del robot con respecto a la posción inicial de una
articulación. Primero activa la rigidez del motor que mueve la articulación, realiza el movimiento en dos
segundos y luego desactiva la rigidez del motor.

\end{fulllineitems}

\index{move\_robot() (método de nao\_robot.Robot)}

\begin{fulllineitems}
\phantomsection\label{\detokenize{nao_firebase:nao_robot.Robot.move_robot}}\pysiglinewithargsret{\sphinxbfcode{\sphinxupquote{move\_robot}}}{\emph{\_x}, \emph{\_y}, \emph{\_theta}}{}
Método para ejecutar hacer caminar utilizando \sphinxcode{\sphinxupquote{moveToward}} de la
API de NAOqi.
\begin{quote}\begin{description}
\item[{Parámetros}] \leavevmode\begin{itemize}
\item {} 
\sphinxstyleliteralstrong{\sphinxupquote{\_x}} \textendash{} La velocidad (-1, 1) sobre el eje \sphinxstylestrong{X}

\item {} 
\sphinxstyleliteralstrong{\sphinxupquote{\_y}} \textendash{} La velocidad (-1, 1) sobre el eje \sphinxstylestrong{Y}

\item {} 
\sphinxstyleliteralstrong{\sphinxupquote{\_theta}} \textendash{} La velocidad (-1, 1) sobre el eje \sphinxstylestrong{Z}

\end{itemize}

\end{description}\end{quote}

\end{fulllineitems}

\index{say\_speech() (método de nao\_robot.Robot)}

\begin{fulllineitems}
\phantomsection\label{\detokenize{nao_firebase:nao_robot.Robot.say_speech}}\pysiglinewithargsret{\sphinxbfcode{\sphinxupquote{say\_speech}}}{\emph{text}}{}
Ejecuta la función \sphinxcode{\sphinxupquote{say}} de \sphinxcode{\sphinxupquote{ALTextToSpeech}}. Dice la cadena
enviada como parámetro.
\begin{quote}\begin{description}
\item[{Parámetros}] \leavevmode
\sphinxstyleliteralstrong{\sphinxupquote{text}} \textendash{} El texto que debe decir el robot

\end{description}\end{quote}

\end{fulllineitems}

\index{set\_move\_config() (método de nao\_robot.Robot)}

\begin{fulllineitems}
\phantomsection\label{\detokenize{nao_firebase:nao_robot.Robot.set_move_config}}\pysiglinewithargsret{\sphinxbfcode{\sphinxupquote{set\_move\_config}}}{\emph{move\_config\_map}}{}
Cambia los parámetros de caminado del robot. Recibe un diccionario
con los nuevos parámetros y sus valores. Los procesa para ver si son válidos
y cambiarlos.

\end{fulllineitems}

\index{stop\_movement() (método de nao\_robot.Robot)}

\begin{fulllineitems}
\phantomsection\label{\detokenize{nao_firebase:nao_robot.Robot.stop_movement}}\pysiglinewithargsret{\sphinxbfcode{\sphinxupquote{stop\_movement}}}{}{}
Método para detener los movimientos del robot, llama a \sphinxcode{\sphinxupquote{stopMove()}}
de \sphinxcode{\sphinxupquote{ALMotion}}.

\end{fulllineitems}

\index{subscribe\_to\_camera() (método de nao\_robot.Robot)}

\begin{fulllineitems}
\phantomsection\label{\detokenize{nao_firebase:nao_robot.Robot.subscribe_to_camera}}\pysiglinewithargsret{\sphinxbfcode{\sphinxupquote{subscribe\_to\_camera}}}{}{}
Se suscribe a la cámara del robot, con los parámetros que se definieron
en los atributos de la clase.

\end{fulllineitems}

\index{subscribe\_to\_sonar() (método de nao\_robot.Robot)}

\begin{fulllineitems}
\phantomsection\label{\detokenize{nao_firebase:nao_robot.Robot.subscribe_to_sonar}}\pysiglinewithargsret{\sphinxbfcode{\sphinxupquote{subscribe\_to\_sonar}}}{}{}
Se suscribe al sonar para que se actualicen los valores en la memoria
con las distancias a obstáculos.

\end{fulllineitems}

\index{unsubscribe\_to\_camera() (método de nao\_robot.Robot)}

\begin{fulllineitems}
\phantomsection\label{\detokenize{nao_firebase:nao_robot.Robot.unsubscribe_to_camera}}\pysiglinewithargsret{\sphinxbfcode{\sphinxupquote{unsubscribe\_to\_camera}}}{}{}
Da de baja al suscriptor de la cámara

\end{fulllineitems}

\index{unsubscribe\_to\_sonar() (método de nao\_robot.Robot)}

\begin{fulllineitems}
\phantomsection\label{\detokenize{nao_firebase:nao_robot.Robot.unsubscribe_to_sonar}}\pysiglinewithargsret{\sphinxbfcode{\sphinxupquote{unsubscribe\_to\_sonar}}}{}{}
Se apagan los sonares.

\end{fulllineitems}


\end{fulllineitems}



\paragraph{Conectando la API REST de Firebase y NAOqi}
\label{\detokenize{nao_firebase:conectando-la-api-rest-de-firebase-y-naoqi}}
Después de contar con una biblioteca cliente para realizar peticiones a la
API REST de Firebase y con un módulo que utilice la API de NAOqi sólo
resta unirlos en una aplicación que sea fácil de ejecutar y que responda con
funciones de módulos de NAOqi a actualizaciones enviadas por Firebase o
que envíe información a ubicaciones de la base de datos.

Este aplicación debe poder ejecutarse local y remotamente (dentro de la misma red)
en el robot.

Para mantener simple la ejecución del script que conecta al robot con Firebase
la aplicación está dividida en cinco módulos. Estos incluyen a los módulos
descritos anteriormente, \sphinxcode{\sphinxupquote{firebase}} y \sphinxcode{\sphinxupquote{nao\_robot}}.


\subsubsection{app.py}
\label{\detokenize{nao_firebase:app-py}}\label{\detokenize{nao_firebase:module-app}}\index{app (módulo)}\index{main() (en el módulo app)}

\begin{fulllineitems}
\phantomsection\label{\detokenize{nao_firebase:app.main}}\pysiglinewithargsret{\sphinxcode{\sphinxupquote{app.}}\sphinxbfcode{\sphinxupquote{main}}}{}{}
Inicia la aplicación, obtiene la dirección ip y puerto del robot a través
de los argumentos enviados en la línea de comandos, luego llama a la Función
\sphinxcode{\sphinxupquote{run()}} del módulo \sphinxcode{\sphinxupquote{fire\_nao}}.

\end{fulllineitems}



\subsubsection{confg.py}
\label{\detokenize{nao_firebase:confg-py}}
Este archivo unicamente tiene un diccionario con la configuración para Firebase
con el siguiente formato:

\fvset{hllines={, ,}}%
\begin{sphinxVerbatim}[commandchars=\\\{\}]
\PYG{n}{config} \PYG{o}{=} \PYG{p}{\PYGZob{}}
    \PYG{l+s+s2}{\PYGZdq{}}\PYG{l+s+s2}{databaseURL}\PYG{l+s+s2}{\PYGZdq{}}\PYG{p}{:} \PYG{l+s+s2}{\PYGZdq{}}\PYG{l+s+s2}{https://url\PYGZhy{}de\PYGZhy{}firebase.com/}\PYG{l+s+s2}{\PYGZdq{}}\PYG{p}{,}
    \PYG{l+s+s2}{\PYGZdq{}}\PYG{l+s+s2}{auth}\PYG{l+s+s2}{\PYGZdq{}}\PYG{p}{:} \PYG{l+s+s2}{\PYGZdq{}}\PYG{l+s+s2}{API KEY del proyecto de Firebase}\PYG{l+s+s2}{\PYGZdq{}}\PYG{p}{,}
    \PYG{l+s+s2}{\PYGZdq{}}\PYG{l+s+s2}{robotUID}\PYG{l+s+s2}{\PYGZdq{}}\PYG{p}{:} \PYG{l+s+s2}{\PYGZdq{}}\PYG{l+s+s2}{identificador único del robot generado por Firebase}\PYG{l+s+s2}{\PYGZdq{}}
\PYG{p}{\PYGZcb{}}
\end{sphinxVerbatim}


\subsubsection{fire\_nao.py}
\label{\detokenize{nao_firebase:fire-nao-py}}
Este módulo es quien integra a \sphinxcode{\sphinxupquote{firebase{}`{}`{}`y {}`{}`nao\_robot}} para comunicar al robot
con la aplicación web a través de Firebase.

\phantomsection\label{\detokenize{nao_firebase:module-fire_nao}}\index{fire\_nao (módulo)}\index{REF\_TO\_COMMANDS (en el módulo fire\_nao)}

\begin{fulllineitems}
\phantomsection\label{\detokenize{nao_firebase:fire_nao.REF_TO_COMMANDS}}\pysigline{\sphinxcode{\sphinxupquote{fire\_nao.}}\sphinxbfcode{\sphinxupquote{REF\_TO\_COMMANDS}}\sphinxbfcode{\sphinxupquote{ = \textless{}firebase.FirebaseDatabase object\textgreater{}}}}
Referencia a la ubicación de los comandos en la base de datos

\end{fulllineitems}

\index{REF\_TO\_LIVE\_IMAGE (en el módulo fire\_nao)}

\begin{fulllineitems}
\phantomsection\label{\detokenize{nao_firebase:fire_nao.REF_TO_LIVE_IMAGE}}\pysigline{\sphinxcode{\sphinxupquote{fire\_nao.}}\sphinxbfcode{\sphinxupquote{REF\_TO\_LIVE\_IMAGE}}\sphinxbfcode{\sphinxupquote{ = \textless{}firebase.FirebaseDatabase object\textgreater{}}}}
Referencia a la ubicación donde se enviarán las imágenes de la cámara del robot

\end{fulllineitems}

\index{REF\_TO\_LOGS (en el módulo fire\_nao)}

\begin{fulllineitems}
\phantomsection\label{\detokenize{nao_firebase:fire_nao.REF_TO_LOGS}}\pysigline{\sphinxcode{\sphinxupquote{fire\_nao.}}\sphinxbfcode{\sphinxupquote{REF\_TO\_LOGS}}\sphinxbfcode{\sphinxupquote{ = \textless{}firebase.FirebaseDatabase object\textgreater{}}}}
Referencia donde se guardan los logs del robot en la base de datos

\end{fulllineitems}

\index{REF\_TO\_STATUS (en el módulo fire\_nao)}

\begin{fulllineitems}
\phantomsection\label{\detokenize{nao_firebase:fire_nao.REF_TO_STATUS}}\pysigline{\sphinxcode{\sphinxupquote{fire\_nao.}}\sphinxbfcode{\sphinxupquote{REF\_TO\_STATUS}}\sphinxbfcode{\sphinxupquote{ = \textless{}firebase.FirebaseDatabase object\textgreater{}}}}
Referencia a la ubicación con el estado de la conexión del robot y el nivel de batería

\end{fulllineitems}

\index{ROBOT (en el módulo fire\_nao)}

\begin{fulllineitems}
\phantomsection\label{\detokenize{nao_firebase:fire_nao.ROBOT}}\pysigline{\sphinxcode{\sphinxupquote{fire\_nao.}}\sphinxbfcode{\sphinxupquote{ROBOT}}\sphinxbfcode{\sphinxupquote{ = None}}}
La instancia del objeto Robot, para ejecutar funciones de NAOqi. Se incializa con valor nulo.

\end{fulllineitems}

\index{ROBOT\_OPTIONS (en el módulo fire\_nao)}

\begin{fulllineitems}
\phantomsection\label{\detokenize{nao_firebase:fire_nao.ROBOT_OPTIONS}}\pysigline{\sphinxcode{\sphinxupquote{fire\_nao.}}\sphinxbfcode{\sphinxupquote{ROBOT\_OPTIONS}}\sphinxbfcode{\sphinxupquote{ = \{'changePosture': \textless{}function change\_posture at 0x7effe8e46488\textgreater{}, 'gaitParameters': \textless{}function set\_move\_config at 0x7effe8e46840\textgreater{}, 'moveJoint': \textless{}function move\_joint at 0x7effe8e46378\textgreater{}, 'speech': \textless{}function say at 0x7effe8e46730\textgreater{}, 'walk': \textless{}function walk at 0x7effe8e46510\textgreater{}, 'walkSpeed': \textless{}function set\_velocity at 0x7effe8e46400\textgreater{}\}}}}
El diccionario que mapea los valores recibidos de Firebase a una función que ejecute el comando.

\end{fulllineitems}

\index{ROOT\_DB\_REF (en el módulo fire\_nao)}

\begin{fulllineitems}
\phantomsection\label{\detokenize{nao_firebase:fire_nao.ROOT_DB_REF}}\pysigline{\sphinxcode{\sphinxupquote{fire\_nao.}}\sphinxbfcode{\sphinxupquote{ROOT\_DB\_REF}}\sphinxbfcode{\sphinxupquote{ = \textless{}firebase.FirebaseDatabase object\textgreater{}}}}
La referencia a la base de datos de Firebase

\end{fulllineitems}

\index{change\_posture() (en el módulo fire\_nao)}

\begin{fulllineitems}
\phantomsection\label{\detokenize{nao_firebase:fire_nao.change_posture}}\pysiglinewithargsret{\sphinxcode{\sphinxupquote{fire\_nao.}}\sphinxbfcode{\sphinxupquote{change\_posture}}}{\emph{*args}}{}
Función que se llama cuando se recibe una actualización que solicita
el cambio de postura del robot.
\begin{quote}\begin{description}
\item[{Parámetros}] \leavevmode
\sphinxstyleliteralstrong{\sphinxupquote{*args}} \textendash{} Se esperan dos argumentos, el primero debe ser nulo y el segundo la postura.

\item[{Args{[}1{]}}] \leavevmode
(\sphinxcode{\sphinxupquote{str}}) La postura del robot (Stand, Crouch)

\end{description}\end{quote}

\end{fulllineitems}

\index{commands\_handler() (en el módulo fire\_nao)}

\begin{fulllineitems}
\phantomsection\label{\detokenize{nao_firebase:fire_nao.commands_handler}}\pysiglinewithargsret{\sphinxcode{\sphinxupquote{fire\_nao.}}\sphinxbfcode{\sphinxupquote{commands\_handler}}}{\emph{**kwargs}}{}
La función del escuchador suscrito a la ubicación de Firebase
\sphinxcode{\sphinxupquote{REF\_TO\_COMMANDS}}. Parse la respuesta y luego de acuerdo al comando enviado
ejecuta la función que corresponde según \sphinxcode{\sphinxupquote{ROBOT\_OPTIONS}}.

\end{fulllineitems}

\index{init\_subscribers() (en el módulo fire\_nao)}

\begin{fulllineitems}
\phantomsection\label{\detokenize{nao_firebase:fire_nao.init_subscribers}}\pysiglinewithargsret{\sphinxcode{\sphinxupquote{fire\_nao.}}\sphinxbfcode{\sphinxupquote{init\_subscribers}}}{}{}
Inicializa los suscriptores. Por ahora sólo la cámara.

\end{fulllineitems}

\index{move\_joint() (en el módulo fire\_nao)}

\begin{fulllineitems}
\phantomsection\label{\detokenize{nao_firebase:fire_nao.move_joint}}\pysiglinewithargsret{\sphinxcode{\sphinxupquote{fire\_nao.}}\sphinxbfcode{\sphinxupquote{move\_joint}}}{\emph{*args}}{}
Función que se llama cuando se recibe una actualización que solicita
mover una articulación.
\begin{quote}\begin{description}
\item[{Parámetros}] \leavevmode
\sphinxstyleliteralstrong{\sphinxupquote{*args}} \textendash{} Se esperan dos argumentos, el nombre de la articulación y el ángulo que debe moverse

\item[{Args{[}0{]}}] \leavevmode
(\sphinxcode{\sphinxupquote{str}}) El nombre de la articulación

\item[{Args{[}1{]}}] \leavevmode
(\sphinxcode{\sphinxupquote{float}}) El valor en grados sexagesimales del ángulo.

\end{description}\end{quote}

\end{fulllineitems}

\index{off\_subscribers() (en el módulo fire\_nao)}

\begin{fulllineitems}
\phantomsection\label{\detokenize{nao_firebase:fire_nao.off_subscribers}}\pysiglinewithargsret{\sphinxcode{\sphinxupquote{fire\_nao.}}\sphinxbfcode{\sphinxupquote{off\_subscribers}}}{}{}
Da de baja los suscriptores.

\end{fulllineitems}

\index{run() (en el módulo fire\_nao)}

\begin{fulllineitems}
\phantomsection\label{\detokenize{nao_firebase:fire_nao.run}}\pysiglinewithargsret{\sphinxcode{\sphinxupquote{fire\_nao.}}\sphinxbfcode{\sphinxupquote{run}}}{\emph{robot\_ip}, \emph{robot\_port}}{}
La función que se llama al ejecutar la aplicación, inicializa la variable
\sphinxcode{\sphinxupquote{ROBOT}} como una instancia de la clase \sphinxcode{\sphinxupquote{Robot()}}. Se añade el agente que
escucha los cambios en una referencia de la base de datos (\sphinxcode{\sphinxupquote{REF\_TO\_COMMANDS}}).
Inician los suscriptores. Luego con declara un bucle que se ejecuta
hasta que el usuario interrumpe el programa. Mientras no se detenga
envía constantes actualizaciones a Firebase. Cuando se detiene,
se dan de baja los suscriptores y se elimina el escuchador.

\end{fulllineitems}

\index{say() (en el módulo fire\_nao)}

\begin{fulllineitems}
\phantomsection\label{\detokenize{nao_firebase:fire_nao.say}}\pysiglinewithargsret{\sphinxcode{\sphinxupquote{fire\_nao.}}\sphinxbfcode{\sphinxupquote{say}}}{\emph{*args}}{}
Función que se llama cuando se recibe una actualización que solicita
que el robot diga un discurso.
\begin{quote}\begin{description}
\item[{Parámetros}] \leavevmode
\sphinxstyleliteralstrong{\sphinxupquote{*args}} \textendash{} Se esperan dos argumentos, el primero debe ser nulo y el segundo el texto que debe repetir el robot.

\item[{Args{[}1{]}}] \leavevmode
(\sphinxcode{\sphinxupquote{str}}) El texto que debe decir el robot.

\end{description}\end{quote}

\end{fulllineitems}

\index{set\_move\_config() (en el módulo fire\_nao)}

\begin{fulllineitems}
\phantomsection\label{\detokenize{nao_firebase:fire_nao.set_move_config}}\pysiglinewithargsret{\sphinxcode{\sphinxupquote{fire\_nao.}}\sphinxbfcode{\sphinxupquote{set\_move\_config}}}{\emph{*args}}{}
Función que se llama cuando se recibe una actualización que solicita
que el robot cambie los parámetros de caminado
\begin{quote}\begin{description}
\item[{Parámetros}] \leavevmode
\sphinxstyleliteralstrong{\sphinxupquote{*args}} \textendash{} Se esperan dos argumentos, el primero debe ser nulo y el segundo un diccionario con los parámetros que se quieren actualizar y los nuevos valores.

\item[{Args{[}1{]}}] \leavevmode
(\sphinxcode{\sphinxupquote{dict}}). Los nuevos valores para los parámetros de caminado

\end{description}\end{quote}

\end{fulllineitems}

\index{set\_velocity() (en el módulo fire\_nao)}

\begin{fulllineitems}
\phantomsection\label{\detokenize{nao_firebase:fire_nao.set_velocity}}\pysiglinewithargsret{\sphinxcode{\sphinxupquote{fire\_nao.}}\sphinxbfcode{\sphinxupquote{set\_velocity}}}{\emph{*args}}{}
Función que se llama cuando se recibe una actualización que solicita
que el robot cambie la velocidad de caminado
\begin{quote}\begin{description}
\item[{Parámetros}] \leavevmode
\sphinxstyleliteralstrong{\sphinxupquote{*args}} \textendash{} Se esperan dos argumentos, el primero debe ser nulo y el segundo la velocidad.

\item[{Args{[}1{]}}] \leavevmode
(\sphinxcode{\sphinxupquote{float}}) La velocidad de caminado (-1, 1)

\end{description}\end{quote}

\end{fulllineitems}

\index{walk() (en el módulo fire\_nao)}

\begin{fulllineitems}
\phantomsection\label{\detokenize{nao_firebase:fire_nao.walk}}\pysiglinewithargsret{\sphinxcode{\sphinxupquote{fire\_nao.}}\sphinxbfcode{\sphinxupquote{walk}}}{\emph{*args}}{}
Función que se llama cuando se recibe una actualización que solicita
que el robot camine o se detenga.
\begin{quote}\begin{description}
\item[{Parámetros}] \leavevmode
\sphinxstyleliteralstrong{\sphinxupquote{*args}} \textendash{} Se esperan dos argumentos, el primero debe ser nulo y el segundo la dirección a del caminado o la bandera de \sphinxcode{\sphinxupquote{stop}} para que se detenga.

\item[{Args{[}1{]}}] \leavevmode
(\sphinxcode{\sphinxupquote{str}}) La dirección del caminado (forward, backward, right, left, turnLeft, turnRight, stop)

\end{description}\end{quote}

\end{fulllineitems}
