\subsection{Firebase y el robot NAO}\label{\detokenize{firebase-nao-robot}}
El robot NAO dentro de la arquitectura CloudNAO es sobre quien se aplican las
otras tecnologías desarrolladas. La aplicación móvil, utilizando el SDK para Android de NAOqi,
controla al robot, recibe imágenes para sus procesamiento en la nube, almacena
logs para sus posteriores análisis usando la Firebase, también en la nube, y muchas
funcionalidades más. Sin embargo, en esta sección se describe el robot como componente
que no sólo recibe órdenes y envía datos sin ejecutar un programa localmente.

% El cómputo en la nube como ya se habló en en secciones pasadas, incluye también
% que el robot aunque no realice todo el procesamiento que conlleva el ejecutar
% un modelo de aprendizaje profundo para analizar imágenes, procese la salida
% de un algoritmo para realizar cierta tarea que desee.

En esta sección se describe cómo conectar al robot a la aplicación web utilizando
como intermediario a Firebase.

Para conectar remotamente a la aplicación web con el robot NAO, es necesario
un intermediario, ya que se quiere que la conexión no sea dentro de una misma red
y la respuesta sea inmediata, Ese intermediario es Firebase, que sirve como
backend para conectar estas plataformas.

El robot es una plataforma con recursos muy limitados, no hay un SDK
de Firebase
específicamente hecho para el robot. Por lo anterior, se utiliza la API REST
de Firebase Realtime Database para enviar y recibir información a través de
internet. Sin embargo, a pesar de ser una API REST, fue necesaria la creación
de un módulo escrito en Python que sirviera como una interfaz de alto nivel,
donde hacer una petición HTTP fuera igual de sencillo que llamar una función en
los SDK disponibles.


\paragraph{Módulo auxiliar para la API REST}
\label{\detokenize{nao_firebase:modulo-auxiliar-para-la-api-rest}}
Para utilizar la API REST de manera más simple se desarrolló una biblioteca
que encapsulara los métodos de lectura, escritura y el streaming de datos.
Es un módulo de Python, compuesto por dos partes principales; la primera parte
es aquella que maneja las peticiones de escritura y lectura usando los métodos
HTTP antes mencionados, la segunda parte es la encargada de recibir las
notificaciones de parte del servidor. Para la primera el uso de la biblioteca
\sphinxcode{\sphinxupquote{requests}} es suficiente. Para la segunda, se emplean las bibliotecas
\sphinxcode{\sphinxupquote{threading}}, para que en un hilo se escuchen los cambios, \sphinxcode{\sphinxupquote{sseclient}},
un cliente del protocolo \sphinxstylestrong{SSE}, y \sphinxcode{\sphinxupquote{socket}}, el hilo del cliente \sphinxstylestrong{SSE}.
A continuación se describe cada elemnto del módulo \sphinxcode{\sphinxupquote{firebase}}.
\newline

\codedocumentation{\sphinxbfcode{\sphinxupquote{class }}\sphinxcode{\sphinxupquote{firebase.}}\sphinxbfcode{\sphinxupquote{ClosableSSEClient}}(\emph{*args}, \emph{**kwargs})}
Una clase pública desarrollada por el equipo de Firebase.
Añade una funcionalidad al módulo \sphinxcode{\sphinxupquote{SSEClient}} para
abandonar correctamente el streaming.

\codedocumentation{{\sphinxbfcode{\sphinxupquote{class }}\sphinxcode{\sphinxupquote{firebase.}}\sphinxbfcode{\sphinxupquote{FirebaseDatabase}}}(\emph{url})}
Una clase para usar utilizar la API REST de Firebase con métodos de alto nivel.
Sirve para crear algo similar a las \sphinxstyleemphasis{Referencias} definidas por Firebase.
Un objeto de esta clase representa una ubicación específica en la base de datos.
Con los métodos se realizan operaciones de escritura y lectura sobre esa ubicación.

El constructor de la clase se encarga de definir la ubicación de la base de datos
sobre la que se harán las operaciones.
Necesita el diccionario \sphinxcode{\sphinxupquote{config}} que contiene el valor del parámetro de autorización.

\begin{fulllineitems}
\phantomsection\label{\detokenize{nao_firebase:firebase.FirebaseDatabase.addeventlistener}}\pysiglinewithargsret{\sphinxbfcode{\sphinxupquote{add\_event\_listener}}}{\emph{handler\_function}}{}~

Este método crea un hilo para escuchar los cambios en la ubicación actual
de la base de datos, y manejar esos cambios en una función.
\begin{quote}\begin{description}
\item[{Parámetros}] \leavevmode
\sphinxstyleliteralstrong{\sphinxupquote{handler\_function}} \textendash{} La función que maneja los eventos enviados por firebase.

\end{description}\end{quote}
\end{fulllineitems}

\begin{fulllineitems}
\phantomsection\label{\detokenize{nao_firebase:firebase.FirebaseDatabase.child}}\pysiglinewithargsret{\sphinxbfcode{\sphinxupquote{child}}}{\emph{path}}{}~
Se encarga de crear una nueva instancia con una ubicación relativa a la actual.
\begin{quote}\begin{description}
\item[{Parámetros}] \leavevmode
\sphinxstyleliteralstrong{\sphinxupquote{path}} \textendash{} La ruta que se añade a la ubicación actual.

\item[{Devuelve}] \leavevmode
Un nuevo objeto de \sphinxcode{\sphinxupquote{FirebaseDatabase}}

\end{description}\end{quote}

\end{fulllineitems}

\index{get() (método de firebase.FirebaseDatabase)}

\begin{fulllineitems}
\phantomsection\label{\detokenize{nao_firebase:firebase.FirebaseDatabase.get}}\pysiglinewithargsret{\sphinxbfcode{\sphinxupquote{get}}}{}{}~
Este método obtiene los datos asociados con la ubicación del objeto a través
del método \sphinxcode{\sphinxupquote{GET}} del protocolo HTTP.

\end{fulllineitems}

\index{push() (método de firebase.FirebaseDatabase)}

\begin{fulllineitems}
\phantomsection\label{\detokenize{nao_firebase:firebase.FirebaseDatabase.push}}\pysiglinewithargsret{\sphinxbfcode{\sphinxupquote{push}}}{\emph{data}}{}~
Genera una nueva ubicación hija de la actual. Esta ubicación tiene un
llave aleatoria única y se escriben los datos en esta.
Por ejemplo, al hacer un \sphinxcode{\sphinxupquote{push(\{"name" : "Rick Deckard"\})}} a \sphinxcode{\sphinxupquote{/users/}}
se produce \sphinxcode{\sphinxupquote{/users/-L43aJQ\\FQNzAIXry8\_6g/name/}} con un valor \sphinxcode{\sphinxupquote{Rick Deckard}}.
Simplemente hace un \sphinxcode{\sphinxupquote{POST}} con los datos a enviar  a la URL del objeto. Recibe como parámetro cualquier estructura válida en un JSON.

\end{fulllineitems}

\index{remove() (método de firebase.FirebaseDatabase)}

\begin{fulllineitems}
\phantomsection\label{\detokenize{nao_firebase:firebase.FirebaseDatabase.remove}}\pysiglinewithargsret{\sphinxbfcode{\sphinxupquote{remove}}}{}{}~
Elimina todos los datos en la ubicación actual.

\end{fulllineitems}

\index{set() (método de firebase.FirebaseDatabase)}

\begin{fulllineitems}
\phantomsection\label{\detokenize{nao_firebase:firebase.FirebaseDatabase.set}}\pysiglinewithargsret{\sphinxbfcode{\sphinxupquote{set}}}{\emph{data}}{}~
Escribe datos en la ubicación actual. Esto sobrescribe cualquier información
que contenga.
Hace un \sphinxcode{\sphinxupquote{PUT}} sobre la ubicación actual.
Recibe cualquier tipo de objeto válido en un JSON.

\end{fulllineitems}

\index{update() (método de firebase.FirebaseDatabase)}

\begin{fulllineitems}
\phantomsection\label{\detokenize{nao_firebase:firebase.FirebaseDatabase.update}}\pysiglinewithargsret{\sphinxbfcode{\sphinxupquote{update}}}{\emph{data}}{}~
Escribe múltiples valores en la ubicación actual de la base de datos. A diferencia
de {\hyperref[\detokenize{nao_firebase:firebase.FirebaseDatabase.set}]{\sphinxcrossref{\sphinxcode{\sphinxupquote{set()}}}}} solo actualiza los valores que se desean de acuerdo a la ubicación
actual. Se hace un \sphinxcode{\sphinxupquote{PATCH}} al URL.
\newline
\end{fulllineitems}

\codedocumentation{{\sphinxbfcode{\sphinxupquote{class }}\sphinxcode{\sphinxupquote{firebase.}}\sphinxbfcode{\sphinxupquote{RemoteThread}}}(\emph{url}, \emph{handler})}
Una clase creada por el equipo de Firebase para que un hilo
se encargue de los eventos enviados por Firebase, con una
funcionalidad que añadí para que una función callback maneje
los datos enviados por Firebase.



\paragraph{Módulo para utilizar la API de NAOqi}
\label{\detokenize{nao_firebase:modulo-para-utiizar-modulos-de-naoqi}}
Este módulo es el que se ocupa de integrar la API de NAOqi en la aplicación
que se corre sobre el robot. Aquí se crean los proxies a los módulos de NAOqi,
y los métodos que se utilizan para obtener datos de la memoria del robot,
realizar un movimiento, hablar, entre otros.
\newline

\codedocumentation{{\sphinxbfcode{\sphinxupquote{class }}\sphinxcode{\sphinxupquote{nao\_robot.}}\sphinxbfcode{\sphinxupquote{Robot}}}(\emph{ip\_address}, \emph{port})}
Una clase para encapsular todos los métodos del robot
que van a recibir o enviar información a Firebase.
Los atributos de clase y sus valores son los siguientes:

\fvset{hllines={, ,}}%
\begin{sphinxVerbatim}[commandchars=\\\{\}]
\PYG{n}{cam\PYGZus{}idx} \PYG{o}{=} \PYG{l+m+mi}{0}
\PYG{n}{resolution} \PYG{o}{=} \PYG{l+m+mi}{0}
\PYG{n}{color\PYGZus{}space} \PYG{o}{=} \PYG{l+m+mi}{11}
\PYG{n}{fps} \PYG{o}{=} \PYG{l+m+mi}{10}
\PYG{n}{camera\PYGZus{}subscriber} \PYG{o}{=} \PYG{l+s+s1}{\PYGZsq{}}\PYG{l+s+s1}{cameraSubscriber0001}\PYG{l+s+s1}{\PYGZsq{}}
\PYG{n}{sonar\PYGZus{}subscriber} \PYG{o}{=} \PYG{l+s+s1}{\PYGZsq{}}\PYG{l+s+s1}{sonarSubscriber0001}\PYG{l+s+s1}{\PYGZsq{}}
\PYG{n}{move\PYGZus{}config} \PYG{o}{=} \PYG{p}{[}
    \PYG{p}{[}\PYG{l+s+s1}{\PYGZsq{}}\PYG{l+s+s1}{MaxStepFrequency}\PYG{l+s+s1}{\PYGZsq{}}\PYG{p}{,} \PYG{l+m+mf}{0.5}\PYG{p}{]}\PYG{p}{,}
    \PYG{p}{[}\PYG{l+s+s1}{\PYGZsq{}}\PYG{l+s+s1}{MaxStepTheta}\PYG{l+s+s1}{\PYGZsq{}}\PYG{p}{,} \PYG{l+m+mf}{0.349}\PYG{p}{]}\PYG{p}{,}
    \PYG{p}{[}\PYG{l+s+s1}{\PYGZsq{}}\PYG{l+s+s1}{MaxStepX}\PYG{l+s+s1}{\PYGZsq{}}\PYG{p}{,} \PYG{l+m+mf}{0.04}\PYG{p}{]}\PYG{p}{,}
    \PYG{p}{[}\PYG{l+s+s1}{\PYGZsq{}}\PYG{l+s+s1}{MaxStepY}\PYG{l+s+s1}{\PYGZsq{}}\PYG{p}{,} \PYG{l+m+mf}{0.14}\PYG{p}{]}\PYG{p}{,}
    \PYG{p}{[}\PYG{l+s+s1}{\PYGZsq{}}\PYG{l+s+s1}{StepHeight}\PYG{l+s+s1}{\PYGZsq{}}\PYG{p}{,} \PYG{l+m+mf}{0.02}\PYG{p}{]}
\PYG{p}{]}
\end{sphinxVerbatim}

El constructor de la clase crea los proxies de las API de NAOqi.
Recibe la dirección IP y el puerto del robot. A continuación
se describen los métodos de esta clase.

\begin{fulllineitems}
\phantomsection\label{\detokenize{nao_firebase:nao_robot.Robot.change_posture}}\pysiglinewithargsret{\sphinxbfcode{\sphinxupquote{change\_posture}}}{\emph{posture}}{}
Cambia entre dos posturas del robot, \sphinxstylestrong{Stand} y \sphinxstylestrong{Crouch}.
\end{fulllineitems}

\index{get\_battery\_level() (método de nao\_robot.Robot)}

\begin{fulllineitems}
\phantomsection\label{\detokenize{nao_firebase:nao_robot.Robot.get_battery_level}}\pysiglinewithargsret{\sphinxbfcode{\sphinxupquote{get\_battery\_level}}}{}{}
Solicita el valor del nivel de la batería desde la memoria del robot
y lo retorna.

\end{fulllineitems}

\begin{fulllineitems}
\phantomsection\label{\detokenize{nao_firebase:nao_robot.Robot.get_image_from_robot}}\pysiglinewithargsret{\sphinxbfcode{\sphinxupquote{get\_image\_from\_robot}}}{}{}
Obtiene la imagen del robot codificada en base 64. Primero con
\sphinxcode{\sphinxupquote{getImageRemo\\te()}} recibe el contenedor de la imagen, y luego,
el arreglo de bytes de la imagen en el contenedor, se guarda en una
estructura de \sphinxcode{\sphinxupquote{PIL}} para que pueda convertirse en una cadena.

\end{fulllineitems}

\index{get\_values\_from\_memory() (método de nao\_robot.Robot)}

\begin{fulllineitems}
\phantomsection\label{\detokenize{nao_firebase:nao_robot.Robot.get_values_from_memory}}\pysiglinewithargsret{\sphinxbfcode{\sphinxupquote{get\_values\_from\_memory}}}{}{}
Obtiene valores de la memoria del robot y retorna un diccionario
con sus llaves y valores.

\end{fulllineitems}

\index{move\_joint() (método de nao\_robot.Robot)}

\begin{fulllineitems}
\phantomsection\label{\detokenize{nao_firebase:nao_robot.Robot.move_joint}}\pysiglinewithargsret{\sphinxbfcode{\sphinxupquote{move\_joint}}}{\emph{joint\_name}, \emph{angle}}{}
Mueve una articulación del robot con respecto a la posición inicial de una
articulación. Primero activa la rigidez del motor que mueve la articulación, realiza el movimiento en dos
segundos y luego desactiva la rigidez del motor.

\end{fulllineitems}

\index{move\_robot() (método de nao\_robot.Robot)}

\begin{fulllineitems}
\phantomsection\label{\detokenize{nao_firebase:nao_robot.Robot.move_robot}}\pysiglinewithargsret{\sphinxbfcode{\sphinxupquote{move\_robot}}}{\emph{\_x}, \emph{\_y}, \emph{\_theta}}{}
Método para ejecutar hacer caminar utilizando \sphinxcode{\sphinxupquote{moveToward}} de la
API de NAOqi.Recibe como parámetros la velocidad sobre los tres
ejes en el intervalo $(-1, 1)$.

\end{fulllineitems}

\begin{fulllineitems}
\phantomsection\label{\detokenize{nao_firebase:nao_robot.Robot.say_speech}}\pysiglinewithargsret{\sphinxbfcode{\sphinxupquote{say\_speech}}}{\emph{text}}{}
Ejecuta la función \sphinxcode{\sphinxupquote{say}} de \sphinxcode{\sphinxupquote{ALTextToSpeech}}. Repite la cadena
enviada como parámetro.

\end{fulllineitems}

\index{set\_move\_config() (método de nao\_robot.Robot)}

\begin{fulllineitems}
\phantomsection\label{\detokenize{nao_firebase:nao_robot.Robot.set_move_config}}\pysiglinewithargsret{\sphinxbfcode{\sphinxupquote{set\_move\_config}}}{\emph{move\_config\_map}}{}
Cambia los parámetros de caminado del robot. Recibe un diccionario
con los nuevos parámetros y sus valores. Los procesa para ver si son válidos
y cambiarlos.

\end{fulllineitems}

\index{stop\_movement() (método de nao\_robot.Robot)}

\begin{fulllineitems}
\phantomsection\label{\detokenize{nao_firebase:nao_robot.Robot.stop_movement}}\pysiglinewithargsret{\sphinxbfcode{\sphinxupquote{stop\_movement}}}{}{}
Método para detener los movimientos del robot, llama a \sphinxcode{\sphinxupquote{stopMove()}}
de \sphinxcode{\sphinxupquote{ALMotion}}.

\end{fulllineitems}

\index{subscribe\_to\_camera() (método de nao\_robot.Robot)}

\begin{fulllineitems}
\phantomsection\label{\detokenize{nao_firebase:nao_robot.Robot.subscribe_to_camera}}\pysiglinewithargsret{\sphinxbfcode{\sphinxupquote{subscribe\_to\_camera}}}{}{}
Se suscribe a la cámara del robot, con los parámetros que se definieron
en los atributos de la clase.

\end{fulllineitems}

\index{subscribe\_to\_sonar() (método de nao\_robot.Robot)}

\begin{fulllineitems}
\phantomsection\label{\detokenize{nao_firebase:nao_robot.Robot.subscribe_to_sonar}}\pysiglinewithargsret{\sphinxbfcode{\sphinxupquote{subscribe\_to\_sonar}}}{}{}
Se suscribe al sonar para que se actualicen los valores en la memoria
con las distancias a obstáculos.

\end{fulllineitems}

\index{unsubscribe\_to\_camera() (método de nao\_robot.Robot)}

\begin{fulllineitems}
\phantomsection\label{\detokenize{nao_firebase:nao_robot.Robot.unsubscribe_to_camera}}\pysiglinewithargsret{\sphinxbfcode{\sphinxupquote{unsubscribe\_to\_camera}}}{}{}
Da de baja al suscriptor de la cámara

\end{fulllineitems}

\index{unsubscribe\_to\_sonar() (método de nao\_robot.Robot)}

\begin{fulllineitems}
\phantomsection\label{\detokenize{nao_firebase:nao_robot.Robot.unsubscribe_to_sonar}}\pysiglinewithargsret{\sphinxbfcode{\sphinxupquote{unsubscribe\_to\_sonar}}}{}{}
Se apagan los sonares.

\end{fulllineitems}




\paragraph{Conexión de la API REST de Firebase y NAOqi}
\label{\detokenize{nao_firebase:conectando-la-api-rest-de-firebase-y-naoqi}}
Después de contar con una biblioteca cliente para realizar peticiones a la
API REST de Firebase y con un módulo que utilice la API de NAOqi sólo
resta unirlos en una aplicación que sea fácil de ejecutar y que responda con
funciones de módulos de NAOqi a actualizaciones enviadas por Firebase o
que envíe información a ubicaciones de la base de datos.

Este aplicación debe poder ejecutarse local y remotamente (dentro de la misma red)
en el robot.

Para mantener simple la ejecución del script que conecta al robot con Firebase
la aplicación está dividida en cinco módulos. Estos incluyen a los módulos
descritos anteriormente, \sphinxcode{\sphinxupquote{firebase}} y \sphinxcode{\sphinxupquote{nao\_robot}}.


\subsubsection{app.py}
\label{\detokenize{nao_firebase:app-py}}\label{\detokenize{nao_firebase:module-app}}\index{app (módulo)}\index{main() (en el módulo app)}

\begin{fulllineitems}
\phantomsection\label{\detokenize{nao_firebase:app.main}}\pysiglinewithargsret{\sphinxcode{\sphinxupquote{app.}}\sphinxbfcode{\sphinxupquote{main}}}{}{}
Inicia la aplicación, obtiene la dirección ip y puerto del robot a través
de los argumentos enviados en la línea de comandos, luego llama a la función
\sphinxcode{\sphinxupquote{run()}} del módulo \sphinxcode{\sphinxupquote{fire\_nao}}.

\end{fulllineitems}



\subsubsection{confg.py}
\label{\detokenize{nao_firebase:confg-py}}
Este archivo únicamente tiene un diccionario con la configuración para Firebase
con la siguiente estructura:

\fvset{hllines={, ,}}%
\begin{sphinxVerbatim}[commandchars=\\\{\}]
\PYG{n}{config} \PYG{o}{=} \PYG{p}{\PYGZob{}}
    \PYG{l+s+s2}{\PYGZdq{}}\PYG{l+s+s2}{databaseURL}\PYG{l+s+s2}{\PYGZdq{}}\PYG{p}{:} \PYG{l+s+s2}{\PYGZdq{}}\PYG{l+s+s2}{https://url\PYGZhy{}de\PYGZhy{}firebase.com/}\PYG{l+s+s2}{\PYGZdq{}}\PYG{p}{,}
    \PYG{l+s+s2}{\PYGZdq{}}\PYG{l+s+s2}{auth}\PYG{l+s+s2}{\PYGZdq{}}\PYG{p}{:} \PYG{l+s+s2}{\PYGZdq{}}\PYG{l+s+s2}{API KEY del proyecto de Firebase}\PYG{l+s+s2}{\PYGZdq{}}\PYG{p}{,}
    \PYG{l+s+s2}{\PYGZdq{}}\PYG{l+s+s2}{robotUID}\PYG{l+s+s2}{\PYGZdq{}}\PYG{p}{:} \PYG{l+s+s2}{\PYGZdq{}}\PYG{l+s+s2}{identificador único del robot generado por Firebase}\PYG{l+s+s2}{\PYGZdq{}}
\PYG{p}{\PYGZcb{}}
\end{sphinxVerbatim}


\subsubsection{fire\_nao.py}
\label{\detokenize{nao_firebase:fire-nao-py}}
Este módulo es quien integra los módulos \texttt{firebase} y  para comunicar al robot
con la aplicación web a través de Firebase.
