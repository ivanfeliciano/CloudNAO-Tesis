


\subsection{Servidor LAR}
\label{\detokenize{chapter_two/desc_cloudnao:servidor-lar}}
El servidor LAR es cliente de los servicios web de terceros y a la vez
ofrece los mantenidos por el LAR.
El resultado de la unión cliente-servidor se
entrega por medio de una
API REST.
Todos los servicios brindados tienen en común que están basados en modelos de
aprendizaje automático, específicamente de aprendizaje profundo. Estos dan solución
a problemas de visión computacional como la detección de rostros o el
reconocimiento óptico de caracteres, el reconocimiento
de voz, el procesamiento de lenguaje natural y la traducción automática
neuronal. A continuación se enlistan los servicios de terceros que el servidor
consume.
\begin{itemize}
\item {} 
\sphinxhref{https://cloud.google.com/vision/}{API Vision de Google Cloud}, permite comprender el contenido de una imagen. Son dos las funcionalidades ocupadas; el etiquetado de imágenes en miles de categorías y el reconocimiento óptico de caracteres (OCR).

% \item {} 
% \sphinxhref{https://cloud.google.com/speech/}{API Speech de Google Cloud}, convierte audio a texto.

\item {} 
\sphinxhref{https://cloud.google.com/translate/}{API Translation de Google Cloud}, se encarga de traducir una cadena arbitraria en cualquier idioma admitido.

\item {} 
\sphinxhref{https://www.kairos.com/}{Kairos}, es una API de reconocimiento facial. Esta cuenta con varios métodos, de los cuales se utilizaron \sphinxstyleemphasis{enroll} y \sphinxstyleemphasis{recognize}. El primero para añadir a un nuevo rostro a una base de datos junto con un identificador, el segundo para encontrar a un usuario cuya cara ha sido almacenada.

\item {} 
\sphinxhref{https://wit.ai/}{Wit.ai}, una API para procesamiento de lenguaje natural capaz de convertir oraciones en información estructurada.

\end{itemize}

El servidor mantiene la ejecución de un contenedor de
Docker en el que se encuentra corriendo una aplicación web desarrollada en Python.
Dicha aplicación es la API REST que se comunica con el robot y la aplicación móvil.

La \hyperref[\detokenize{chapter_two/desc_cloudnao:cn-server-lar-diagram}]{Figura \ref{\detokenize{chapter_two/desc_cloudnao:cn-server-lar-diagram}}} muestra de manera general la estructura del
servidor.

\begin{figure}[htbp]
\centering
\capstart

\noindent\sphinxincludegraphics[scale=0.35]{{CN_server_lar_diagram}.png}
\caption{Diagrama del servidor}\label{\detokenize{chapter_two/desc_cloudnao:cn-server-lar-diagram}}\end{figure}
