
\subsection{Aplicación web de CloudNAO}
\label{\detokenize{nao_web:introduccion}}
La aplicación web para la arquitectura CloudNAO es un componente que permite a
usuarios interactuar con el robot NAO, enviando comandos para que éste ejecute
algunos movimientos, diga algo, o actualice ciertos parámetros en su memoria;
y además reciba valores de la memoria del robot para generar un historial
de información del robot o simplemente recibir una imagen de lo que el robot
visualiza a través de su cámara. Esta aplicación forma parte de lo que
llamamos \sphinxstyleemphasis{Robot logs} y \sphinxstyleemphasis{Robot Remote Control}.

A pesar de que el objetivo principal de esta aplicación web es ser el
front end de la conexión entre el robot NAO directamente con la Firebase
Realtime Database, la unión de estos tres elementos prentende
mostrar un caso de uso del cómputo en la nube, que no es el procesamiento de
imágenes o la ejecución de un algoritmo de aprendizaje automático que el robot
dificilmente podría ejecutar con los recursos que cuenta. En cambio,
qie Firebase con su Realtime database, hosting y Authentication, permiten
generar grandes volúmenes de datos globalmente disponibles, y protegidos.

Como se mencionó, la aplicación es una herramienta más que nada de tipo
front end, simplemente recibe y envía información de la Realtime Database de
Firebase todo a través de una interfaz gráfica sencilla para los usuarios
del robot NAO.

Además de depender del SDK para Web de Firebase, otras dos dependencias están
detrás de la aplicación, React, una biblioteca de Javascript para contruir
interfaces de usuario y Semantic UI, esta última no se mencionará ya que
es sólo es la encargada del diseño de la interfaz.

La aplicación tiene una estructura muy sencilla, cumple con los elementos
básicos de una aplicación web moderna. Un login de usuario, registro de uno
nuevo, y un tablero donde el usuario interactúe con el robot.

En las siguientes secciones se describen a detalle los elementos que confoman
este componente, Firebase y React como herramientas de desarrollo,
una descripción más detallada de la aplicación y finalmente la documentación
del código de ésta.


\subsubsection{Firebase para aplicaciones Web}
\label{\detokenize{firebase_web:firebase-para-aplicaciones-web}}\label{\detokenize{firebase_web::doc}}

\paragraph{Agregar Firebase a un proyecto de JavaScript}
\label{\detokenize{firebase_web:agregar-firebase-a-un-proyecto-de-javascript}}

\subparagraph{Agregar Firebase a tu aplicación}
\label{\detokenize{firebase_web:agregar-firebase-a-tu-aplicacion}}
Para agregar Firebase a tu aplicación, necesitarás un proyecto de Firebase, el SDK de Firebase y un fragmento corto del código de inicialización que incluya algunos detalles sobre tu proyecto.
\begin{enumerate}
\item {} 
Crea un proyecto en Firebase console si no lo hiciste antes.

\item {} 
Selecciona la opción para agregar Firebase a tu aplicación web.

\item {} 
El SDK de Firebase está disponible a través npm. Instala el paquete npm \sphinxcode{\sphinxupquote{firebase}} y guárdalo en el archivo \sphinxcode{\sphinxupquote{package.json}}.

\end{enumerate}

\fvset{hllines={, ,}}%
\begin{sphinxVerbatim}[commandchars=\\\{\}]
npm install firebase \PYGZhy{}\PYGZhy{}save
\end{sphinxVerbatim}

Para usar el módulo en tu aplicación, usa la función require en cualquier archivo JavaScript:

\fvset{hllines={, ,}}%
\begin{sphinxVerbatim}[commandchars=\\\{\}]
\PYG{n}{var} \PYG{n}{firebase} \PYG{o}{=} \PYG{n}{require}\PYG{p}{(}\PYG{l+s+s2}{\PYGZdq{}}\PYG{l+s+s2}{firebase}\PYG{l+s+s2}{\PYGZdq{}}\PYG{p}{)}\PYG{p}{;}
\end{sphinxVerbatim}

Si estás usando ES2015, también puedes usar la función import para importar el módulo:

\fvset{hllines={, ,}}%
\begin{sphinxVerbatim}[commandchars=\\\{\}]
\PYG{k+kn}{import} \PYG{o}{*} \PYG{k}{as} \PYG{n}{firebase} \PYG{k+kn}{from} \PYG{l+s+s2}{\PYGZdq{}}\PYG{l+s+s2}{firebase}\PYG{l+s+s2}{\PYGZdq{}}\PYG{p}{;}
\end{sphinxVerbatim}

Luego, inicia el SDK de Firebase con el fragmento de código anterior, que debería tener el siguiente aspecto:

\fvset{hllines={, ,}}%
\begin{sphinxVerbatim}[commandchars=\\\{\}]
\PYG{o}{/}\PYG{o}{/} \PYG{n}{Inicializa} \PYG{n}{Firebase}
\PYG{o}{/}\PYG{o}{/} \PYG{n}{Sólo} \PYG{n}{falta} \PYG{n}{remplazar} \PYG{n}{los} \PYG{n}{campos} \PYG{n}{con} \PYG{n}{la} \PYG{n}{información} \PYG{n}{de} \PYG{n}{la} \PYG{n}{aplicación}
\PYG{n}{var} \PYG{n}{config} \PYG{o}{=} \PYG{p}{\PYGZob{}}
  \PYG{n}{apiKey}\PYG{p}{:} \PYG{l+s+s2}{\PYGZdq{}}\PYG{l+s+s2}{\PYGZlt{}API\PYGZus{}KEY\PYGZgt{}}\PYG{l+s+s2}{\PYGZdq{}}\PYG{p}{,}
  \PYG{n}{authDomain}\PYG{p}{:} \PYG{l+s+s2}{\PYGZdq{}}\PYG{l+s+s2}{\PYGZlt{}PROJECT\PYGZus{}ID\PYGZgt{}.firebaseapp.com}\PYG{l+s+s2}{\PYGZdq{}}\PYG{p}{,}
  \PYG{n}{databaseURL}\PYG{p}{:} \PYG{l+s+s2}{\PYGZdq{}}\PYG{l+s+s2}{https://\PYGZlt{}DATABASE\PYGZus{}NAME\PYGZgt{}.firebaseio.com}\PYG{l+s+s2}{\PYGZdq{}}\PYG{p}{,}
  \PYG{n}{storageBucket}\PYG{p}{:} \PYG{l+s+s2}{\PYGZdq{}}\PYG{l+s+s2}{\PYGZlt{}BUCKET\PYGZgt{}.appspot.com}\PYG{l+s+s2}{\PYGZdq{}}\PYG{p}{,}
\PYG{p}{\PYGZcb{}}\PYG{p}{;}
\PYG{n}{firebase}\PYG{o}{.}\PYG{n}{initializeApp}\PYG{p}{(}\PYG{n}{config}\PYG{p}{)}\PYG{p}{;}
\end{sphinxVerbatim}


\subparagraph{Usa servicios de Firebase}
\label{\detokenize{firebase_web:usa-servicios-de-firebase}}
Una \sphinxcode{\sphinxupquote{App}} de Firebase puede usar varios servicios de Firebase. Se puede acceder
a cada servicio desde el espacio de nombres de \sphinxcode{\sphinxupquote{firebase}}:
\begin{itemize}
\item {} 
\sphinxcode{\sphinxupquote{firebase.auth()}} - Authentication

\item {} 
\sphinxcode{\sphinxupquote{firebase.storage()}} - Cloud Storage

\item {} 
\sphinxcode{\sphinxupquote{firebase.database()}} - Realtime Database

\item {} 
\sphinxcode{\sphinxupquote{firebase.firestore()}} - Cloud Firestore

\end{itemize}


\subparagraph{Ejecuta un servidor web local para el desarrollo}
\label{\detokenize{firebase_web:ejecuta-un-servidor-web-local-para-el-desarrollo}}
Si estás compilando una aplicación web, notarás que algunas partes de Firebase JavaScript SDK necesitan que tu aplicación web esté asociada con un servidor en lugar de un sistema de archivos local. Puedes usar Firebase CLI para ejecutar un servidor local como el siguiente:

\fvset{hllines={, ,}}%
\begin{sphinxVerbatim}[commandchars=\\\{\}]
\PYG{n}{npm} \PYG{n}{install} \PYG{o}{\PYGZhy{}}\PYG{n}{g} \PYG{n}{firebase}\PYG{o}{\PYGZhy{}}\PYG{n}{tools}
\PYG{n}{firebase} \PYG{n}{init}    \PYG{c+c1}{\PYGZsh{} Genera un archivo firebase.json (REQUERIDO)}
\PYG{n}{firebase} \PYG{n}{serve}   \PYG{c+c1}{\PYGZsh{} Inicia el servidor para desarrollo}
\end{sphinxVerbatim}


\paragraph{Firebase Realtime Database}
\label{\detokenize{firebase_web:firebase-realtime-database}}

\subparagraph{Obtener una referencia a una base de datos}
\label{\detokenize{firebase_web:obtener-una-referencia-a-una-base-de-datos}}
Para leer la base de datos o escribir en ella, necesitas una instancia de
\sphinxcode{\sphinxupquote{firebase.database.Reference}}:

\fvset{hllines={, ,}}%
\begin{sphinxVerbatim}[commandchars=\\\{\}]
\PYG{o}{/}\PYG{o}{/} \PYG{n}{Obtiene} \PYG{n}{una} \PYG{n}{referecia} \PYG{n}{al} \PYG{n}{servicio} \PYG{n}{de} \PYG{n}{la} \PYG{n}{base} \PYG{n}{de} \PYG{n}{datos}
\PYG{n}{var} \PYG{n}{database} \PYG{o}{=} \PYG{n}{firebase}\PYG{o}{.}\PYG{n}{database}\PYG{p}{(}\PYG{p}{)}\PYG{p}{;}
\end{sphinxVerbatim}


\subparagraph{Lectura y escritura de datos}
\label{\detokenize{firebase_web:lectura-y-escritura-de-datos}}
Para recuperar los datos de Firebase, se debe agregar un escuchador
asíncrono a \sphinxcode{\sphinxupquote{firebase.database.Reference}}. El escuchador se activa una vez para
el estado inicial de los datos y otra vez cuando los datos cambian.


\textbf{Operaciones básicas de escritura}
\label{\detokenize{firebase_web:operaciones-basicas-de-escritura}}
Para ejecutar operaciones de escritura básicas, puedes usar \sphinxcode{\sphinxupquote{set()}} para guardar
datos en una referencia que especifiques y reemplazar los datos existentes en
esa ruta de acceso. Por ejemplo si se desea añadir un usuario a la base de datos,
una opción es la siguiente:

\fvset{hllines={, ,}}%
\begin{sphinxVerbatim}[commandchars=\\\{\}]
\PYG{n}{function} \PYG{n}{writeUserData}\PYG{p}{(}\PYG{n}{userId}\PYG{p}{,} \PYG{n}{name}\PYG{p}{,} \PYG{n}{email}\PYG{p}{,} \PYG{n}{imageUrl}\PYG{p}{)} \PYG{p}{\PYGZob{}}
\PYG{n}{firebase}\PYG{o}{.}\PYG{n}{database}\PYG{p}{(}\PYG{p}{)}\PYG{o}{.}\PYG{n}{ref}\PYG{p}{(}\PYG{l+s+s1}{\PYGZsq{}}\PYG{l+s+s1}{users/}\PYG{l+s+s1}{\PYGZsq{}} \PYG{o}{+} \PYG{n}{userId}\PYG{p}{)}\PYG{o}{.}\PYG{n}{set}\PYG{p}{(}\PYG{p}{\PYGZob{}}
    \PYG{n}{username}\PYG{p}{:} \PYG{n}{name}\PYG{p}{,}
    \PYG{n}{email}\PYG{p}{:} \PYG{n}{email}\PYG{p}{,}
    \PYG{n}{profile\PYGZus{}picture} \PYG{p}{:} \PYG{n}{imageUrl}
  \PYG{p}{\PYGZcb{}}\PYG{p}{)}\PYG{p}{;}
\PYG{p}{\PYGZcb{}}
\end{sphinxVerbatim}

\sphinxcode{\sphinxupquote{set()}} sobrescribe los datos en la ubicación que se especifíca, incluidos
los nodos secundarios.


\textbf{Detecta eventos en valores}
\label{\detokenize{firebase_web:detecta-eventos-en-valores}}
Si deseas leer datos de una ruta de acceso y escuchar para detectar cambios,
usa los métodos \sphinxcode{\sphinxupquote{on()}} o \sphinxcode{\sphinxupquote{once()}} de \sphinxcode{\sphinxupquote{firebase.database.Reference}}
para observar eventos.

El evento más común es \sphinxcode{\sphinxupquote{value}} y permite leer una instantánea estática del
contenido de una ruta de acceso determinada, en el estado en que se encontraba
en el momento del evento. Este método se activa cuando se vincula el escuchador
y se vuelve a activar cada vez que cambian los datos (incluidos los de
nivel secundario). La devolución de llamada del evento recibe una instantánea
que contiene todos los datos de dicha ubicación, incluidos los datos
secundarios. Si no hay datos, la instantánea tiene un valor nulo.

En el siguiente ejemplo se muestra una aplicación donde se recupera el valor
de un recurso en la vase de datos:

\fvset{hllines={, ,}}%
\begin{sphinxVerbatim}[commandchars=\\\{\}]
\PYG{n}{var} \PYG{n}{resourceRef} \PYG{o}{=} \PYG{n}{firebase}\PYG{o}{.}\PYG{n}{database}\PYG{p}{(}\PYG{p}{)}\PYG{o}{.}\PYG{n}{ref}\PYG{p}{(}\PYG{l+s+s1}{\PYGZsq{}}\PYG{l+s+s1}{resource}\PYG{l+s+s1}{\PYGZsq{}}\PYG{p}{)}\PYG{p}{;}
\PYG{n}{resourceRef}\PYG{o}{.}\PYG{n}{on}\PYG{p}{(}\PYG{l+s+s1}{\PYGZsq{}}\PYG{l+s+s1}{value}\PYG{l+s+s1}{\PYGZsq{}}\PYG{p}{,} \PYG{n}{function}\PYG{p}{(}\PYG{n}{snapshot}\PYG{p}{)} \PYG{p}{\PYGZob{}}
  \PYG{n}{showValueOnScreen}\PYG{p}{(}\PYG{n}{snapshot}\PYG{o}{.}\PYG{n}{val}\PYG{p}{(}\PYG{p}{)}\PYG{p}{)}\PYG{p}{;}
\PYG{p}{\PYGZcb{}}\PYG{p}{)}\PYG{p}{;}
\end{sphinxVerbatim}

El escuchador recibe una \sphinxcode{\sphinxupquote{snapshot}} que contiene los datos de la ubicación
especifíca en la base de datos en el momento del evento. Se pueden recuperar
los datos de la \sphinxcode{\sphinxupquote{snapshot}} con el método \sphinxcode{\sphinxupquote{val()}}.


\textbf{Leer los datos una sola vez}
\label{\detokenize{firebase_web:leer-los-datos-una-sola-vez}}
En algunos casos, deseas tener una instantánea de los datos sin detectar cambios,
como cuando inicializas un elemento de IU que no debería cambiar. Puedes usar
el método once() para simplificar esta situación: se activa una vez y no vuelve
a activarse.

Por ejemplo, en una aplicación de recursos cualesquiera, queremos mostrar un
recurso en la IU que sólo se carga una vez:

\fvset{hllines={, ,}}%
\begin{sphinxVerbatim}[commandchars=\\\{\}]
\PYG{n}{firebase}\PYG{o}{.}\PYG{n}{database}\PYG{p}{(}\PYG{p}{)}\PYG{o}{.}\PYG{n}{ref}\PYG{p}{(}\PYG{l+s+s1}{\PYGZsq{}}\PYG{l+s+s1}{uiResource}\PYG{l+s+s1}{\PYGZsq{}}\PYG{p}{)}\PYG{o}{.}\PYG{n}{once}\PYG{p}{(}\PYG{l+s+s1}{\PYGZsq{}}\PYG{l+s+s1}{value}\PYG{l+s+s1}{\PYGZsq{}}\PYG{p}{)}\PYG{o}{.}\PYG{n}{then}\PYG{p}{(}\PYG{n}{function}\PYG{p}{(}\PYG{n}{snapshot}\PYG{p}{)} \PYG{p}{\PYGZob{}}
  \PYG{n}{showOnScreen}\PYG{p}{(}\PYG{n}{snapshot}\PYG{o}{.}\PYG{n}{val}\PYG{p}{(}\PYG{p}{)}\PYG{p}{)}
\PYG{p}{\PYGZcb{}}\PYG{p}{)}\PYG{p}{;}
\end{sphinxVerbatim}


\textbf{Actualizar o borrar datos}
\label{\detokenize{firebase_web:actualizar-o-borrar-datos}}
Para escribir de forma simultánea en elementos secundarios específicos de un
nodo sin sobrescribir otros nodos secundarios, usa el método \sphinxcode{\sphinxupquote{update()}}.

También existe el método \sphinxcode{\sphinxupquote{push()}} para añadir una entrada con un identificador
único sobre una referencia.

La forma más sencilla de borrar datos es llamar a \sphinxcode{\sphinxupquote{remove()}} en una referencia a
la ubicación de los datos. Para borrar, también puedes especificar nulo como
el valor de otra operación de escritura, como \sphinxcode{\sphinxupquote{set()}} o \sphinxcode{\sphinxupquote{update()}}.


\paragraph{Firebase Authentication}
\label{\detokenize{firebase_web:firebase-authentication}}
Firebase Authentication proporciona servicios de backend, SDK fáciles de usar y bibliotecas de IU ya elaboradas para autenticar a los usuarios en tu aplicación. Admite la autenticación mediante contraseñas, números de teléfono, proveedores de identidad federados populares, como Google, Facebook y Twitter, y mucho más.


\subparagraph{Autenticación basada en correo electrónico y contraseña}
\label{\detokenize{firebase_web:autenticacion-basada-en-correo-electronico-y-contrasena}}
Autentica a los usuarios con sus direcciones de correo electrónico y contraseñas. El SDK de Firebase Authentication proporciona métodos para crear y administrar usuarios que utilizan sus direcciones de correo electrónico y sus contraseñas para acceder. Firebase Authentication también maneja el envío de mensajes de correo electrónico para restablecer la contraseña.


\textbf{Crear una cuenta basada en contraseña}
\label{\detokenize{firebase_web:crear-una-cuenta-basada-en-contrasena}}
Para crear una cuenta de usuario nueva con una contraseña, completa con los
siguientes pasos en la actitividad de acceso de tu aplicación:
\begin{enumerate}
\item {} 
Cuando un usuario nuevo se registre mediante el formulario de registro de la aplicación, realiza los pasos de validación de la cuenta nueva necesarios (como verificar que se haya escrito correctamente la contraseña y que cumpla con los requisitos de complejidad).

\item {} 
Pasa la dirección de correo electrónico y la contraseña del nuevo usuario a createUserWithEmailAndPassword para crear la cuenta nueva.

\end{enumerate}

Ejemplo:

\fvset{hllines={, ,}}%
\begin{sphinxVerbatim}[commandchars=\\\{\}]
\PYG{n}{firebase}\PYG{o}{.}\PYG{n}{auth}\PYG{p}{(}\PYG{p}{)}\PYG{o}{.}\PYG{n}{createUserWithEmailAndPassword}\PYG{p}{(}\PYG{n}{email}\PYG{p}{,} \PYG{n}{password}\PYG{p}{)}\PYG{o}{.}\PYG{n}{catch}\PYG{p}{(}\PYG{n}{function}\PYG{p}{(}\PYG{n}{error}\PYG{p}{)} \PYG{p}{\PYGZob{}}
  \PYG{o}{/}\PYG{o}{/} \PYG{n}{El} \PYG{n}{manejo} \PYG{n}{de} \PYG{n}{errores} \PYG{n}{va} \PYG{n}{aquí}
  \PYG{n}{var} \PYG{n}{errorCode} \PYG{o}{=} \PYG{n}{error}\PYG{o}{.}\PYG{n}{code}\PYG{p}{;}
  \PYG{n}{var} \PYG{n}{errorMessage} \PYG{o}{=} \PYG{n}{error}\PYG{o}{.}\PYG{n}{message}\PYG{p}{;}
\PYG{p}{\PYGZcb{}}\PYG{p}{)}\PYG{p}{;}
\end{sphinxVerbatim}


\textbf{Acceso con una dirección de correo electrónico y una contraseña}
\label{\detokenize{firebase_web:acceso-con-una-direccion-de-correo-electronico-y-una-contrasena}}
Los pasos para que un usuario acceda con una contraseña son similares a los pasos para crear una cuenta nueva. En la página de acceso de tu aplicación, haz lo siguiente:

Cuando un usuario accede a tus apps, pasa la dirección de correo electrónico y la contraseña a signInWithEmailAndPassword:

\fvset{hllines={, ,}}%
\begin{sphinxVerbatim}[commandchars=\\\{\}]
\PYG{n}{firebase}\PYG{o}{.}\PYG{n}{auth}\PYG{p}{(}\PYG{p}{)}\PYG{o}{.}\PYG{n}{signInWithEmailAndPassword}\PYG{p}{(}\PYG{n}{email}\PYG{p}{,} \PYG{n}{password}\PYG{p}{)}\PYG{o}{.}\PYG{n}{catch}\PYG{p}{(}\PYG{n}{function}\PYG{p}{(}\PYG{n}{error}\PYG{p}{)} \PYG{p}{\PYGZob{}}
\PYG{o}{/}\PYG{o}{/} \PYG{n}{Aquí} \PYG{n}{se} \PYG{n}{manejan} \PYG{n}{los} \PYG{n}{errores}
\PYG{n}{var} \PYG{n}{errorCode} \PYG{o}{=} \PYG{n}{error}\PYG{o}{.}\PYG{n}{code}\PYG{p}{;}
\PYG{n}{var} \PYG{n}{errorMessage} \PYG{o}{=} \PYG{n}{error}\PYG{o}{.}\PYG{n}{message}\PYG{p}{;}
\PYG{o}{/}\PYG{o}{/} \PYG{o}{.}\PYG{o}{.}\PYG{o}{.}
\PYG{p}{\PYGZcb{}}\PYG{p}{)}\PYG{p}{;}
\end{sphinxVerbatim}


\textbf{Obtener el usuario con sesión activa}
\label{\detokenize{firebase_web:obtener-el-usuario-con-sesion-activa}}
La manera recomendada de obtener el usuario actual es establecer un observador en el objeto \sphinxcode{\sphinxupquote{Auth}}:

\fvset{hllines={, ,}}%
\begin{sphinxVerbatim}[commandchars=\\\{\}]
\PYG{n}{firebase}\PYG{o}{.}\PYG{n}{auth}\PYG{p}{(}\PYG{p}{)}\PYG{o}{.}\PYG{n}{onAuthStateChanged}\PYG{p}{(}\PYG{n}{function}\PYG{p}{(}\PYG{n}{user}\PYG{p}{)} \PYG{p}{\PYGZob{}}
\PYG{k}{if} \PYG{p}{(}\PYG{n}{user}\PYG{p}{)} \PYG{p}{\PYGZob{}}
  \PYG{o}{/}\PYG{o}{/} \PYG{n}{El} \PYG{n}{usuario} \PYG{n}{ha} \PYG{n}{iniciado} \PYG{n}{sesión}
\PYG{p}{\PYGZcb{}} \PYG{k}{else} \PYG{p}{\PYGZob{}}
  \PYG{o}{/}\PYG{o}{/} \PYG{n}{No} \PYG{n}{hay} \PYG{n}{un} \PYG{n}{usuario} \PYG{n}{activo}
\PYG{p}{\PYGZcb{}}
\PYG{p}{\PYGZcb{}}\PYG{p}{)}\PYG{p}{;}
\end{sphinxVerbatim}

También se puede usar la propiedad \sphinxcode{\sphinxupquote{currentUser}} para obtener el usuario que
accedió. Si un usuario no accedió a su cuenta, \sphinxcode{\sphinxupquote{currentUser}} mostrará un
resultado nulo:

\fvset{hllines={, ,}}%
\begin{sphinxVerbatim}[commandchars=\\\{\}]
\PYG{n}{var} \PYG{n}{user} \PYG{o}{=} \PYG{n}{firebase}\PYG{o}{.}\PYG{n}{auth}\PYG{p}{(}\PYG{p}{)}\PYG{o}{.}\PYG{n}{currentUser}\PYG{p}{;}
\PYG{k}{if} \PYG{p}{(}\PYG{n}{user}\PYG{p}{)} \PYG{p}{\PYGZob{}}
  \PYG{o}{/}\PYG{o}{/} \PYG{n}{El} \PYG{n}{usuario} \PYG{n}{ha} \PYG{n}{iniciado} \PYG{n}{sesión}
\PYG{p}{\PYGZcb{}} \PYG{k}{else} \PYG{p}{\PYGZob{}}
  \PYG{o}{/}\PYG{o}{/} \PYG{n}{No} \PYG{n}{hay} \PYG{n}{un} \PYG{n}{usuario} \PYG{n}{activo}
\PYG{p}{\PYGZcb{}}
\end{sphinxVerbatim}


\textbf{Obtener el perfil de un usuario}
\label{\detokenize{firebase_web:obtener-el-perfil-de-un-usuario}}
Para obtener la información del perfil de un usuario, puedes usar las propiedades de una instancia de User. Por ejemplo:

\fvset{hllines={, ,}}%
\begin{sphinxVerbatim}[commandchars=\\\{\}]
\PYG{n}{var} \PYG{n}{user} \PYG{o}{=} \PYG{n}{firebase}\PYG{o}{.}\PYG{n}{auth}\PYG{p}{(}\PYG{p}{)}\PYG{o}{.}\PYG{n}{currentUser}\PYG{p}{;}
\PYG{n}{var} \PYG{n}{name}\PYG{p}{,} \PYG{n}{email}\PYG{p}{,} \PYG{n}{photoUrl}\PYG{p}{,} \PYG{n}{uid}\PYG{p}{,} \PYG{n}{emailVerified}\PYG{p}{;}

\PYG{k}{if} \PYG{p}{(}\PYG{n}{user} \PYG{o}{!=} \PYG{n}{null}\PYG{p}{)} \PYG{p}{\PYGZob{}}
  \PYG{n}{name} \PYG{o}{=} \PYG{n}{user}\PYG{o}{.}\PYG{n}{displayName}\PYG{p}{;}
  \PYG{n}{email} \PYG{o}{=} \PYG{n}{user}\PYG{o}{.}\PYG{n}{email}\PYG{p}{;}
  \PYG{n}{photoUrl} \PYG{o}{=} \PYG{n}{user}\PYG{o}{.}\PYG{n}{photoURL}\PYG{p}{;}
  \PYG{n}{emailVerified} \PYG{o}{=} \PYG{n}{user}\PYG{o}{.}\PYG{n}{emailVerified}\PYG{p}{;}
  \PYG{n}{uid} \PYG{o}{=} \PYG{n}{user}\PYG{o}{.}\PYG{n}{uid}\PYG{p}{;}  \PYG{o}{/}\PYG{o}{/} \PYG{n}{El} \PYG{n+nb}{id} \PYG{k}{del} \PYG{n}{usuario} \PYG{n}{es} \PYG{n}{único} \PYG{n}{en} \PYG{n}{un} \PYG{n}{proyecto} \PYG{n}{de} \PYG{n}{Firebase}

\PYG{p}{\PYGZcb{}}
\end{sphinxVerbatim}


\textbf{Cierre de sesión}
\label{\detokenize{firebase_web:cierre-de-sesion}}
Para salir de la sesión de un usuario, se llama a \sphinxcode{\sphinxupquote{signOut}}:

\fvset{hllines={, ,}}%
\begin{sphinxVerbatim}[commandchars=\\\{\}]
\PYG{n}{firebase}\PYG{o}{.}\PYG{n}{auth}\PYG{p}{(}\PYG{p}{)}\PYG{o}{.}\PYG{n}{signOut}\PYG{p}{(}\PYG{p}{)}\PYG{o}{.}\PYG{n}{then}\PYG{p}{(}\PYG{n}{function}\PYG{p}{(}\PYG{p}{)} \PYG{p}{\PYGZob{}}
\PYG{o}{/}\PYG{o}{/} \PYG{n}{Se} \PYG{n}{cierra} \PYG{n}{la} \PYG{n}{sesión} \PYG{n}{exitosamente}
\PYG{p}{\PYGZcb{}}\PYG{p}{)}\PYG{o}{.}\PYG{n}{catch}\PYG{p}{(}\PYG{n}{function}\PYG{p}{(}\PYG{n}{error}\PYG{p}{)} \PYG{p}{\PYGZob{}}
  \PYG{o}{/}\PYG{o}{/} \PYG{n}{Un} \PYG{n}{error} \PYG{n}{ocurrió}
\PYG{p}{\PYGZcb{}}\PYG{p}{)}\PYG{p}{;}
\end{sphinxVerbatim}


\paragraph{Firebase Hosting}
\label{\detokenize{firebase_web:firebase-hosting}}
Este servicio permite implementar y alojar los recursos estáticos de una aplicación
fácilmente (HTML, CSS, JavaScript, etc.). Todo el contenido se transmite a través
de HTTPS y está respaldado por una CDN global.

La ruta de implementación se muestra en las siguientes subsecciones.


\subparagraph{Instalar Firebase CLI}
\label{\detokenize{firebase_web:instalar-firebase-cli}}
La CLI (interfaz de línea de comandos) de Firebase necesita Node.js y npm.

Para instalar Firebase CLI a través de npm se escribe los siguiente:

\fvset{hllines={, ,}}%
\begin{sphinxVerbatim}[commandchars=\\\{\}]
\PYG{n}{npm} \PYG{n}{install} \PYG{o}{\PYGZhy{}}\PYG{n}{g} \PYG{n}{firebase}\PYG{o}{\PYGZhy{}}\PYG{n}{tools}
\end{sphinxVerbatim}

Esto instala el comando \sphinxcode{\sphinxupquote{firebase}} de manera global.


\subparagraph{Inicializar la aplicación}
\label{\detokenize{firebase_web:inicializar-la-aplicacion}}
Si se ejecuta el comando \sphinxcode{\sphinxupquote{firebase init}}, se crea un archivo de configuración
\sphinxcode{\sphinxupquote{firebase.json}} en la raíz del directorio del proyecto.


\subparagraph{Agregar un archivo}
\label{\detokenize{firebase_web:agregar-un-archivo}}
Cuando se inicializa una aplicación, se pedirá proporcionar un directorio para
usarlo como raíz pública (el valor predeterminado es «public»). Si aún no
se tiene un archivo index.html válido en tu directorio raíz público, se creará
uno.


\subparagraph{Implementar un sitio web}
\label{\detokenize{firebase_web:implementar-un-sitio-web}}
Ejecutar \sphinxcode{\sphinxupquote{firebase deploy}} implementará la aplicación en el dominio
\sphinxcode{\sphinxupquote{\textless{}APP-DE-FIREBASE\textgreater{}.firebaseapp.com}}


\subparagraph{Administar y revertir versiones}
\label{\detokenize{firebase_web:administar-y-revertir-versiones}}
En el panel Hosting de Firebase console, se puede ver un historial completo de
implementaciones. Para revertir a una implementación previa, basta con
desplazarset sobre su entrada en la lista, hacer clic en el ícono del menú
ampliado y luego en «Rollback».


\subsubsection{React}
\label{\detokenize{reactjs::doc}}\label{\detokenize{reactjs:react}}
React es una biblioteca de Javascript para crear interfaces de usuario
lanzada por Facebook en 2013.
tiene tres características que definen y distinguen las distinguen de otras
biblioteca o de frameworks y son:
\begin{itemize}
\item {} 
Es \sphinxstylestrong{declarativa}, ya que permite crear interfaces de manera sencilla. Sólo se diseña una vista para cada estado de la aplicación y React actualizará de manera eficiente y \sphinxstylestrong{rederizará} los componentes correctos cuando los datos cambian.

\item {} 
Está \sphinxstylestrong{basado en componentes}, contruye componentes encapsulados que manejan su propio estado, después se integran con otros para hacer IU más complejas.

\item {} 
\sphinxstylestrong{Aprende una vez, escribe donde sea}, React puede \sphinxstyleemphasis{renderizar} sobre un servidor usando node o en dispositivos móviles con React Native.

\end{itemize}


\paragraph{Requisitos}
\label{\detokenize{reactjs:requisitos}}

\subparagraph{Node y NPM}
\label{\detokenize{reactjs:node-y-npm}}
Node.js es un entorno de ejecución para Javascript construido con el motor
Javascript V8 de Chrome. Node.js usa un modelo de operaciones de entrada y
salida sin bloqueo y orientado a eventos, que lo hace liviano y eficiente.
En pocas palabras es un intérprete de Javascript del lado del servidor pero
con ciertas características que lo hacen especial. npm (node package manager
en inglés) es el ecosistema de paquetes de Node.js.

Para comenzar utilizar React es necesaria la instalación antes que nada
de node y npm. El manejador de paquetes de node permite la instalación de
paquetes de node externos desde la línea de comandos. Estos paquetes  pueden
ser un conjunto de funciones, bibliotecas o frameworks completos. Estas son
dependencias de la aplicación que se desarrolla. Se pueden instalar estos
paquetes de manera global o local en la carpeta del proyecto. La diferencia
de estos dos últimos es que la instalación de un paquete de manera global
hace que esté accesible desde cualquier parte de la terminal.

Para instalar un paquete de manera global se hace de la siguiente manera:

\fvset{hllines={, ,}}%
\begin{sphinxVerbatim}[commandchars=\\\{\}]
npm install \PYGZhy{}g \PYGZlt{}paquete\PYGZgt{}
\end{sphinxVerbatim}

La bandera \sphinxcode{\sphinxupquote{-g}} le dice a npm que instale de manera global ese paquete. Los
paquetes locales son los que se usan dentro de una aplicación. Por ejemplo
si queremos usar React sólo en nuestra aplicación escribimos:

\fvset{hllines={, ,}}%
\begin{sphinxVerbatim}[commandchars=\\\{\}]
npm installl react
\end{sphinxVerbatim}

El paquete instalado aparecerá en un folder llamado \sphinxstyleemphasis{node\_modules/} y estará
enlistado en el archivo \sphinxstyleemphasis{package.json} junto con las otras dependencias.

Para poder crear el directorio y el archivo antes mencionados y por tanto
poder instalar paquetes de manera local a través de npm, éste cuenta con el
siguiente comando:

\fvset{hllines={, ,}}%
\begin{sphinxVerbatim}[commandchars=\\\{\}]
npm init
\end{sphinxVerbatim}

Después de realizar la configuración de tu proyecto, es posible instalar nuevos
paquetes usando \sphinxcode{\sphinxupquote{npm install \textless{}package\textgreater{}}}. No está de más mencionar que todo
esto se hace dentro del directorio del proyecto sobre el que se trabaja.


\subparagraph{Instalación de React}
\label{\detokenize{reactjs:instalacion-de-react}}
Cuando tu aplicación cuenta con un archivo \sphinxstyleemphasis{package.json}, se puede instalar
\sphinxstyleemphasis{react} y \sphinxstyleemphasis{react-dom} desde la línea de comandos. El único requisito es que el
directorio esté inicializado como un proyecto de npm usando \sphinxcode{\sphinxupquote{npm init}}. Por
tanto la instrucción para instalar react y react-dom es la siguiente:

\fvset{hllines={, ,}}%
\begin{sphinxVerbatim}[commandchars=\\\{\}]
npm instll react react\PYGZhy{}dom
\end{sphinxVerbatim}

Sin embargo lo descrito anteriormente no es suficiente. Se debe lidiar con
\sphinxstylestrong{Babel} para que la aplicación pueda usar JSX (la sintaxis de React) y
Javascript ES6. Babel traduce y compila tu código para que los navegadores
puedan interpretar Javascript ES6 y JSX. No todos los navegadores tienen la
capacidad de interpretar las sintaxis. La puesta en marcha
incluye una gran cantidad de configuración y herramientas.

Por la razón anterior, Facebook introdujo \sphinxstyleemphasis{create-react-app} a como una solución
para lanzar React sin ninguna configuración.


\subparagraph{Puesta en marcha sin configuraciones}
\label{\detokenize{reactjs:puesta-en-marcha-sin-configuraciones}}
\sphinxstylestrong{create-react-app} es un kit inicial sin configuraciones para React lanzado
por Facebook en 2016. Para iniciar con esta herramienta en la línea de comandos
se escribe lo siguiente:

\fvset{hllines={, ,}}%
\begin{sphinxVerbatim}[commandchars=\\\{\}]
npm install \PYGZhy{}g create\PYGZhy{}react\PYGZhy{}app
\end{sphinxVerbatim}

Si se intala de manera global, siempre estará disponible el comando
\sphinxstyleemphasis{create-react-app}.

Ahora se puede crear y lanzar una aplicación con React. Esto toma unos segundos,
después solo se navega al directorio que se creo.

\fvset{hllines={, ,}}%
\begin{sphinxVerbatim}[commandchars=\\\{\}]
create\PYGZhy{}react\PYGZhy{}app ejemplo\PYGZus{}react\PYGZus{}app
\PYG{n+nb}{cd} react\PYGZus{}ejemplo\PYGZus{}app
\end{sphinxVerbatim}

La estructura del directorio creado es la siguiente:

\fvset{hllines={, ,}}%
\begin{sphinxVerbatim}[commandchars=\\\{\}]
ejemplo\PYGZus{}react\PYGZus{}app/
├── node\PYGZus{}modules
├── package.json
├── package\PYGZhy{}lock.json
├── public
│   ├── favicon.ico
│   ├── index.html
│   └── manifest.json
├── README.md
└── src
    ├── App.css
    ├── App.js
    ├── App.test.js
    ├── index.css
    ├── index.js
    ├── logo.svg
    └── registerServiceWorker.js
\end{sphinxVerbatim}
\begin{itemize}
\item {} 
\sphinxstylestrong{node\_modules/}. Contiene todos los paquetes de node que fueron instalados vía npm. Como se usó create-react-app, algunos módulos ya fueron instalados.

\item {} 
\sphinxstylestrong{package.json}. El archivo que muestra las lista de dependecias y otras opciones de configuracón del proyecto.

\item {} 
\sphinxstylestrong{.gitignore}. Este archivo indica todos los archivos y carpetas que nos deben añadirse a un repositorio remoto cuando se usa git.

\item {} 
\sphinxstylestrong{public/}. Esta carpeta guarda todos los archivos cuando se despliega un proyecto en modo de producción.

\end{itemize}

En un principio todo lo que se necesita está localizado en la carpeta \sphinxstyleemphasis{src/}.
El enfoque principal está dirigido al archivo \sphinxstyleemphasis{src/App.js} para implementar los
componentes de React. Este se usa para implementar la aplicación, pero con el
crecimiento de un proyecto siempre es necesario dividir los componentes en
múltiples archivos, donde cada archivo mantiene uno o algunos componentes por
su cuenta.

La aplicación generada con \sphinxstyleemphasis{create-react-app} es un proyecto de npm. Se puede usar
npm para instalar o eliminar paqueres de node. Además viene con los siguientes
scripts de npm para la línea de comandos.

\fvset{hllines={, ,}}%
\begin{sphinxVerbatim}[commandchars=\\\{\}]
\PYG{c+c1}{\PYGZsh{} Ejecuta la aplicación en http://localhost:3000}
npm start

\PYG{c+c1}{\PYGZsh{} Ejecutar los tests}
npm \PYG{n+nb}{test}

\PYG{c+c1}{\PYGZsh{} Construye la aplicación para producción}
npm run build
\end{sphinxVerbatim}

Esos scripts están definidos en el \sphinxstyleemphasis{package.json}.


\paragraph{Un ejemplo básico}
\label{\detokenize{reactjs:un-ejemplo-basico}}
La aplicación más simple en React luce como la siguiente:

\fvset{hllines={, ,}}%
\begin{sphinxVerbatim}[commandchars=\\\{\}]
\PYG{n}{ReactDOM}\PYG{o}{.}\PYG{n}{render}\PYG{p}{(}
  \PYG{o}{\PYGZlt{}}\PYG{n}{h1}\PYG{o}{\PYGZgt{}}\PYG{n}{Hola} \PYG{n}{mundo} \PYG{o}{\PYGZlt{}}\PYG{o}{/}\PYG{n}{h1}\PYG{o}{\PYGZgt{}}
  \PYG{n}{document}\PYG{o}{.}\PYG{n}{getElementById}\PYG{p}{(}\PYG{l+s+s1}{\PYGZsq{}}\PYG{l+s+s1}{root}\PYG{l+s+s1}{\PYGZsq{}}\PYG{p}{)}
\PYG{p}{)}\PYG{p}{;}
\end{sphinxVerbatim}

Esto renderiza un encabezado que dice «Hola mundo» en una página.
Si se quiere probar este ejemplo sobre una aplicación creada con
\sphinxcode{\sphinxupquote{create-react-app}} solo se debe modificar el archivo \sphinxcode{\sphinxupquote{index.js}} dentro
de \sphinxstyleemphasis{src/} cambiando \sphinxcode{\sphinxupquote{\textless{}App \textbackslash{}\textgreater{}}} por el \sphinxcode{\sphinxupquote{\textless{}h1\textgreater{}Hola mundo \textless{}/h1\textgreater{}}}, si la aplicación
se está ejecutando solo se mostrará el mensaje y ningún otro elemento de los
generados por \sphinxcode{\sphinxupquote{create-react-app}}.


\paragraph{JSX}
\label{\detokenize{reactjs:jsx}}
Considera la siguiente declaracoión de una variable:

\fvset{hllines={, ,}}%
\begin{sphinxVerbatim}[commandchars=\\\{\}]
\PYG{n}{const} \PYG{n}{element} \PYG{o}{=} \PYG{o}{\PYGZlt{}}\PYG{n}{h1}\PYG{o}{\PYGZgt{}}\PYG{n}{Hola} \PYG{n}{mundo}\PYG{o}{\PYGZlt{}}\PYG{o}{/}\PYG{n}{h1}\PYG{o}{\PYGZgt{}}\PYG{p}{;}
\end{sphinxVerbatim}

La etiqueta \sphinxcode{\sphinxupquote{h1}} no es ni una cadena ni HTML. Se llama JSX. JSX es una
extensión de Javascript. Parece un lenguage de plantillas, sin embargo, pero
incluye todo el poder de Javascript.

React adopta el hecho de que la lógica de renderizado está inherentemente
acoplada con otra lógica de la IU: cómo se manjean los eventos, cómo cambia el
estado a lo largo del tiempo y como se preparan los datos para su visualización.

En vez de separar \sphinxstyleemphasis{tecnologías}, poniendo el lenguaje de marcado
y la la lógica en archivos diferentes, la
\sphinxhref{https://es.wikipedia.org/wiki/Separaci\%C3\%B3n\_de\_intereses}{separación de intereses}
de React es con con unidades débilemente acopladas llamadas \sphinxstylestrong{componentes}
que contienen ambas tecnologías.

React no requere usar JSX, pero es útil visualmente cuando se trabaja con el lado
de la IU usando código de Javascript. Además permite a React mostrar mensajes
de error o advertencias.

Incrustando expresiones en JSX

Se puede incrustar cualquier expresión de Javascript en JSX encerrándola entre
llaves.

Por ejemplos una operación matemática \sphinxcode{\sphinxupquote{2019 + 30}}, funciones \sphinxcode{\sphinxupquote{foo(x)}},
obtener el valor de un elemento en un objeto \sphinxcode{\sphinxupquote{user.firstName}} y todas las
expresiones válidas:

\fvset{hllines={, ,}}%
\begin{sphinxVerbatim}[commandchars=\\\{\}]
\PYG{n}{function} \PYG{n}{foo}\PYG{p}{(}\PYG{n}{x}\PYG{p}{)} \PYG{p}{\PYGZob{}}
  \PYG{k}{return} \PYG{l+s+s2}{\PYGZdq{}}\PYG{l+s+s2}{La respuesta es }\PYG{l+s+s2}{\PYGZdq{}} \PYG{o}{+} \PYG{n}{x}\PYG{p}{;}
\PYG{p}{\PYGZcb{}}
\PYG{n}{const} \PYG{n}{ans} \PYG{o}{=} \PYG{l+m+mi}{42}\PYG{p}{;}
\PYG{n}{const} \PYG{n}{element} \PYG{o}{=} \PYG{p}{(}\PYG{o}{\PYGZlt{}}\PYG{n}{h1}\PYG{o}{\PYGZgt{}} \PYG{p}{\PYGZob{}}\PYG{n}{foo}\PYG{p}{(}\PYG{n}{ans}\PYG{p}{)}\PYG{p}{\PYGZcb{}} \PYG{o}{\PYGZlt{}}\PYG{o}{/}\PYG{n}{h1}\PYG{o}{\PYGZgt{}}\PYG{p}{)}\PYG{p}{;}
\end{sphinxVerbatim}

JSX tiene las siguientes características:
\begin{itemize}
\item {} 
Las expresines de JSX se vuelven llamadas a funciones y objetos de Javascript regulares.

\item {} 
Se pueden especificar atributos con JSX.

\item {} 
Se pueden especificar hijos. Las etiquetas de JSX pueden contener hijos.

\item {} 
Previene ataques de inyección.

\item {} 
JSX representa objetos.

\end{itemize}


\paragraph{Renderizado de elementos}
\label{\detokenize{reactjs:renderizado-de-elementos}}
Los elementos son los bloques más básicos en las aplicaciones de React.
Un elemento describe tolo que se quiere visualizar en la pantalla:

\fvset{hllines={, ,}}%
\begin{sphinxVerbatim}[commandchars=\\\{\}]
\PYG{n}{consts} \PYG{n}{element} \PYG{o}{=} \PYG{o}{\PYGZlt{}}\PYG{n}{h1}\PYG{o}{\PYGZgt{}}\PYG{n}{Hola} \PYG{n}{mundo}\PYG{o}{\PYGZlt{}}\PYG{o}{/}\PYG{n}{h1}\PYG{o}{\PYGZgt{}}
\end{sphinxVerbatim}

A diferencia de los elementos del DOM, los elementos de React son objetos
planos, y no gastan recursos al crearlos. \sphinxstylestrong{React DOM} se ocupa de la
actualización de DOM para encajar con los elementos de React.


\subparagraph{Renderizando un elementos en el DOM}
\label{\detokenize{reactjs:renderizando-un-elementos-en-el-dom}}
Supongamos que existe un \sphinxcode{\sphinxupquote{\textless{}div\textgreater{}}} en alguna parte de tu archivo HTML.

\fvset{hllines={, ,}}%
\begin{sphinxVerbatim}[commandchars=\\\{\}]
\PYG{p}{\PYGZlt{}}\PYG{n+nt}{div} \PYG{n+na}{id}\PYG{o}{=}\PYG{l+s}{\PYGZdq{}root\PYGZdq{}}\PYG{p}{\PYGZgt{}}\PYG{p}{\PYGZlt{}}\PYG{p}{/}\PYG{n+nt}{div}\PYG{p}{\PYGZgt{}}
\end{sphinxVerbatim}

Llamos este un nodo raíz del DOM (root en ingés) porque dentro de éste todo
será manejado por \sphinxstylestrong{React DOM}.

Las aplicaciones construidad con React usualmente tiene un solo nodo raíz del
DOM.

Para renderizar un elemento de React en el nodo raíz, se pasan ambos al método
\sphinxcode{\sphinxupquote{ReactDOM.render()}}:

\fvset{hllines={, ,}}%
\begin{sphinxVerbatim}[commandchars=\\\{\}]
\PYG{n}{const} \PYG{n}{element} \PYG{o}{=} \PYG{o}{\PYGZlt{}}\PYG{n}{h1}\PYG{o}{\PYGZgt{}}\PYG{n}{Hola} \PYG{n}{mundo}\PYG{o}{\PYGZlt{}}\PYG{o}{/}\PYG{n}{h1}\PYG{o}{\PYGZgt{}}\PYG{p}{;}
\PYG{n}{ReactDOM}\PYG{o}{.}\PYG{n}{render}\PYG{p}{(}\PYG{n}{element}\PYG{p}{,} \PYG{n}{document}\PYG{o}{.}\PYG{n}{getElementById}\PYG{p}{(}\PYG{l+s+s1}{\PYGZsq{}}\PYG{l+s+s1}{root}\PYG{l+s+s1}{\PYGZsq{}}\PYG{p}{)}\PYG{p}{)}\PYG{p}{;}
\end{sphinxVerbatim}

Esto mostrará un «Hola mundo» en la pantalla.


\subparagraph{Acualizando un elemento rederizado}
\label{\detokenize{reactjs:acualizando-un-elemento-rederizado}}
Los elementos de Reac son inmutables. Una vez que creas una elemento, no se
pueden cambiar sus atributos hijos. Un elementos es como un fotograma en
una película: representa la IU en cierto punto en el tiempo.

React solamente actualiza lo que es necesario. React Dom compara el elemento y
sus hijos con el anterior, y sólo aplica las actualizaciones necesarias del DOM
para brindale el estado deseado.


\paragraph{Componentes y propiedades}
\label{\detokenize{reactjs:componentes-y-propiedades}}
Los \sphinxstylestrong{componentes} dividen la IU en piezas independientes y reutilizables; además
permite pensar en cada una de manera aislada.

Conceptualmente, los componentes son como funciones de Javascript. Aceptan
entradas arbitrarias (\sphinxstyleemphasis{propiedades},  llamadas \sphinxstylestrong{props}) y regresan elementos
de React describiendo lo que debe aparecer en la pantalla.


\subparagraph{Componentes funcionales y clases}
\label{\detokenize{reactjs:componentes-funcionales-y-clases}}
La manera más simple de definir un componente es escribiend una función
de Javascript:

\fvset{hllines={, ,}}%
\begin{sphinxVerbatim}[commandchars=\\\{\}]
\PYG{n}{function} \PYG{n}{foo}\PYG{p}{(}\PYG{n}{props}\PYG{p}{)} \PYG{p}{\PYGZob{}}
  \PYG{k}{return} \PYG{o}{\PYGZlt{}}\PYG{n}{h1}\PYG{o}{\PYGZgt{}} \PYG{n}{Hola} \PYG{p}{\PYGZob{}}\PYG{n}{props}\PYG{o}{.}\PYG{n}{name}\PYG{p}{\PYGZcb{}} \PYG{o}{\PYGZlt{}}\PYG{o}{/}\PYG{n}{h1}\PYG{o}{\PYGZgt{}}
\PYG{p}{\PYGZcb{}}
\end{sphinxVerbatim}

Esta función es un componente válido de React porque acepta una
objeto \sphinxstylestrong{props} como parámetro con información y retorna un elemento de React.
A esos componentes se les llama \sphinxstylestrong{funcionales} porque literalmente son funciones
de Javascript.

También se puede usar clase de ES6 para definir un componente:

\fvset{hllines={, ,}}%
\begin{sphinxVerbatim}[commandchars=\\\{\}]
\PYG{k}{class} \PYG{n+nc}{Foo} \PYG{n}{extends} \PYG{n}{React}\PYG{o}{.}\PYG{n}{Component} \PYG{p}{\PYGZob{}}
  \PYG{n}{render} \PYG{p}{(}\PYG{p}{)} \PYG{p}{\PYGZob{}}
    \PYG{k}{return} \PYG{o}{\PYGZlt{}}\PYG{n}{h1}\PYG{o}{\PYGZgt{}} \PYG{n}{Hola} \PYG{p}{\PYGZob{}}\PYG{n}{this}\PYG{o}{.}\PYG{n}{props}\PYG{o}{.}\PYG{n}{name}\PYG{p}{\PYGZcb{}} \PYG{o}{\PYGZlt{}}\PYG{o}{/}\PYG{n}{h1}\PYG{o}{\PYGZgt{}}
  \PYG{p}{\PYGZcb{}}
\PYG{p}{\PYGZcb{}}
\end{sphinxVerbatim}

Ambos componentes son equivalentes desde el punto de vista de React. Aunque las
clases tiene características adicionales.


\subparagraph{Renderizando un componente}
\label{\detokenize{reactjs:renderizando-un-componente}}
Previamente sólo se han mostrado elementos de React que representa etiquetas
del DOM:

\fvset{hllines={, ,}}%
\begin{sphinxVerbatim}[commandchars=\\\{\}]
\PYG{n}{const} \PYG{n}{element} \PYG{o}{=}  \PYG{o}{\PYGZlt{}}\PYG{n}{div} \PYG{o}{/}\PYG{o}{\PYGZgt{}}
\end{sphinxVerbatim}

Sin embargo, los elementos pueden representar componentes definidos por el
usuario:

\fvset{hllines={, ,}}%
\begin{sphinxVerbatim}[commandchars=\\\{\}]
\PYG{n}{const} \PYG{n}{element} \PYG{o}{=} \PYG{o}{\PYGZlt{}}\PYG{n}{Foo} \PYG{n}{name}\PYG{o}{=}\PYG{l+s+s2}{\PYGZdq{}}\PYG{l+s+s2}{K}\PYG{l+s+s2}{\PYGZdq{}} \PYG{o}{/}\PYG{o}{\PYGZgt{}}
\end{sphinxVerbatim}

Cuando React ve un elemento que representa un componente definido por el usuario,
pasa atributos JSX a este componente como un solo objeto. Llamamos a este
objeto \sphinxstylestrong{props}.

Por ejemplo el siguiente código renderiza «Hola K» en una pagina:

\fvset{hllines={, ,}}%
\begin{sphinxVerbatim}[commandchars=\\\{\}]
\PYG{n}{function} \PYG{n}{Welcome}\PYG{p}{(}\PYG{n}{props}\PYG{p}{)} \PYG{p}{\PYGZob{}}
  \PYG{k}{return} \PYG{o}{\PYGZlt{}}\PYG{n}{h1}\PYG{o}{\PYGZgt{}}\PYG{n}{Hola} \PYG{p}{\PYGZob{}}\PYG{n}{props}\PYG{o}{.}\PYG{n}{name}\PYG{p}{\PYGZcb{}}\PYG{o}{\PYGZlt{}}\PYG{o}{/}\PYG{n}{h1}\PYG{o}{\PYGZgt{}}\PYG{p}{;}
\PYG{p}{\PYGZcb{}}

\PYG{n}{const} \PYG{n}{element} \PYG{o}{=} \PYG{o}{\PYGZlt{}}\PYG{n}{Welcome} \PYG{n}{name}\PYG{o}{=}\PYG{l+s+s2}{\PYGZdq{}}\PYG{l+s+s2}{K}\PYG{l+s+s2}{\PYGZdq{}} \PYG{o}{/}\PYG{o}{\PYGZgt{}}\PYG{p}{;}
\PYG{n}{ReactDOM}\PYG{o}{.}\PYG{n}{render}\PYG{p}{(}
  \PYG{n}{element}\PYG{p}{,}
  \PYG{n}{document}\PYG{o}{.}\PYG{n}{getElementById}\PYG{p}{(}\PYG{l+s+s1}{\PYGZsq{}}\PYG{l+s+s1}{root}\PYG{l+s+s1}{\PYGZsq{}}\PYG{p}{)}
\PYG{p}{)}\PYG{p}{;}
\end{sphinxVerbatim}


\subparagraph{Composición de componentes}
\label{\detokenize{reactjs:composicion-de-componentes}}
Componentes pueden hacer referencia a otros componentes en su salida. Esto
permite usar la misma abstracción del componente en cualquier nivel de detalle.
Un butón, un formulario, diaĺogo, una pantall: en aplicaciones de React, todos
estos son comunmente expresados como componentes.

Por ejemplo, cuando creamos un componente \sphinxcode{\sphinxupquote{App}} que renderiza \sphinxcode{\sphinxupquote{Welcom}}
muchas veces:

\fvset{hllines={, ,}}%
\begin{sphinxVerbatim}[commandchars=\\\{\}]
\PYG{n}{function} \PYG{n}{Welcome}\PYG{p}{(}\PYG{n}{props}\PYG{p}{)} \PYG{p}{\PYGZob{}}
  \PYG{k}{return} \PYG{o}{\PYGZlt{}}\PYG{n}{h1}\PYG{o}{\PYGZgt{}}\PYG{n}{Hola} \PYG{p}{\PYGZob{}}\PYG{n}{props}\PYG{o}{.}\PYG{n}{name}\PYG{p}{\PYGZcb{}}\PYG{o}{\PYGZlt{}}\PYG{o}{/}\PYG{n}{h1}\PYG{o}{\PYGZgt{}}\PYG{p}{;}
\PYG{p}{\PYGZcb{}}

\PYG{n}{function} \PYG{n}{App}\PYG{p}{(}\PYG{p}{)} \PYG{p}{\PYGZob{}}
  \PYG{k}{return} \PYG{p}{(}
    \PYG{o}{\PYGZlt{}}\PYG{n}{div}\PYG{o}{\PYGZgt{}}
      \PYG{o}{\PYGZlt{}}\PYG{n}{Welcome} \PYG{n}{name}\PYG{o}{=}\PYG{l+s+s2}{\PYGZdq{}}\PYG{l+s+s2}{K}\PYG{l+s+s2}{\PYGZdq{}} \PYG{o}{/}\PYG{o}{\PYGZgt{}}
      \PYG{o}{\PYGZlt{}}\PYG{n}{Welcome} \PYG{n}{name}\PYG{o}{=}\PYG{l+s+s2}{\PYGZdq{}}\PYG{l+s+s2}{Rick}\PYG{l+s+s2}{\PYGZdq{}} \PYG{o}{/}\PYG{o}{\PYGZgt{}}
      \PYG{o}{\PYGZlt{}}\PYG{n}{Welcome} \PYG{n}{name}\PYG{o}{=}\PYG{n}{Rachael}\PYG{l+s+s2}{\PYGZdq{}}\PYG{l+s+s2}{\PYGZdq{}} \PYG{o}{/}\PYG{o}{\PYGZgt{}}
    \PYG{o}{\PYGZlt{}}\PYG{o}{/}\PYG{n}{div}\PYG{o}{\PYGZgt{}}
  \PYG{p}{)}\PYG{p}{;}
\PYG{p}{\PYGZcb{}}

\PYG{n}{ReactDOM}\PYG{o}{.}\PYG{n}{render}\PYG{p}{(}
  \PYG{o}{\PYGZlt{}}\PYG{n}{App} \PYG{o}{/}\PYG{o}{\PYGZgt{}}\PYG{p}{,}
  \PYG{n}{document}\PYG{o}{.}\PYG{n}{getElementById}\PYG{p}{(}\PYG{l+s+s1}{\PYGZsq{}}\PYG{l+s+s1}{root}\PYG{l+s+s1}{\PYGZsq{}}\PYG{p}{)}
\PYG{p}{)}\PYG{p}{;}
\end{sphinxVerbatim}

Comúnmente, nuevas aplicaciones de React tiene una solo componente \sphinxcode{\sphinxupquote{App}}
en el nivel más alto.


\subparagraph{Extrayendo de componentes}
\label{\detokenize{reactjs:extrayendo-de-componentes}}
La extraccción de componentes se refiere a dividir componentes en más pequeños.

Extraer componentes permite tener una paleta de componentes reutilizables en
aplicaciones grandes. Una regla genera es que que si una parte de tu IU es
usada variasa veces (un formulario, un avatar para un usuario, una lista),
o es lo suficientemente compleja por su cuenta (App, un perfil de un usuario),
es candidato para ser un componente reutilizable.


\subparagraph{Props son de solo lectura}
\label{\detokenize{reactjs:props-son-de-solo-lectura}}
Cuando se declara un componente como una función o una clase, nunca debe modificar
su \sphinxstylestrong{props}. Considera la función de suma:

\fvset{hllines={, ,}}%
\begin{sphinxVerbatim}[commandchars=\\\{\}]
\PYG{n}{function} \PYG{n+nb}{sum}\PYG{p}{(}\PYG{n}{a}\PYG{p}{,} \PYG{n}{b}\PYG{p}{)} \PYG{p}{\PYGZob{}}
  \PYG{k}{return} \PYG{n}{a} \PYG{o}{+} \PYG{n}{b}\PYG{p}{;}
\PYG{p}{\PYGZcb{}}
\end{sphinxVerbatim}

Funciones como esta son llamadas \sphinxstyleemphasis{puras} porque no intentan cambiar sus entradas
y siempre regresan el mismo resultado para las mismas entradas.

En contraste, esta función es impura poque cambia su propia entrada:

\fvset{hllines={, ,}}%
\begin{sphinxVerbatim}[commandchars=\\\{\}]
\PYG{n}{function} \PYG{n}{changeUserEmail}\PYG{p}{(}\PYG{n}{user}\PYG{p}{,} \PYG{n}{newEmail}\PYG{p}{)} \PYG{p}{\PYGZob{}}
  \PYG{n}{user}\PYG{o}{.}\PYG{n}{email} \PYG{o}{=} \PYG{n}{newEmail}\PYG{p}{;}
\PYG{p}{\PYGZcb{}}
\end{sphinxVerbatim}

React tiene la regla de que todos los componentes actúan como funciones puras
con respecto a sus props.


\paragraph{Estados y ciclo de vida}
\label{\detokenize{reactjs:estados-y-ciclo-de-vida}}
Consideremos el siguiente ejemplo:
\begin{quote}

Tenemos un componente que muestra la la hora en la pantalla (un formato en
horas, minutos y segundos). Este componente debe ser reutilizable y estar
encapsulado. Defininirá su propio temporizador y se actualizará cada segund.o
Podemos empezar definiendo una versión del componente que cumple con el
requisito de estar encapsulado:

\fvset{hllines={, ,}}%
\begin{sphinxVerbatim}[commandchars=\\\{\}]
\PYG{n}{function} \PYG{n}{Clock}\PYG{p}{(}\PYG{n}{props}\PYG{p}{)} \PYG{p}{\PYGZob{}}
  \PYG{k}{return} \PYG{p}{(}
    \PYG{o}{\PYGZlt{}}\PYG{n}{div}\PYG{o}{\PYGZgt{}}
      \PYG{o}{\PYGZlt{}}\PYG{n}{h2}\PYG{o}{\PYGZgt{}}\PYG{p}{\PYGZob{}}\PYG{n}{props}\PYG{o}{.}\PYG{n}{date}\PYG{o}{.}\PYG{n}{toLocaleTimeString}\PYG{p}{(}\PYG{p}{)}\PYG{p}{\PYGZcb{}}\PYG{o}{.}\PYG{o}{\PYGZlt{}}\PYG{o}{/}\PYG{n}{h2}\PYG{o}{\PYGZgt{}}
    \PYG{o}{\PYGZlt{}}\PYG{o}{/}\PYG{n}{div}\PYG{o}{\PYGZgt{}}
  \PYG{p}{)}\PYG{p}{;}
\PYG{p}{\PYGZcb{}}

\PYG{n}{function} \PYG{n}{tick}\PYG{p}{(}\PYG{p}{)} \PYG{p}{\PYGZob{}}
  \PYG{n}{ReactDOM}\PYG{o}{.}\PYG{n}{render}\PYG{p}{(}
    \PYG{o}{\PYGZlt{}}\PYG{n}{Clock} \PYG{n}{data}\PYG{o}{=}\PYG{p}{\PYGZob{}}\PYG{n}{new} \PYG{n}{Date}\PYG{p}{(}\PYG{p}{)}\PYG{p}{\PYGZcb{}}\PYG{o}{\PYGZgt{}}\PYG{p}{,}
    \PYG{n}{document}\PYG{o}{.}\PYG{n}{getElementById}\PYG{p}{(}\PYG{l+s+s1}{\PYGZsq{}}\PYG{l+s+s1}{root}\PYG{l+s+s1}{\PYGZsq{}}\PYG{p}{)}
  \PYG{p}{)}\PYG{p}{;}
\PYG{p}{\PYGZcb{}}

\PYG{n}{setInterval}\PYG{p}{(}\PYG{n}{tick}\PYG{p}{,} \PYG{l+m+mi}{1000}\PYG{p}{)}\PYG{p}{;}
\end{sphinxVerbatim}
\end{quote}

En el ejemplo ejecutamos la función \sphinxcode{\sphinxupquote{tick}} cada segundo para renderizar al
componente \sphinxcode{\sphinxupquote{Clock}} y mostrar en la pantalla la hora actual. A pesar de
que el componente anterior funciona, es un requisito esencial: el hecho
de que \sphinxcode{\sphinxupquote{Clock}} configure un temporizador y actualice la IU cada segundo
debe ser un detalle de implementación del \sphinxcode{\sphinxupquote{Clock}}.

Es ideal escribir una vez lo siguiente y que \sphinxcode{\sphinxupquote{Clock}} se actualice a sí mismo:

\fvset{hllines={, ,}}%
\begin{sphinxVerbatim}[commandchars=\\\{\}]
\PYG{n}{ReactDOM}\PYG{o}{.}\PYG{n}{render}\PYG{p}{(}
  \PYG{o}{\PYGZlt{}}\PYG{n}{Clock} \PYG{o}{/}\PYG{o}{\PYGZgt{}}\PYG{p}{,}
  \PYG{n}{document}\PYG{o}{.}\PYG{n}{getElementById}\PYG{p}{(}\PYG{l+s+s1}{\PYGZsq{}}\PYG{l+s+s1}{root}\PYG{l+s+s1}{\PYGZsq{}}\PYG{p}{)}
\PYG{p}{)}\PYG{p}{;}
\end{sphinxVerbatim}

Para implementar esto, es necesario añadir un \sphinxstylestrong{estado} al componente \sphinxcode{\sphinxupquote{Clock}}.

Un estado es similar a los props, pero es privado y únicamente controlado por su
componente.

Los componentes definidos como clases tiene características adicionales. Los
estados locales son un funcionalidad exclusiva de las clases.


\subparagraph{Convertir componente funcional a una clase}
\label{\detokenize{reactjs:convertir-componente-funcional-a-una-clase}}
Se puede convertir un componente funcional como \sphinxcode{\sphinxupquote{Clock}} en una clase en
cinco pasos:
\begin{enumerate}
\item {} 
Crear una clase de \sphinxstylestrong{ES6}, con el mismo nombre, que herede de \sphinxcode{\sphinxupquote{React.Component}}.

\item {} 
Añader un método vacío llamado \sphinxcode{\sphinxupquote{render()}}.

\item {} 
Mover el cuerpo de la función dentro del método \sphinxcode{\sphinxupquote{render()}}.

\item {} 
Remplazar \sphinxcode{\sphinxupquote{props}} con \sphinxcode{\sphinxupquote{this.props}} en el cuerpo de \sphinxcode{\sphinxupquote{render()}}.

\item {} 
Elimnar la declaración de la función vacía restante.

\end{enumerate}

Siguiendo los pasos sobre \sphinxcode{\sphinxupquote{Clock}}, resulta:

\fvset{hllines={, ,}}%
\begin{sphinxVerbatim}[commandchars=\\\{\}]
\PYG{k}{class} \PYG{n+nc}{Clock} \PYG{n}{extends} \PYG{n}{React}\PYG{o}{.}\PYG{n}{Component} \PYG{p}{\PYGZob{}}
  \PYG{n}{render}\PYG{p}{(}\PYG{p}{)} \PYG{p}{\PYGZob{}}
    \PYG{k}{return} \PYG{p}{(}
      \PYG{o}{\PYGZlt{}}\PYG{n}{div}\PYG{o}{\PYGZgt{}}
        \PYG{o}{\PYGZlt{}}\PYG{n}{h2}\PYG{o}{\PYGZgt{}}\PYG{p}{\PYGZob{}}\PYG{n}{this}\PYG{o}{.}\PYG{n}{props}\PYG{o}{.}\PYG{n}{date}\PYG{o}{.}\PYG{n}{toLocaleTimeString}\PYG{p}{(}\PYG{p}{)}\PYG{p}{\PYGZcb{}}\PYG{o}{.}\PYG{o}{\PYGZlt{}}\PYG{o}{/}\PYG{n}{h2}\PYG{o}{\PYGZgt{}}
      \PYG{o}{\PYGZlt{}}\PYG{o}{/}\PYG{n}{div}\PYG{o}{\PYGZgt{}}
    \PYG{p}{)}\PYG{p}{;}
  \PYG{p}{\PYGZcb{}}
\PYG{p}{\PYGZcb{}}
\end{sphinxVerbatim}


\subparagraph{Adición de un estado local a la clase}
\label{\detokenize{reactjs:adicion-de-un-estado-local-a-la-clase}}
Siguiendo el ejemplo, se moverá \sphinxcode{\sphinxupquote{data}} de las props a un estado como sigue:
\begin{enumerate}
\item {} 
Remplazar \sphinxcode{\sphinxupquote{this.props.date}} con \sphinxcode{\sphinxupquote{this.state.date}} ene el método \sphinxcode{\sphinxupquote{render()}}.

\item {} 
Añadir un constructor a la clase que asigne un valor al objeto \sphinxcode{\sphinxupquote{this.state}}. Pasar como parámetro del contructor base \sphinxcode{\sphinxupquote{props}}. Los componentes como clases deben siempre llamar al constructor base con \sphinxcode{\sphinxupquote{props}} como argumento.

\item {} 
Eliminar la prop \sphinxcode{\sphinxupquote{date}} del elemento \sphinxcode{\sphinxupquote{\textless{}Clock/\textgreater{}}}.

\end{enumerate}

El resultado es el siguiente:

\fvset{hllines={, ,}}%
\begin{sphinxVerbatim}[commandchars=\\\{\}]
\PYG{k}{class} \PYG{n+nc}{Clock} \PYG{n}{extends} \PYG{n}{React}\PYG{o}{.}\PYG{n}{Component} \PYG{p}{\PYGZob{}}
  \PYG{n}{constructor}\PYG{p}{(}\PYG{n}{props}\PYG{p}{)} \PYG{p}{\PYGZob{}}
    \PYG{n+nb}{super}\PYG{p}{(}\PYG{n}{props}\PYG{p}{)}\PYG{p}{;}
    \PYG{n}{this}\PYG{o}{.}\PYG{n}{state} \PYG{o}{=} \PYG{p}{\PYGZob{}}\PYG{n}{date}\PYG{p}{:} \PYG{n}{new} \PYG{n}{Date}\PYG{p}{(}\PYG{p}{)}\PYG{p}{\PYGZcb{}}\PYG{p}{;}
  \PYG{p}{\PYGZcb{}}

  \PYG{n}{render}\PYG{p}{(}\PYG{p}{)} \PYG{p}{\PYGZob{}}
    \PYG{k}{return} \PYG{p}{(}
      \PYG{o}{\PYGZlt{}}\PYG{n}{div}\PYG{o}{\PYGZgt{}}
        \PYG{o}{\PYGZlt{}}\PYG{n}{h2}\PYG{o}{\PYGZgt{}}\PYG{p}{\PYGZob{}}\PYG{n}{this}\PYG{o}{.}\PYG{n}{state}\PYG{o}{.}\PYG{n}{date}\PYG{o}{.}\PYG{n}{toLocaleTimeString}\PYG{p}{(}\PYG{p}{)}\PYG{p}{\PYGZcb{}}\PYG{o}{.}\PYG{o}{\PYGZlt{}}\PYG{o}{/}\PYG{n}{h2}\PYG{o}{\PYGZgt{}}
      \PYG{o}{\PYGZlt{}}\PYG{o}{/}\PYG{n}{div}\PYG{o}{\PYGZgt{}}
    \PYG{p}{)}\PYG{p}{;}
  \PYG{p}{\PYGZcb{}}
\PYG{p}{\PYGZcb{}}

\PYG{n}{ReactDOM}\PYG{o}{.}\PYG{n}{render}\PYG{p}{(}
  \PYG{o}{\PYGZlt{}}\PYG{n}{Clock} \PYG{o}{/}\PYG{o}{\PYGZgt{}}\PYG{p}{,}
  \PYG{n}{document}\PYG{o}{.}\PYG{n}{getElementById}\PYG{p}{(}\PYG{l+s+s1}{\PYGZsq{}}\PYG{l+s+s1}{root}\PYG{l+s+s1}{\PYGZsq{}}\PYG{p}{)}
\PYG{p}{)}\PYG{p}{;}
\end{sphinxVerbatim}

Por ahora el código el componente no ha definido su temporizador y actualización
cada segundo.


\subparagraph{Métodos de ciclo de vida para una clase}
\label{\detokenize{reactjs:metodos-de-ciclo-de-vida-para-una-clase}}
Cada componente tiene varios \sphinxstylestrong{métodos de ciclo de vida} que se pueden
sobreescribir para ejecutar código en determinados momentos. Los métodos
prefijados con \sphinxstylestrong{will} son llamados antes de que algo suceda, y los métodos
con el prefijo \sphinxstylestrong{did} son llamados justo después de que algo pase.


\textbf{Montaje (Mounting)}
\label{\detokenize{reactjs:montaje-mounting}}
Estos métodos son llamados cuando una instancia de un componente está siendo
creada e insertada en el DOM:
\begin{itemize}
\item {} 
\sphinxcode{\sphinxupquote{constructor()}}

\item {} 
\sphinxcode{\sphinxupquote{componentWillMount()}}

\item {} 
\sphinxcode{\sphinxupquote{render()}}

\item {} 
\sphinxcode{\sphinxupquote{componentDidMount()}}

\end{itemize}


\textbf{Actualización (Updating)}
\label{\detokenize{reactjs:actualizacion-updating}}
Una actualización puede ser causada por cambios en el estado o en las props.
Estos métodos son llamadas cuando un componente está siendo re-renderizado:
\begin{itemize}
\item {} 
\sphinxcode{\sphinxupquote{componentWillReceiveProps()}}

\item {} 
\sphinxcode{\sphinxupquote{shouldComponentUpdate()}}

\item {} 
\sphinxcode{\sphinxupquote{componentWillUpdate()}}

\item {} 
\sphinxcode{\sphinxupquote{render()}}

\item {} 
\sphinxcode{\sphinxupquote{componentDidUpdate()}}

\end{itemize}


\textbf{Desmontaje (Unmounting)}
\label{\detokenize{reactjs:desmontaje-unmounting}}
Este método es llamado cuando un componente se elimina del DOM.
\begin{itemize}
\item {} 
\sphinxcode{\sphinxupquote{componentWillUnmount()}}

\end{itemize}


\textbf{Sobrescribiendo lo métodos del cliclo de vida}
\label{\detokenize{reactjs:sobrescribiendo-lo-metodos-del-cliclo-de-vida}}
Para el ejemplo del componente \sphinxcode{\sphinxupquote{Clock}} queremos definir un temporizador cuando
sea renderizado en el DOM la primera vez. Tambien queremos limpiar el
temporizador cuando sea que el DOM generado por \sphinxcode{\sphinxupquote{Clock}} se elimine. Lo métodos
para cada son \sphinxcode{\sphinxupquote{componentDidMount()}} y \sphinxcode{\sphinxupquote{componentWillUnmount()}}
respectivamente.

También necesitamos un método encargado de cambiar el estado de la hora obteniendo
la hora actual.

Entonces, el componente crea un temporizador que ejecuta un método
dentro de la clase que actualiza un estado con \sphinxcode{\sphinxupquote{this.setState}} cuando se monta.
Cuando se desmonta se elimina el temporizador.

Finalmente el componente queda como sigue y cumple con los requisitos de
estar encapsulado y que es reutilizable:

\fvset{hllines={, ,}}%
\begin{sphinxVerbatim}[commandchars=\\\{\}]
class Clock extends React.Component \PYGZob{}
  constructor(props) \PYGZob{}
    super(props);
    this.state = \PYGZob{}date: new Date()\PYGZcb{};
  \PYGZcb{}

  componentDidMount() \PYGZob{}
    this.timerID = setInterval(
      () =\PYGZgt{} this.tick(),
      1000
    );
  \PYGZcb{}

  componentWillUnmount() \PYGZob{}
    clearInterval(this.timerID);
  \PYGZcb{}

  tick() \PYGZob{}
    this.setState(\PYGZob{}
      date: new Date()
    \PYGZcb{});
  \PYGZcb{}

  render() \PYGZob{}
    return (
      \PYGZlt{}div\PYGZgt{}
        \PYGZlt{}h1\PYGZgt{}Hello, world!\PYGZlt{}/h1\PYGZgt{}
        \PYGZlt{}h2\PYGZgt{}It is \PYGZob{}this.state.date.toLocaleTimeString()\PYGZcb{}.\PYGZlt{}/h2\PYGZgt{}
      \PYGZlt{}/div\PYGZgt{}
    );
  \PYGZcb{}
\PYGZcb{}

ReactDOM.render(
  \PYGZlt{}Clock /\PYGZgt{},
  document.getElementById(\PYGZsq{}root\PYGZsq{})
);
\end{sphinxVerbatim}

Hay tres cosas importantes que hay que saber sobre los estados:
\begin{enumerate}
\item {} 
No se deben modificar directamente. Siempre usar \sphinxcode{\sphinxupquote{setState}}.

\item {} 
Las actualizaciones de los estados pueden ser asíncronas.

\item {} 
Las actualizaciones de los estaados se unen.

\end{enumerate}


\paragraph{Manejando eventos}
\label{\detokenize{reactjs:manejando-eventos}}
Manejar eventos con elementos de Reacr es muy parecido a manejar eventos en
el DOM Hay alagunas diferencias sintácticas:
\begin{itemize}
\item {} 
El nombre de los eventos de React se hace usando el estilo camelCase y no con minúsculas.

\item {} 
Con JSX se pasa una función como manejador del evento, y no una cadena.

\end{itemize}

Cuando se define un componente unsando una clase de ES6, un patrón cómun es que
un manejador de un evento sea método de la clase. Por ejemplo, el siguiente
\sphinxcode{\sphinxupquote{Toggle}} renderiza un butón que permite cambiar al usuario entre estados
«ON» y «OFF»:

\fvset{hllines={, ,}}%
\begin{sphinxVerbatim}[commandchars=\\\{\}]
class Toggle extends React.Component \PYGZob{}
  constructor(props) \PYGZob{}
    super(props);
    this.state = \PYGZob{}isToggleOn: true\PYGZcb{};

    // El enlace (binding) es necesario para hacer que {}`this{}` funcione en el callback
    this.handleClick = this.handleClick.bind(this);
  \PYGZcb{}

  handleClick() \PYGZob{}
    this.setState(prevState =\PYGZgt{} (\PYGZob{}
      isToggleOn: !prevState.isToggleOn
    \PYGZcb{}));
  \PYGZcb{}

  render() \PYGZob{}
    return (
      \PYGZlt{}button onClick=\PYGZob{}this.handleClick\PYGZcb{}\PYGZgt{}
        \PYGZob{}this.state.isToggleOn ? \PYGZsq{}ON\PYGZsq{} : \PYGZsq{}OFF\PYGZsq{}\PYGZcb{}
      \PYGZlt{}/button\PYGZgt{}
    );
  \PYGZcb{}
\PYGZcb{}

ReactDOM.render(
  \PYGZlt{}Toggle /\PYGZgt{},
  document.getElementById(\PYGZsq{}root\PYGZsq{})
);
\end{sphinxVerbatim}


\subparagraph{Pasando argumentos a los manejadores de eventos}
\label{\detokenize{reactjs:pasando-argumentos-a-los-manejadores-de-eventos}}
Las siguientes dos líneas son equivalentes, permiten pasar un parámetro extra al
manejador de un evento. La primera usa una \sphinxstylestrong{función flecha} y la segunda
\sphinxstylestrong{Function.prototype.bind}:

\fvset{hllines={, ,}}%
\begin{sphinxVerbatim}[commandchars=\\\{\}]
\PYG{o}{\PYGZlt{}}\PYG{n}{button} \PYG{n}{onClick}\PYG{o}{=}\PYG{p}{\PYGZob{}}\PYG{p}{(}\PYG{n}{e}\PYG{p}{)} \PYG{o}{=}\PYG{o}{\PYGZgt{}} \PYG{n}{this}\PYG{o}{.}\PYG{n}{deleteRow}\PYG{p}{(}\PYG{n+nb}{id}\PYG{p}{,} \PYG{n}{e}\PYG{p}{)}\PYG{p}{\PYGZcb{}}\PYG{o}{\PYGZgt{}}\PYG{n}{Delete} \PYG{n}{Row}\PYG{o}{\PYGZlt{}}\PYG{o}{/}\PYG{n}{button}\PYG{o}{\PYGZgt{}}
\PYG{o}{\PYGZlt{}}\PYG{n}{button} \PYG{n}{onClick}\PYG{o}{=}\PYG{p}{\PYGZob{}}\PYG{n}{this}\PYG{o}{.}\PYG{n}{deleteRow}\PYG{o}{.}\PYG{n}{bind}\PYG{p}{(}\PYG{n}{this}\PYG{p}{,} \PYG{n+nb}{id}\PYG{p}{)}\PYG{p}{\PYGZcb{}}\PYG{o}{\PYGZgt{}}\PYG{n}{Delete} \PYG{n}{Row}\PYG{o}{\PYGZlt{}}\PYG{o}{/}\PYG{n}{button}\PYG{o}{\PYGZgt{}}
\end{sphinxVerbatim}

En ambos casos, el argumento \sphinxcode{\sphinxupquote{e}} será pasado como segundo argumento después
del \sphinxcode{\sphinxupquote{id}}. Con una función flecha se tiene que pasar explícitamente, con
\sphinxcode{\sphinxupquote{bind}} cualquier argumento adicional se reenvía automáticamente.


\paragraph{Componentes de alto orden}
\label{\detokenize{reactjs:componentes-de-alto-orden}}
Un componente de alto orden (HOC por sus siglas en inglés) es una técnica
avanzada en React para reutilizar la lógica de componentes. Los HOC no son
parte de la API de React. Son un patrón que surge de la naturaleza de
composición  de React.

De manera concreta, un componente de alto nivel es una función que toma como
parámetro un componente y retorna un nuevo componente.

\fvset{hllines={, ,}}%
\begin{sphinxVerbatim}[commandchars=\\\{\}]
\PYG{k+kr}{const} \PYG{n+nx}{ComponenteMejorado} \PYG{o}{=} \PYG{n+nx}{componenteDeAltoNivel}\PYG{p}{(}\PYG{n+nx}{ComponenteEnvuelto}\PYG{p}{)}
\end{sphinxVerbatim}

Donde un componente transforma props en un IU, un HOC transforma un
componente en otro componente.


\subsubsection{Documentación para usuarios}
\label{\detokenize{users_docs:documentacion-para-usuarios}}

\paragraph{Descripción de la aplicación}
\label{\detokenize{descripcion-de-la-aplicacion}}
La aplicación es bastante simple e intuitiva, el flujo que se sigue para
utilizarla es el siguiente:
\begin{enumerate}
\item {} 
El usuario inicia sesión o se registra si no ha creado una cuenta.

\item {} 
Es redireccionado a un panel para interactuar con el robot, después de haber seleccionado alguno de una lista o de haber creado uno nuevo.

\item {} 
Puede cerrar la sesión o esta se mantiene activa.

\end{enumerate}

\begin{figure}[htbp]
\centering
\capstart

\noindent\sphinxincludegraphics[scale=0.2]{{webappflow}.png}
\caption{El usuario inicia sesión o se registra, es redireccionado a un panel, el usuario debe elegir un robot de la lista para poder interactuar con él.}\label{\detokenize{users_docs:webappflow}}\end{figure}

Como se puede ver, la aplicación web cuenta con algunas características
muy similares a la aplicación móvil. Cuenta con las funcionalidades para
el registro de usuarios, que cada usuario pueda utilizar varios robots e
interactuar con ellos usando un control remoto, o recibiendo valores enviados
por el mimos Robot a Firebase.

Para comenzar a utilizarla simplementes se hace lo siguiente:


\subparagraph{Inicio de sesión}
\label{\detokenize{inicio-de-sesion}}
Un usuario puede iniciar sesión si ya se ha registrado, sino tiene la opción
de crear una nueva cuenta. El usuario no necesariamente debe crearse dentro
de la aplicación web, si se ha registrado usando la aplicación móvil, entonces
puede usar el mismo nombre de usuario y contraseña.


\subparagraph{Panel de inteacción con el robot}
\label{\detokenize{users_docs:panel-de-inteaccion-con-el-robot}}
La principal función de esta aplicación es conectarse remótamente a un robot
NAO usando como intermediario a Firebase. Si se desea enviar una orden al robot
primero se escribe esa orden en una ubicación en la base de datos, el robot
recibe dicha actualización y la procesa para ejecutar tal comando.

El panel cuenta la estructura mostrada en la fugura \hyperref[\detokenize{users_docs:webappstruct}]{Figura \ref{\detokenize{users_docs:webappstruct}}}:

\begin{figure}[htbp]
\centering
\capstart

\noindent\sphinxincludegraphics[scale=0.5]{{webappstruct}.png}
\caption{Estructura del panel dentro de la aplicación}\label{\detokenize{users_docs:webappstruct}}\end{figure}
\begin{enumerate}
\item {} 
Lista de los robots que ha añadido el usuario, aquí también puede agregar uno. Primero hay que seleccionar un robot de la lista para poder ver sus logs, o la última imagen que envío de sus cámara, etc.

\item {} 
Aquí el usuario puede escribir un texto breve para que el robot lo repita oralmente.

\item {} 
Control remoto del robot para realizar algunos movimientos como caminar, cambiar entre dos posturas y mover las articulaciones de la cabeza. Se puede cambiar la velocidad con la que camina y otros parámetros de caminado definidos por la API de NAOqi.

\item {} 
Muestra una imagen tomada con la cámara superior del robot, funciona para obtener una imágen en vivo enviada por el robot cuando éste está conectado a Firebase.

\item {} 
Aquí aparece el estado de la conexión con el robot, si el robot ha ejecutado el programa que se conecta a Firebase, se muestra un estado en línea y el nivel de la batería.

\item {} 
Una tabla con valores de \sphinxcode{\sphinxupquote{ALMemory}} enviados por el robot. También permite descargar el historial completo de logs.

\end{enumerate}


\subparagraph{Cierre de sesión}
\label{\detokenize{users_docs:cierre-de-sesion}}
El usuario puede cerrar su sesión o esta se mantiene activa en el navegador
hasta que se desee.


\subparagraph{Notas}
\label{\detokenize{users_docs:notas}}
Cabe señalar nuevamente que para usar todas las funcionalidades de esta
aplicación se debe ejecutar el programa del lado del robot que se conecta a
Firebase para recibir y enviar información, si esto no se lleva a cabo, sólo
se muestran los últimos datos enviados por el robot o por la aplicación móvil.

Cada robot que registra un usuario tiene un identificador único, este identificador
le sirve para escribir en las ubicaciones den la base de datos que le corresponden.
Al pasar el puntero del ratón sobre un robot en la lista de robots, se muestra
su identificador único que es el que se añade al archivo de configuración
del programa que se corre sobre el robot.


\subsubsection{Documentación para desarrolladores}
\label{\detokenize{code_docs:documentacion-para-desarrolladores}}\label{\detokenize{code_docs::doc}}
En la aplicación se implementa únicamente la parte del front end, de la parte
del back end, se encarga Firebase. Por esto, las únicas herramientas que se
usaron fueron React y el SDK de Firebase para la autenticación de usuarios y la base
de datos.


\paragraph{Creación del proyecto}
\label{\detokenize{code_docs:creacion-del-proyecto}}
Para la creación del proyecto se utiliza la herramienta \sphinxcode{\sphinxupquote{creat-react-app}}
que crea un proyecto con todas las dependencias de React, después de hacer un
\sphinxcode{\sphinxupquote{create-react-app cloudnao-web}}, navegar el nuevo directorio creado y
ejecutar \sphinxcode{\sphinxupquote{npm start}}, la aplicación está disponible en \sphinxcode{\sphinxupquote{http://localhost:3000}}.

Con esto ya tenemos una aplicación de React, pero aún no se puede
utilizar la base de datos de Firebase, la autenticación o Firebase Hosting.

Para poder desplegar la aplicación usando Firebase Hosting se debe instalar la
interfaz para la línea de comandos de Firebase, usando el comando
\sphinxcode{\sphinxupquote{npm install firebase-tools}}. Después de instalar esta herramienta, se ejecuta
\sphinxcode{\sphinxupquote{firebase init}}, que conectará nuestro proyecto local con un proyecto
creado en Firebase.
\begin{itemize}
\item {} 
Paso uno. Seleccionar las características de Firebase que se desean utilizar.

\item {} 
Paso dos. Seleccionar el proyecto correspondiente de la lista de proyectos de Firebase.

\item {} 
Paso tres. Mantener las reglas para la base de datos por defecto.

\item {} 
Paso cuatro. Crear y seleccionar el nombre del directorio donde se encuentra la aplicación construida para producción.

\item {} 
Paso cinco. Seleccionar la configuración de una aplicación de una sola página.

\item {} 
Paso seis. Mantener el archivo \sphinxcode{\sphinxupquote{index.html}} generado por nuestro proceso de construcción.

\end{itemize}

Para usar la base de datos de Firebase y la autenticación en la aplicación,
hace falta importar la biblioteca de Firebase y configurarla con las llaves
de la aplicación de Firebase (se encuentran en la Consola de Firebase).
Dentro del proyecto creado se instala el paquete de Firebase con el comando
\sphinxcode{\sphinxupquote{npm install firebase}}. Ahora ya se pueden importar la biblioteca de Firebase
para configurarla.

A partir de ahora ya se hablará de los archivos que componen al proyecto y
que están en la carpeta \sphinxcode{\sphinxupquote{src}} generada por \sphinxcode{\sphinxupquote{creact-react-app}}.
El primero es \sphinxcode{\sphinxupquote{fire.js}}, que es el archivo de configuración para utilizar la
base de datos y la autenticación de Firebase.

En el archivo primero se importa firebase

\fvset{hllines={, ,}}%
\begin{sphinxVerbatim}[commandchars=\\\{\}]
\PYG{k+kr}{import} \PYG{o}{*} \PYG{n+nx}{as} \PYG{n+nx}{firebase} \PYG{n+nx}{from} \PYG{l+s+s1}{\PYGZsq{}firebase\PYGZsq{}}\PYG{p}{;}
\end{sphinxVerbatim}

Se define el objeto que define las opciones de configuración para la aplicación.
Contiene llaves, y el URL de la base de datos.

\fvset{hllines={, ,}}%
\begin{sphinxVerbatim}[commandchars=\\\{\}]
\PYG{k+kr}{const} \PYG{n+nx}{config} \PYG{o}{=} \PYG{p}{\PYGZob{}}
  \PYG{n+nx}{apiKey}\PYG{o}{:} \PYG{l+s+s2}{\PYGZdq{}API\PYGZus{}KEY\PYGZdq{}}\PYG{p}{,}
  \PYG{n+nx}{authDomain}\PYG{o}{:} \PYG{l+s+s2}{\PYGZdq{}nao\PYGZhy{}firebase\PYGZhy{}2a34d.firebaseapp.com\PYGZdq{}}\PYG{p}{,}
  \PYG{n+nx}{databaseURL}\PYG{o}{:} \PYG{l+s+s2}{\PYGZdq{}https://nao\PYGZhy{}firebase\PYGZhy{}2a34d.firebaseio.com\PYGZdq{}}\PYG{p}{,}
  \PYG{n+nx}{projectId}\PYG{o}{:} \PYG{l+s+s2}{\PYGZdq{}nao\PYGZhy{}firebase\PYGZhy{}2a34d\PYGZdq{}}\PYG{p}{,}
  \PYG{n+nx}{storageBucket}\PYG{o}{:} \PYG{l+s+s2}{\PYGZdq{}nao\PYGZhy{}firebase\PYGZhy{}2a34d.appspot.com\PYGZdq{}}\PYG{p}{,}
  \PYG{n+nx}{messagingSenderId}\PYG{o}{:} \PYG{l+s+s2}{\PYGZdq{}1111111111111111\PYGZdq{}}
\PYG{p}{\PYGZcb{}}\PYG{p}{;}
\end{sphinxVerbatim}

Finalmente se definen los objetos que se ocuparán en los otros archivos,
para eso se usa \sphinxcode{\sphinxupquote{export}}.

\fvset{hllines={, ,}}%
\begin{sphinxVerbatim}[commandchars=\\\{\}]
\PYG{k+kr}{export} \PYG{k+kr}{const} \PYG{n+nx}{fire} \PYG{o}{=} \PYG{n+nx}{firebase}\PYG{p}{.}\PYG{n+nx}{initializeApp}\PYG{p}{(}\PYG{n+nx}{config}\PYG{p}{)}\PYG{p}{;}
\PYG{k+kr}{export} \PYG{k+kr}{const} \PYG{n+nx}{db} \PYG{o}{=} \PYG{n+nx}{firebase}\PYG{p}{.}\PYG{n+nx}{database}\PYG{p}{(}\PYG{p}{)}\PYG{p}{;}
\PYG{k+kr}{export} \PYG{k+kr}{const} \PYG{n+nx}{auth} \PYG{o}{=} \PYG{n+nx}{firebase}\PYG{p}{.}\PYG{n+nx}{auth}\PYG{p}{(}\PYG{p}{)}\PYG{p}{;}
\end{sphinxVerbatim}

Con esto ya se puede utilizar la base de datos y la autenticación en cualquier
otro módulo dentro de la aplicación.


\paragraph{Descripción de los componentes}
\label{\detokenize{code_docs:descripcion-de-los-componentes}}
El código de la aplicación está en el directorio \sphinxcode{\sphinxupquote{src}} y éste tiene la siguiente
estructura:

\fvset{hllines={, ,}}%
\begin{sphinxVerbatim}[commandchars=\\\{\}]
src/
├── App.js
├── App.test.js
├── auth.js
├── components
│   ├── dashboard.js
│   ├── error\PYGZus{}msg.js
│   ├── landing\PYGZus{}page.js
│   ├── live\PYGZus{}image\PYGZus{}from\PYGZus{}robot.js
│   ├── log\PYGZus{}table.js
│   ├── navigation\PYGZus{}bar.js
│   ├── not\PYGZus{}found.js
│   ├── robot\PYGZus{}commands.js
│   ├── robot\PYGZus{}log.js
│   ├── robot\PYGZus{}selection.js
│   ├── robot\PYGZus{}status.js
│   ├── signin.js
│   ├── signout.js
│   ├── signup.js
│   ├── text\PYGZus{}to\PYGZus{}speech.js
│   ├── with\PYGZus{}authentication.js
│   └── with\PYGZus{}authorization.js
├── constants
│   └── routes.js
├── fire.js
├── index.css
├── index.js
└── registerServiceWorker.js
\end{sphinxVerbatim}


\subparagraph{App.js}
\label{\detokenize{code_docs:app-js}}
Se define el componente principal de la aplicación, que se renderizará
al iniciarla. Se define el elemento \sphinxcode{\sphinxupquote{Router}} de React-Router
que define las rutas de los componentes que se renderizarán de acuerdo a lo
elegido por el usuario.
\index{App() (función incorporada)}

\begin{fulllineitems}
\phantomsection\label{\detokenize{code_docs:App}}\pysiglinewithargsret{\sphinxbfcode{\sphinxupquote{App}}}{}{}
El componente principal de la aplicación, se define las rutas de la aplicación
y un menú donde el usuario selecciona un componente.

\end{fulllineitems}



\subparagraph{auth.js}
\label{\detokenize{code_docs:auth-js}}
Aquí se definen las funciones para el inicio de sesión, registro de un usuario
y el cierre de sesión.
\index{doCreateUserWithEmailAndPassword() (función incorporada)}

\begin{fulllineitems}
\phantomsection\label{\detokenize{code_docs:doCreateUserWithEmailAndPassword}}\pysiglinewithargsret{\sphinxbfcode{\sphinxupquote{doCreateUserWithEmailAndPassword}}}{}{}
Llama al métoodo  de Firebase para crear un usuario

\end{fulllineitems}

\index{doSignInWithEmailAndPassword() (función incorporada)}

\begin{fulllineitems}
\phantomsection\label{\detokenize{code_docs:doSignInWithEmailAndPassword}}\pysiglinewithargsret{\sphinxbfcode{\sphinxupquote{doSignInWithEmailAndPassword}}}{}{}
Llama al métoodo de Firebase para iniciar sesión

\end{fulllineitems}

\index{doSignOut() (función incorporada)}

\begin{fulllineitems}
\phantomsection\label{\detokenize{code_docs:doSignOut}}\pysiglinewithargsret{\sphinxbfcode{\sphinxupquote{doSignOut}}}{}{}
Llama al métoodo  de Firebase para cerrar la sesión del usuario

\end{fulllineitems}



\subparagraph{fire.js}
\label{\detokenize{code_docs:fire-js}}
Aquí está la configuración de Firebase para usar los servicios de la base de datos
y de la autenticación.
\index{config() (función incorporada)}

\begin{fulllineitems}
\phantomsection\label{\detokenize{code_docs:config}}\pysiglinewithargsret{\sphinxbfcode{\sphinxupquote{config}}}{}{}
El objeto que define las opciones de configuración para la aplicación.
Contiene llaves, y el URL de la base de datos.

\end{fulllineitems}

\index{fire() (función incorporada)}

\begin{fulllineitems}
\phantomsection\label{\detokenize{code_docs:fire}}\pysiglinewithargsret{\sphinxbfcode{\sphinxupquote{fire}}}{}{}~\begin{description}
\item[{Se inicializa la aplicación y con export se permite importar en otros archivos dentro del}] \leavevmode
proyecto.

\end{description}

\end{fulllineitems}

\index{db() (función incorporada)}

\begin{fulllineitems}
\phantomsection\label{\detokenize{code_docs:db}}\pysiglinewithargsret{\sphinxbfcode{\sphinxupquote{db}}}{}{}
Crea el objeto para utilizar la base de datos de Firebase

\end{fulllineitems}

\index{auth() (función incorporada)}

\begin{fulllineitems}
\phantomsection\label{\detokenize{code_docs:auth}}\pysiglinewithargsret{\sphinxbfcode{\sphinxupquote{auth}}}{}{}
Crea el objeto para utilizar el servicio de autenticación de Firebae

\end{fulllineitems}



\subparagraph{index.js}
\label{\detokenize{code_docs:index-js}}
En este archivo se llama al método render de ReactDOM para renderizar al
componente \sphinxcode{\sphinxupquote{App}}.


\subparagraph{components/}
\label{\detokenize{code_docs:components}}

\textbf{dashboard.js}
\label{\detokenize{code_docs:dashboard-js}}
Este es el componente que muestra el panel de interacción con el robot.
Como estados define al usuario actual, un id y el nombre de un robot, el robot
que se selecciona en la lista de robots.

En el método \sphinxcode{\sphinxupquote{render()}} incluye otros componentes, la lista de robots del
usuario, la cámara en vivo del robot seleccionado, sus logs, etc.

\fvset{hllines={, ,}}%
\begin{sphinxVerbatim}[commandchars=\\\{\}]
\PYG{c+cm}{/**}
\PYG{c+cm}{ * Esta clase representa al componente que muestra el panel de interacción con}
\PYG{c+cm}{ * el robot.}
\PYG{c+cm}{ */}

\PYG{k+kr}{class} \PYG{n+nx}{Dashboard} \PYG{k+kr}{extends} \PYG{n+nx}{Component} \PYG{p}{\PYGZob{}}

  \PYG{c+cm}{/**}
\PYG{c+cm}{   * El constructor de la clase para inicializar los estados del componente.}
\PYG{c+cm}{   */}
  \PYG{n+nx}{constructor}\PYG{p}{(}\PYG{n+nx}{props}\PYG{p}{)} \PYG{p}{\PYGZob{}}
    \PYG{k+kr}{super}\PYG{p}{(}\PYG{n+nx}{props}\PYG{p}{)}\PYG{p}{;}
    \PYG{k}{this}\PYG{p}{.}\PYG{n+nx}{state} \PYG{o}{=} \PYG{p}{\PYGZob{}}
      \PYG{n+nx}{currentUserId} \PYG{o}{:} \PYG{k+kc}{null}\PYG{p}{,}
      \PYG{n+nx}{robotId}\PYG{o}{:} \PYG{k+kc}{null}\PYG{p}{,}
      \PYG{n+nx}{robotName} \PYG{o}{:} \PYG{k+kc}{null}\PYG{p}{,}
    \PYG{p}{\PYGZcb{}}\PYG{p}{;}
  \PYG{p}{\PYGZcb{}}
  \PYG{c+cm}{/**}
\PYG{c+cm}{   * Una función callback para actualizar los estados que almacenas el id}
\PYG{c+cm}{   * y el nombre del robot que se seleccionó en el componente {}`{}`RobotList{}`{}`}
\PYG{c+cm}{   */}
  \PYG{n+nx}{setRobotId} \PYG{o}{=} \PYG{p}{(}\PYG{n+nx}{robotIdFromRobotList}\PYG{p}{,} \PYG{n+nx}{robotNameFromRobotList}\PYG{p}{)} \PYG{p}{=\PYGZgt{}} \PYG{p}{\PYGZob{}}
    \PYG{k}{this}\PYG{p}{.}\PYG{n+nx}{setState}\PYG{p}{(}\PYG{p}{\PYGZob{}}
      \PYG{n+nx}{robotId}\PYG{o}{:} \PYG{n+nx}{robotIdFromRobotList}\PYG{p}{,}
      \PYG{n+nx}{robotName} \PYG{o}{:} \PYG{n+nx}{robotNameFromRobotList}\PYG{p}{,}
    \PYG{p}{\PYGZcb{}}\PYG{p}{,} \PYG{p}{(}\PYG{p}{)} \PYG{p}{=\PYGZgt{}} \PYG{p}{\PYGZob{}}
      \PYG{n+nx}{console}\PYG{p}{.}\PYG{n+nx}{log}\PYG{p}{(}\PYG{l+s+s2}{\PYGZdq{}robotId set to \PYGZdq{}} \PYG{o}{+} \PYG{k}{this}\PYG{p}{.}\PYG{n+nx}{state}\PYG{p}{.}\PYG{n+nx}{robotId}\PYG{p}{)}\PYG{p}{;}
    \PYG{p}{\PYGZcb{}}\PYG{p}{)}\PYG{p}{;}
  \PYG{p}{\PYGZcb{}}

  \PYG{c+cm}{/**}
\PYG{c+cm}{   * Método del ciclo de vida del componente que se ejecuta después de montar}
\PYG{c+cm}{   * al componente. Se asigna el valor del usuario actual a un estado del componente.}
\PYG{c+cm}{   */}
  \PYG{n+nx}{componentWillMount}\PYG{p}{(}\PYG{p}{)} \PYG{p}{\PYGZob{}}
    \PYG{k+kr}{const} \PYG{n+nx}{user}\PYG{o}{=} \PYG{n+nx}{auth}\PYG{p}{.}\PYG{n+nx}{currentUser}\PYG{p}{;}
    \PYG{k}{this}\PYG{p}{.}\PYG{n+nx}{setState}\PYG{p}{(}\PYG{p}{\PYGZob{}}\PYG{n+nx}{currentUserId} \PYG{o}{:} \PYG{n+nx}{user}\PYG{p}{.}\PYG{n+nx}{uid}\PYG{p}{\PYGZcb{}}\PYG{p}{)}\PYG{p}{;}
  \PYG{p}{\PYGZcb{}}
  \PYG{c+cm}{/**}
\PYG{c+cm}{   * Renderiza los componentes que arman a este componente.}
\PYG{c+cm}{   */}
  \PYG{n+nx}{render}\PYG{p}{(}\PYG{p}{)} \PYG{p}{\PYGZob{}}
      \PYG{k}{return} \PYG{p}{(}\PYG{p}{...}\PYG{p}{)}\PYG{p}{;}
  \PYG{p}{\PYGZcb{}}
\PYG{p}{\PYGZcb{}}
\end{sphinxVerbatim}


\textbf{error\_msg.js}
\label{\detokenize{code_docs:error-msg-js}}
Un mensaje de error utilizado en el SignUp y SignIn del usuario, pero puede
adaptarse a cualquier contexto pues simplemente recibe el texto del error
generado.

\fvset{hllines={, ,}}%
\begin{sphinxVerbatim}[commandchars=\\\{\}]
\PYG{c+cm}{/**}
\PYG{c+cm}{ * Muestra una mensaje de error con un formato de Semantic UI}
\PYG{c+cm}{ */}
\PYG{k+kr}{const} \PYG{n+nx}{ErrorMessage} \PYG{o}{=} \PYG{p}{(}\PYG{n+nx}{props}\PYG{p}{)} \PYG{p}{=\PYGZgt{}} \PYG{p}{\PYGZob{}}
    \PYG{k}{return} \PYG{p}{(}\PYG{p}{...}\PYG{p}{)}\PYG{p}{;}
\PYG{p}{\PYGZcb{}}\PYG{p}{;}
\end{sphinxVerbatim}


\textbf{landing\_page.js}
\label{\detokenize{code_docs:landing-page-js}}
El componente que se muestra al abrir la aplicación, estando o no una sesión
activa.

\fvset{hllines={, ,}}%
\begin{sphinxVerbatim}[commandchars=\\\{\}]
\PYG{c+cm}{/**}
\PYG{c+cm}{ * El componente que se muestra al iniciar la aplicación, esté o no activa la sesión de un usuario.}
\PYG{c+cm}{ */}
\PYG{k+kr}{class} \PYG{n+nx}{Landing} \PYG{k+kr}{extends} \PYG{n+nx}{Component} \PYG{p}{\PYGZob{}}
  \PYG{n+nx}{render}\PYG{p}{(}\PYG{p}{)} \PYG{p}{\PYGZob{}}
    \PYG{k}{return} \PYG{p}{(}\PYG{p}{...}\PYG{p}{)}\PYG{p}{;}
  \PYG{p}{\PYGZcb{}}
\PYG{p}{\PYGZcb{}}
\end{sphinxVerbatim}


\textbf{live\_image\_from\_robot.js}
\label{\detokenize{code_docs:live-image-from-robot-js}}
Muestra la última imagen enviada por la cámara del robot. Tiene una imagen por
defecto, y espera una imagen codificada en base64 obtendida desde Firebase.

\fvset{hllines={, ,}}%
\begin{sphinxVerbatim}[commandchars=\\\{\}]
\PYG{c+cm}{/**}
\PYG{c+cm}{ * La clase del componente para obtener una imagen enviada a Firebase por el robot.}
\PYG{c+cm}{ * Utiliza Semantic UI para el contenedor de la imagen y una referencia a la base de datos}
\PYG{c+cm}{ * para suscribirse a la ubicación donde se recibe la imagen codificada en base 64.}
\PYG{c+cm}{ */}
\PYG{k+kr}{class} \PYG{n+nx}{LiveImage} \PYG{k+kr}{extends} \PYG{n+nx}{Component} \PYG{p}{\PYGZob{}}

  \PYG{c+cm}{/**}
\PYG{c+cm}{   * @constructor El constructor de la clase inicializa el estado que}
\PYG{c+cm}{   * contiene a la imagen recibida de Firebase.}
\PYG{c+cm}{   */}
  \PYG{n+nx}{constructor}\PYG{p}{(}\PYG{n+nx}{props}\PYG{p}{)} \PYG{p}{\PYGZob{}}
    \PYG{k+kr}{super}\PYG{p}{(}\PYG{n+nx}{props}\PYG{p}{)}\PYG{p}{;}
    \PYG{k}{this}\PYG{p}{.}\PYG{n+nx}{state} \PYG{o}{=} \PYG{p}{\PYGZob{}}
      \PYG{n+nx}{currentImage} \PYG{o}{:} \PYG{l+s+s2}{\PYGZdq{}https://semantic\PYGZhy{}ui.com/images/wireframe/image.png\PYGZdq{}}
    \PYG{p}{\PYGZcb{}}\PYG{p}{;}
  \PYG{p}{\PYGZcb{}}

  \PYG{c+cm}{/**}
\PYG{c+cm}{   * El método del ciclo de vida del componente que se ejecuta después de que}
\PYG{c+cm}{   * reciba propiedades del componente padre. La única propiedad por ahora}
\PYG{c+cm}{   * es el id del robot.}
\PYG{c+cm}{   */}
  \PYG{n+nx}{componentWillReceiveProps}\PYG{p}{(}\PYG{n+nx}{nextProps}\PYG{p}{)} \PYG{p}{\PYGZob{}}
    \PYG{c+c1}{//console.log(\PYGZdq{}live image :\PYGZdq{} + nextProps.robotId);}
    \PYG{k}{this}\PYG{p}{.}\PYG{n+nx}{setState}\PYG{p}{(}\PYG{p}{\PYGZob{}} \PYG{n+nx}{currentImage} \PYG{o}{:} \PYG{k+kc}{null} \PYG{p}{\PYGZcb{}}\PYG{p}{)}\PYG{p}{;}
    \PYG{k+kr}{const} \PYG{n+nx}{imageFromRobotRef} \PYG{o}{=} \PYG{n+nx}{db}\PYG{p}{.}\PYG{n+nx}{ref}\PYG{p}{(}\PYG{l+s+s1}{\PYGZsq{}liveImages/\PYGZsq{}} \PYG{o}{+} \PYG{n+nx}{nextProps}\PYG{p}{.}\PYG{n+nx}{robotId} \PYG{o}{+} \PYG{l+s+s1}{\PYGZsq{}/currentImage\PYGZsq{}}\PYG{p}{)}\PYG{p}{;}
    \PYG{n+nx}{imageFromRobotRef}\PYG{p}{.}\PYG{n+nx}{on}\PYG{p}{(}\PYG{l+s+s1}{\PYGZsq{}value\PYGZsq{}}\PYG{p}{,} \PYG{n+nx}{snap} \PYG{p}{=\PYGZgt{}} \PYG{p}{\PYGZob{}}
      \PYG{k}{this}\PYG{p}{.}\PYG{n+nx}{setState}\PYG{p}{(}\PYG{p}{\PYGZob{}}
        \PYG{n+nx}{currentImage} \PYG{o}{:} \PYG{n+nx}{snap}\PYG{p}{.}\PYG{n+nx}{val}\PYG{p}{(}\PYG{p}{)} \PYG{o}{?} \PYG{l+s+s2}{\PYGZdq{}data:image/png;base64, \PYGZdq{}} \PYG{o}{+} \PYG{n+nx}{snap}\PYG{p}{.}\PYG{n+nx}{val}\PYG{p}{(}\PYG{p}{)} \PYG{o}{:} \PYG{l+s+s2}{\PYGZdq{}https://semantic\PYGZhy{}ui.com/images/wireframe/image.png\PYGZdq{}}
      \PYG{p}{\PYGZcb{}}\PYG{p}{)}\PYG{p}{;}
    \PYG{p}{\PYGZcb{}}\PYG{p}{)}\PYG{p}{;}
  \PYG{p}{\PYGZcb{}}

  \PYG{c+cm}{/**}
\PYG{c+cm}{   * Muestra en un Image de Semantic UI, el valor del estado que contiene la imagen}
\PYG{c+cm}{   * enviada por el robot.}
\PYG{c+cm}{   */}
  \PYG{n+nx}{render}\PYG{p}{(}\PYG{p}{)} \PYG{p}{\PYGZob{}}
    \PYG{k}{return} \PYG{p}{(}\PYG{p}{...}\PYG{p}{)}\PYG{p}{;}
  \PYG{p}{\PYGZcb{}}
\PYG{p}{\PYGZcb{}}
\end{sphinxVerbatim}


\textbf{log\_table.js}
\label{\detokenize{code_docs:log-table-js}}
Un componente que genera una tabla a partir de un JSON. Dos columnas lo
componen, en la primera están las llaves y en la segunda los valores de los
atributos del objeto de javascript. Se utiliza para mostrar los logs adquiridos
de Firebase.

\fvset{hllines={, ,}}%
\begin{sphinxVerbatim}[commandchars=\\\{\}]
\PYG{c+cm}{/**}
\PYG{c+cm}{ * Este elemento llena la tabla donde se muestran los logs del robot.}
\PYG{c+cm}{ */}
\PYG{k+kr}{const} \PYG{n+nx}{TableFromObject} \PYG{o}{=} \PYG{p}{(}\PYG{n+nx}{props}\PYG{p}{)} \PYG{p}{=\PYGZgt{}} \PYG{p}{\PYGZob{}}

  \PYG{k+kr}{const} \PYG{n+nx}{tableItems} \PYG{o}{=} \PYG{n+nb}{Object}\PYG{p}{.}\PYG{n+nx}{keys}\PYG{p}{(}\PYG{n+nx}{props}\PYG{p}{.}\PYG{n+nx}{data}\PYG{p}{)}\PYG{p}{.}\PYG{n+nx}{map}\PYG{p}{(} \PYG{n+nx}{key} \PYG{p}{=\PYGZgt{}} \PYG{p}{\PYGZob{}}
      \PYG{k}{return} \PYG{p}{(}\PYG{p}{...}\PYG{p}{)}\PYG{p}{;}
  \PYG{p}{\PYGZcb{}}\PYG{p}{)}\PYG{p}{;}
  \PYG{k}{return} \PYG{p}{(}\PYG{p}{...}\PYG{p}{)}\PYG{p}{;}
\PYG{p}{\PYGZcb{}}\PYG{p}{;}
\end{sphinxVerbatim}


\textbf{navigation\_bar.js}
\label{\detokenize{code_docs:navigation-bar-js}}
Hay tres componentes dentro de este archivo, el menú de navegación cuando el
usuario no ha iniciado sesión y el cuando ha iniciado sesión. El tercer
componente muestra al menú que corresponda según el estado del usuario.

\fvset{hllines={, ,}}%
\begin{sphinxVerbatim}[commandchars=\\\{\}]
\PYG{c+cm}{/**}
\PYG{c+cm}{ * Es el menú que se muestra cuando el usuario ha iniciado sesión.}
\PYG{c+cm}{ */}
\PYG{k+kr}{const} \PYG{n+nx}{NavigationAuth} \PYG{o}{=} \PYG{p}{(}\PYG{p}{)} \PYG{p}{=\PYGZgt{}} \PYG{p}{\PYGZob{}}\PYG{p}{...}\PYG{p}{\PYGZcb{}}

\PYG{c+cm}{/**}
\PYG{c+cm}{ * Es el menú que se muestra cuando el usuario no ha iniciado sesión.}
\PYG{c+cm}{ */}
\PYG{k+kr}{const} \PYG{n+nx}{NavigationNonAuth} \PYG{o}{=} \PYG{p}{(}\PYG{p}{)} \PYG{p}{=\PYGZgt{}} \PYG{p}{\PYGZob{}}\PYG{p}{...}\PYG{p}{\PYGZcb{}}

\PYG{c+cm}{/**}
\PYG{c+cm}{ * Este componente es el menú superior en la aplicación, que permite cambiar}
\PYG{c+cm}{ * entre las rutas de la aplicación. Elige entre NavigationAuth o NavigationNonAuth}
\PYG{c+cm}{ * dependiendo si el usuario inició o no sesión.}
\PYG{c+cm}{ */}
\PYG{k+kr}{const} \PYG{n+nx}{NavigationBar} \PYG{o}{=} \PYG{p}{(}\PYG{n+nx}{props}\PYG{p}{,} \PYG{p}{\PYGZob{}} \PYG{n+nx}{authUser} \PYG{p}{\PYGZcb{}}\PYG{p}{)} \PYG{p}{=\PYGZgt{}}
    \PYG{p}{\PYGZob{}} \PYG{n+nx}{authUser} \PYG{o}{?} \PYG{o}{\PYGZlt{}}\PYG{n+nx}{NavigationAuth} \PYG{o}{/}\PYG{o}{\PYGZgt{}} \PYG{o}{:} \PYG{o}{\PYGZlt{}}\PYG{n+nx}{NavigationNonAuth} \PYG{o}{/}\PYG{o}{\PYGZgt{}}\PYG{p}{\PYGZcb{}}
\end{sphinxVerbatim}


\textbf{robot\_commands.js}
\label{\detokenize{code_docs:robot-commands-js}}
Este componente es el control remoto del robot, aquí se envía información a la
de Firebase para que el robot realiza acciones como caminar, cambiar de postura,
o mover la cabeza, también se pueden actualizar algunos parámetros de caminado.

\fvset{hllines={, ,}}%
\begin{sphinxVerbatim}[commandchars=\\\{\}]
\PYG{c+cm}{/**}
\PYG{c+cm}{ * El componente que envía comandos al robot a través de Firebase.}
\PYG{c+cm}{ */}
\PYG{k+kr}{class} \PYG{n+nx}{RobotCommands} \PYG{k+kr}{extends} \PYG{n+nx}{Component} \PYG{p}{\PYGZob{}}
  \PYG{c+cm}{/**}
\PYG{c+cm}{   * El constructor inicializa los estados que contendrán los valores que}
\PYG{c+cm}{   * se guardarán en Firebase y que el robot procesará.}
\PYG{c+cm}{   */}
  \PYG{n+nx}{constructor}\PYG{p}{(}\PYG{n+nx}{props}\PYG{p}{)} \PYG{p}{\PYGZob{}}
    \PYG{k+kr}{super}\PYG{p}{(}\PYG{n+nx}{props}\PYG{p}{)}\PYG{p}{;}
    \PYG{k}{this}\PYG{p}{.}\PYG{n+nx}{state} \PYG{o}{=} \PYG{n+nx}{INITIAL\PYGZus{}STATE}\PYG{p}{;}
    \PYG{k}{this}\PYG{p}{.}\PYG{n+nx}{moveRobot} \PYG{o}{=} \PYG{k}{this}\PYG{p}{.}\PYG{n+nx}{moveRobot}\PYG{p}{.}\PYG{n+nx}{bind}\PYG{p}{(}\PYG{k}{this}\PYG{p}{)}\PYG{p}{;}
    \PYG{k}{this}\PYG{p}{.}\PYG{n+nx}{stopRobot} \PYG{o}{=} \PYG{k}{this}\PYG{p}{.}\PYG{n+nx}{stopRobot}\PYG{p}{.}\PYG{n+nx}{bind}\PYG{p}{(}\PYG{k}{this}\PYG{p}{)}\PYG{p}{;}
    \PYG{k}{this}\PYG{p}{.}\PYG{n+nx}{handleJointRangeInput} \PYG{o}{=} \PYG{k}{this}\PYG{p}{.}\PYG{n+nx}{handleJointRangeInput}\PYG{p}{.}\PYG{n+nx}{bind}\PYG{p}{(}\PYG{k}{this}\PYG{p}{)}\PYG{p}{;}
    \PYG{k}{this}\PYG{p}{.}\PYG{n+nx}{changePosture} \PYG{o}{=} \PYG{k}{this}\PYG{p}{.}\PYG{n+nx}{changePosture}\PYG{p}{.}\PYG{n+nx}{bind}\PYG{p}{(}\PYG{k}{this}\PYG{p}{)}\PYG{p}{;}
    \PYG{k}{this}\PYG{p}{.}\PYG{n+nx}{setVelocity} \PYG{o}{=} \PYG{k}{this}\PYG{p}{.}\PYG{n+nx}{setVelocity}\PYG{p}{.}\PYG{n+nx}{bind}\PYG{p}{(}\PYG{k}{this}\PYG{p}{)}\PYG{p}{;}
    \PYG{k}{this}\PYG{p}{.}\PYG{n+nx}{onSubmitGaitParameters} \PYG{o}{=} \PYG{k}{this}\PYG{p}{.}\PYG{n+nx}{onSubmitGaitParameters}\PYG{p}{.}\PYG{n+nx}{bind}\PYG{p}{(}\PYG{k}{this}\PYG{p}{)}\PYG{p}{;}
  \PYG{p}{\PYGZcb{}}
  \PYG{c+cm}{/**}
\PYG{c+cm}{   * Maneja el evento de mantener el mouse presionado sobre un botón,}
\PYG{c+cm}{   * si este evento ocurre escribe en la ubicación de /walk, la dirección}
\PYG{c+cm}{   * que debe seguir el robot.}
\PYG{c+cm}{   */}
  \PYG{n+nx}{moveRobot}\PYG{p}{(}\PYG{n+nx}{param}\PYG{p}{,} \PYG{n+nx}{e}\PYG{p}{)} \PYG{p}{\PYGZob{}}
    \PYG{n+nx}{e}\PYG{p}{.}\PYG{n+nx}{preventDefault}\PYG{p}{(}\PYG{p}{)}\PYG{p}{;}
    \PYG{n+nx}{db}\PYG{p}{.}\PYG{n+nx}{ref}\PYG{p}{(}\PYG{l+s+s1}{\PYGZsq{}commands/\PYGZsq{}} \PYG{o}{+} \PYG{k}{this}\PYG{p}{.}\PYG{n+nx}{props}\PYG{p}{.}\PYG{n+nx}{robotId} \PYG{o}{+} \PYG{l+s+s1}{\PYGZsq{}/walk\PYGZsq{}}\PYG{p}{)}\PYG{p}{.}\PYG{n+nx}{set}\PYG{p}{(}\PYG{n+nx}{param}\PYG{p}{)}\PYG{p}{;}
    \PYG{n+nx}{console}\PYG{p}{.}\PYG{n+nx}{log}\PYG{p}{(}\PYG{l+s+s1}{\PYGZsq{}Parameter\PYGZsq{}}\PYG{p}{,} \PYG{n+nx}{param}\PYG{p}{)}\PYG{p}{;}
  \PYG{p}{\PYGZcb{}}

  \PYG{c+cm}{/**}
\PYG{c+cm}{  * Maneja el evento de soltar un botón, si eso ocurre escribe en la ubicación}
\PYG{c+cm}{  * /walk el valor de stop para que el robot detenga sus movimientos.}
\PYG{c+cm}{  */}
  \PYG{n+nx}{stopRobot}\PYG{p}{(}\PYG{n+nx}{e}\PYG{p}{)} \PYG{p}{\PYGZob{}}
    \PYG{n+nx}{e}\PYG{p}{.}\PYG{n+nx}{preventDefault}\PYG{p}{(}\PYG{p}{)}\PYG{p}{;}
    \PYG{n+nx}{db}\PYG{p}{.}\PYG{n+nx}{ref}\PYG{p}{(}\PYG{l+s+s1}{\PYGZsq{}commands/\PYGZsq{}} \PYG{o}{+} \PYG{k}{this}\PYG{p}{.}\PYG{n+nx}{props}\PYG{p}{.}\PYG{n+nx}{robotId} \PYG{o}{+} \PYG{l+s+s1}{\PYGZsq{}/walk\PYGZsq{}}\PYG{p}{)}\PYG{p}{.}\PYG{n+nx}{set}\PYG{p}{(}\PYG{l+s+s1}{\PYGZsq{}stop\PYGZsq{}}\PYG{p}{)}\PYG{p}{;}
    \PYG{n+nx}{console}\PYG{p}{.}\PYG{n+nx}{log}\PYG{p}{(}\PYG{l+s+s1}{\PYGZsq{}stopRobot\PYGZsq{}}\PYG{p}{)}\PYG{p}{;}
  \PYG{p}{\PYGZcb{}}
  \PYG{c+cm}{/**}
\PYG{c+cm}{   * Maneja los cambios en el input range, si este cambia, escribe los nuevos}
\PYG{c+cm}{   * valores en Firebase, donde corresponda a la articulación de la cabeza que}
\PYG{c+cm}{   * se seleccionó.}
\PYG{c+cm}{   */}
  \PYG{n+nx}{handleJointRangeInput}\PYG{p}{(}\PYG{n+nx}{jointName}\PYG{p}{,} \PYG{n+nx}{e}\PYG{p}{)} \PYG{p}{\PYGZob{}}
    \PYG{n+nx}{db}\PYG{p}{.}\PYG{n+nx}{ref}\PYG{p}{(}\PYG{l+s+s1}{\PYGZsq{}commands/\PYGZsq{}} \PYG{o}{+} \PYG{k}{this}\PYG{p}{.}\PYG{n+nx}{props}\PYG{p}{.}\PYG{n+nx}{robotId} \PYG{o}{+} \PYG{l+s+s1}{\PYGZsq{}/moveJoint/\PYGZsq{}} \PYG{o}{+} \PYG{n+nx}{jointName}\PYG{p}{)}\PYG{p}{.}\PYG{n+nx}{set}\PYG{p}{(}\PYG{n+nx}{e}\PYG{p}{.}\PYG{n+nx}{target}\PYG{p}{.}\PYG{n+nx}{value}\PYG{p}{)}\PYG{p}{;}
    \PYG{n+nx}{console}\PYG{p}{.}\PYG{n+nx}{log}\PYG{p}{(}\PYG{n+nx}{e}\PYG{p}{.}\PYG{n+nx}{target}\PYG{p}{.}\PYG{n+nx}{value}\PYG{p}{)}\PYG{p}{;}
  \PYG{p}{\PYGZcb{}}

  \PYG{c+cm}{/**}
\PYG{c+cm}{   * Maneja los eventos de los botones que cambian la postura del robot.}
\PYG{c+cm}{   */}
  \PYG{n+nx}{changePosture}\PYG{p}{(}\PYG{n+nx}{posture}\PYG{p}{,} \PYG{n+nx}{e}\PYG{p}{)} \PYG{p}{\PYGZob{}}
    \PYG{n+nx}{e}\PYG{p}{.}\PYG{n+nx}{preventDefault}\PYG{p}{(}\PYG{p}{)}
    \PYG{n+nx}{db}\PYG{p}{.}\PYG{n+nx}{ref}\PYG{p}{(}\PYG{l+s+s1}{\PYGZsq{}commands/\PYGZsq{}} \PYG{o}{+} \PYG{k}{this}\PYG{p}{.}\PYG{n+nx}{props}\PYG{p}{.}\PYG{n+nx}{robotId} \PYG{o}{+} \PYG{l+s+s1}{\PYGZsq{}/changePosture\PYGZsq{}}\PYG{p}{)}\PYG{p}{.}\PYG{n+nx}{set}\PYG{p}{(}\PYG{n+nx}{posture}\PYG{p}{)}\PYG{p}{;}
  \PYG{p}{\PYGZcb{}}
  \PYG{c+cm}{/**}
\PYG{c+cm}{   * Escribe en la base de datos el valor de la velocidad que se configuró}
\PYG{c+cm}{   * en el input range correspondiente.}
\PYG{c+cm}{   */}
  \PYG{n+nx}{setVelocity}\PYG{p}{(}\PYG{n+nx}{e}\PYG{p}{)} \PYG{p}{\PYGZob{}}
    \PYG{n+nx}{db}\PYG{p}{.}\PYG{n+nx}{ref}\PYG{p}{(}\PYG{l+s+s1}{\PYGZsq{}commands/\PYGZsq{}} \PYG{o}{+} \PYG{k}{this}\PYG{p}{.}\PYG{n+nx}{props}\PYG{p}{.}\PYG{n+nx}{robotId} \PYG{o}{+} \PYG{l+s+s1}{\PYGZsq{}/walkSpeed\PYGZsq{}}\PYG{p}{)}\PYG{p}{.}\PYG{n+nx}{set}\PYG{p}{(}\PYG{n+nx}{e}\PYG{p}{.}\PYG{n+nx}{target}\PYG{p}{.}\PYG{n+nx}{value}\PYG{p}{)}\PYG{p}{;}
  \PYG{p}{\PYGZcb{}}

  \PYG{c+cm}{/**}
\PYG{c+cm}{   * Maneja el submit que actualiza los parámetros de caminado del robot.}
\PYG{c+cm}{   */}
  \PYG{n+nx}{onSubmitGaitParameters} \PYG{o}{=} \PYG{p}{(}\PYG{n+nx}{e}\PYG{p}{)} \PYG{p}{=\PYGZgt{}} \PYG{p}{\PYGZob{}}
    \PYG{n+nx}{e}\PYG{p}{.}\PYG{n+nx}{preventDefault}\PYG{p}{(}\PYG{p}{)}\PYG{p}{;}
    \PYG{n+nx}{db}\PYG{p}{.}\PYG{n+nx}{ref}\PYG{p}{(}\PYG{l+s+s1}{\PYGZsq{}commands/\PYGZsq{}} \PYG{o}{+} \PYG{k}{this}\PYG{p}{.}\PYG{n+nx}{props}\PYG{p}{.}\PYG{n+nx}{robotId} \PYG{o}{+} \PYG{l+s+s1}{\PYGZsq{}/gaitParameters\PYGZsq{}}\PYG{p}{)}\PYG{p}{.}\PYG{n+nx}{update}\PYG{p}{(}\PYG{k}{this}\PYG{p}{.}\PYG{n+nx}{state}\PYG{p}{)}

  \PYG{p}{\PYGZcb{}}
  \PYG{c+cm}{/**}
\PYG{c+cm}{   * Métod del ciclo de vida que se llamará justo después de que se reciban}
\PYG{c+cm}{   * propiedades del componente padre. Inicializa los estados.}
\PYG{c+cm}{   */}
  \PYG{n+nx}{componentWillReceiveProps}\PYG{p}{(}\PYG{n+nx}{nextProps}\PYG{p}{)} \PYG{p}{\PYGZob{}}
    \PYG{n+nx}{console}\PYG{p}{.}\PYG{n+nx}{log}\PYG{p}{(}\PYG{l+s+s2}{\PYGZdq{}Robot commands \PYGZdq{}} \PYG{o}{+} \PYG{n+nx}{nextProps}\PYG{p}{.}\PYG{n+nx}{robotId}\PYG{p}{)}\PYG{p}{;}
    \PYG{k}{this}\PYG{p}{.}\PYG{n+nx}{setState}\PYG{p}{(}\PYG{n+nx}{INITIAL\PYGZus{}STATE}\PYG{p}{)}\PYG{p}{;}
  \PYG{p}{\PYGZcb{}}

  \PYG{c+cm}{/**}
\PYG{c+cm}{   * Renderiza al componente}
\PYG{c+cm}{   */}
  \PYG{n+nx}{render}\PYG{p}{(}\PYG{p}{)} \PYG{p}{\PYGZob{}}

    \PYG{k}{return} \PYG{p}{(}\PYG{p}{...}\PYG{p}{)}\PYG{p}{;}
  \PYG{p}{\PYGZcb{}}
\PYG{p}{\PYGZcb{}}
\end{sphinxVerbatim}


\textbf{robot\_log.js}
\label{\detokenize{code_docs:robot-log-js}}
Este componente renderiza la tabla de logs con la información que se obtiene de
Firebase, también permite descargar el historial completo de logs para cada
robot.

\fvset{hllines={, ,}}%
\begin{sphinxVerbatim}[commandchars=\\\{\}]
\PYG{c+cm}{/**}
\PYG{c+cm}{ * El componente que muestra una tabla con los últimos logs enviados por el robot.}
\PYG{c+cm}{ * También permite descargar el historial completo de todos lo registros almacenados.}
\PYG{c+cm}{ */}
\PYG{k+kr}{class} \PYG{n+nx}{RobotLogs} \PYG{k+kr}{extends} \PYG{n+nx}{Component} \PYG{p}{\PYGZob{}}
  \PYG{n+nx}{constructor}\PYG{p}{(}\PYG{n+nx}{props}\PYG{p}{)} \PYG{p}{\PYGZob{}}
    \PYG{k+kr}{super}\PYG{p}{(}\PYG{n+nx}{props}\PYG{p}{)}\PYG{p}{;}
    \PYG{k}{this}\PYG{p}{.}\PYG{n+nx}{state} \PYG{o}{=} \PYG{p}{\PYGZob{}}
      \PYG{n+nx}{valuesFromRobot} \PYG{o}{:} \PYG{p}{\PYGZob{}}\PYG{p}{\PYGZcb{}}\PYG{p}{,}
      \PYG{n+nx}{robotId} \PYG{o}{:} \PYG{k+kc}{null}\PYG{p}{,}
      \PYG{n+nx}{logsHistory} \PYG{o}{:} \PYG{k+kc}{null}\PYG{p}{,}
      \PYG{n+nx}{loadingJsonFile} \PYG{o}{:} \PYG{k+kc}{null}\PYG{p}{,}
    \PYG{p}{\PYGZcb{}}\PYG{p}{;}
    \PYG{k}{this}\PYG{p}{.}\PYG{n+nx}{prepareDowload} \PYG{o}{=} \PYG{k}{this}\PYG{p}{.}\PYG{n+nx}{prepareDowload}\PYG{p}{.}\PYG{n+nx}{bind}\PYG{p}{(}\PYG{k}{this}\PYG{p}{)}\PYG{p}{;}
  \PYG{p}{\PYGZcb{}}
  \PYG{c+cm}{/**}
\PYG{c+cm}{   * Método del ciclo de vida del componente que se ejecuta una vez que recibió}
\PYG{c+cm}{   * el id del robot. Se suscribe a la ubicación de los logs del robot seleccionado.}
\PYG{c+cm}{   */}
  \PYG{n+nx}{componentWillReceiveProps}\PYG{p}{(}\PYG{n+nx}{nextProps}\PYG{p}{)} \PYG{p}{\PYGZob{}}
    \PYG{n+nx}{console}\PYG{p}{.}\PYG{n+nx}{log}\PYG{p}{(}\PYG{l+s+s2}{\PYGZdq{}logs\PYGZdq{}} \PYG{o}{+} \PYG{n+nx}{nextProps}\PYG{p}{.}\PYG{n+nx}{robotId}\PYG{p}{)}\PYG{p}{;}
    \PYG{k}{this}\PYG{p}{.}\PYG{n+nx}{setState}\PYG{p}{(}\PYG{p}{\PYGZob{}}\PYG{n+nx}{valuesFromRobot} \PYG{o}{:} \PYG{p}{\PYGZob{}}\PYG{p}{\PYGZcb{}}\PYG{p}{,} \PYG{n+nx}{robotId} \PYG{o}{:} \PYG{n+nx}{nextProps}\PYG{p}{.}\PYG{n+nx}{robotId}\PYG{p}{,} \PYG{n+nx}{logsHistory} \PYG{o}{:} \PYG{k+kc}{null}\PYG{p}{\PYGZcb{}}\PYG{p}{)}\PYG{p}{;}
    \PYG{k+kr}{const} \PYG{n+nx}{myLogsRootRef} \PYG{o}{=} \PYG{n+nx}{db}\PYG{p}{.}\PYG{n+nx}{ref}\PYG{p}{(}\PYG{p}{)}\PYG{p}{.}\PYG{n+nx}{child}\PYG{p}{(}\PYG{l+s+s1}{\PYGZsq{}logs/\PYGZsq{}}\PYG{o}{+} \PYG{n+nx}{nextProps}\PYG{p}{.}\PYG{n+nx}{robotId}\PYG{p}{)}
                          \PYG{p}{.}\PYG{n+nx}{orderByKey}\PYG{p}{(}\PYG{p}{)}\PYG{p}{.}\PYG{n+nx}{limitToLast}\PYG{p}{(}\PYG{l+m+mi}{1}\PYG{p}{)}\PYG{p}{;}
    \PYG{n+nx}{myLogsRootRef}\PYG{p}{.}\PYG{n+nx}{on}\PYG{p}{(}\PYG{l+s+s1}{\PYGZsq{}child\PYGZus{}added\PYGZsq{}}\PYG{p}{,} \PYG{n+nx}{snap} \PYG{p}{=\PYGZgt{}} \PYG{p}{\PYGZob{}}
      \PYG{k}{this}\PYG{p}{.}\PYG{n+nx}{setState}\PYG{p}{(}\PYG{p}{\PYGZob{}}
        \PYG{n+nx}{valuesFromRobot} \PYG{o}{:} \PYG{n+nx}{snap}\PYG{p}{.}\PYG{n+nx}{val}\PYG{p}{(}\PYG{p}{)}
      \PYG{p}{\PYGZcb{}}\PYG{p}{)}\PYG{p}{;}
    \PYG{p}{\PYGZcb{}}\PYG{p}{)}\PYG{p}{;}
  \PYG{p}{\PYGZcb{}}
  \PYG{c+cm}{/**}
\PYG{c+cm}{   * Obtiene todo el historial de logs guardados en Firebase, y permite su descarga.}
\PYG{c+cm}{   */}
  \PYG{n+nx}{prepareDowload}\PYG{p}{(}\PYG{p}{)} \PYG{p}{\PYGZob{}}
    \PYG{k+kd}{let} \PYG{n+nx}{data} \PYG{o}{=} \PYG{l+s+s2}{\PYGZdq{}\PYGZdq{}}\PYG{p}{;}
    \PYG{k}{this}\PYG{p}{.}\PYG{n+nx}{setState}\PYG{p}{(}\PYG{p}{\PYGZob{}}\PYG{n+nx}{loadingJsonFile} \PYG{o}{:} \PYG{k+kc}{true}\PYG{p}{\PYGZcb{}}\PYG{p}{)}\PYG{p}{;}
    \PYG{n+nx}{db}\PYG{p}{.}\PYG{n+nx}{ref}\PYG{p}{(}\PYG{p}{)}\PYG{p}{.}\PYG{n+nx}{child}\PYG{p}{(}\PYG{l+s+s1}{\PYGZsq{}logs/\PYGZsq{}}\PYG{o}{+} \PYG{k}{this}\PYG{p}{.}\PYG{n+nx}{state}\PYG{p}{.}\PYG{n+nx}{robotId}\PYG{p}{)}\PYG{p}{.}\PYG{n+nx}{once}\PYG{p}{(}\PYG{l+s+s1}{\PYGZsq{}value\PYGZsq{}}\PYG{p}{)}\PYG{p}{.}\PYG{n+nx}{then}\PYG{p}{(}\PYG{p}{(}\PYG{n+nx}{snapshot}\PYG{p}{)} \PYG{p}{=\PYGZgt{}} \PYG{p}{\PYGZob{}}
      \PYG{n+nx}{data} \PYG{o}{=} \PYG{l+s+s2}{\PYGZdq{}data:text/json;charset=utf\PYGZhy{}8,\PYGZdq{}} \PYG{o}{+} \PYG{n+nb}{encodeURIComponent}\PYG{p}{(}\PYG{n+nx}{JSON}\PYG{p}{.}\PYG{n+nx}{stringify}\PYG{p}{(}\PYG{n+nx}{snapshot}\PYG{p}{.}\PYG{n+nx}{val}\PYG{p}{(}\PYG{p}{)}\PYG{p}{)}\PYG{p}{)}\PYG{p}{;}
      \PYG{n+nx}{console}\PYG{p}{.}\PYG{n+nx}{log}\PYG{p}{(}\PYG{l+s+s2}{\PYGZdq{}data\PYGZdq{}} \PYG{o}{+} \PYG{n+nx}{data}\PYG{p}{)}\PYG{p}{;}
      \PYG{k}{this}\PYG{p}{.}\PYG{n+nx}{setState}\PYG{p}{(}\PYG{p}{\PYGZob{}}\PYG{n+nx}{logsHistory} \PYG{o}{:} \PYG{n+nx}{data}\PYG{p}{,} \PYG{n+nx}{loadingJsonFile} \PYG{o}{:} \PYG{k+kc}{false}\PYG{p}{\PYGZcb{}}\PYG{p}{)}\PYG{p}{;}
    \PYG{p}{\PYGZcb{}}\PYG{p}{)}\PYG{p}{;}
  \PYG{p}{\PYGZcb{}}
  \PYG{c+cm}{/**}
\PYG{c+cm}{   * Renderiza el componente}
\PYG{c+cm}{   */}
  \PYG{n+nx}{render}\PYG{p}{(}\PYG{p}{)}\PYG{p}{\PYGZob{}}\PYG{p}{...}\PYG{p}{\PYGZcb{}}\PYG{p}{;}
  \PYG{p}{\PYGZcb{}}
\PYG{p}{\PYGZcb{}}
\end{sphinxVerbatim}


\textbf{robot\_selection.js}
\label{\detokenize{code_docs:robot-selection-js}}
Este componente es quien maneja la selección de un robot y actualiza al
\sphinxcode{\sphinxupquote{Dashboard}} para que envíe a sus hijos el id del robot seleccionado.

\fvset{hllines={, ,}}%
\begin{sphinxVerbatim}[commandchars=\\\{\}]
\PYG{k+kr}{class} \PYG{n+nx}{RobotList} \PYG{k+kr}{extends} \PYG{n+nx}{Component} \PYG{p}{\PYGZob{}}

  \PYG{n+nx}{robotsToList} \PYG{o}{=} \PYG{p}{(}\PYG{n+nx}{data}\PYG{p}{)} \PYG{p}{=\PYGZgt{}} \PYG{p}{\PYGZob{}}
    \PYG{k}{return} \PYG{n+nb}{Object}\PYG{p}{.}\PYG{n+nx}{keys}\PYG{p}{(}\PYG{n+nx}{data}\PYG{p}{)}\PYG{p}{.}\PYG{n+nx}{map}\PYG{p}{(}\PYG{n+nx}{key} \PYG{p}{=\PYGZgt{}} \PYG{p}{\PYGZob{}}
      \PYG{k}{return} \PYG{p}{(}\PYG{p}{...}\PYG{p}{)}\PYG{p}{;}
    \PYG{p}{\PYGZcb{}}\PYG{p}{)}\PYG{p}{;}
  \PYG{p}{\PYGZcb{}}

  \PYG{n+nx}{constructor}\PYG{p}{(}\PYG{n+nx}{props}\PYG{p}{)} \PYG{p}{\PYGZob{}}
    \PYG{k+kr}{super}\PYG{p}{(}\PYG{n+nx}{props}\PYG{p}{)}\PYG{p}{;}
    \PYG{k}{this}\PYG{p}{.}\PYG{n+nx}{state} \PYG{o}{=} \PYG{p}{\PYGZob{}}
      \PYG{n+nx}{newRobot} \PYG{o}{:} \PYG{l+s+s1}{\PYGZsq{}\PYGZsq{}}\PYG{p}{,}
      \PYG{n+nx}{listOfRobots}\PYG{o}{:} \PYG{p}{\PYGZob{}}\PYG{p}{\PYGZcb{}}\PYG{p}{,}
      \PYG{n+nx}{currentRobot} \PYG{o}{:} \PYG{k+kc}{null}\PYG{p}{,}
    \PYG{p}{\PYGZcb{}}\PYG{p}{;}
    \PYG{k}{this}\PYG{p}{.}\PYG{n+nx}{robotSelectionHandler} \PYG{o}{=} \PYG{k}{this}\PYG{p}{.}\PYG{n+nx}{robotSelectionHandler}\PYG{p}{.}\PYG{n+nx}{bind}\PYG{p}{(}\PYG{k}{this}\PYG{p}{)}\PYG{p}{;}
    \PYG{k}{this}\PYG{p}{.}\PYG{n+nx}{addNewRobot} \PYG{o}{=} \PYG{k}{this}\PYG{p}{.}\PYG{n+nx}{addNewRobot}\PYG{p}{.}\PYG{n+nx}{bind}\PYG{p}{(}\PYG{k}{this}\PYG{p}{)}\PYG{p}{;}
  \PYG{p}{\PYGZcb{}}

  \PYG{n+nx}{robotSelectionHandler} \PYG{o}{=} \PYG{p}{(}\PYG{n+nx}{robotKey}\PYG{p}{,} \PYG{n+nx}{robotName}\PYG{p}{,} \PYG{n+nx}{e}\PYG{p}{)} \PYG{p}{=\PYGZgt{}} \PYG{p}{\PYGZob{}}
    \PYG{n+nx}{e}\PYG{p}{.}\PYG{n+nx}{preventDefault}\PYG{p}{(}\PYG{p}{)}\PYG{p}{;}
    \PYG{k}{this}\PYG{p}{.}\PYG{n+nx}{setState}\PYG{p}{(}\PYG{p}{\PYGZob{}}\PYG{n+nx}{currentRobot} \PYG{o}{:} \PYG{n+nx}{robotName}\PYG{p}{\PYGZcb{}}\PYG{p}{)}\PYG{p}{;}
    \PYG{k}{this}\PYG{p}{.}\PYG{n+nx}{props}\PYG{p}{.}\PYG{n+nx}{robotSelection}\PYG{p}{(}\PYG{n+nx}{robotKey}\PYG{p}{,} \PYG{n+nx}{robotName}\PYG{p}{)}\PYG{p}{;}
  \PYG{p}{\PYGZcb{}}

  \PYG{n+nx}{addNewRobot} \PYG{o}{=} \PYG{p}{(}\PYG{n+nx}{e}\PYG{p}{)} \PYG{p}{=\PYGZgt{}} \PYG{p}{\PYGZob{}}
    \PYG{n+nx}{e}\PYG{p}{.}\PYG{n+nx}{preventDefault}\PYG{p}{(}\PYG{p}{)}\PYG{p}{;}
    \PYG{k}{if} \PYG{p}{(}\PYG{k}{this}\PYG{p}{.}\PYG{n+nx}{state}\PYG{p}{.}\PYG{n+nx}{newRobot} \PYG{o}{\PYGZam{}\PYGZam{}} \PYG{n+nx}{auth}\PYG{p}{.}\PYG{n+nx}{currentUser}\PYG{p}{)} \PYG{p}{\PYGZob{}}
      \PYG{k+kr}{const} \PYG{n+nx}{robotName} \PYG{o}{=} \PYG{k}{this}\PYG{p}{.}\PYG{n+nx}{state}\PYG{p}{.}\PYG{n+nx}{newRobot}\PYG{p}{;}
      \PYG{n+nx}{db}\PYG{p}{.}\PYG{n+nx}{ref}\PYG{p}{(}\PYG{l+s+s1}{\PYGZsq{}robots\PYGZsq{}}\PYG{p}{)}\PYG{p}{.}\PYG{n+nx}{push}\PYG{p}{(}\PYG{p}{\PYGZob{}} \PYG{n+nx}{name} \PYG{o}{:} \PYG{n+nx}{robotName}\PYG{p}{\PYGZcb{}}\PYG{p}{)}
        \PYG{p}{.}\PYG{n+nx}{then}\PYG{p}{(}\PYG{p}{(}\PYG{n+nx}{snap}\PYG{p}{)} \PYG{p}{=\PYGZgt{}} \PYG{p}{\PYGZob{}}
          \PYG{k+kr}{const} \PYG{n+nx}{newRobotkey} \PYG{o}{=} \PYG{n+nx}{snap}\PYG{p}{.}\PYG{n+nx}{key}\PYG{p}{;}
          \PYG{n+nx}{db}\PYG{p}{.}\PYG{n+nx}{ref}\PYG{p}{(}\PYG{l+s+s1}{\PYGZsq{}users/\PYGZsq{}} \PYG{o}{+} \PYG{n+nx}{auth}\PYG{p}{.}\PYG{n+nx}{currentUser}\PYG{p}{.}\PYG{n+nx}{uid} \PYG{o}{+} \PYG{l+s+s1}{\PYGZsq{}/robots/\PYGZsq{}} \PYG{o}{+} \PYG{n+nx}{newRobotkey}\PYG{p}{)}
            \PYG{p}{.}\PYG{n+nx}{set}\PYG{p}{(}\PYG{n+nx}{robotName}\PYG{p}{)}\PYG{p}{;}
      \PYG{p}{\PYGZcb{}}\PYG{p}{)}\PYG{p}{;}
    \PYG{p}{\PYGZcb{}}
    \PYG{k}{this}\PYG{p}{.}\PYG{n+nx}{setState}\PYG{p}{(}\PYG{p}{\PYGZob{}} \PYG{n+nx}{newRobot} \PYG{o}{:} \PYG{l+s+s1}{\PYGZsq{}\PYGZsq{}} \PYG{p}{\PYGZcb{}}\PYG{p}{)}
  \PYG{p}{\PYGZcb{}}

  \PYG{n+nx}{componentDidMount}\PYG{p}{(}\PYG{p}{)} \PYG{p}{\PYGZob{}}
    \PYG{n+nx}{console}\PYG{p}{.}\PYG{n+nx}{log}\PYG{p}{(}\PYG{l+s+s2}{\PYGZdq{}robot sele \PYGZdq{}} \PYG{o}{+} \PYG{k}{this}\PYG{p}{.}\PYG{n+nx}{props}\PYG{p}{.}\PYG{n+nx}{robotId}\PYG{p}{)}\PYG{p}{;}
    \PYG{n+nx}{db}\PYG{p}{.}\PYG{n+nx}{ref}\PYG{p}{(}\PYG{l+s+s1}{\PYGZsq{}users/\PYGZsq{}} \PYG{o}{+} \PYG{n+nx}{auth}\PYG{p}{.}\PYG{n+nx}{currentUser}\PYG{p}{.}\PYG{n+nx}{uid} \PYG{o}{+} \PYG{l+s+s1}{\PYGZsq{}/robots\PYGZsq{}}\PYG{p}{)}
      \PYG{p}{.}\PYG{n+nx}{on}\PYG{p}{(}\PYG{l+s+s1}{\PYGZsq{}value\PYGZsq{}}\PYG{p}{,} \PYG{n+nx}{snap} \PYG{p}{=\PYGZgt{}} \PYG{p}{\PYGZob{}}
        \PYG{k}{this}\PYG{p}{.}\PYG{n+nx}{setState}\PYG{p}{(}\PYG{p}{\PYGZob{}} \PYG{n+nx}{listOfRobots} \PYG{o}{:} \PYG{n+nx}{snap}\PYG{p}{.}\PYG{n+nx}{val}\PYG{p}{(}\PYG{p}{)} \PYG{p}{\PYGZcb{}}\PYG{p}{)}
      \PYG{p}{\PYGZcb{}}
    \PYG{p}{)}\PYG{p}{;}
  \PYG{p}{\PYGZcb{}}

  \PYG{n+nx}{render}\PYG{p}{(}\PYG{p}{)} \PYG{p}{\PYGZob{}}\PYG{p}{...}\PYG{p}{\PYGZcb{}}
\PYG{p}{\PYGZcb{}}
\end{sphinxVerbatim}


\textbf{signin.js}
\label{\detokenize{code_docs:signin-js}}
Este componente maneja el inicio de sesión de un usuario, utiliza al
objeto \sphinxcode{\sphinxupquote{auth}} para que el usuario utilice su dirección de correo y una
contraseña.

\fvset{hllines={, ,}}%
\begin{sphinxVerbatim}[commandchars=\\\{\}]
\PYG{c+cm}{/**}
\PYG{c+cm}{ * El componente que se encarga del inicio de sesión de un usuario.}
\PYG{c+cm}{ */}
\PYG{k+kr}{class} \PYG{n+nx}{SignIn} \PYG{k+kr}{extends} \PYG{n+nx}{Component} \PYG{p}{\PYGZob{}}

  \PYG{n+nx}{constructor}\PYG{p}{(}\PYG{n+nx}{props}\PYG{p}{)} \PYG{p}{\PYGZob{}}
    \PYG{k+kr}{super}\PYG{p}{(}\PYG{n+nx}{props}\PYG{p}{)}\PYG{p}{;}
    \PYG{k}{this}\PYG{p}{.}\PYG{n+nx}{state} \PYG{o}{=} \PYG{p}{\PYGZob{}}
      \PYG{n+nx}{INITIAL\PYGZus{}STATE}
    \PYG{p}{\PYGZcb{}}
  \PYG{p}{\PYGZcb{}}
  \PYG{c+cm}{/**}
\PYG{c+cm}{   * La función que se ejecuta al subir el formulario con el nombre de usuario}
\PYG{c+cm}{   * y la contraseña de este.}
\PYG{c+cm}{   */}
  \PYG{n+nx}{onSubmit} \PYG{o}{=} \PYG{p}{(}\PYG{n+nx}{event}\PYG{p}{)} \PYG{p}{=\PYGZgt{}} \PYG{p}{\PYGZob{}}
    \PYG{n+nx}{event}\PYG{p}{.}\PYG{n+nx}{preventDefault}\PYG{p}{(}\PYG{p}{)}\PYG{p}{;}
    \PYG{k+kr}{const} \PYG{n+nx}{history} \PYG{o}{=} \PYG{k}{this}\PYG{p}{.}\PYG{n+nx}{props}\PYG{p}{.}\PYG{n+nx}{history}\PYG{p}{;}

    \PYG{n+nx}{auth}\PYG{p}{.}\PYG{n+nx}{doSignInWithEmailAndPassword}\PYG{p}{(}\PYG{k}{this}\PYG{p}{.}\PYG{n+nx}{state}\PYG{p}{.}\PYG{n+nx}{email}\PYG{p}{,} \PYG{k}{this}\PYG{p}{.}\PYG{n+nx}{state}\PYG{p}{.}\PYG{n+nx}{password}\PYG{p}{)}
      \PYG{p}{.}\PYG{n+nx}{then}\PYG{p}{(}\PYG{p}{(}\PYG{p}{)} \PYG{p}{=\PYGZgt{}} \PYG{p}{\PYGZob{}}
        \PYG{k}{this}\PYG{p}{.}\PYG{n+nx}{setState}\PYG{p}{(}\PYG{p}{(}\PYG{p}{)} \PYG{p}{=\PYGZgt{}} \PYG{p}{(}\PYG{p}{\PYGZob{}} \PYG{n+nx}{INITIAL\PYGZus{}STATE} \PYG{p}{\PYGZcb{}}\PYG{p}{)}\PYG{p}{)}\PYG{p}{;}
        \PYG{n+nx}{history}\PYG{p}{.}\PYG{n+nx}{push}\PYG{p}{(}\PYG{n+nx}{routes}\PYG{p}{.}\PYG{n+nx}{ROBOTS}\PYG{p}{)}\PYG{p}{;}
      \PYG{p}{\PYGZcb{}}\PYG{p}{)}
      \PYG{p}{.}\PYG{k}{catch}\PYG{p}{(}\PYG{n+nx}{error} \PYG{p}{=\PYGZgt{}} \PYG{p}{\PYGZob{}}
        \PYG{k}{this}\PYG{p}{.}\PYG{n+nx}{setState}\PYG{p}{(}\PYG{p}{\PYGZob{}} \PYG{n+nx}{error} \PYG{o}{:} \PYG{n+nx}{error} \PYG{p}{\PYGZcb{}}\PYG{p}{)}\PYG{p}{;}
      \PYG{p}{\PYGZcb{}}\PYG{p}{)}\PYG{p}{;}
  \PYG{p}{\PYGZcb{}}
  \PYG{c+cm}{/**}
\PYG{c+cm}{   * Renderiza al componente}
\PYG{c+cm}{   */}
  \PYG{n+nx}{render}\PYG{p}{(}\PYG{p}{)} \PYG{p}{\PYGZob{}}
    \PYG{k}{return} \PYG{p}{(}\PYG{p}{...}\PYG{p}{)}\PYG{p}{;}
  \PYG{p}{\PYGZcb{}}
\PYG{p}{\PYGZcb{}}
\end{sphinxVerbatim}


\textbf{signout.js}
\label{\detokenize{code_docs:signout-js}}
Es un componente dentro del menú de navegación para cerrar la sesión del usuario.

\fvset{hllines={, ,}}%
\begin{sphinxVerbatim}[commandchars=\\\{\}]
\PYG{c+cm}{/**}
\PYG{c+cm}{ * El componente en el menú para cerrar la sesión del usuario.}
\PYG{c+cm}{ */}
\PYG{k+kr}{const} \PYG{n+nx}{SignOutMenuItem} \PYG{o}{=} \PYG{p}{(}\PYG{p}{)} \PYG{p}{=\PYGZgt{}}
  \PYG{p}{\PYGZob{}}\PYG{p}{...}\PYG{p}{\PYGZcb{}}
\end{sphinxVerbatim}


\textbf{signup.js}
\label{\detokenize{code_docs:signup-js}}
El componente donde un nuevo usuario se registra, utiliza las funciones
de autenticación de Firebase y además almacena al nuevo usuario a la base de
datos.

\fvset{hllines={, ,}}%
\begin{sphinxVerbatim}[commandchars=\\\{\}]
\PYG{c+cm}{/**}
\PYG{c+cm}{ * Componente cuya función es el registro de un nuevo usuario.}
\PYG{c+cm}{ */}
\PYG{k+kr}{class} \PYG{n+nx}{SignUp} \PYG{k+kr}{extends} \PYG{n+nx}{Component} \PYG{p}{\PYGZob{}}

  \PYG{n+nx}{constructor}\PYG{p}{(}\PYG{n+nx}{props}\PYG{p}{)} \PYG{p}{\PYGZob{}}
    \PYG{k+kr}{super}\PYG{p}{(}\PYG{n+nx}{props}\PYG{p}{)}\PYG{p}{;}
    \PYG{k}{this}\PYG{p}{.}\PYG{n+nx}{state} \PYG{o}{=} \PYG{p}{\PYGZob{}}
      \PYG{n+nx}{INITIAL\PYGZus{}STATE}
    \PYG{p}{\PYGZcb{}}
  \PYG{p}{\PYGZcb{}}
  \PYG{c+cm}{/**}
\PYG{c+cm}{   * Verifica que las dos contraseñas que se solicitan no estén vacías y}
\PYG{c+cm}{   * sean la misma cadena.}
\PYG{c+cm}{   */}
  \PYG{n+nx}{isValid}\PYG{p}{(}\PYG{p}{)} \PYG{p}{\PYGZob{}}
    \PYG{k}{if} \PYG{p}{(}\PYG{k}{this}\PYG{p}{.}\PYG{n+nx}{state}\PYG{p}{.}\PYG{n+nx}{passwordOne} \PYG{o}{\PYGZam{}\PYGZam{}} \PYG{k}{this}\PYG{p}{.}\PYG{n+nx}{state}\PYG{p}{.}\PYG{n+nx}{passwordTwo}\PYG{p}{)} \PYG{p}{\PYGZob{}}
      \PYG{k}{if} \PYG{p}{(}\PYG{k}{this}\PYG{p}{.}\PYG{n+nx}{state}\PYG{p}{.}\PYG{n+nx}{passwordTwo} \PYG{o}{!==} \PYG{k}{this}\PYG{p}{.}\PYG{n+nx}{state}\PYG{p}{.}\PYG{n+nx}{passwordOne}\PYG{p}{)} \PYG{p}{\PYGZob{}}
        \PYG{k}{this}\PYG{p}{.}\PYG{n+nx}{setState}\PYG{p}{(}\PYG{p}{\PYGZob{}}\PYG{n+nx}{error} \PYG{o}{:} \PYG{p}{\PYGZob{}}\PYG{n+nx}{message} \PYG{o}{:} \PYG{l+s+s2}{\PYGZdq{}Passwords must match\PYGZdq{}}\PYG{p}{\PYGZcb{}}\PYG{p}{\PYGZcb{}}\PYG{p}{)}\PYG{p}{;}
        \PYG{k}{return} \PYG{k+kc}{false}\PYG{p}{;}
      \PYG{p}{\PYGZcb{}}
      \PYG{k}{return} \PYG{k+kc}{true}\PYG{p}{;}
    \PYG{p}{\PYGZcb{}}
    \PYG{k}{this}\PYG{p}{.}\PYG{n+nx}{setState}\PYG{p}{(}\PYG{p}{\PYGZob{}}\PYG{n+nx}{error} \PYG{o}{:} \PYG{p}{\PYGZob{}}\PYG{n+nx}{message} \PYG{o}{:} \PYG{l+s+s2}{\PYGZdq{}Invalid password\PYGZdq{}}\PYG{p}{\PYGZcb{}}\PYG{p}{\PYGZcb{}}\PYG{p}{)}\PYG{p}{;}
    \PYG{k}{return} \PYG{k+kc}{false}\PYG{p}{;}
  \PYG{p}{\PYGZcb{}}
  \PYG{c+cm}{/**}
\PYG{c+cm}{   * Método que maneja al evento de subir el formulario de registro}
\PYG{c+cm}{   */}
  \PYG{n+nx}{onSubmit} \PYG{o}{=} \PYG{p}{(}\PYG{n+nx}{event}\PYG{p}{)} \PYG{p}{=\PYGZgt{}} \PYG{p}{\PYGZob{}}
    \PYG{n+nx}{event}\PYG{p}{.}\PYG{n+nx}{preventDefault}\PYG{p}{(}\PYG{p}{)}\PYG{p}{;}
    \PYG{k+kr}{const} \PYG{n+nx}{history} \PYG{o}{=} \PYG{k}{this}\PYG{p}{.}\PYG{n+nx}{props}\PYG{p}{.}\PYG{n+nx}{history}\PYG{p}{;}
    \PYG{k+kr}{const} \PYG{n+nx}{email} \PYG{o}{=} \PYG{k}{this}\PYG{p}{.}\PYG{n+nx}{state}\PYG{p}{.}\PYG{n+nx}{email}\PYG{p}{;}
    \PYG{k}{if} \PYG{p}{(}\PYG{o}{!}\PYG{k}{this}\PYG{p}{.}\PYG{n+nx}{isValid}\PYG{p}{(}\PYG{p}{)}\PYG{p}{)} \PYG{p}{\PYGZob{}}
      \PYG{k}{return}\PYG{p}{;}
    \PYG{p}{\PYGZcb{}}

    \PYG{n+nx}{auth}\PYG{p}{.}\PYG{n+nx}{doCreateUserWithEmailAndPassword}\PYG{p}{(}\PYG{k}{this}\PYG{p}{.}\PYG{n+nx}{state}\PYG{p}{.}\PYG{n+nx}{email}\PYG{p}{,} \PYG{k}{this}\PYG{p}{.}\PYG{n+nx}{state}\PYG{p}{.}\PYG{n+nx}{passwordOne}\PYG{p}{)}
      \PYG{p}{.}\PYG{n+nx}{then}\PYG{p}{(}\PYG{n+nx}{authUser} \PYG{p}{=\PYGZgt{}} \PYG{p}{\PYGZob{}}
        \PYG{n+nx}{db}\PYG{p}{.}\PYG{n+nx}{ref}\PYG{p}{(}\PYG{l+s+s1}{\PYGZsq{}users/\PYGZsq{}} \PYG{o}{+} \PYG{n+nx}{authUser}\PYG{p}{.}\PYG{n+nx}{uid}\PYG{p}{)}\PYG{p}{.}\PYG{n+nx}{set}\PYG{p}{(}\PYG{p}{\PYGZob{}}\PYG{n+nx}{email} \PYG{o}{:} \PYG{n+nx}{email}\PYG{p}{\PYGZcb{}}\PYG{p}{)}
          \PYG{p}{.}\PYG{n+nx}{then}\PYG{p}{(}\PYG{p}{(}\PYG{p}{)} \PYG{p}{=\PYGZgt{}} \PYG{p}{\PYGZob{}}
            \PYG{k}{this}\PYG{p}{.}\PYG{n+nx}{setState}\PYG{p}{(}\PYG{p}{(}\PYG{p}{)} \PYG{p}{=\PYGZgt{}} \PYG{p}{(}\PYG{p}{\PYGZob{}} \PYG{n+nx}{INITIAL\PYGZus{}STATE} \PYG{p}{\PYGZcb{}}\PYG{p}{)}\PYG{p}{)}\PYG{p}{;}
            \PYG{n+nx}{history}\PYG{p}{.}\PYG{n+nx}{push}\PYG{p}{(}\PYG{n+nx}{routes}\PYG{p}{.}\PYG{n+nx}{ROBOTS}\PYG{p}{)}\PYG{p}{;}
          \PYG{p}{\PYGZcb{}}\PYG{p}{)}
          \PYG{p}{.}\PYG{k}{catch}\PYG{p}{(}\PYG{n+nx}{error} \PYG{p}{=\PYGZgt{}} \PYG{p}{\PYGZob{}}
            \PYG{k}{this}\PYG{p}{.}\PYG{n+nx}{setState}\PYG{p}{(}\PYG{p}{\PYGZob{}} \PYG{n+nx}{error} \PYG{o}{:} \PYG{n+nx}{error} \PYG{p}{\PYGZcb{}}\PYG{p}{)}\PYG{p}{;}
          \PYG{p}{\PYGZcb{}}\PYG{p}{)}
      \PYG{p}{\PYGZcb{}}\PYG{p}{)}
      \PYG{p}{.}\PYG{k}{catch}\PYG{p}{(}\PYG{n+nx}{error} \PYG{p}{=\PYGZgt{}} \PYG{p}{\PYGZob{}}
        \PYG{n+nx}{console}\PYG{p}{.}\PYG{n+nx}{log}\PYG{p}{(}\PYG{n+nx}{error}\PYG{p}{)}\PYG{p}{;}
        \PYG{k}{this}\PYG{p}{.}\PYG{n+nx}{setState}\PYG{p}{(}\PYG{p}{\PYGZob{}} \PYG{n+nx}{error} \PYG{o}{:} \PYG{n+nx}{error} \PYG{p}{\PYGZcb{}}\PYG{p}{)}\PYG{p}{;}
      \PYG{p}{\PYGZcb{}}\PYG{p}{)}\PYG{p}{;}

  \PYG{p}{\PYGZcb{}}
  \PYG{c+cm}{/**}
\PYG{c+cm}{   * Renderiza al componente}
\PYG{c+cm}{   */}
  \PYG{n+nx}{render}\PYG{p}{(}\PYG{p}{)} \PYG{p}{\PYGZob{}}
    \PYG{k}{return} \PYG{p}{(}\PYG{p}{...}\PYG{p}{)}\PYG{p}{;}
  \PYG{p}{\PYGZcb{}}
\PYG{p}{\PYGZcb{}}
\end{sphinxVerbatim}


\textbf{text\_to\_speech.js}
\label{\detokenize{code_docs:text-to-speech-js}}
Este componente envía una cadena de texto a Firebase para que el robot la
replique oralmente.

\fvset{hllines={, ,}}%
\begin{sphinxVerbatim}[commandchars=\\\{\}]
\PYG{c+cm}{/**}
\PYG{c+cm}{ * Componente que envía el texto que debe repetir el robot.}
\PYG{c+cm}{ */}
\PYG{k+kr}{class} \PYG{n+nx}{TextToSpeech} \PYG{k+kr}{extends} \PYG{n+nx}{Component} \PYG{p}{\PYGZob{}}

  \PYG{n+nx}{constructor}\PYG{p}{(}\PYG{n+nx}{props}\PYG{p}{)} \PYG{p}{\PYGZob{}}
    \PYG{k+kr}{super}\PYG{p}{(}\PYG{n+nx}{props}\PYG{p}{)}\PYG{p}{;}
    \PYG{k}{this}\PYG{p}{.}\PYG{n+nx}{state} \PYG{o}{=} \PYG{p}{\PYGZob{}}
      \PYG{n+nx}{robotId} \PYG{o}{:} \PYG{k+kc}{null}\PYG{p}{,}
    \PYG{p}{\PYGZcb{}}\PYG{p}{;}
    \PYG{k}{this}\PYG{p}{.}\PYG{n+nx}{sendSpeech} \PYG{o}{=} \PYG{k}{this}\PYG{p}{.}\PYG{n+nx}{sendSpeech}\PYG{p}{.}\PYG{n+nx}{bind}\PYG{p}{(}\PYG{k}{this}\PYG{p}{)}\PYG{p}{;}
  \PYG{p}{\PYGZcb{}}
  \PYG{c+cm}{/**}
\PYG{c+cm}{   * Maneja el evento de que el usuario escriba en el input el texto que el robot}
\PYG{c+cm}{   * debe decir.}
\PYG{c+cm}{   */}
  \PYG{n+nx}{sendSpeech}\PYG{p}{(}\PYG{n+nx}{e}\PYG{p}{)} \PYG{p}{\PYGZob{}}
    \PYG{k}{if} \PYG{p}{(}\PYG{n+nx}{e}\PYG{p}{.}\PYG{n+nx}{keyCode} \PYG{o}{===} \PYG{l+m+mi}{13}\PYG{p}{)} \PYG{p}{\PYGZob{}}
      \PYG{k}{if} \PYG{p}{(}\PYG{k}{this}\PYG{p}{.}\PYG{n+nx}{state}\PYG{p}{.}\PYG{n+nx}{robotId}\PYG{p}{)} \PYG{p}{\PYGZob{}}
        \PYG{n+nx}{db}\PYG{p}{.}\PYG{n+nx}{ref}\PYG{p}{(}\PYG{l+s+s1}{\PYGZsq{}commands/\PYGZsq{}} \PYG{o}{+} \PYG{k}{this}\PYG{p}{.}\PYG{n+nx}{state}\PYG{p}{.}\PYG{n+nx}{robotId} \PYG{o}{+} \PYG{l+s+s1}{\PYGZsq{}/speech\PYGZsq{}}\PYG{p}{)}\PYG{p}{.}\PYG{n+nx}{set}\PYG{p}{(}\PYG{n+nx}{e}\PYG{p}{.}\PYG{n+nx}{target}\PYG{p}{.}\PYG{n+nx}{value}\PYG{p}{)}\PYG{p}{;}
      \PYG{p}{\PYGZcb{}}
      \PYG{n+nx}{e}\PYG{p}{.}\PYG{n+nx}{target}\PYG{p}{.}\PYG{n+nx}{value} \PYG{o}{=} \PYG{l+s+s1}{\PYGZsq{}\PYGZsq{}}\PYG{p}{;}
    \PYG{p}{\PYGZcb{}}
  \PYG{p}{\PYGZcb{}}
  \PYG{c+cm}{/**}
\PYG{c+cm}{   * Actualiza el id del robot, para escribir en la ubicación de Firebase que}
\PYG{c+cm}{   * le corresponde.}
\PYG{c+cm}{   */}
  \PYG{n+nx}{componentWillReceiveProps}\PYG{p}{(}\PYG{n+nx}{nextProps}\PYG{p}{)} \PYG{p}{\PYGZob{}}
    \PYG{k}{this}\PYG{p}{.}\PYG{n+nx}{setState}\PYG{p}{(}\PYG{p}{\PYGZob{}}\PYG{n+nx}{robotId} \PYG{o}{:} \PYG{n+nx}{nextProps}\PYG{p}{.}\PYG{n+nx}{robotId}\PYG{p}{\PYGZcb{}}\PYG{p}{)}\PYG{p}{;}
  \PYG{p}{\PYGZcb{}}
  \PYG{c+cm}{/**}
\PYG{c+cm}{   * Renderiza el componente}
\PYG{c+cm}{   */}
  \PYG{n+nx}{render}\PYG{p}{(}\PYG{p}{)} \PYG{p}{\PYGZob{}}\PYG{p}{...}\PYG{p}{\PYGZcb{}}
\PYG{p}{\PYGZcb{}}
\end{sphinxVerbatim}


\subparagraph{constants/}
\label{\detokenize{code_docs:constants}}
En este directorio contiene el archivo de las rutas que tendrá la aplicación

\fvset{hllines={, ,}}%
\begin{sphinxVerbatim}[commandchars=\\\{\}]
\PYG{k+kr}{export} \PYG{k+kr}{const} \PYG{n+nx}{SIGN\PYGZus{}IN} \PYG{o}{=} \PYG{l+s+s1}{\PYGZsq{}/signin\PYGZsq{}}\PYG{p}{;}
\PYG{k+kr}{export} \PYG{k+kr}{const} \PYG{n+nx}{SIGN\PYGZus{}UP} \PYG{o}{=} \PYG{l+s+s1}{\PYGZsq{}/signup\PYGZsq{}}\PYG{p}{;}
\PYG{k+kr}{export} \PYG{k+kr}{const} \PYG{n+nx}{LANDING} \PYG{o}{=} \PYG{l+s+s1}{\PYGZsq{}/\PYGZsq{}}\PYG{p}{;}
\PYG{k+kr}{export} \PYG{k+kr}{const} \PYG{n+nx}{ROBOTS} \PYG{o}{=} \PYG{l+s+s1}{\PYGZsq{}/robots\PYGZsq{}}\PYG{p}{;}
\end{sphinxVerbatim}

La aplicación inicia mostrando el componente de la ruta \sphinxcode{\sphinxupquote{LANDING}}, que es
\sphinxcode{\sphinxupquote{\textless{}Landing /\textgreater{}}}, el usuario selecciona \sphinxcode{\sphinxupquote{Sign In}} en el menú de navegación
se direcciona a la ruta \sphinxcode{\sphinxupquote{SIGN\_IN}}, si es un nuevo usuario puede dar click
en el botón que envía a la ruta \sphinxcode{\sphinxupquote{SIGN\_UP}}. Después de realzar un inicio de
sesión de cualquiera de las dos formas anteriores se redirecciona a la ruta
\sphinxcode{\sphinxupquote{ROBOTS}} que muestra al componente \sphinxcode{\sphinxupquote{\textless{}Dashboard /\textgreater{}}}


\paragraph{Manejo de sesiones y autorización de usuarios}
\label{\detokenize{code_docs:manejo-de-sesiones-y-autorizacion-de-usuarios}}
Para manejar las sesión de un usuario, es decir, verificar que si este ha
iniciado sesión y así saber qué componentes se pueden mostrar, se utilizan
\sphinxstylestrong{componentes de alto orden} y el patrón proveedor de React.

Los componentes de alto nivel de cumplen con una sola función. Separa la lógica
del negocio de los componentes. Así, un componente se mantiene ligero.

El patrón proveedor de React es un concepto que ayuda a pasar propiedades
en la aplicación usando el contexto de React. Permite pasar propiedades
implícitamente entre componentes sin importar su jerarquía.

Utilizando un HOC, se envuelve al componente \sphinxcode{\sphinxupquote{App}} para mejorarlo y
añadir la funcionalidad del manejo de sesión.

El componente de alto nivel está dentro del directorio \sphinxcode{\sphinxupquote{components}} en el
archivo \sphinxcode{\sphinxupquote{with\_authentication}}.

\fvset{hllines={, ,}}%
\begin{sphinxVerbatim}[commandchars=\\\{\}]
\PYG{c+cm}{/**}
\PYG{c+cm}{ * El componente de alto nivel para manejar el inicio de sesión y enviar al}
\PYG{c+cm}{ * usuario actual a otros componentes.}
\PYG{c+cm}{ */}
\PYG{k+kr}{const} \PYG{n+nx}{withAuthentication} \PYG{o}{=} \PYG{p}{(}\PYG{n+nx}{MyComponent}\PYG{p}{)} \PYG{p}{=\PYGZgt{}} \PYG{p}{\PYGZob{}}
  \PYG{k+kr}{class} \PYG{n+nx}{WithAuthentication} \PYG{k+kr}{extends} \PYG{n+nx}{Component} \PYG{p}{\PYGZob{}}
    \PYG{n+nx}{constructor}\PYG{p}{(}\PYG{n+nx}{props}\PYG{p}{)} \PYG{p}{\PYGZob{}}
      \PYG{k+kr}{super}\PYG{p}{(}\PYG{n+nx}{props}\PYG{p}{)}\PYG{p}{;}
      \PYG{k}{this}\PYG{p}{.}\PYG{n+nx}{state} \PYG{o}{=} \PYG{p}{\PYGZob{}}
        \PYG{n+nx}{authUser}\PYG{o}{:} \PYG{k+kc}{null}
      \PYG{p}{\PYGZcb{}}
    \PYG{p}{\PYGZcb{}}
    \PYG{c+cm}{/**}
\PYG{c+cm}{     * Permite enviar a los componentes hijos el objeto que contiene al usuario}
\PYG{c+cm}{     * activo}
\PYG{c+cm}{     */}
    \PYG{n+nx}{getChildContext}\PYG{p}{(}\PYG{p}{)} \PYG{p}{\PYGZob{}}
      \PYG{k}{return} \PYG{p}{\PYGZob{}}\PYG{n+nx}{authUser}\PYG{o}{:} \PYG{k}{this}\PYG{p}{.}\PYG{n+nx}{state}\PYG{p}{.}\PYG{n+nx}{authUser}\PYG{p}{\PYGZcb{}}\PYG{p}{;}
    \PYG{p}{\PYGZcb{}}

    \PYG{n+nx}{componentDidMount}\PYG{p}{(}\PYG{p}{)} \PYG{p}{\PYGZob{}}
      \PYG{n+nx}{auth}\PYG{p}{.}\PYG{n+nx}{onAuthStateChanged}\PYG{p}{(}\PYG{n+nx}{user} \PYG{p}{=\PYGZgt{}} \PYG{p}{\PYGZob{}}
        \PYG{k}{this}\PYG{p}{.}\PYG{n+nx}{setState}\PYG{p}{(}\PYG{p}{\PYGZob{}}
          \PYG{n+nx}{authUser}\PYG{o}{:} \PYG{n+nx}{user}
            \PYG{o}{?} \PYG{n+nx}{user}
            \PYG{o}{:} \PYG{k+kc}{null}
        \PYG{p}{\PYGZcb{}}\PYG{p}{)}\PYG{p}{;}
      \PYG{p}{\PYGZcb{}}\PYG{p}{)}
    \PYG{p}{\PYGZcb{}}

    \PYG{n+nx}{render}\PYG{p}{(}\PYG{p}{)} \PYG{p}{\PYGZob{}}
      \PYG{k}{return} \PYG{p}{(}\PYG{o}{\PYGZlt{}}\PYG{n+nx}{MyComponent}\PYG{o}{/}\PYG{o}{\PYGZgt{}}\PYG{p}{)}\PYG{p}{;}
    \PYG{p}{\PYGZcb{}}
  \PYG{p}{\PYGZcb{}}
  \PYG{c+cm}{/**}
\PYG{c+cm}{   * Parte del patrón proveedor de React con el objeto context}
\PYG{c+cm}{   */}
  \PYG{n+nx}{WithAuthentication}\PYG{p}{.}\PYG{n+nx}{childContextTypes} \PYG{o}{=} \PYG{p}{\PYGZob{}}
    \PYG{n+nx}{authUser}\PYG{o}{:} \PYG{n+nx}{PropTypes}\PYG{p}{.}\PYG{n+nx}{object}
  \PYG{p}{\PYGZcb{}}\PYG{p}{;}

  \PYG{k}{return} \PYG{n+nx}{WithAuthentication}\PYG{p}{;}
\PYG{p}{\PYGZcb{}}

\PYG{k+kr}{export} \PYG{k}{default} \PYG{n+nx}{withAuthentication}\PYG{p}{;}
\end{sphinxVerbatim}

Para la parte de la autorización que se utliza para proteger las rutas a las que el usuario no puede a menos que haya inciado sesión
se vuelve a usar un compoente de alto nivel, en este caso no «mejora» al componente \sphinxcode{\sphinxupquote{App}}, pero si a \sphinxcode{\sphinxupquote{Dashboard}}.

\fvset{hllines={, ,}}%
\begin{sphinxVerbatim}[commandchars=\\\{\}]
\PYG{k+kr}{const} \PYG{n+nx}{withAuthorization} \PYG{o}{=} \PYG{p}{(}\PYG{n+nx}{authCondition}\PYG{p}{)} \PYG{p}{=\PYGZgt{}} \PYG{p}{(}\PYG{n+nx}{MyComponent}\PYG{p}{)} \PYG{p}{=\PYGZgt{}} \PYG{p}{\PYGZob{}}
  \PYG{k+kr}{class} \PYG{n+nx}{WithAuthorization} \PYG{k+kr}{extends} \PYG{n+nx}{Component} \PYG{p}{\PYGZob{}}

    \PYG{c+cm}{/**}
\PYG{c+cm}{     * Cada vez que el usuario cambia de estado, verifica la condición de autenticación (authCondition).}
\PYG{c+cm}{     * Si la autorización falla el HOC redirecciona a la pagína de inicio de sesión.}
\PYG{c+cm}{     */}
    \PYG{n+nx}{componentDidMount}\PYG{p}{(}\PYG{p}{)} \PYG{p}{\PYGZob{}}
      \PYG{n+nx}{auth}\PYG{p}{.}\PYG{n+nx}{onAuthStateChanged}\PYG{p}{(}\PYG{n+nx}{authUser} \PYG{p}{=\PYGZgt{}} \PYG{p}{\PYGZob{}}
        \PYG{k}{if} \PYG{p}{(}\PYG{o}{!}\PYG{n+nx}{authCondition}\PYG{p}{(}\PYG{n+nx}{authUser}\PYG{p}{)}\PYG{p}{)} \PYG{p}{\PYGZob{}}
          \PYG{k}{this}\PYG{p}{.}\PYG{n+nx}{props}\PYG{p}{.}\PYG{n+nx}{history}\PYG{p}{.}\PYG{n+nx}{push}\PYG{p}{(}\PYG{n+nx}{routes}\PYG{p}{.}\PYG{n+nx}{SIGN\PYGZus{}IN}\PYG{p}{)}\PYG{p}{;}
        \PYG{p}{\PYGZcb{}}
      \PYG{p}{\PYGZcb{}}\PYG{p}{)}\PYG{p}{;}
    \PYG{p}{\PYGZcb{}}
    \PYG{c+cm}{/**}
\PYG{c+cm}{     * Rederiza al componente que se pasa como parámetro (Dashboard por ejemplo) o nada.}
\PYG{c+cm}{     */}
    \PYG{n+nx}{render}\PYG{p}{(}\PYG{p}{)} \PYG{p}{\PYGZob{}}
      \PYG{k}{return} \PYG{k}{this}\PYG{p}{.}\PYG{n+nx}{context}\PYG{p}{.}\PYG{n+nx}{authUser} \PYG{o}{?} \PYG{o}{\PYGZlt{}}\PYG{n+nx}{MyComponent} \PYG{o}{/}\PYG{o}{\PYGZgt{}} \PYG{o}{:} \PYG{k+kc}{null}\PYG{p}{;}
    \PYG{p}{\PYGZcb{}}
  \PYG{p}{\PYGZcb{}}

  \PYG{n+nx}{WithAuthorization}\PYG{p}{.}\PYG{n+nx}{contextTypes} \PYG{o}{=} \PYG{p}{\PYGZob{}}
    \PYG{n+nx}{authUser}\PYG{o}{:} \PYG{n+nx}{PropTypes}\PYG{p}{.}\PYG{n+nx}{object}\PYG{p}{,}
  \PYG{p}{\PYGZcb{}}\PYG{p}{;}

  \PYG{k}{return} \PYG{n+nx}{withRouter}\PYG{p}{(}\PYG{n+nx}{WithAuthorization}\PYG{p}{)}\PYG{p}{;}
\PYG{p}{\PYGZcb{}}
\end{sphinxVerbatim}

Con este HOC de autorización se puede definir una condiciaón de autorización, que es que el objeto que representa al usuario actual
no sea nulo.

Finalmente se añade la funcionalidad de autorización en el componente \sphinxcode{\sphinxupquote{Dashboard}}
con la condición de que el usuario no sea nulo.

\fvset{hllines={, ,}}%
\begin{sphinxVerbatim}[commandchars=\\\{\}]
\PYG{k+kr}{const} \PYG{n+nx}{authCondition} \PYG{o}{=} \PYG{p}{(}\PYG{n+nx}{authUser}\PYG{p}{)} \PYG{p}{=\PYGZgt{}} \PYG{o}{!}\PYG{o}{!}\PYG{n+nx}{authUser}\PYG{p}{;}
\PYG{k+kr}{export} \PYG{k}{default} \PYG{n+nx}{withAuthorization}\PYG{p}{(}\PYG{n+nx}{authCondition}\PYG{p}{)}\PYG{p}{(}\PYG{n+nx}{Dashboard}\PYG{p}{)}\PYG{p}{;}
\end{sphinxVerbatim}


\paragraph{Despliegue de la aplicación en Firebase Hosting}
\label{\detokenize{code_docs:despliegue-de-la-aplicacion-en-firebase-hosting}}
Para implementar la aplicación a Firebase Hosting son dos simples comandos los
que se ejecutan, el primero contruye la aplicación para producción y
el segundo envía los archivos de la carpeta \sphinxcode{\sphinxupquote{build/}} al sevicio de alojamiento:

\fvset{hllines={, ,}}%
\begin{sphinxVerbatim}[commandchars=\\\{\}]
\PYG{n}{npm} \PYG{n}{run} \PYG{n}{build}
\PYG{n}{firebase} \PYG{n}{deploy}
\end{sphinxVerbatim}

