


\subsection{Firebase Realtime Database y CloudNAO}
\label{\detokenize{chapter_two/desc_cloudnao:firebase-realtime-database-y-cloudnao}}
Como componente dentro de la arquitectura CloudNAO, la Realtime Database tiene
dos funciones principales; almacenar un registro de datos generados por el robot,
y servir como interfaz para enviar desde un navegador web comandos que puedan
ser ejecutados por el robot de manera remota, sin la necesidad de estar
conectados a la misma red. A la primera
funcionalidad la llamaremos \sphinxstyleemphasis{Robot Log} y a la segunda \sphinxstyleemphasis{Robot Remote Control}.

Firebase Realtime Database es el elemento que relaciona al robot, la aplicación
web y la móvil. En la \hyperref[\detokenize{chapter_two/desc_cloudnao:rtd-components-diagram}]{Figura \ref{\detokenize{chapter_two/desc_cloudnao:rtd-components-diagram}}} se muestran las relaciones.
La unión del robot, la aplicación móvil y Firebase Realtime Database forma parte de
la función de \sphinxstyleemphasis{Robot Log}. La conexión entre el robot, la aplicación web y
Firebase Realtime Database son la otra parte del \sphinxstyleemphasis{Robot Log}, y toda la
estructura del \sphinxstyleemphasis{Robot Remote Control}.

\begin{figure}[htbp]
\centering
\capstart

\noindent\sphinxincludegraphics[scale=0.3]{{realtime_database_diagram}.png}
\caption{Diagrama de la relación entre los componentes de la arquitectura y Firebase Realtime Database}\label{\detokenize{chapter_two/desc_cloudnao:rtd-components-diagram}}\end{figure}

De la \hyperref[\detokenize{chapter_two/desc_cloudnao:rtd-components-diagram}]{Figura \ref{\detokenize{chapter_two/desc_cloudnao:rtd-components-diagram}}} se muestran identificadas por un número
las relaciones entre cada elemento. A continuación se describe cada una de estas:

1. \sphinxstylestrong{Robot - Firebase Realtime Database}, el robot envía información
a la base de datos mediante su API REST.
Para la parte de \sphinxstyleemphasis{Robot Log}, envía datos de \sphinxcode{ALMemory} o de \sphinxcode{ALVideDevice}
y para \sphinxstyleemphasis{Robot Remote Control} establece una conexión constante para recibir
actualizaciones que son interpretadas como comandos.
La API REST soporta el protocolo
\sphinxhref{https://www.w3.org/TR/eventsource/}{Server-sent events}, la cual es una
tecnología donde un cliente recibe actualizaciones automáticamente desde el
servidor mediante una conexión HTTP. Con esta herramienta el robot se mantiene
«escuchando» los cambios realizados a una ubicación en la base de datos. Al
recibir un mensaje el robot lo maneja como un comando para ejecutar cierta
acción, como puede ser un movimiento o enviar una imagen en vivo. Esta es la parte principal del \sphinxstyleemphasis{Robot Remote Control}.

2. \sphinxstylestrong{Aplicación móvil - Robot}, utilizando el SDK de NAOqi para Java,
específicamente para Android, se ejecutan remotamente módulos de la API de NAOqi.
Para la parte de Realtime Database se solicitan datos de \sphinxcode{ALMemory}. El robot
envía la información a la aplicación y ésta escribe los datos en una ubicación
única para cada robot. Esta relación es una de las partes que generan los datos
de \sphinxstyleemphasis{Robot Log}.

3. \sphinxstylestrong{Firebase Realtime Database - Aplicación móvil}, el primer elemento
envía los últimos valores guardados en la base de datos a una lista que se muestra
dentro de una actividad en la aplicación cuya función es mostrar valores
obtenidos de \sphinxcode{ALMemory}, esta parte corresponde a \sphinxstyleemphasis{Robot Log}. Usando
el SDK para Android de Firebase Realtime Database se hacen realizan las
operaciones de escritura y lectura de los datos de \sphinxcode{ALMemory}
obtenidos del robot.

4. \sphinxstylestrong{Firebase Realtime Database - Aplicación web}, el navegador web escribe y
lee los datos con el SDK web de Firebase Realtime Database. Se vale de
la funcionalidad de \sphinxstyleemphasis{Robot Log} para mostrar los últimos valores de \sphinxcode{ALMemory}
enviados por el robot y escribe información a una ubicación de la base de
datos a la que el robot se habrá suscrito para detectar cambios. Dichos cambios
son comandos que el robot ejecutará. Esto último es una parte de
\sphinxstyleemphasis{Robot Remote Control}.

En las secciones subsecuentes se explica cómo implementar el servicio de
Firebase Realtime Database en cada una de las plataformas utilizadas, el robot
NAO, la aplicación móvil y la web.


\subsubsection{Diseño de la base de datos}
\label{\detokenize{chapter_two/desc_cloudnao:diseno-de-la-base-de-datos}}
Es muy importante estructurar de manera correcta la información en una base
de datos de Firebase.

Los datos dentro de la Firebase Realtime Database se almacenan como objetos
JSON. La base de datos puede conceptualizarse como un árbol JSON alojado en la
nube. A diferencia de una base de datos de SQL, no hay tablas ni registros.
Cuando le agregas datos al árbol JSON, estos se convierten en un nodo de la
estructura JSON existente con una clave asociada.

La documentación de Firebase recomienda algunos puntos que se deben tomar
en cuenta para estructurar los datos dentro del JSON.
\begin{itemize}
\item {} 
\sphinxstylestrong{Evitar anidación de datos}. Firebase Realtime Database permite anidar datos en un máximo de 32 niveles, sin embargo, cuando obtienes datos de una ubicación de la base de datos, también se recuperan todos los nodos secundarios.

\item {} 
\sphinxstylestrong{Compactar las estructuras de datos}. Si los datos se dividen en rutas de acceso independientes, que también se conocen como no normalizadas, se pueden descargar de manera eficaz en llamadas separadas, según sea necesario.

\item {} 
\sphinxstylestrong{Crear datos escalables}. Siempre es mejor descargar un subconjunto de una lista de registros.

\end{itemize}

Siguiendo estas recomendaciones, se diseñó una base de datos, evitando anidar
elementos separándolos como fuera conveniente y manteniendo relaciones que
eran necesarias para realizar consultas de manera simple.

En las siguientes secciones primero se describe la base de datos con un enfoque
clásico con el modelo entidad relación y después como se puede transformar ese
modelo base en un JSON que se pueda almacenar en la nube.

El diagrama de la \hyperref[\detokenize{chapter_two/desc_cloudnao:entity-relationship-model-cn-firebase}]{Figura \ref{\detokenize{chapter_two/desc_cloudnao:entity-relationship-model-cn-firebase}}}, muestra el
modelo sobre el que se basa el JSON que se almacena en Firebase.

Un usuario puede tener varios robots, cada robot cuenta está relacionado
con otras entidades como: un texto que representa el discurso que tiene que decir,
la última imagen enviada por el robot, un largo historial de
logs con valores de \sphinxcode{ALMemory}, el estado del robot (el nivel de batería y
su conexión a la red), y comandos para ejecutar ciertas tareas.

\begin{figure}[htbp]
\centering
\capstart

\noindent\sphinxincludegraphics[scale=0.75]{{entity_relationship_model_cn_firebase}.png}
\caption{Diagrama entidad relación de la base de datos para Firebase}\label{\detokenize{chapter_two/desc_cloudnao:entity-relationship-model-cn-firebase}}\end{figure}


\paragraph{Estructura del JSON para Firebase}
\label{\detokenize{chapter_two/desc_cloudnao:estructura-del-json-para-firebase}}
La estructura de la base de datos descrita en el modelo de la \hyperref[\detokenize{chapter_two/desc_cloudnao:entity-relationship-model-cn-firebase}]{Figura \ref{\detokenize{chapter_two/desc_cloudnao:entity-relationship-model-cn-firebase}}}
es la base del JSON y es por eso que puede no cumplir los requisitos de normalización,
simplemente ilustra el diseño de la base de datos para que sea más simple su
visualización.

Con una idea de cómo está diseñada la base de datos la estructura del JSON
queda como en la \hyperref[\detokenize{chapter_two/desc_cloudnao:json-structure}]{Figura \ref{\detokenize{chapter_two/desc_cloudnao:json-structure}}}. Es importante señalar que esta base de datos
es muy flexible, los campos en las «entidades» que aquí son los atributos
del JSON, pueden cambiar: eliminándolos, agregando nuevos, etcétera.

\begin{figure}[htbp]
\centering
\capstart

\noindent\sphinxincludegraphics[scale=0.50]{{Json_structure}.png}
\caption{La estructura del JSON donde se almacenan los datos}\label{\detokenize{chapter_two/desc_cloudnao:json-structure}}\end{figure}

Para entender mejor el JSON de la \hyperref[\detokenize{chapter_two/desc_cloudnao:json-structure}]{Figura \ref{\detokenize{chapter_two/desc_cloudnao:json-structure}}}, se describen
algunos de sus principales atributos:
\begin{itemize}
\item {} 
\sphinxstylestrong{nao-firebase}, es la raíz del objeto.

\item {} 
\sphinxstylestrong{commands}, es parte del segundo nivel, los robots se suscriben esta ubicación para manejar actualizaciones. Está compuesto por objetos que corresponde a cada robot.

\item {} 
\sphinxstylestrong{liveImages}, es parte del segundo nivel, y almacena imágenes enviadas por cada robot. Está compuesto por objetos que corresponde a cada robot.

\item {} 
\sphinxstylestrong{logs}, está en el segundo nivel, y aquí se encuentra el historial de logs de \sphinxcode{ALMemory} de cada robot. Está compuesto por objetos que corresponde a cada robot.

\item {} 
\sphinxstylestrong{robots}, definido en el segundo nivel, conjunto de todos los robots que se añaden a la aplicación.

\item {} 
\sphinxstylestrong{speeches}, está en el segundo nivel y contiene el texto que debe ser enviado al robot y éste repetir oralmente. Está compuesto por objetos que corresponde a cada robot.

\item {} 
\sphinxstylestrong{users}, en el segundo nivel, el conjunto de todos los usuarios de la aplicación, ordenados por su identificador generado por Firebase Auth. Sólo se almacena su dirección de correo electrónico y un objeto con los robots que le pertenecen.

\item {} 
\sphinxstylestrong{robotId}, es un identificador único generado por Firebase, cada robot que agrega un usuario tiene un id único. Es una cadena que sirve para que cada robot tenga su propia ubicación de comandos, imágenes, etc.

\item {} 
\sphinxstylestrong{userId}, un id obtenido de Firebase Authentication.

\end{itemize}

Los demás atributos son autodescriptivos.

