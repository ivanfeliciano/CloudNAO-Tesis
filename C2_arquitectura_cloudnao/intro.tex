\section{Descripción de la arquitectura CloudNAO}
\label{\detokenize{chapter_two/desc_cloudnao:descripcion-de-la-arquitectura-cloudnao}}\label{\detokenize{chapter_two/desc_cloudnao::doc}}
CloudNAO integra servicios web de
terceros, servicios web desarrollados por el LAR, una aplicación móvil para
dispositivos con el sistema operativo Android, la plataforma como servicio
Firebase, una aplicación web y al robot humanoide NAO. Todos estos componentes
se comunican a través de internet. Cada componente tiene subcomponentes que
permiten la comunicación y el procesamiento de diversas tareas. En la parte del
backend están los servicios web de terceros, los desarrollados
por el LAR y Firebase. 
%Los elementos restantes corresponden entonces al
%frontend.

Los servicios web de terceros utilizados son brindados por Google Cloud,
Kairos y Wit.ai. Se integran a algunos elementos de la arquitectura a través
de sus API REST.

Los servicios provistos por el LAR, son ejemplos de algunas aplicaciones sobre el
robot que le permitan realizar tareas complejas, como el procesamiento
de imágenes para detección de objetos o clasificación de escenarios.

Los servicios de terceros junto con los del LAR se integran dentro de un servidor al
que llamaremos \sphinxstyleemphasis{servidor LAR}. Éste cuenta
con una API REST, que tiene como clientes al robot y a la aplicación móvil.

Las funcionalidades de Firebase utilizadas son: la base de datos en tiempo real,
para envío de información entre el robot y las aplicaciones móvil y web,
la autenticación, para el acceso a la aplicación móvil y web, y hosting para
la aplicación web.

La aplicación móvil es una herramienta frontend para que los usuarios
interactúen con el robot sin necesidad de instalar o ejecutar un programa de
manera local en éste. Hace peticiones a la API REST del servidor del LAR y
se comunica con Firebase. Usando el SDK de Java de NAOqi la
aplicación recibe datos del robot y ejecuta comandos de manera remota.

De manera similar, la aplicación web, es una parte del frontend para enlazar
al usuario con el robot, sin embargo, esta interacción tiene como intermediario
a Firebase, por lo que la aplicación utiliza su API web. En este caso el robot
sí necesita comunicarse con Firebase a través de un programa corriendo local
o remotamente.

El robot es quien se conecta con el mayor número de componentes, usando las API
web de los servicios o plataformas, y con el framework de NAOqi.

La \hyperref[\detokenize{chapter_two/desc_cloudnao:cn-components-diagram}]{Figura \ref{\detokenize{chapter_two/desc_cloudnao:cn-components-diagram}}} es un diagrama con los
componentes de la arquitectura y de las interfaces que brindan o necesita cada
uno de los elementos.

\begin{figure}[htbp]
\centering
\capstart

\noindent\sphinxincludegraphics[scale=0.3]{{CN_components_diagram}.png}
\caption{Diagrama de los componentes de la arquitectura.}\label{\detokenize{chapter_two/desc_cloudnao:cn-components-diagram}}\end{figure}

En las siguientes secciones se describen con mayor detalle cada uno de los
componentes y subcomponentes que forman esta arquitectura.

