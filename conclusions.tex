\chapter*{Conclusiones}
\label{\detokenize{conclusion:cloudnao-una-arquitectura-de-software-para-la-integracion-de-computo-en-la-nube-con-robots-nao}}\label{\detokenize{conclusion:conclusion}}\label{\detokenize{conclusion::doc}}
\addcontentsline{toc}{chapter}{Conclusiones}

El primer objetivo del trabajo fue
desarrollar una arquitectura de software para la
integración de cómputo en la nube con robots NAO.
% fue el diseño e 
% implementación de
% la arquitectura sobre la cual robots NAO pueden
% utilizar tecnologías emergentes como son el cómputo en la nube
% y el aprendizaje 
% profundo.
Esto se logró con la creación de una arquitectura llamada
CloudNAO. 
De manera general, los elementos que interactúan en la arquitectura son:
una aplicación móvil y una web, una API REST, el robot NAO y Firebase.

Entre las características más importantes dentro 
de la arquitectura CloudNAO están las brindadas por 
la aplicación móvil, la API REST y Firebase. Por medio
del consumo de servicios en la aplicación móvil móvil 
el robot NAO pueda realizar tareas
que por sus capacidades, no podría llevar a cabo
localmente.
Con la aplicación, el robot puede generar datos que
se guardan en la nube o tomar una fotografía,
enviarla a un servicio para su procesamiento y a partir
de la respuesta obtenida realizar una acción; todo
esto en cuestión de un par de segundos.
% Sin embargo, un problema pendiente de la aplicación móvil
% es la compatibilidad del SDK de NAOqi con nuevos sistemas
% Android.
Como los servicios brindados dentro de la arquitectura son 
RESTful, no están limitados a una sola plataforma
de consumo, cualquier dispositivo que pueda solicitar y
manejar mensajes HTTP puede hacer uso de los servicios
de la API REST de CloudNAO.
Gracias a los servicios que ofrece Firebase, como la
base de datos en tiempo real, se pueden mandar comandos al 
robot de manera inmediata, almacenar grandes volúmenes de datos
producidos por los sensores de éste y sincronizarlos entre los 
dispositivos conectados.

El segundo objetivo del trabajo el cual fue desarrollar un servicio de cómputo en la nube que utilizara
algoritmos de aprendizaje automático, para robots NAO, se 
alcanzó exitosamente con el desarrollo
de una red neuronal convolucional para clasificación
de imágenes. Este clasificador fue inspirado
por el problema de localización dentro de la planeación 
automática de movimiento.
En el Capítulo \ref{\detokenize{chapter_three::doc}} se explicó
la implementación de un servicio, misma que se puede 
replicar para la creación de nuevos servicios
enfocados a las plataformas NAO u otros dispositivos.
% De la experimentación, encontramos modelos
% cuyo tiempo de entrenamiento es relativamente bajo, sólo unos 
% cuantos pares de minutos, y cuya precisión es muy alta, que 
% aunque no significa que sean excelentes clasificadores,
% las pruebas sobre el robot funcionaron bien.
% Esto en parte se debe a que
% TensorFlow hace simples y eficientes
% la definición, el entrenamiento y la carga de modelos de 
% aprendizaje automático. 
% Un trabajo a futuro sería desarrollar un caso de estudio 
% donde el modelo sirva como auxiliar en la navegación
% del robot.

Entre los principales trabajos a futuro están:

\begin{itemize}
    \item Uso de nuevos dispositivos
que interactúen dentro de la arquitectura como
kits de microcontroladores (Arduino), ordenadores
de placa reducida (Raspberry Pi) o incluso 
vehículos aéreos no tripulados. Hacer pruebas con estos 
dispositivos es posible gracias a las tecnologías 
utilizadas en la arquitectura CloudNAO.
\item Un caso de estudio interesante es la navegación y localización del robot NAO, por ejemplo, en el espacio de trabajo del LAR.
Queda como tarea ulterior la integración del servicio de clasificacióń de imágenes a la solución de navegación y localización del robot NAO. El clasificador ofrecido como un servicio es un herramienta simple y 
eficiente que tiene como objetivo asistir a la planeación de movimiento de robots NAO dentro del LAR.
\item Crear nuevos servicios para la arquitectura CloudNAO. Estos 
servicios no tienen que aplicarse necesariamente
sobre tareas específicas de los robots NAO, pueden estar enfocados
hacia otras plataformas. Adicionalmente, se propone utilizar otras
técnicas, dentro del campo de la inteligencias artificial, como
las redes bayesianas.
\end{itemize}

% La principal desventaja de la arquitectura es, como
% en la mayoría de los servicios en la nube, la dependencia
% a una conexión de internet para acceder a todos
% los servicios dentro de CloudNAO. Esto se compensa con
