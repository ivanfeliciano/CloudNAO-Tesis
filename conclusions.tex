\chapter*{Conclusiones}
\label{\detokenize{conclusion:cloudnao-una-arquitectura-de-software-para-la-integracion-de-computo-en-la-nube-con-robots-nao}}\label{\detokenize{conclusion:conclusion}}\label{\detokenize{conclusion::doc}}
\addcontentsline{toc}{chapter}{Conclusiones}


El primer objetivo de la tesis fue el diseño e 
implementación de
la arquitectura sobre la cual robots NAO pueden
utilizar tecnologías emergentes como son el cómputo en la nube
y el aprendizaje 
profundo. Esto se logró uniendo los siguientes componentes:
una aplicación móvil y una web, una API REST y al robot NAO.

En la aplicación
móvil se muestran algunos casos de uso del consumo de
servicios para que el robot NAO pueda realizar tareas
que por su capacidad de procesamiento no podría 
llevar a cabo de manera autónoma.
Con ésta, el robot puede generar datos que
se guardan en la nube o tomar una fotografía,
enviarla a un servicio para su procesamiento y a partir
de la respuesta obtenida realizar una acción; todo
esto en cuestión de segundos.
% Sin embargo, un problema pendiente de la aplicación móvil
% es la compatibilidad del SDK de NAOqi con nuevos sistemas
% Android.

Como los servicios brindados dentro de la arquitectura son 
RESTful, no están limitados a una sola plataforma
de consumo, cualquier dispositivo que pueda solicitar y
manejar mensajes HTTP puede hacer uso de los servicios
de la API REST de CloudNAO.
El papel de Firebase es importante porque muestra
los principales beneficios de la nube en el desarrollo
de aplicaciones modernas. Cuenta con una base de datos en tiempo
real que permite mandar comandos al robot de manera
inmediata, almacenar grandes volúmenes de datos
producidos por los sensores de éste y sincronizarlos entre los dispositivos
conectados.

El segundo objetivo del trabajo fue 
alcanzado exitosamente con el desarrollo
de una red neuronal convolucional para la clasificación
de imágenes en diferentes escenarios dentro del laboratorio.
De la experimentación, encontramos modelos
cuyo tiempo de entrenamiento es relativamente bajo, sólo unos 
cuantos pares de minutos, y cuya precisión es muy alta, que 
aunque no significa que sean excelentes clasificadores,
las pruebas sobre el robot funcionaron bien.
Esto en parte se debe a que
TensorFlow hace simples y eficientes
la definición, el entrenamiento y la carga de modelos de 
aprendizaje automático. 
% Un trabajo a futuro sería desarrollar un caso de estudio 
% donde el modelo sirva como auxiliar en la navegación
% del robot.

Como nota final puedo decir que las tecnologías empleadas
facilitan la adición
de nuevas funcionalidades sobre una plataforma con recursos
limitados, como los son los robots NAO,
permitiendo introducir al robot a otros campos de estudio.
