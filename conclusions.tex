\chapter*{Conclusiones}
\label{\detokenize{conclusion:cloudnao-una-arquitectura-de-software-para-la-integracion-de-computo-en-la-nube-con-robots-nao}}\label{\detokenize{conclusion:conclusion}}\label{\detokenize{conclusion::doc}}
\addcontentsline{toc}{chapter}{Conclusiones}


El objetivo principal de esta tesis fue el diseño e implementación de
la arquitectura sobre la cual robots NAO pueden
utilizar tecnologías emergentes como son el cómputo en la nube y
la inteligencia artificial, en específico el aprendizaje profundo.

Una pregunta que inspiró el trabajo
fue si se podían integrar servicios en la nube
en el robot NAO para conectarse a bases de datos, realizar tareas como el procesamiento de imágenes y de lenguaje natural. En la aplicación
móvil se muestran algunos casos de uso de estos servicios
sin necesidad de ejecutar un programa de manera local
en el robot. Sin embargo, un problema pendiente de ésta
es la compatibilidad del SDK de NAOqi con nuevos sistemas
Android.

Otra parte fundamental del trabajo fue el desarrollo
de una red neuronal convolucional para la clasificación
de imágenes en diferentes escenarios dentro del laboratorio.
Aunque se cumplió el objetivo de implementar un modelo
de aprendizaje profundo y de lanzarlo como un servicio, 
a partir del tiempo de entrenamiento y la precisión de las
distintas arquitecturas de la red experimentadas,
parece exagerado utilizar bibliotecas como TensorFlow 
y que además el robot acceda al modelo como
un servicio.
Encontramos modelos
cuya precisión es muy alta, que aunque no significa que sean excelentes clasificadores,
las pruebas sobre el robot funcionaron correctamente.
TensorFlow hace simples y eficientes
las definición y entrenamiento de modelos de aprendizaje automático,
dejando la nueva pregunta de si podríamos implementar un modelo 
directamente sobre el robot, sin la dependencia a una conexión
a internet para solicitar recursos. Si bien TensorFlow
no tiene soporte para plataformas como el robot NAO, podríamos
desarrollar una herramienta específica para éstos. 

Como nota final puedo decir que las tecnologías empleadas
facilitan la adición
de nuevas funcionalidades sobre una plataforma con recursos
limitados, como los son los robots NAO, y aunque existen desventajas,
permiten la introducción
del robot a otros campos de estudio.