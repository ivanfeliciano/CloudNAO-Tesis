\chapter*{Conclusiones}
\label{\detokenize{conclusion:cloudnao-una-arquitectura-de-software-para-la-integracion-de-computo-en-la-nube-con-robots-nao}}\label{\detokenize{conclusion:conclusion}}\label{\detokenize{conclusion::doc}}
\addcontentsline{toc}{chapter}{Conclusiones}


La arquitectura CloudNAO permite
la integración de tecnologías que cada vez se vuelven
más comunes entre los usuarios; la nube, la inteligencia artificial
y las apps. 
Los elementos principales de este 
trabajo son: la conexión y acceso
entre múltiples dispositivos a un	 sistema
utilizando Firebase, la solución de problemas relativos al campo 
de la inteligencia artificial utilizando modelos
de aprendizaje profundo para procesamiento de imágenes y de lenguaje natural, 
y la entrega de esos modelos como un servicio de cómputo en la nube.



1) Se debe enfatizar el mensaje principal y los
resultados que lo soportan; en relación con el resultado(s) 
principal se establece la conclusión (es). Se describen
las ideas importantes del estudio.

2 y 3) Comparar los resultados principales  con estudios
publicados por otros autores estableciendo su postura;
otorgar crédito a otros autores de publicaciones previas cuyo trabajo
ha sido confirmado. Ser justo con los autores cuyos resultados difieren de
su estudio y explique, si es posible, las
diferencias con imparcialidad.

4) Se describen los resultados secundarios, que son hallazfos sin relación directa con el mensaje principal; con frecuencia los resultados
secundarios ofrecen la oportunidad para realizar otras investigaciones.
Los resultados secundarios también deben compararse con lo
publicado por otros autores.

5) Se deben hacer explícitas las fortalezas y debilidades del 
estudio.

6) El último párrafo debe referirse a las conclusiones 
del estudio relacionadas con el mensaje principal, así como 
las recomendaciones para aplicar los resultados de la investigación
y /o la necesidad de estudio futuros.; se deben evitar hacer conclusines
insuficientemente soportadas por los datos, en particular,
las afirmaciones sobre beneficios económics a menos que su 
manuscrito incluya datos económicos y el análisis apropiado.
Proponga nuevas hipótesis cuando esté justificado, pero identificándolas
claramente como tales. Igual que en el 
primer párrafo se describen las ideas importantes del estudio.