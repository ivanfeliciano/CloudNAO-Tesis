\newglossaryentry{NAO}{%
  name={NAO},%
  description={robot humanoide autónomo y programable desarrollado por la empresa Aldebaran Robotics}}
\newglossaryentry{Ejemplos o muestras}{name={Ejemplos o muestras}, description={Objetos o instancias de información usadas para aprender o evaluar}}
\newglossaryentry{features}{name={Características}, description={El conjunto de atributos, comúnmente representadas como un vector, asociado a una muestra}}
\newglossaryentry{Etiquetas}{name={Etiquetas}, description={Valores o categorías
asignadas a muestras}}
\newglossaryentry{Muestra de entrenamiento}{name={Muestra de entrenamiento}, description={Las muestras usadas
para entrenar un algoritmo}}
\newglossaryentry{validation_set}{name={Muestra de validación}, description={Muestras usadas para ajustar los parámetros de un algoritmo de aprendizaje cuando se trabaja con datos etiquetados. Los algoritmos de aprendizaje usualmente tienen uno o más parámetros libres, y la muestra de validación es usada para seleccionar los valores apropiados para los parámetros del modelo}}
\newglossaryentry{Muestra de prueba}{name={Muestra de prueba}, description={Muestras usadas para evaluar el desempeño de un algoritmo de aprendizaje. La muestra de prueba está separada del conjunto de entrenamiento y validación, y no está disponible en la etapa de aprendizaje}}
\newglossaryentry{cost_function}{name={Función de costo}, description={También llamada función de error o de pérdida. 
Es una función que mide la
diferencia, o pérdida, entre la etiqueta predicha y la verdadera.
Denotando al conjunto de todas las etiquetas como $Y$ y al conjunto de
todas las posibles predicciones como $Y'$, una función $C$ es una mapeo $C: Y
 \times Y' \rightarrow \mathbb{R}^+$. En la mayoría de los casos $Y = Y'$ y la función
de pérdida está acotada, pero estas condiciones no se cumplen siempre.
Algunos ejemplos comunes de funciones de pérdida incluyen la $0-1$,
definida como $C(y, y') = 1$ si $y \neq y'$ y la pérdida cuadrada definida por
$C(y, y') = (y' - y)^2$ 
}}
\newglossaryentry{hypothesis_set}{name={Conjunto de hipótesis}, description={Un conjunto de
funciones que mapean características (vectores de características) al
conjunto de etiquetas $Y$. De manera más general, las hipótesis pueden ser
funciones que mapean las características a un conjunto diferente $Y'$}}

\newglossaryentry{activity}{name={Actividad}, description={Una actividad es una instancia
de \texttt{Activity}, una clase del SDK de Android. Una actividad es la encargada de
manejar la interacción del usuario con una pantalla de información}}

\newglossaryentry{component}{name={Componente},
description={Dentro del contexto de desarrollo
de aplicaciones web, son lo elementos de React
que dividen la IU en piezas independientes y 
reutilizables}}

\newglossaryentry{kernel}{name={Kernel}, description=
{Generalmente una función o matriz pequeña de números que
se utiliza en la convolución de imágenes}}

\newacronym{api}{API}{Interfaz de programación de aplicaciones}
\newacronym{sdk}{SDK}{Kit de desarrollo de software}
\newacronym{fc}{FC}{Completamente conectada}
\newacronym{cnn}{CNN}{Red Neuronal Convolucional}
\newacronym{saas}{SaaS}{Software as a Service}
\newacronym{baas}{BaaS}{Backend as a Service}
\newacronym{paas}{PaaS}{Platform as a Service}
\newacronym{iaas}{IaaS}{Infrastructure as a Service}
\newacronym{http}{HTTP}{Protocolo de Transferencia de Hipertexto}
\newacronym{uri}{URI}{Identificador de Recursos Uniforme}
\newacronym{rest}{REST}{Transferencia de Estado Representacional}
\newacronym{json}{JSON}{JavaScript Object Notation}
\newacronym{ocr}{OCR}{Reconocimiento óptico de caracteres}
\newacronym{orm}{ORM}{Mapeo objeto relacional}