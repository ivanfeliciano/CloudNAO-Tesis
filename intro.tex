

\chapter*{Introducción}
\label{\detokenize{introduction:cloudnao-una-arquitectura-de-software-para-la-integracion-de-computo-en-la-nube-con-robots-nao}}\label{\detokenize{introduction:introduccion}}\label{\detokenize{introduction::doc}}
\addcontentsline{toc}{chapter}{Introducción}

% Interconnection of sensing and actuating devices providing the ability to share information across platforms through a unified framework, developing a common operating picture for enabling innovative applications. This is achieved by seamless ubiquitous sensing, data analytics and information representation with Cloud computing as the unifying framework.

Actualmente existen diferentes plataformas que 
ofrecen servicios
de cómputo en la nube para solucionar problemas de 
visión computacional \cite{googlevision2018} o
procesamiento de lenguaje natural \cite{witaidocs2018} usando modelos
de aprendizaje profundo. 
%Éstas también incluyen servicios 
% para entrenar y lanzar
% modelos creados por los mismos usuarios \cite{tensorflowgooglecloud2018}.
Todos estos recursos se entregan a través de 
interfaces que
permiten que casi cualquier dispositivo con 
internet
pueda acceder a éstos.
El cómputo en la nube es un estilo de cómputo que 
permite el acceso a 
recursos informáticos a través de internet cuando un
usuario lo demanda \cite{borkofurhtarmandoescalante2010}. El aprendizaje profundo es un 
tipo
de aprendizaje automático donde las computadoras 
construyen
conceptos complejos a partir de conceptos más 
simples \cite{iangoodfellowyoshuabengioaaroncourville2017}.


%¿Qué es lo que no sabemos?\\
El cómputo en la nube es el marco de trabajo que
unifica los elementos que permiten que exista el
internet de las cosas \cite{jayavardhanagubbiarajkumarbuyyabslavenmarusicamarimuthupalaniswamia2013}. Queremos que el robot NAO 
sea parte de esa interconexión de sensores y actuadores
que comparten información entre plataformas
a través de internet.
NAO es un robot humanoide programable y autónomo
que cuenta con
un poder de procesamiento suficiente para realizar 
tareas como
jugar un partido de fútbol con otros robots \cite{splinfo2018}. Sin embargo,
las aplicaciones modernas
exigen que los dispositivos se adapten constantemente a nuevas necesidades
de los usuarios, como el acceso a los datos
del robot desde cualquier lugar, el uso del aprendizaje profundo para resolver problemas como la clasificación de imágenes,
% en múltiples categorías
encontrar
la posición de objetos en una imagen, reconocimiento
de personas y el 
procesamiento de lenguaje natural de un discurso \cite{benkehoesachinpatilpieterabbeelkengoldberg2014}.

%¿Qué pretendemos averiguar? (objetivo)

Los objetivos de esta investigación son: desarrollar una 
arquitectura de software para  la  
integración  de  cómputo  en  la  nube  con  robots  NAO
e implementar un  servicio  de  cómputo  en  la  nube  que  utilice  
algoritmos  de  aprendizaje  automático y que pueda
ser usado por robots NAO.
Algunos recursos son brindados por las plataformas
Google Cloud, Wit.ai, Kairos, Firebase y otros
por el Laboratorio de Algoritmos para la
Robótica.
Los servicios en la nube permitirán al robot NAO
realizar tareas como el procesamiento de lenguaje
natural de un discurso oral, el procesamiento 
de imágenes para reconocimiento de rostros
y objetos, la clasificación de
imágenes en escenarios, la conexión con el robot desde diferentes dispositivos y el acceso a datos generados por
éste desde cualquier lugar.

