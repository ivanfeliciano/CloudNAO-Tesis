

\chapter*{Introducción}
\label{\detokenize{introduction:cloudnao-una-arquitectura-de-software-para-la-integracion-de-computo-en-la-nube-con-robots-nao}}\label{\detokenize{introduction:introduccion}}\label{\detokenize{introduction::doc}}
\addcontentsline{toc}{chapter}{Introducción}
%\documentclass[10pt,a4paper]{report}
%\usepackage[utf8]{inputenc}
%\usepackage{amsmath}
%\usepackage{amsfonts}
%\usepackage{amssymb}
%\begin{document}
%¿Qué es lo que sabemos?
%\\
Actualmente existen diferentes plataformas que 
ofrecen servicios
de cómputo en la nube para solucionar problemas de 
visión computacional o
procesamiento de lenguaje natural usando modelos
de aprendizaje profundo; también incluyen servicios 
para crear, entrenar y lanzar
modelos creados por los desarrolladores.
Todos estos recursos se entregan a través de 
interfaces que
permiten que casi cualquier dispositivo con acceso a 
internet
pueda acceder a éstos.
El cómputo en la nube es un estilo de cómputo que 
permite el acceso a 
recursos informáticos a través de internet cuando un
usuario lo demanda. El aprendizaje profundo es un 
tipo
de aprendizaje automático donde las computadoras 
construyen
conceptos complejos a partir de conceptos más 
simples.


%¿Qué es lo que no sabemos?\\

NAO es un robot humanoide programable y autónomo
que a pesar de contar con
un poder de procesamiento suficiente para realizar 
tareas como
jugar un partido de fútbol con otros robots,
la aplicaciones modernas
exigen que los dispositivos se adapten constantemente a nuevas necesidades
de los usuarios, como el acceso a los datos
del robot desde cualquier lugar, la 
clasificación de imágenes y detección de
objetos en múltiples categorías, el 
procesamiento de lenguaje natural, etc.


%¿Qué pretendemos averiguar? (objetivo)

El objetivo de esta investigación es desarrollar e implementar un sistema
que permita la integración de servicios en la nube
con robots NAO.
Los recursos son brindados por las plataformas
Google Cloud, Wit.ai, Kairos, Firebase y por
el mismo Laboratorio de Algoritmos para la
Robótica.
Los servicios en la nube permitirán al robot NAO
realizar tareas como el procesamiento de lenguaje
natural de un discurso oral, el procesamiento 
de imágenes para reconocimiento de rostros
y objetos, la clasificación de escenarios en una
imagen, la conexión desde diferentes dispositivos
con el robot y el acceso a datos generados por
éste.


%Poniendo el ejemplo del partido de fútbol,
%la detección de una pelota blanca con figuras negras usando métodos de visión tradicionales
%puede volverse muy complejo por variables como la luz, los obstrucción del objeto,
%distorción, etc. Un modelo de aprendizaje profundo que haya sido entrenado con 
%un conjunto de imágenes de pelotas puede inferir después si hay o no una pelota
%en una nueva fotografía. Ese modelo puede entrenarse con nuevas imágenes
%de otros tipos de objetos. Sin embargo, como mencionamos, el robot NAO
%tiene un procesamiento limitado, el entrenamiento de un modelo, puede
%durar horas, días o semanas en un dispositivo con esas características.

%A partir de lo descrito 
%
%Integrar servicios sobre el robot nao.
%
%

%Desarrollar una arquitectura de software para la
%integración de cómputo en la nube con robots NAO.
%Desarrollar un servicio de cómputo en la nube que utilice
%algoritmos de aprendizaje automático, para robots NAO.
%\\
%
%La hipótesis es:Es posible crear y adecuar servicios de cómputo en la nube,
%para solucionar problemas de aplicación en la robótica móvil.
%El deep learning 
%El robot nao
%porque REST
%aplciaciones web y móviles
%servicios en la nube
%
%\end{document}