

\chapter*{Introducción}
\label{\detokenize{introduction:cloudnao-una-arquitectura-de-software-para-la-integracion-de-computo-en-la-nube-con-robots-nao}}\label{\detokenize{introduction:introduccion}}\label{\detokenize{introduction::doc}}
\addcontentsline{toc}{chapter}{Introducción}

% Interconnection of sensing and actuating devices providing the ability to share information across platforms through a unified framework, developing a common operating picture for enabling innovative applications. This is achieved by seamless ubiquitous sensing, data analytics and information representation with Cloud computing as the unifying framework.

% Actualmente existen diferentes plataformas que 
% ofrecen servicios
% de cómputo en la nube para solucionar problemas de 
% visión computacional \cite{googlevision2018} o
% procesamiento de lenguaje natural \cite{witaidocs2018} usando modelos
% de aprendizaje profundo. 
% %Éstas también incluyen servicios 
% % para entrenar y lanzar
% % modelos creados por los mismos usuarios \cite{tensorflowgooglecloud2018}.
% Todos estos recursos se entregan a través de 
% interfaces que
% permiten que casi cualquier dispositivo con 
% internet
% pueda acceder a éstos.
% El cómputo en la nube es un estilo de cómputo que 
% permite el acceso a 
% recursos informáticos a través de internet cuando un
% usuario lo demanda \cite{borkofurhtarmandoescalante2010}. El aprendizaje profundo es un 
% tipo
% de aprendizaje automático donde las computadoras 
% construyen
% conceptos complejos a partir de conceptos más 
% simples \cite{iangoodfellowyoshuabengioaaroncourville2017}.


% %¿Qué es lo que no sabemos?\\
% El cómputo en la nube es el marco de trabajo que
% unifica los elementos que permiten que exista el
% internet de las cosas \cite{jayavardhanagubbiarajkumarbuyyabslavenmarusicamarimuthupalaniswamia2013}. Queremos que el robot NAO 
% sea parte de esa interconexión de sensores y actuadores
% que comparten información entre plataformas
% a través de internet.
% NAO es un robot humanoide programable y autónomo
% que cuenta con
% un poder de procesamiento suficiente para realizar 
% tareas como
% jugar un partido de fútbol con otros robots \cite{splinfo2018}. Sin embargo,
% las aplicaciones modernas
% exigen que los dispositivos se adapten constantemente a nuevas necesidades
% de los usuarios, como el acceso a los datos
% del robot desde cualquier lugar, el uso del aprendizaje profundo para resolver problemas como la clasificación de imágenes,
% % en múltiples categorías
% encontrar
% la posición de objetos en una imagen, reconocimiento
% de personas y el 
% procesamiento de lenguaje natural de un discurso \cite{benkehoesachinpatilpieterabbeelkengoldberg2014}.

% %¿Qué pretendemos averiguar? (objetivo)

% Los objetivos de esta investigación son: desarrollar una 
% arquitectura de software para  la  
% integración  de  cómputo  en  la  nube  con  robots  NAO
% e implementar un  servicio  de  cómputo  en  la  nube  que  utilice  
% algoritmos  de  aprendizaje  automático y que pueda
% ser usado por robots NAO.
% Algunos recursos son brindados por las plataformas
% Google Cloud, Wit.ai, Kairos, Firebase y otros
% por el Laboratorio de Algoritmos para la
% Robótica.
% Los servicios en la nube permitirán al robot NAO
% realizar tareas como el procesamiento de lenguaje
% natural de un discurso oral, el procesamiento 
% de imágenes para reconocimiento de rostros
% y objetos, la clasificación de
% imágenes en escenarios, la conexión con el robot desde diferentes dispositivos y el acceso a datos generados por
% éste desde cualquier lugar.

El cómputo en la nube se ha convertido en una
tendencia tecnológica importante y se espera que este paradigma modifique los procesos de las tecnologías de
la información. Con el cómputo en la nube, los usuarios 
 utilizan una variedad de dispositivos como 
 computadoras personales y teléfonos inteligentes,
 para acceder a programas, almacenamiento y plataformas
 para el desarrollo de aplicaciones a través de internet,
 usando servicios ofrecidos por proveedores de cómputo
 en la nube \cite{borkofurhtarmandoescalante2010}. 
%  Las ventajas de esta tecnología 
%  incluyen ahorro en el costo, alta disponibilidad y fácil
%  escalabilidad .
 
 
El cómputo en la nube puede definirse como un modelo 
que permite el acceso ubicuo, conveniente y bajo demanda, 
a un conjunto compartido de recursos informáticos 
configurables (por ejemplo, redes, servidores, 
almacenamiento, aplicaciones y servicios) que se pueden 
unir y lanzar rápidamente con un mínimo esfuerzo de 
gestión o interacción con el proveedor de servicios \cite{mell_peter_tim_2011}.


Existen diferentes plataformas 
de cómputo en la nube que 
ofrecen servicios de aprendizaje profundo
para la solución de problemas de visión
computacional \cite{googlevision2018} o
procesamiento de lenguaje natural \cite{witaidocs2018}.
El aprendizaje profundo es un 
tipo de aprendizaje automático donde las computadoras 
construyen conceptos complejos a partir de conceptos más 
simples \cite{iangoodfellowyoshuabengioaaroncourville2017}.
Todos estos recursos se entregan a través de 
interfaces que
permiten que casi cualquier dispositivo con internet, un robot
por ejemplo, pueda acceder a éstos.


Un robot es un dispositivo mecánico versátil (por ejemplo,
una brazo manipulador, una mano con múltiples dedos 
articulados, un vehículo con piernas o llantas, una
plataforma de vuelo, o una combinación de estos)
equipado con actuadores y sensores bajo el control
de un sistema computacional. 
Éste opera en un espacio de trabajo en el mundo real.
Este espacio de trabajo está poblado con objetos físicos
y sujeto a las leyes de la naturaleza.
El robot realiza tareas ejecutando movimientos en el 
espacio de trabajo \cite{latombe_1993}.
Los robots autónomos son máquinas inteligentes
con la habilidad de realizar tareas por ellos
mismos, sin el control explícito del humano \cite{bekey_2017}.
Algunos de los retos más importantes 
a los que se enfrenta el área de estudio de los
robots autónomos recaen en la planeación automática de movimiento. 
El objetivo es poder especificar una tarea en un lenguaje de alto nivel y hacer que el robot compile automáticamente esta especificación
en un conjunto de movimientos primitivos de bajo nivel, o controladores de retroalimentación, para realizar la tarea. 
La tarea prototípica es encontrar el camino 
para un robot, ya sea un brazo robótico o un robot móvil,
de un punto a otro evitando obstáculos.
Existen cuatro tareas fundamentales dentro 
de la planeación de movimiento: navegación, cobertura, localización y mapeo.
La navegación es el problema de encontrar un movimiento
libre de colisiones de un estado del robot a otro.
La cobertura es el problema de pasar un sensor o
herramienta sobre todos los puntos en un espacio.
La localización es el problema de utilizar un mapa
para interpretar la información de los sensores para
determinar el estado del robot. El mapeo es el problema de 
explorar y sensar un ambiente desconocido. La combinación
de localización y mapeo es conocido como SLAM (mapeo y localización simultáneos)  \cite{choset_2005}.

El Laboratorio de Algoritmos para la Robótica (LAR) es un 
grupo de investigación perteneciente a la Facultad de Estudios
Superiores (FES) Acatlán, de la Universidad Nacional 
Autónoma de México (UNAM), que se dedica, entre otras cosas, 
al diseño de algoritmos con aplicación en la planeación de 
movimientos de robots móviles. El LAR cuenta con un robot 
NAO, un robot humanoide programable y autónomo, conocido
por ser la plataforma utilizada en la competencia de RoboCup Standard Platform League \cite{splinfo2018}. 
Debido a las 
características del hardware (véanse las especificaciones 
técnicas en \cite{aldebaranrobotics}) y a que sus componentes 
no son personalizables, es una plataforma con recursos 
bastante limitados.


% Utilizando esta plataforma el laboratorio ha participado
% en torneos de robótica en la categoría RoboCup Standard 
% Platform League \cite{tmr2017}, una competencia de fútbol 
% en la que todos los equipos participantes utilizan al robot NAO y éste juega de manera autónoma.
%¿Qué es lo que no sabemos?\\

Trabajos actuales como 
\cite{benkehoesachinpatilpieterabbeelkengoldberg2014}
indican que el cómputo en la nube tienen el potencial para
proveer beneficios significativos a robots y
sistemas autónomos.
Además, el cómputo en la nube es el marco de trabajo que
unifica los elementos que permiten que exista el
internet de las cosas \cite{jayavardhanagubbiarajkumarbuyyabslavenmarusicamarimuthupalaniswamia2013}. Utilizar este modelo de cómputo sobre el robot NAO permite
que forme parte de esa interconexión de sensores y actuadores que comparten información entre plataformas
a través de internet. Adicionalmente, los proveedores de servicios de 
computación en la nube brindan herramientas para el acceso 
a los datos del robot desde cualquier lugar y el uso de 
aprendizaje profundo para resolver problemas 
como el reconocimiento de rostros, detección de objetos en
imágenes, procesamiento de lenguaje natural e incluso el 
problema de SLAM \cite{tateno2017cnn}.
Estos servicios en la nube permiten que
todo el sensado, cálculo y memoria no estén
integrados en un solo sistema, lo que permite
extender las capacidades del robot NAO.
%¿Qué pretendemos averiguar? (objetivo)

Este trabajo de investigación tiene dos objetivos principales:

\begin{itemize}
    \item Desarrollar una arquitectura de software para la
integración de cómputo en la nube con robots NAO.
    \item Desarrollar un servicio de cómputo en la nube que
utilice algoritmos de aprendizaje automático, para robots NAO.
\end{itemize}

Para cumplir el primer objetivo se creó una arquitectura, llamada \textit{CloudNAO}, diseñada para que el robot sea capaz de utilizar los servicios
de las plataformas: Google Cloud \cite{googlevision2018}, Wit.ai \cite{witaidocs2018}, Kairos \cite{kairosdevdocs2018}, Firebase \cite{firebasedocs2018}
y los lanzados por el LAR.
Los servicios en la nube permitirán al robot NAO
realizar tareas como el procesamiento de lenguaje
natural de un discurso oral, el procesamiento 
de imágenes para reconocimiento de rostros
y objetos, la clasificación de
imágenes en escenarios, la conexión con el robot desde diferentes dispositivos y el acceso a datos generados por
éste en cualquier momento y desde cualquier lugar.

Para cubrir el segundo objetivo se desarrolló 
una red neuronal convolucional, tal que pudiera ser consumida
como un servicio. Esta red neuronal profunda es un clasificador
que sirve como herramienta auxiliar a una pequeña parte de la solución 
del problema de localización. Aquí el principal objetivo
no es la calidad ni el desempeño del clasificador,
sino la implementación de este servicio para ser consumido
por cualquier robot NAO.

En el Capítulo \ref{\detokenize{chapter_one::doc}} de este 
documento se describen las tecnologías sobre las que está
basada la arquitectura de software propuesta como son: el 
robot NAO y su marco de trabajo para desarrollo, la 
arquitectura REST (una arquitectura de software que provee 
estándares entre sistemas computacionales en la web, 
facilitando la comunicación entro ellos), 
el cómputo en la nube, la plataforma de Backend as a Service 
Firebase, ejemplos de servicios utilizados dentro de la 
arquitectura y además de conceptos fundamentales del 
aprendizaje automático y redes neuronales, esto último para la
implementación del servicio.

En el Capítulo \ref{\detokenize{chapter_two::doc}} se 
desarrolla la arquitectura de software
CloudNAO. Se detalla cada uno de los elementos de la arquitectura, sus funcionalidades, implementación y relación
entre ellos.
Entre los elementos que integran la arquitectura se 
encuentran: el robot NAO, una API REST la cual maneja los 
servicios que consume el robot, una aplicación móvil, una 
aplicación web y Firebase.
Lo descrito
en esta parte del trabajo sirve 
como documentación tanto como para desarrolladores como
para usuarios finales. Además, al final de este capítulo
se incluyen algunos casos de estudio en los que
explotamos algunas de las funcionalidades 
ofrecidas dentro de la arquitectura.


En el Capítulo \ref{\detokenize{chapter_three::doc}} se 
desarrolla un servicio de cómputo en la nube,
que emplea algoritmos de aprendizaje profundo, para ser 
utilizado sobre la arquitectura de software CloudNAO.
Aquí se especifica cómo se cumplió el segundo
objetivo de la tesis. Se describe desde el diseño de
la red neuronal convolucional, su implementación, 
la realización de algunos experimentos, la evaluación
del modelo, su integración a la arquitectura CloudNAO
y algunas pruebas sobre el robot NAO.
