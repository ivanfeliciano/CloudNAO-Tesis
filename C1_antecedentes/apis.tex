


\section{Interfaces de Programación de Aplicaciones de Transferencia de Estado Representacional}
\label{\detokenize{chapter_one/apis_rest:interfaces-de-programacion-de-aplicaciones-de-transferencia-de-estado-representacional}}\label{\detokenize{chapter_one/apis_rest::doc}}
Para entender mejor el concepto de una API REST conviene poner como ejemplos
los servicios web RESTful que se utilizan en este proyecto. Se accede
a esos servicios a través de sus API REST y se enlistan a continuación:
\begin{itemize}
\item {} 
Google Cloud Vision

\item {} 
Kairos

\item {} 
Wit.ai

\item {} 
Firebase Realtime Database

\end{itemize}


\subsection{API de Google Cloud Vision}
\label{\detokenize{chapter_one/apis_rest:api-de-google-cloud-vision}}
La API Vision de Google Cloud permite que los desarrolladores comprendan el
contenido de una imagen mediante el encapsulado de potentes modelos de
aprendizaje automático en una API REST fácil de usar. La API clasifica
imágenes rápidamente en miles de categorías (por ejemplo, «barco de vela»,
«león» o «torre Eiffel»), detecta objetos y caras individuales dentro de las
imágenes, además de buscar y leer palabras impresas en ellas.

El recurso que se solicita para el procesamiento de imagenes es \sphinxcode{images} y
el URL que lo identifica es \sphinxcode{https://vision.googleapis.com/v1/images:annotate}.
La API únicamente tiene definida la operación con el método HTTP POST,
por lo que hacer un POST a la URL del recurso ejecuta la detección de algunas
características en una o varias imágenes.

El cuerpo del mensaje del petición HTTP es un JSON con la siguente estructura
(por simplicidad se omiten varios objetos del JSON que no se utilizaron en el
proyecto):

\begin{sphinxVerbatim}[commandchars=\\\{\}]
\PYG{p}{\PYGZob{}}
  \PYG{l+s+s2}{\PYGZdq{}requests\PYGZdq{}} \PYG{o}{:} \PYG{p}{[}
    \PYG{p}{\PYGZob{}}
      \PYG{l+s+s2}{\PYGZdq{}image\PYGZdq{}} \PYG{o}{:} \PYG{p}{\PYGZob{}}
        \PYG{l+s+s2}{\PYGZdq{}content\PYGZdq{}} \PYG{o}{:} \PYG{n+nx}{string}\PYG{p}{,}
        \PYG{l+s+s2}{\PYGZdq{}source\PYGZdq{}} \PYG{o}{:} \PYG{p}{\PYGZob{}}
          \PYG{l+s+s2}{\PYGZdq{}gcsImageUri\PYGZdq{}} \PYG{o}{:} \PYG{n+nx}{string}\PYG{p}{,}
          \PYG{l+s+s2}{\PYGZdq{}imageUri\PYGZdq{}} \PYG{o}{:} \PYG{n+nx}{string}
        \PYG{p}{\PYGZcb{}}
      \PYG{p}{\PYGZcb{}}
      \PYG{l+s+s2}{\PYGZdq{}features\PYGZdq{}} \PYG{o}{:} \PYG{p}{[}
        \PYG{p}{\PYGZob{}}
          \PYG{l+s+s2}{\PYGZdq{}type\PYGZdq{}} \PYG{o}{:} \PYG{n+nx}{string}\PYG{p}{,}
          \PYG{l+s+s2}{\PYGZdq{}maxResults\PYGZdq{}} \PYG{o}{:} \PYG{n+nx}{number}
        \PYG{p}{\PYGZcb{}}
      \PYG{p}{]}
    \PYG{p}{\PYGZcb{}}
  \PYG{p}{]}
\PYG{p}{\PYGZcb{}}
\end{sphinxVerbatim}

\sphinxcode{requests} es una lista de objetos, donde cada objeto representa una imagen
y las caracterísiticas que se piden detectar. \sphinxcode{image} es la representación
de la imagen, que puede estar en tres formas distintas: codificada en base 64,
guardada en el servicio de almacenamiento en la nuble de Google Cloud Storage
o a través de un enlace en algún servidor. Para el primer caso a \sphinxcode{content}
se le asigna la image codificada; en cambio, si está en un contenedor de Google Cloud Storage
entonces se pasa a \sphinxcode{gcsImageUri} el valor de la URL de la imagen. Si se encuentra
en algún otro servicio de almacenamiento simplemente el valor de \sphinxcode{imageUri}
es el URL de la imagen.

\sphinxcode{features} es un arreglo de las caracterísiticas que se pretenden reconocer
en la imagen. Por cada objeto se tiene \sphinxcode{type} que es el tipo de caracterísitica
y \sphinxcode{maxResults} que es el número máximo de ese tipo. \sphinxcode{type} puede tomar
los siguientes valores:
\begin{itemize}
\item {} 
\sphinxcode{TYPE\_UNSPECIFIED}: No se específica la característica.

\item {} 
\sphinxcode{FACE\_DETECTION}: Ejecuta la detección de rostros.

\item {} 
\sphinxcode{LANDMARK\_DETECTION}: Ejecuta la detección de puntos de referencia.

\item {} 
\sphinxcode{LOGO\_DETECTION}: Ejecuta la detección de logotipos.

\item {} 
\sphinxcode{LABEL\_DETECTION}: Ejecuta la detección de etiquetas.

\item {} 
\sphinxcode{TEXT\_DETECTION}: Ejecuta el reconocimiento óptico de caracteres (OCR).

\item {} 
\sphinxcode{DOCUMENT\_TEXT\_DETECTION}: Ejecuta el OCR sobre grandes documentos de texto.

\item {} 
\sphinxcode{SAFE\_SEARCH\_DETECTION}: Ejecuta la búqueda segura para detectar contenido pontencialmente inseguro o no deseable.

\item {} 
\sphinxcode{IMAGE\_PROPERTIES}: Calcula un conjunto de propiedades en una imagen, como los colores dominantes.

\item {} 
\sphinxcode{CROP\_HINTS}: Ejecuta la sugerencia de recortes de una imagen.

\item {} 
\sphinxcode{WEB\_DETECTION}: Ejecuta la detección web.

\end{itemize}

De todas estas posibles caracterísiticas, las que se utilizan dentro de la arquitectura
CloudNAO son \sphinxcode{TEXT\_DETECTION} y \sphinxcode{LABEL\_DETECTION}.

El cuerpo del mensaje de respuesta, si todo salió bien (se obtiene una código
de estado 200), contiene información
con la siguiente estructura:

\begin{sphinxVerbatim}[commandchars=\\\{\}]
\PYG{p}{\PYGZob{}}
  \PYG{l+s+s2}{\PYGZdq{}responses\PYGZdq{}}\PYG{o}{:} \PYG{p}{[}
    \PYG{p}{\PYGZob{}}
        \PYG{l+s+s2}{\PYGZdq{}labelAnnotations\PYGZdq{}}\PYG{o}{:} \PYG{p}{[}
          \PYG{p}{\PYGZob{}}
            \PYG{l+s+s2}{\PYGZdq{}mid\PYGZdq{}}\PYG{o}{:} \PYG{n+nx}{string}\PYG{p}{,}
            \PYG{l+s+s2}{\PYGZdq{}description\PYGZdq{}}\PYG{o}{:} \PYG{n+nx}{string}\PYG{p}{,}
            \PYG{l+s+s2}{\PYGZdq{}score\PYGZdq{}}\PYG{o}{:} \PYG{n+nx}{number}
          \PYG{p}{\PYGZcb{}}
        \PYG{p}{]}\PYG{p}{,}
        \PYG{l+s+s2}{\PYGZdq{}textAnnotations\PYGZdq{}}\PYG{o}{:} \PYG{p}{[}
          \PYG{p}{\PYGZob{}}
              \PYG{l+s+s2}{\PYGZdq{}mid\PYGZdq{}}\PYG{o}{:} \PYG{n+nx}{string}\PYG{p}{,}
              \PYG{l+s+s2}{\PYGZdq{}locale\PYGZdq{}}\PYG{o}{:} \PYG{n+nx}{string}\PYG{p}{,}
              \PYG{l+s+s2}{\PYGZdq{}score\PYGZdq{}}\PYG{o}{:} \PYG{n+nx}{number}\PYG{p}{,}
              \PYG{l+s+s2}{\PYGZdq{}boundingPoly\PYGZdq{}}\PYG{o}{:} \PYG{p}{\PYGZob{}}
                \PYG{l+s+s2}{\PYGZdq{}vertices\PYGZdq{}}\PYG{o}{:} \PYG{p}{[}
                  \PYG{p}{\PYGZob{}}
                    \PYG{l+s+s2}{\PYGZdq{}x\PYGZdq{}}\PYG{o}{:} \PYG{n+nx}{number}\PYG{p}{,}
                    \PYG{l+s+s2}{\PYGZdq{}y\PYGZdq{}}\PYG{o}{:} \PYG{n+nx}{number}
                  \PYG{p}{\PYGZcb{}}
                \PYG{p}{]}
              \PYG{p}{\PYGZcb{}}
          \PYG{p}{\PYGZcb{}}
        \PYG{p}{]}
    \PYG{p}{\PYGZcb{}}
  \PYG{p}{]}
\PYG{p}{\PYGZcb{}}
\end{sphinxVerbatim}

La respuesta obtenida depende de que caracterísiticas se solicitaron,
para el JSON anterior suponemos que se pidió la detección de
categorías y el reconocimiento de texto. \sphinxcode{mid} es un identificador,
\sphinxcode{description} es la descripción textual de la entidad; por ejemplo,
para el caso de \sphinxcode{labelAnnotations}, \sphinxcode{description} puede tener un valor
asociado igual a «psychedelic art», en cambio para \sphinxcode{textAnnotations},
se espera tenga una cadena del texto encontrado como «lateralus».

El \sphinxcode{score} es la precisión de la detección de la entidad en la imagen,
toma valores en el intervalo {[}0, 1{]}. \sphinxcode{boundingPoly} contiene
las coordenadas del polígono que encierra la entidad encontrada.


\subsection{API de Kairos}
\label{\detokenize{chapter_one/apis_rest:api-de-kairos}}
Kairos permite a los desarrolladores integrar un análisis de rostros preciso
y rápido en cualquier aplicación o servicio. Cuenta con un API REST
para ejecutar tareas en la nube como la detección e identificación de rostros,
emociones, raza, edad, entre otras. También cuenta con un SDK, más limitado
pero que puede ejecutarse offline y directamente en dispositivos móviles.
Por ahora sólo nos interesa la API REST, por lo que describiremos cómo funciona,
es decir, cómo se debe hacer una petición y cuál es la respuesta
que envía el servicio.

Para realizar el reconocimiento de caras, la API tiene como URL base
\sphinxcode{https://api.kairos.com} y cuenta con diversos recursos para cada tipo de
reconocimiento. Únicamente nos interesan dos recursos, \sphinxcode{enroll} y
\sphinxcode{recognize}, que son relativos al URL base.
El primero toma una fotografía, encuentra rostros en esta,
y los guarda en una galería, cada rostro necesita un identificador de la
persona. Las galerías sirven para que posteriormente se reconozca a un sujeto
por medio de su fotografía.

El otro recurso que nos interesa es \sphinxcode{recognize}, que toma una fotografía,
encuentra caras e intenta emparejarlas con las que sean guardado
previamente en una galería.

Para hacer ambas peticiones se utiliza el método POST. Al hacer un POST
a \sphinxcode{enroll} se debe enviar un JSON con la siguiente estructura en el cuerpo
del mensaje.

\begin{sphinxVerbatim}[commandchars=\\\{\}]
\PYG{p}{\PYGZob{}}
  \PYG{l+s+s2}{\PYGZdq{}image\PYGZdq{}}\PYG{o}{:} \PYG{n+nx}{string}\PYG{p}{,}
  \PYG{l+s+s2}{\PYGZdq{}subject\PYGZus{}id\PYGZdq{}}\PYG{o}{:} \PYG{n+nx}{string}\PYG{p}{,}
  \PYG{l+s+s2}{\PYGZdq{}gallery\PYGZus{}name\PYGZdq{}}\PYG{o}{:} \PYG{n+nx}{string}
\PYG{p}{\PYGZcb{}}
\end{sphinxVerbatim}

\sphinxcode{image} es un URL de acceso público o la codificación en base 64, \sphinxcode{subject\_id} es un identificador
de una cara, es único y es definido por el desarrollador, \sphinxcode{gallery\_name}
definida también por el desarrollador sirve para identificar la galería.

La respuesta de un POST a \sphinxcode{/enroll}, si todo fue exitoso
es un JSON con la siguiente estructura.

\begin{sphinxVerbatim}[commandchars=\\\{\}]
\PYG{p}{\PYGZob{}}
  \PYG{l+s+s2}{\PYGZdq{}face\PYGZus{}id\PYGZdq{}}\PYG{o}{:} \PYG{n+nx}{string}\PYG{p}{,}
  \PYG{l+s+s2}{\PYGZdq{}images\PYGZdq{}}\PYG{o}{:} \PYG{p}{[}
    \PYG{p}{\PYGZob{}}
      \PYG{l+s+s2}{\PYGZdq{}attributes\PYGZdq{}}\PYG{o}{:} \PYG{p}{\PYGZob{}}
        \PYG{l+s+s2}{\PYGZdq{}lips\PYGZdq{}}\PYG{o}{:} \PYG{n+nx}{string}\PYG{p}{,}
        \PYG{l+s+s2}{\PYGZdq{}asian\PYGZdq{}}\PYG{o}{:} \PYG{n+nx}{number}\PYG{p}{,}
        \PYG{l+s+s2}{\PYGZdq{}gender\PYGZdq{}}\PYG{o}{:} \PYG{p}{\PYGZob{}}
          \PYG{l+s+s2}{\PYGZdq{}type\PYGZdq{}}\PYG{o}{:} \PYG{n+nx}{string}
        \PYG{p}{\PYGZcb{}}\PYG{p}{,}
        \PYG{l+s+s2}{\PYGZdq{}age\PYGZdq{}}\PYG{o}{:} \PYG{n+nx}{number}\PYG{p}{,}
        \PYG{l+s+s2}{\PYGZdq{}hispanic\PYGZdq{}}\PYG{o}{:} \PYG{n+nx}{number}\PYG{p}{,}
        \PYG{l+s+s2}{\PYGZdq{}other\PYGZdq{}}\PYG{o}{:} \PYG{n+nx}{number}\PYG{p}{,}
        \PYG{l+s+s2}{\PYGZdq{}black\PYGZdq{}}\PYG{o}{:} \PYG{n+nx}{number}\PYG{p}{,}
        \PYG{l+s+s2}{\PYGZdq{}white\PYGZdq{}}\PYG{o}{:} \PYG{n+nx}{number}\PYG{p}{,}
        \PYG{l+s+s2}{\PYGZdq{}glasses\PYGZdq{}}\PYG{o}{:} \PYG{n+nx}{string}
      \PYG{p}{\PYGZcb{}}\PYG{p}{,}
      \PYG{l+s+s2}{\PYGZdq{}transaction\PYGZdq{}}\PYG{o}{:} \PYG{p}{\PYGZob{}}
        \PYG{l+s+s2}{\PYGZdq{}status\PYGZdq{}}\PYG{o}{:} \PYG{n+nx}{string}\PYG{p}{,}
        \PYG{l+s+s2}{\PYGZdq{}topLeftX\PYGZdq{}}\PYG{o}{:} \PYG{n+nx}{number}\PYG{p}{,}
        \PYG{l+s+s2}{\PYGZdq{}topLeftY\PYGZdq{}}\PYG{o}{:} \PYG{n+nx}{number}\PYG{p}{,}
        \PYG{l+s+s2}{\PYGZdq{}gallery\PYGZus{}name\PYGZdq{}}\PYG{o}{:} \PYG{n+nx}{string}\PYG{p}{,}
        \PYG{l+s+s2}{\PYGZdq{}timestamp\PYGZdq{}}\PYG{o}{:} \PYG{n+nx}{string}\PYG{p}{,}
        \PYG{l+s+s2}{\PYGZdq{}height\PYGZdq{}}\PYG{o}{:} \PYG{n+nx}{number}\PYG{p}{,}
        \PYG{l+s+s2}{\PYGZdq{}quality\PYGZdq{}}\PYG{o}{:} \PYG{n+nx}{number}\PYG{p}{,}
        \PYG{l+s+s2}{\PYGZdq{}confidence\PYGZdq{}}\PYG{o}{:} \PYG{n+nx}{number}\PYG{p}{,}
        \PYG{l+s+s2}{\PYGZdq{}subject\PYGZus{}id\PYGZdq{}}\PYG{o}{:} \PYG{n+nx}{string}\PYG{p}{,}
        \PYG{l+s+s2}{\PYGZdq{}width\PYGZdq{}}\PYG{o}{:} \PYG{n+nx}{number}\PYG{p}{,}
        \PYG{l+s+s2}{\PYGZdq{}face\PYGZus{}id\PYGZdq{}}\PYG{o}{:} \PYG{n+nx}{number}
      \PYG{p}{\PYGZcb{}}
    \PYG{p}{\PYGZcb{}}
  \PYG{p}{]}
\PYG{p}{\PYGZcb{}}
\end{sphinxVerbatim}

Como se puede ver, los elementos del JSON son autodescriptivos. Para el caso
del POST a \sphinxcode{/recognize}, el cuerpo de la petición lleva un JSON como el
siguiente:

\begin{sphinxVerbatim}[commandchars=\\\{\}]
\PYG{p}{\PYGZob{}}
  \PYG{l+s+s2}{\PYGZdq{}image\PYGZdq{}}\PYG{o}{:} \PYG{n+nx}{string}\PYG{p}{,}
  \PYG{l+s+s2}{\PYGZdq{}gallery\PYGZus{}name\PYGZdq{}}\PYG{o}{:} \PYG{n+nx}{string}
\PYG{p}{\PYGZcb{}}
\end{sphinxVerbatim}

La respuesta enviada por Kairos,  si se todo fue exitoso,
es un JSON con campos autodescriptivos como los
de la respuesta del recurso previo. Este JSON tiene se muestra a continuación.

\begin{sphinxVerbatim}[commandchars=\\\{\}]
\PYG{p}{\PYGZob{}}
  \PYG{l+s+s2}{\PYGZdq{}images\PYGZdq{}}\PYG{o}{:} \PYG{p}{[}
    \PYG{p}{\PYGZob{}}
      \PYG{l+s+s2}{\PYGZdq{}transaction\PYGZdq{}}\PYG{o}{:} \PYG{p}{\PYGZob{}}
        \PYG{l+s+s2}{\PYGZdq{}status\PYGZdq{}}\PYG{o}{:} \PYG{n+nx}{string}\PYG{p}{,}
        \PYG{l+s+s2}{\PYGZdq{}width\PYGZdq{}}\PYG{o}{:} \PYG{n+nx}{number}\PYG{p}{,}
        \PYG{l+s+s2}{\PYGZdq{}topLeftX\PYGZdq{}}\PYG{o}{:} \PYG{n+nx}{number}\PYG{p}{,}
        \PYG{l+s+s2}{\PYGZdq{}topLeftY\PYGZdq{}}\PYG{o}{:} \PYG{n+nx}{number}\PYG{p}{,}
        \PYG{l+s+s2}{\PYGZdq{}gallery\PYGZus{}name\PYGZdq{}}\PYG{o}{:} \PYG{n+nx}{string}\PYG{p}{,}
        \PYG{l+s+s2}{\PYGZdq{}face\PYGZus{}id\PYGZdq{}}\PYG{o}{:} \PYG{n+nx}{number}\PYG{p}{,}
        \PYG{l+s+s2}{\PYGZdq{}confidence\PYGZdq{}}\PYG{o}{:} \PYG{n+nx}{number}\PYG{p}{,}
        \PYG{l+s+s2}{\PYGZdq{}subject\PYGZus{}id\PYGZdq{}}\PYG{o}{:} \PYG{n+nx}{string}\PYG{p}{,}
        \PYG{l+s+s2}{\PYGZdq{}height\PYGZdq{}}\PYG{o}{:} \PYG{n+nx}{number}\PYG{p}{,}
        \PYG{l+s+s2}{\PYGZdq{}quality\PYGZdq{}}\PYG{o}{:} \PYG{n+nx}{number}
      \PYG{p}{\PYGZcb{}}\PYG{p}{,}
      \PYG{l+s+s2}{\PYGZdq{}candidates\PYGZdq{}}\PYG{o}{:} \PYG{p}{[}
        \PYG{p}{\PYGZob{}}
          \PYG{l+s+s2}{\PYGZdq{}subject\PYGZus{}id\PYGZdq{}}\PYG{o}{:} \PYG{n+nx}{string}\PYG{p}{,}
          \PYG{l+s+s2}{\PYGZdq{}confidence\PYGZdq{}}\PYG{o}{:} \PYG{n+nx}{number}\PYG{p}{,}
          \PYG{l+s+s2}{\PYGZdq{}enrollment\PYGZus{}timestamp\PYGZdq{}}\PYG{o}{:} \PYG{n+nx}{string}
        \PYG{p}{\PYGZcb{}}\PYG{p}{,}
        \PYG{p}{\PYGZob{}}
          \PYG{l+s+s2}{\PYGZdq{}subject\PYGZus{}id\PYGZdq{}} \PYG{o}{:} \PYG{n+nx}{string}\PYG{p}{,}
          \PYG{l+s+s2}{\PYGZdq{}confidence\PYGZdq{}} \PYG{o}{:} \PYG{n+nx}{numref}\PYG{p}{,}
          \PYG{l+s+s2}{\PYGZdq{}enrollment\PYGZus{}timestamp\PYGZdq{}}\PYG{o}{:} \PYG{n+nx}{string}
        \PYG{p}{\PYGZcb{}}
      \PYG{p}{]}
    \PYG{p}{\PYGZcb{}}
  \PYG{p}{]}
\PYG{p}{\PYGZcb{}}
\end{sphinxVerbatim}


\subsection{API de Wit.ai}
\label{\detokenize{chapter_one/apis_rest:api-de-wit-ai}}
Wit.ai es una API para procesamiento de lenguaje natural capaz de convertir
oraciones en datos estructurados. Esto quiere decir que puedes crear bots
que interactúen con humanos en plataformas de mensajería. Con Wit.ai los
desarrolladores pueden contruir aplicaciones a la que se les pueda hablar o
escribir. Los bots aprenden conforme se les hable, se vuelven más inteligentes
en cada interacción.

Wit.ai, como muchos servicios, cuenta con diferentes SDK para lenguajes como
JavaScript, Python y Ruby. También tiene una API REST para extraer el significado
de una sentencia o de un discurso de audio. Por ahora nos enfocaremos
al segundo caso, cómo solicitar y cuál es la respuesta al enviar audio
para su procesamiento.

Wit trabaja con \sphinxstyleemphasis{intenciones} y \sphinxstyleemphasis{entidades}. Las intenciones se usan para entender
el significado en general de una oración. Por ejemplo la intención
de la oración «¿A qué hora abren la bibloteca central de CU?»
puede ser las horas de apertura de un lugar. Las entidades
proveen más información para contestar correctamente. Siguiendo el ejemplo,
el usuario no está preguntando a qué hora abren todas las bibliotecas,
sino en específico la biblioteca central de CU, por lo que
una entidad sería el lugar del que se desear obtener la hora de apertura.

La API cuenta con el URL base \sphinxcode{htttps://api.wit.ai/} y el recurso que
solicitamos es \sphinxcode{/speech}. A través de un POST a \sphinxcode{htttps://api.wit.ai/speech}
podemos obtener el significado de un archivo de audio.

A diferencia de las otras API donde no necesariamente tenemos que especificar
un encabezado en el mensaje de petición HTTP, aquí es necesario especificar
el formato del archivo de audio. Se debe pasar como valor al campo
\sphinxcode{Content-type} cualquiera de los siguientes:
\begin{itemize}
\item {} 
\sphinxcode{audio/wav}: si envias un archivo wav.

\item {} 
\sphinxcode{audio/mpeg3}: para un archivo mp3.

\item {} 
\sphinxcode{audio/ulaw}: para archivo G.11 u-law.

\end{itemize}

Wit.ai únicamente procesa audio en formato monoaural no estéreo.

En el cuerpo del mensaje va la información binaria, es decir,
el archivo de audio. La respuesta del servidor
es un JSON con la siguiente estructura:

\begin{sphinxVerbatim}[commandchars=\\\{\}]
\PYG{p}{\PYGZob{}}
  \PYG{l+s+s2}{\PYGZdq{}msg\PYGZus{}id\PYGZdq{}}\PYG{o}{:} \PYG{n+nx}{string}\PYG{p}{,}
  \PYG{l+s+s2}{\PYGZdq{}\PYGZus{}text\PYGZdq{}}\PYG{o}{:} \PYG{n+nx}{string}\PYG{p}{,}
  \PYG{l+s+s2}{\PYGZdq{}entities\PYGZdq{}}\PYG{o}{:} \PYG{p}{\PYGZob{}}
    \PYG{n+nx}{string} \PYG{o}{:} \PYG{p}{[} \PYG{p}{\PYGZob{}}
      \PYG{l+s+s2}{\PYGZdq{}value\PYGZdq{}}\PYG{o}{:} \PYG{n+nx}{string}\PYG{p}{,}
      \PYG{l+s+s2}{\PYGZdq{}confidence\PYGZdq{}}\PYG{o}{:} \PYG{n+nx}{number}
    \PYG{p}{\PYGZcb{}} \PYG{p}{]}\PYG{p}{,}
    \PYG{l+s+s2}{\PYGZdq{}intent\PYGZdq{}}\PYG{o}{:} \PYG{p}{\PYGZob{}}
      \PYG{l+s+s2}{\PYGZdq{}value\PYGZdq{}}\PYG{o}{:} \PYG{n+nx}{string}\PYG{p}{,}
      \PYG{l+s+s2}{\PYGZdq{}confidence\PYGZdq{}}\PYG{p}{;} \PYG{n+nx}{number}
    \PYG{p}{\PYGZcb{}}
  \PYG{p}{\PYGZcb{}}
\PYG{p}{\PYGZcb{}}
\end{sphinxVerbatim}

\sphinxcode{msg\_id} es el id de la respuesta. \sphinxcode{\_text} es la trancripción
del discurso de audio en texto. \sphinxcode{entities} es un objeto con las
entidades e intenciones que se encontraron. Si es una intención
el objeto tiene como llave la cadena \sphinxcode{"intent"} y como
valores los campos \sphinxcode{value} y \sphinxcode{confidence}, donde el primero
es el valor con el que se definió la intención y el segundo es la
probabilidad de que sea tal intención. Si es una
entidad la llave del objeto es el nombre de la entidad y como valor
tiene un arreglo de las diferentes apariciones que tiene
esa entidad en el mensaje. Cada entidad tiene los campos de
\sphinxcode{value} y \sphinxcode{confidence}.


\subsection{API REST de Firebase Database}
\label{\detokenize{chapter_one/apis_rest:api-rest-de-firebase-database}}
La base de datos en tiempo real de Firebase cuenta con una API REST
para realizar operaciones para leer, escribir, modificar o eliminar datos.

Se puede utilizar cualquier URL de la base de datos de Firebase como un endpoint
REST. Todo lo que se necesita es añadir la extensión \sphinxcode{.json} al final del
URL y enviar una petición desde cualquier cliente HTTP.

Para poder utilizar la API REST, el mapeo de los métodos
HTTP y las operaciones sobre Firebase Realtime Database es el siguiente.
\begin{itemize}
\item {} 
\sphinxstylestrong{GET}, para leer información previamente almacenada en la base de datos.

\item {} 
\sphinxstylestrong{PUT}, para escribir datos sobre una ubicación de la base de datos. Si existen datos los sobreescribe. Equivalente a \sphinxcode{set()} de los SDK.

\item {} 
\sphinxstylestrong{POST}, para añadir información en la base de datos. Se asigna un identificador único generado por Firebase. Es equivalente al \sphinxcode{push()} de los SDK.

\item {} 
\sphinxstylestrong{PATCH}, para actualizar datos en una ubicación sin sobreescribir. Equivalente a \sphinxcode{update()} de los SDK.

\item {} 
\sphinxstylestrong{DELETE}, elimina los datos en la ubicación dada.

\end{itemize}

La lista anterior muestra los métodos de lectura y escritura sobre la base
de datos, sin embargo, la API REST soportan el protocolo \sphinxstylestrong{Server-sent events (SSE)},
que define una API para recibir notificaciones push desde un servidor a través
de una conexión HTTP.

Para recibir las actualizaciones sobre una ubicación en la base de datos se
necesitan hacer tres cosas:
\begin{itemize}
\item {} 
Configurar el header \sphinxcode{Accept} igual a \sphinxcode{text/event-stream}. Del lado del cliente.

\item {} 
Respetar los redireccionamientos HTTP.

\item {} 
Si la ubicación de la base de datos requiere permisos, incluir el parámetro \sphinxcode{auth}.

\end{itemize}
