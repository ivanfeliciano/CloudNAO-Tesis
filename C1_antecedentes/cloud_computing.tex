\section{Cómputo en la nube}
\label{\detokenize{chapter_one/cloud_computing:computo-en-la-nube}}\label{\detokenize{chapter_one/cloud_computing::doc}}

\begin{remark}
Según el Instituto Nacional de Estándares y Tecnología (NIST por sus siglas en inglés) de Estados Unidos;
el cómputo en la
nube se define como un modelo que hace posible el acceso ubicuo, conveniente
y bajo demanda a un grupo de recursos informáticos configurables
(p.ej. servidores, almacenamiento, redes, aplicaciones, y servicios) a través
de una red donde pueden ser rápidamente suministrados y
lanzados con un mínimo esfuerzo de mantenimiento
o interacción con los proveedores de servicios.
\end{remark}


El cómputo en la nube también se puede definir como un \textit{estilo de cómputo en el cual
se brindan recursos como servicios, bajo demanda, a través de internet}.
Estos recursos se escalan dinámicamente y a menudo son virtualizados.

Con esta tecnología los usuarios, casi con cualquier dispositivo, pueden acceder mediante internet
a programas, servicios de alojamiento de archivos, plataformas para desarrollo de
aplicaciones, etcétera; usando servicios ofrecidos por
proveedores de cómputo en la nube. Las ventajas del cómputo en la nube incluyen
reducción de costos, alta disponibilidad y la fácil escalabilidad.


\subsection{Capas del cómputo en la nube}
\label{\detokenize{chapter_one/cloud_computing:capas-del-computo-en-la-nube}}
El cómputo en la nube puede verse como una colección de servicios, que puede se
presentada como una arquitectura de cómputo en la nube en capas. Estas
tres capas son: \textit{Software as a Service (SaaS)},
\textit{Infrastructure as a Service (IaaS)} y \textit{Platform as a Service (PaaS)}.

\begin{figure}[ht]
\centering
\capstart

\noindent\sphinxincludegraphics[scale=0.75]{{cloud_computing_layers}.png}
\caption{Arquitectura en capas del cómputo en la nube}\label{\detokenize{chapter_one/cloud_computing:c-c-layers}}\end{figure}


\subsubsection{Software as a Service (SaaS)}
\label{\detokenize{chapter_one/cloud_computing:software-as-a-service-saas}}
El \sphinxstyleemphasis{Software como un Servicio} se refiere simplemente a software que se entrega
bajo demanda para su uso. Pongamos el siguiente ejemplo; antes si alguien necesitaba
un programa para edición de textos tenía que ir a una tienda, comprar algún disco, e instalarlo
en su computadora. Tiempo después una nueva actualización salía al mercado, entonces
se repetía el proceso.
Con SaaS, sólo es necesario ingresar desde el navegador a un programa
alojado en alguna parte. No hay instalación, no hay actualización.


\subsubsection{Infrastructure as a Service (IaaS)}
\label{\detokenize{chapter_one/cloud_computing:infrastructure-as-a-service-iaas}}
La \sphinxstyleemphasis{Infraestructura como un servicio} hace referencia a recursos de cómputo como un servicio.
Esto incluye computadoras
virtualizadas con garantía de poder de procesamiento y ancho de banda reservado para
almacenamiento y acceso a internet. Los servicios más comunes de la IaaS son:
\begin{itemize}
\item {} 
Alojamiento

\item {} 
Balance de carga

\item {} 
Conectividad de red púbica y privada

\item {} 
Firewalls

\item {} 
Almacenamiento

\end{itemize}


\subsubsection{Platform as a Service (PaaS)}
\label{\detokenize{chapter_one/cloud_computing:platform-as-a-service-paas}}
Esta se encuentra en medio de las otras dos,
con IaaS en la parte de abajo y SaaS en la parte más alta.
Se encarga de proveer todo lo necesario para ejecutar
un lenguaje en específico o un \sphinxstyleemphasis{stack de soluciones}. Es similar a IaaS pero incluye
un sistema operativo y servicios requeridos para una aplicación en específico.


\subsection{Servicios en la nube}
\label{\detokenize{chapter_one/cloud_computing:servicios-en-la-nube}}
Existen tres categorías de servicios en la nube. La primera es el servicio en la
nube SaaS, donde la aplicación entera corre en la nube. El cliente utiliza un
navegador para acceder a la aplicación. Un ejemplo de SaaS es Office 365.

El otro tipo de servicio es en el que la aplicación corre del lado del cliente,
sin embargo, accede a algunas funcionalidad y servicios provistos en el la nube.
Un ejemplo es Spotify. La aplicación móvil reproduce la música, mientras que el
servicio en la nube es utilizado para descargar nuevas canciones.

El último tipo es una plataforma en la nube para crear aplicaciones, usada por
desarrolladores. Los desarrolladores crean una nueva aplicación como SaaS  usando
una plataforma en la nube, un ejemplo es Cloud9.

\begin{figure}[ht]
\centering
\capstart

\noindent\sphinxincludegraphics[scale=0.75]{{cloud_services_categories}.png}
\caption{Las categorías de los servicios en la nube.}\label{\detokenize{chapter_one/cloud_computing:c-s-categories}}\end{figure}


\subsection{Tipos de cómputo en la nube}
\label{\detokenize{chapter_one/cloud_computing:tipos-de-computo-en-la-nube}}
Hay tres tipos de cómputo en la nube:
\begin{itemize}
\item {} 
\sphinxstylestrong{Nube pública}

\item {} 
\sphinxstylestrong{Nube privada}

\item {} 
\sphinxstylestrong{Nube híbrida}

\end{itemize}

En la nube pública, los recursos de cómputo son proporcionados dinámicamente a
través de internet por medio de aplicaciones web o servicios web brindados por
terceros. Las nubes públicas son ejecutadas por terceros y la aplicaciones de
diferentes consumidores probablemente se mezclan en los servidores en la nube,
los sistemas de almacenamiento y las redes.

La nube privada es el cómputo en la nube sobre redes privadas. Están construidas
exclusivamente para un cliente, proporcionando un control total sobre los datos,
la seguridad y la calidad del servicio. La nube privada puede ser mantenida por
la organización que la ocupará o por terceros.

Una nube híbrida combina múltiples modelos de nubes privadas y públicas. Añaden
la complejidad de determinar como distribuir aplicaciones entre ambos tipos de nubes.


\subsection{Cómputo en la nube vs servicios en la nube}
\label{\detokenize{chapter_one/cloud_computing:computo-en-la-nube-vs-servicios-en-la-nube}}
En esta sección se presentan dos tablas que muestran las diferencias y los principales
atributos del cómputo en la nube y los servicios en la nube. El cómputo en la nube
consiste de las tecnologías que permiten los servicios en la nube. Los atributos
clave del cómputo en la nube y de los servicios en la nube se muestran en las
tablas  \ref{\detokenize{chapter_one/cloud_computing:key-cloud-computing-attr}} y  \ref{\detokenize{chapter_one/cloud_computing:key-cloud-services-attr}}, respectivamente.


\begin{savenotes}\sphinxattablestart
\centering
\sphinxcapstartof{table}
\caption{Atributos principales del cómputo en la nube \label{\detokenize{chapter_one/cloud_computing:key-cloud-computing-attr}}}
\sphinxaftercaption
\begin{tabulary}{\linewidth}[t]{|T|T|}
\hline
\sphinxstylethead{ 
Atributo
\unskip}\relax &\sphinxstylethead{ 
Descripción
\unskip}\relax \\
\hline
Sistemas de infraestructura
&
Incluye servidores, almacenamiento, y redes que pueden escalarse como el usuario lo demande.
\\
\hline
Aplicación
&
Provee una interfaz de usuario basada en la web., servicios web de API, y una amplia variedad de configuraciones.
\\
\hline
Desarrollo aplicaciones y lanzamiento de software
&
Soporta el desarrollo e integración de una aplicación de software en la nube.
\\
\hline
Software de mantenimiento para sistemas y aplicaciones
&
Soporta una provisión rápida de autoservicio y configuración, además de un monitoreo de uso.
\\
\hline
\end{tabulary}
\par
\sphinxattableend\end{savenotes}


\begin{savenotes}\sphinxattablestart
\centering
\sphinxcapstartof{table}
\caption{Atributos principales de los servicios en la nube \label{\detokenize{chapter_one/cloud_computing:key-cloud-services-attr}}}
\sphinxaftercaption
\begin{tabulary}{\linewidth}[t]{|T|T|}
\hline
\sphinxstylethead{ 
Atributo
\unskip}\relax &\sphinxstylethead{ 
Descripción
\unskip}\relax \\
\hline
\sphinxstyleemphasis{Offsite}. Proveedor externo.
&
En la ejecución en la nube, se asume que hay terceros que proveen servicios. También existe la posibilidad de una entrega de servicios en la nube de manera interna.
\\
\hline
Acceso vía internet
&
Los servicios se acceden por medio de una red universal basada en un estándar. Se puede incluir las opciones de seguridad y calidad del servicio.
\\
\hline
Habilidades de IT nulas o mínimas requeridas
&
Hay una especificación simplificada de requerimientos.
\\
\hline
Precio
&
El precio está basado en la capacidad de uso.
\\
\hline
Interfaz de usuario
&
La interfaz de usuario incluye navegadores para una variedad de dispositivos.
\\
\hline
Interfaz del sistema
&
Las interfaces del sistema están basadas en API web de servicios brindando un framework estándar para acceso e integración entre los servicios en la nube.
\\
\hline
Recursos compartidos
&
Los recursos son compartidos entre los usuarios de los servicios en la nube; sin embargo, a través de opciones configuración, existe la posibilidad de personalizar.
\\
\hline
\end{tabulary}
\par
\sphinxattableend\end{savenotes}


\subsection{Tecnologías que hacen posible el cómputo en la nube}
\label{\detokenize{chapter_one/cloud_computing:tecnologias-que-hacen-posible-el-computo-en-la-nube}}
Algunas de las tecnologías claves que permiten el cómputo en la nube se describen
de manera muy general y breve a continuación.


\subsubsection{Virtualización}
\label{\detokenize{chapter_one/cloud_computing:virtualizacion}}
La ventaja del cómputo en la nube es la habilidad de virtualizar y compartir
recursos entre diferentes aplicaciones con el objetivo de utilizar mejor a los
servidores. En el cómputo no en la nube, por ejemplo, tres plataformas existen para
tres aplicaciones diferentes corriendo en su propio servidor. En la nube,
los servidores pueden ser compartidos o virtualizados, para distintos sistemas
operativos y aplicaciones resultando en menos servidores.

\begin{figure}[ht]
\centering
\capstart

\noindent\sphinxincludegraphics[scale=0.30]{{virtualization_cloud}.png}
\caption{Un ejemplo de virtualización: en un cómputo que no esté en la nube se necesitan tres servidores, en la nube sólo se usan dos servidores.}\label{\detokenize{chapter_one/cloud_computing:c-computing-virtualization}}\end{figure}


\subsubsection{Servicios Web y Arquitectura Orientada a Servicios}
\label{\detokenize{chapter_one/cloud_computing:servicios-web-y-arquitectura-orientada-a-servicios}}
Los servicios en la nube están diseñados típicamente como servicios web (sistemas
de software diseñados para soportar la interacción interoperable máquina a máquina
a través de una red), que siguen los estándares de la industria como SOAP, REST, entre
otros. La arquitectura orientada a servicios (SOA por sus siglas en inglés) organiza y maneja
los servicios web dentro de las nubes. Una SOA también incluye un conjunto
de servicios en la nube, que están disponibles sobre varias plataformas distribuidas.


\subsubsection{Flujo de servicio y flujos de trabajo}
\label{\detokenize{chapter_one/cloud_computing:flujo-de-servicio-y-flujos-de-trabajo}}
El concepto de flujo de servicio y flujos de trabajo se refiere a una
vista integrada de actividades basadas en servicios provistas por la nube.
Los flujos de trabajo se han vuelto una de las áreas importantes
de investigación en el campo de los sistemas de bases de datos
y sistemas de información.


\subsection{Características del cómputo en la nube}
\label{\detokenize{chapter_one/cloud_computing:caracteristicas-del-computo-en-la-nube}}
El cómputo en la nube brinda un número de nuevas características
a otros paradigmas de computación.
\begin{itemize}
\item {} 
Escalabilidad y servicios bajo demanda. El cómputo en la nube  provee recursos y servicios para usuarios bajo demanda. Los recursos son escalables sobre varios centros de datos.

\item {} 
Interfaz centrada en el usuario. Las interfaces de la nube son independientes de la ubicación y pueden ser accedidas a través de servicios web o navegadores.

\item {} 
Calidad de servicio garantizada. El cómputo en la nube garantiza calidad de servicio para los usuarios en términos de desempeño de hardware, ancho de banda y capacidad de la memoria.

\item {} 
Sistemas autónomos. Los sistemas de cómputo en la nube son sistemas autónomos administrados de manera transparente a los usuarios. Sin embargo, el software e información dentro de las nubes pueden reconfigurarse automáticamente y consolidarse en una plataforma simple dependiendo de las necesidades del usuario.

\item {} 
Precio. Los usuarios pagan por los servicios y capacidad que necesitan.

\end{itemize}


\subsection{Robótica en la nube}
\label{\detokenize{chapter_one/cloud_computing:robotica-en-la-nube}}

\begin{remark}
Un robot en la nube se define de manera muy general como
cualquier robot o sistema de automatización  que depende ya sea de datos o
código enviado a través de una red que soporta su operación, es decir, donde
no todo el cómputo, y memoria está integrada en un solo sistema.\\

\end{remark}
Debido a factores como la latencia de la conexión, calidad de servicio variable, etcétera;
un robot en la nube a menudo tiene cierta capacidad para procesamiento local para
respuestas de baja latencia y durante periodos donde el acceso a una red no esté
disponible o no sea confiable.

Algunos beneficios potenciales obtenidos de la nube son:
\begin{itemize}
\item {} 
\sphinxstylestrong{Big data}: Acceso a bibliotecas remotas de imágenes, mapas, trayectorias, y datos de objetos.

\item {} 
\sphinxstylestrong{Cómputo en la nube}: Acceso de cómputo bajo demanda en paralelo en un grid de cómputo para análisis estadístico, aprendizaje y planeación de movimiento.

\item {} 
\sphinxstylestrong{Aprendizaje colectivo de robots}: Robots compartiendo sus trayectorias, políticas de control, y salidas.

\item {} 
\sphinxstylestrong{Cómputo humano}: Acceso a \sphinxstyleemphasis{crowdsourcing} (colaboración abierta distribuida) para utilizar la experiencia y habilidad humana en el análisis de imágenes, o en la recopilación de datos. Y acceso a \sphinxstyleemphasis{call centers} que no es más que la operación remota por humanos.

\end{itemize}


