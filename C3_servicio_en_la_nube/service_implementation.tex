%		\def\sphinxdocclass{report}
%\documentclass[letterpaper,10pt,spanish]{sphinxmanual}
%\ifdefined\pdfpxdimen
%   \let\sphinxpxdimen\pdfpxdimen\else\newdimen\sphinxpxdimen
%\fi \sphinxpxdimen=.75bp\relax
%
%\PassOptionsToPackage{warn}{textcomp}
%\usepackage[utf8]{inputenc}
%\ifdefined\DeclareUnicodeCharacter
% \ifdefined\DeclareUnicodeCharacterAsOptional
%  \DeclareUnicodeCharacter{"00A0}{\nobreakspace}
%  \DeclareUnicodeCharacter{"2500}{\sphinxunichar{2500}}
%  \DeclareUnicodeCharacter{"2502}{\sphinxunichar{2502}}
%  \DeclareUnicodeCharacter{"2514}{\sphinxunichar{2514}}
%  \DeclareUnicodeCharacter{"251C}{\sphinxunichar{251C}}
%  \DeclareUnicodeCharacter{"2572}{\textbackslash}
% \else
%  \DeclareUnicodeCharacter{00A0}{\nobreakspace}
%  \DeclareUnicodeCharacter{2500}{\sphinxunichar{2500}}
%  \DeclareUnicodeCharacter{2502}{\sphinxunichar{2502}}
%  \DeclareUnicodeCharacter{2514}{\sphinxunichar{2514}}
%  \DeclareUnicodeCharacter{251C}{\sphinxunichar{251C}}
%  \DeclareUnicodeCharacter{2572}{\textbackslash}
% \fi
%\fi
%\usepackage{cmap}
%\usepackage[T1]{fontenc}
%\usepackage{amsmath,amssymb,amstext}
%\usepackage{babel}
%\usepackage[Sonny]{fncychap}
%\usepackage{sphinx}
%
%\usepackage{breakurl}
%\usepackage{geometry}
%
%% Include hyperref last.
%\usepackage{hyperref}
%% Fix anchor placement for figures with captions.
%\usepackage{hypcap}% it must be loaded after hyperref.
%% Set up styles of URL: it should be placed after hyperref.
%\urlstyle{same}
%\addto\captionsspanish{\renewcommand{\contentsname}{Contents:}}
%
%\addto\captionsspanish{\renewcommand{\figurename}{Figura}}
%\addto\captionsspanish{\renewcommand{\tablename}{Tabla}}
%\addto\captionsspanish{\renewcommand{\literalblockname}{Lista}}
%
%\addto\captionsspanish{\renewcommand{\literalblockcontinuedname}{proviene de la página anterior}}
%\addto\captionsspanish{\renewcommand{\literalblockcontinuesname}{continues on next page}}
%
%\addto\extrasspanish{\def\pageautorefname{página}}
%

%\begin{document}


Para crear y probar cada una de las arquitecturas que se
describieron en la sección anterior se utilizó la biblioteca
de TensorFlow, se definió una clase que pudiera crear gráficas de modelos
con los cuatro casos antes descritos y luego se automatizó el proceso de
entrenamiento de todas las posibles arquitecturas, variando los parámetros.
Así, cada instancia de la clase, es un modelo diferente. En esta sección se 
describe la implementación de las CNN, su entrenamiento, evaluación
y lanzamiento sobre la API REST de CloudNAO.


El programa en TensorFlow encargado del entrenamiendo de la CNN se puede dividir en tres partes principales; la primera es la carga las imágenes de cada conjunto a partir de su ruta y su etiquetado, en la segunda
se crean las gráficas de cómputo y por último se ejecutan esas gráficas de cómputo.

Para agregar el modelo a la API de CloudNAO sólo se añadió un nuevo módulo
al recurso \texttt{/vision}, en donde se carga la gráfica de cómputo 
con los mejores resultados.
	

\subsection{Hardware y software}

El hardware sobre el que se realizó el entrenamiento de las
diferentes arquitecturas fue un equipo con las siguientes características:

\begin{itemize}

\item Procesador: Intel Core i5-7300HQ CPU 2.50GHz x 4.
\item Memoria RAM: 8GB.
\item GPU:  NVIDIA GeForce GTX 1050 2GB.
\item SSD: 250 GB.
\end{itemize}

Como ya se explicó, la biblioteca para desarrollar el modelo en Python fue TensorFlow, la cual cuenta
con soporte para la plataforma de cómputo paralelo CUDA (Arquitectura
Unificada de Dispositivos de Cómputo), desarrollada por NVIDIA.
La versión de CUDA y de TensorFlow (con sopore para GPU) fueron 8.0 y 1.4.0.

Además de TensorFlow, las otras dos bibliotecas importantes son 
\texttt{numpy} y \texttt{opencv}, principalmente para la carga de
las imágenes en un formato que pudiera recibir la CNN.

\subsection{Preparación del conjunto de datos}

Contamos con un conjunto de entrenamiento de $6000$ imágenes y 
un conjunto de prueba de $2112$.
Como estamos utilizando un algoritmo de aprendizaje supervisado, 
se necesitan etiquetas u objetivos para cada imagen. La representación
de esas etiquetas se hizo con vectores $\mathbf{y}$ en codificación one-hot,
con $y_n = 1$ si la imagen pertenece a la clase $n$ y $y_i = 0$ para todos los
demás casos. Los vectores posibles de acuerdo las cuatro categorías son los siguientes:

\begin{itemize}
\item $(1, 0, 0, 0)$ si la imagen es de la categoría \textit{zona de trabajo}.
\item $(0, 1, 0, 0)$ si la imagen es de la categoría \textit{salida}.
\item $(0, 0, 1, 0)$ si la imagen es de la categoría \textit{cubículo}.
\item $(0, 0, 0, 1)$ si la imagen es de la categoría \textit{cancha}.
\end{itemize}

Esta codificación se puede intepretar como una distribución de probabilidad
sobre las cuatro clases.

La clase que encapsula los métodos para la carga de las imágenes, su 
procesamiento, la definición de la gráfica de cómputo y la ejecución se
llama \texttt{CNNClassifierLAR}.
Dentro de la clase el método\\ {\hyperref[\detokenize{model_desc:cnn_indoor_classifier_model.CNNClassifierLAR.get_training_and_test_images}]{\sphinxcrossref{\sphinxcode{\sphinxupquote{cnn\_indoor\_classifier\_model.CNNClassifierLAR.get\_training\_and\_test\_images()}}}}}
carga y procesa las imágenes para definir los conjuntos de entrenamiento
y de prueba para nuestro modelo.
Primero se obtiene la ruta de cada imágen y se crea la lista con las etiquetas
(vectores one-hot)
de cada imagen dependiendo de la carpeta donde se encontraba, luego con
\sphinxcode{\sphinxupquote{opencv}} se cargan las imágenes a la memoria como arreglos de
\sphinxcode{\sphinxupquote{numpy}}, se redimensionan a un tamaño de 32 * 32 y se normalizan
los valores de los pixeles. Al final se tienen dos arreglos de \sphinxcode{\sphinxupquote{numpy}}
donde cada elemento es un arreglo de tres dimensiones que
representa una imagen. Las operaciones dentro del método
básicamente son las siguientes:

\fvset{hllines={, ,}}%
\begin{sphinxVerbatim}[commandchars=\\\{\}]
\PYG{c+c1}{\PYGZsh{} Carga del conjunto de entrenamiento}
\PYG{n}{training\PYGZus{}paths} \PYG{o}{=} \PYG{n}{np}\PYG{o}{.}\PYG{n}{array}\PYG{p}{(}\PYG{n}{list\PYGZus{}files\PYGZus{}in\PYGZus{}directory}\PYG{p}{(}\PYG{l+s+s1}{\PYGZsq{}}\PYG{l+s+s1}{dataset/training\PYGZus{}set/*}\PYG{l+s+s1}{\PYGZsq{}}\PYG{p}{)}\PYG{p}{)}
\PYG{n}{np}\PYG{o}{.}\PYG{n}{random}\PYG{o}{.}\PYG{n}{shuffle}\PYG{p}{(}\PYG{n}{training\PYGZus{}paths}\PYG{p}{)}
\PYG{n}{labels\PYGZus{}training\PYGZus{}set} \PYG{o}{=} \PYG{n}{get\PYGZus{}labels\PYGZus{}from\PYGZus{}path}\PYG{p}{(}\PYG{n}{training\PYGZus{}paths}\PYG{p}{)}
\PYG{n}{training\PYGZus{}images} \PYG{o}{=} \PYG{n}{np}\PYG{o}{.}\PYG{n}{array}\PYG{p}{(}\PYG{p}{[}\PYG{n}{cv2}\PYG{o}{.}\PYG{n}{cvtColor}\PYG{p}{(}\PYG{n}{cv2}\PYG{o}{.}\PYG{n}{resize}\PYG{p}{(}\PYG{n}{cv2}\PYG{o}{.}\PYG{n}{imread}\PYG{p}{(}\PYG{n}{file\PYGZus{}name}\PYG{p}{)}\PYG{p}{,} \PYG{p}{(}\PYG{l+m+mi}{32}\PYG{p}{,} \PYG{l+m+mi}{32}\PYG{p}{)}\PYG{p}{)}\PYG{p}{,} \PYG{n}{cv2}\PYG{o}{.}\PYG{n}{COLOR\PYGZus{}BGR2RGB}\PYG{p}{)} \PYG{o}{/} \PYG{l+m+mi}{255} \PYG{k}{for} \PYG{n}{file\PYGZus{}name} \PYG{o+ow}{in} \PYG{n}{training\PYGZus{}paths}\PYG{p}{]}\PYG{p}{)}
\PYG{c+c1}{\PYGZsh{} Conjunto de prueba}
\PYG{n}{testing\PYGZus{}paths} \PYG{o}{=} \PYG{n}{np}\PYG{o}{.}\PYG{n}{array}\PYG{p}{(}\PYG{n}{list\PYGZus{}files\PYGZus{}in\PYGZus{}directory}\PYG{p}{(}\PYG{l+s+s1}{\PYGZsq{}}\PYG{l+s+s1}{dataset/test\PYGZus{}set/*}\PYG{l+s+s1}{\PYGZsq{}}\PYG{p}{)}\PYG{p}{)}
\PYG{n}{np}\PYG{o}{.}\PYG{n}{random}\PYG{o}{.}\PYG{n}{shuffle}\PYG{p}{(}\PYG{n}{testing\PYGZus{}paths}\PYG{p}{)}
\PYG{n}{labels\PYGZus{}test\PYGZus{}set} \PYG{o}{=} \PYG{n}{get\PYGZus{}labels\PYGZus{}from\PYGZus{}path}\PYG{p}{(}\PYG{n}{testing\PYGZus{}paths}\PYG{p}{)}
\PYG{n}{testing\PYGZus{}images} \PYG{o}{=} \PYG{n}{np}\PYG{o}{.}\PYG{n}{array}\PYG{p}{(}\PYG{p}{[}\PYG{n}{cv2}\PYG{o}{.}\PYG{n}{cvtColor}\PYG{p}{(}\PYG{n}{cv2}\PYG{o}{.}\PYG{n}{resize}\PYG{p}{(}\PYG{n}{cv2}\PYG{o}{.}\PYG{n}{imread}\PYG{p}{(}\PYG{n}{file\PYGZus{}name}\PYG{p}{)}\PYG{p}{,} \PYG{p}{(}\PYG{l+m+mi}{32}\PYG{p}{,} \PYG{l+m+mi}{32}\PYG{p}{)}\PYG{p}{)}\PYG{p}{,} \PYG{n}{cv2}\PYG{o}{.}\PYG{n}{COLOR\PYGZus{}BGR2RGB}\PYG{p}{)} \PYG{o}{/}\PYG{l+m+mi}{255} \PYG{k}{for} \PYG{n}{file\PYGZus{}name} \PYG{o+ow}{in} \PYG{n}{testing\PYGZus{}paths}\PYG{p}{]}\PYG{p}{)}
\end{sphinxVerbatim}

Aquí estaba
%
\subsection{Creación y ejecución de la gráfica de TensorFlow}
\label{\detokenize{model_desc:creacion-y-ejecucion-de-la-grafica-de-tensorflow}}
Para la implementación de la CNN en TensorFlow
se utilizó una clase que representara una
arquitectura. Esto fue para que experimentar con
una estructura y parámetros diferentes sólo
implicara instaciar un objeto de la clase.
La clase se llama \sphinxcode{\sphinxupquote{CNNClassifierLAR}} y
está dentro del módulo {\hyperref[\detokenize{model_desc:module-cnn_indoor_classifier_model}]{\sphinxcrossref{\sphinxcode{\sphinxupquote{cnn\_indoor\_classifier\_model}}}}}.
El módulo anterior además de la clase contiene múltiples
funciones auxiliares para facilitar la definición
de los parámetros de aprendizaje, la operaciones entre los
tensores, la obtención de las rutas de las imágenes
y la generación de los vectores one-hot de sus etiquetas.

Se pueden contruir hasta cuatro tipos distintos de
arquitecturas, con dos capas convolucionales con una o dos
completamente conectadas, y con una capa de convolución
y uno o dos completamente conectadas.
La construcción de éstas se hace con los métodos
{\hyperref[\detokenize{model_desc:cnn_indoor_classifier_model.CNNClassifierLAR.create_graph_2_convo_layers}]{\sphinxcrossref{\sphinxcode{\sphinxupquote{cnn\_indoor\_classifier\_model.CNNClassifierLAR.create\_graph\_2\_convo\_layers()}}}}}
y\\ {\hyperref[\detokenize{model_desc:cnn_indoor_classifier_model.CNNClassifierLAR.create_graph_1_convo_layer}]{\sphinxcrossref{\sphinxcode{\sphinxupquote{cnn\_indoor\_classifier\_model.CNNClassifierLAR.create\_graph\_1\_convo\_layer()}}}}}
que reciben como argumento una bandera para contruir su gráfica
de cómputo con dos o una capa de completamente conectada. A
continuación se desglosan los pasos del método para
crear la gráfica con dos capas de convolución.

1. Creación de la primera de convolución, recibe como entradas el placeholder
con las imágenes y a la función se le pasan como parámetros una lista con las dimensiones
de los kernels que se aplican y el número de mapas de características. Por ejemplo
un lista \sphinxcode{\sphinxupquote{{[}5, 5, 3, 16{]}}} indica que se deben aplicar kernels de 5*5 a
una entrada con 3 canales (la imagen RGB), para obtener
16 mapas de características, después se aplica un agrupamiento para disminuir las
dimensiones de los 16 mapas a la mitad:

\fvset{hllines={, ,}}%
\begin{sphinxVerbatim}[commandchars=\\\{\}]
\PYG{n}{convo\PYGZus{}1} \PYG{o}{=} \PYG{n}{convolutional\PYGZus{}layer}\PYG{p}{(}\PYG{n}{input\PYGZus{}x\PYGZus{}ph}\PYG{p}{,} \PYG{n}{shape}\PYG{o}{=}\PYG{n}{shape\PYGZus{}convo\PYGZus{}layers}\PYG{p}{[}\PYG{l+m+mi}{0}\PYG{p}{]}\PYG{p}{)}
\PYG{n}{convo\PYGZus{}1\PYGZus{}pooling} \PYG{o}{=} \PYG{n}{max\PYGZus{}pool\PYGZus{}2by2}\PYG{p}{(}\PYG{n}{convo\PYGZus{}1}\PYG{p}{)}
\end{sphinxVerbatim}

2. Luego sigue la otra capa de convolución que tiene como entrada las salidas
de la capa anterior. Se aplica un agrupamiento para reducir de nuevo
la imagen a la mitad:

\fvset{hllines={, ,}}%
\begin{sphinxVerbatim}[commandchars=\\\{\}]
\PYG{n}{convo\PYGZus{}2} \PYG{o}{=} \PYG{n}{convolutional\PYGZus{}layer}\PYG{p}{(}\PYG{n}{convo\PYGZus{}1\PYGZus{}pooling}\PYG{p}{,} \PYG{n}{shape}\PYG{o}{=}\PYG{n}{shape\PYGZus{}convo\PYGZus{}layers}\PYG{p}{[}\PYG{l+m+mi}{1}\PYG{p}{]}\PYG{p}{)}
\PYG{n}{convo\PYGZus{}2\PYGZus{}pooling} \PYG{o}{=} \PYG{n}{max\PYGZus{}pool\PYGZus{}2by2}\PYG{p}{(}\PYG{n}{convo\PYGZus{}2}\PYG{p}{)}
\end{sphinxVerbatim}

3. Ahora, se redimensionan los
mapas de características para tener una red neuronal donde cada unidad es un
elemento dentro del mapa de características. Esto es equivalente a aplicar
una convolución con kernel de 1*1. Se obtiene una capa, donde
dependiendo de la bandera se conecta a una capa oculta o directamente
a la capa de salida con 4 unidades:

\fvset{hllines={, ,}}%
\begin{sphinxVerbatim}[commandchars=\\\{\}]
\PYG{n}{last\PYGZus{}maps\PYGZus{}of\PYGZus{}features} \PYG{o}{=} \PYG{n}{shape\PYGZus{}convo\PYGZus{}layers}\PYG{p}{[}\PYG{l+m+mi}{1}\PYG{p}{]}\PYG{p}{[}\PYG{l+m+mi}{3}\PYG{p}{]}
\PYG{n}{convo\PYGZus{}2\PYGZus{}flat} \PYG{o}{=} \PYG{n}{tf}\PYG{o}{.}\PYG{n}{reshape}\PYG{p}{(}\PYG{n}{convo\PYGZus{}2\PYGZus{}pooling}\PYG{p}{,} \PYG{p}{[}\PYG{o}{\PYGZhy{}}\PYG{l+m+mi}{1}\PYG{p}{,} \PYG{l+m+mi}{8} \PYG{o}{*} \PYG{l+m+mi}{8} \PYG{o}{*} \PYG{n}{last\PYGZus{}maps\PYGZus{}of\PYGZus{}features}\PYG{p}{]}\PYG{p}{)}
\PYG{k}{if} \PYG{n}{two\PYGZus{}fc}\PYG{p}{:}
    \PYG{n}{full\PYGZus{}layer\PYGZus{}one} \PYG{o}{=} \PYG{n}{tf}\PYG{o}{.}\PYG{n}{nn}\PYG{o}{.}\PYG{n}{relu}\PYG{p}{(}\PYG{n}{normal\PYGZus{}full\PYGZus{}layer}\PYG{p}{(}\PYG{n}{convo\PYGZus{}2\PYGZus{}flat}\PYG{p}{,} \PYG{n}{units\PYGZus{}fc}\PYG{p}{)}\PYG{p}{)}
\PYG{k}{else}\PYG{p}{:}
    \PYG{n}{full\PYGZus{}layer\PYGZus{}one} \PYG{o}{=} \PYG{n}{convo\PYGZus{}2\PYGZus{}flat}
\end{sphinxVerbatim}

4. La última capa es el vector predicho, que se pasa por la función softmax para
medir el error con la entropía cruzada. Finalmete se tiene un la operación
\sphinxcode{\sphinxupquote{train}} que minimiza la función de costo utilizando el algoritmo de
retropropagación:

\fvset{hllines={, ,}}%
\begin{sphinxVerbatim}[commandchars=\\\{\}]
\PYG{n}{y\PYGZus{}pred} \PYG{o}{=} \PYG{n}{tf}\PYG{o}{.}\PYG{n}{identity}\PYG{p}{(}\PYG{n}{normal\PYGZus{}full\PYGZus{}layer}\PYG{p}{(}\PYG{n}{full\PYGZus{}layer\PYGZus{}one}\PYG{p}{,} \PYG{l+m+mi}{4}\PYG{p}{)}\PYG{p}{,} \PYG{n}{name}\PYG{o}{=}\PYG{l+s+s2}{\PYGZdq{}}\PYG{l+s+s2}{y\PYGZus{}pred}\PYG{l+s+s2}{\PYGZdq{}}\PYG{p}{)}
\PYG{n}{cross\PYGZus{}entropy} \PYG{o}{=} \PYG{n}{tf}\PYG{o}{.}\PYG{n}{reduce\PYGZus{}mean}\PYG{p}{(}\PYG{n}{tf}\PYG{o}{.}\PYG{n}{nn}\PYG{o}{.}\PYG{n}{softmax\PYGZus{}cross\PYGZus{}entropy\PYGZus{}with\PYGZus{}logits}\PYG{p}{(}\PYG{n}{labels}\PYG{o}{=}\PYG{n}{y\PYGZus{}true\PYGZus{}ph}\PYG{p}{,} \PYG{n}{logits}\PYG{o}{=}\PYG{n}{y\PYGZus{}pred}\PYG{p}{)}\PYG{p}{)}
\PYG{n}{optimizer} \PYG{o}{=} \PYG{n}{tf}\PYG{o}{.}\PYG{n}{train}\PYG{o}{.}\PYG{n}{GradientDescentOptimizer}\PYG{p}{(}\PYG{n}{learning\PYGZus{}rate}\PYG{o}{=}\PYG{n}{learning\PYGZus{}rate}\PYG{p}{)}
\PYG{n}{train} \PYG{o}{=} \PYG{n}{optimizer}\PYG{o}{.}\PYG{n}{minimize}\PYG{p}{(}\PYG{n}{loss}\PYG{o}{=}\PYG{n}{cross\PYGZus{}entropy}\PYG{p}{,} \PYG{n}{global\PYGZus{}step}\PYG{o}{=}\PYG{n}{tf}\PYG{o}{.}\PYG{n}{train}\PYG{o}{.}\PYG{n}{get\PYGZus{}global\PYGZus{}step}\PYG{p}{(}\PYG{p}{)}\PYG{p}{)}
\end{sphinxVerbatim}

La ejecución de la gráfica de cómputo de una instancia
es con el método {\hyperref[\detokenize{model_desc:cnn_indoor_classifier_model.CNNClassifierLAR.run_graph}]{\sphinxcrossref{\sphinxcode{\sphinxupquote{cnn\_indoor\_classifier\_model.CNNClassifierLAR.run\_graph()}}}}}.
Se pasan como parámetro el número de épocas y el tamaño del lote
para el aprendizaje.
Este método ejecuta la operación
\sphinxcode{\sphinxupquote{train}} que a su vez corre todas las operaciones que la preceden.
Cada 100 épocas se evalúa la precisión del modelo.

\phantomsection\label{\detokenize{model_desc:module-cnn_indoor_classifier_model}}\index{cnn\_indoor\_classifier\_model (módulo)}\phantomsection\label{\detokenize{model_desc:module-cnn_indoor_classifier_model}}\index{cnn\_indoor\_classifier\_model (módulo)}\index{CNNClassifierLAR (clase en cnn\_indoor\_classifier\_model)}

\begin{fulllineitems}
\phantomsection\label{\detokenize{model_desc:cnn_indoor_classifier_model.CNNClassifierLAR}}\pysiglinewithargsret{\sphinxbfcode{\sphinxupquote{class }}\sphinxcode{\sphinxupquote{cnn\_indoor\_classifier\_model.}}\sphinxbfcode{\sphinxupquote{CNNClassifierLAR}}}{\emph{shape\_convo\_layers}, \emph{units\_fc}, \emph{learning\_rate}}{}
La clase que representa una arquitectura de red convolucional con a lo más
dos capas de convolución y dos capas completamente conectadas.

La red está creada para específicamente aceptar entradas de dimensiones
de 32 pixeles por 32 pixeles por 3 canales. El constructor recibe tres parámetros
una lista con las dimensiones de los kernels, entradas de la convolución y
número de mapas de características, un entero con el número de unidades en la
capa oculta y un valor flotante que es la tasa de aprendizaje.
En el siguiente ejemplo se crea una red con dos capas de convolución,
una capa oculta con 2048 unidades y una tasa de aprendizaje de 0.01. La ejecución de la gráfica
para el aprendizaje por lotes recibe como parámetros 200 y 3000,
el tamaño del lote y el número de épocas.

\fvset{hllines={, ,}}%
\begin{sphinxVerbatim}[commandchars=\\\{\}]
\PYG{g+gp}{\PYGZgt{}\PYGZgt{}\PYGZgt{} }\PYG{k+kn}{from} \PYG{n+nn}{cnn\PYGZus{}indoor\PYGZus{}classifier\PYGZus{}model} \PYG{k}{import} \PYG{n}{CNNClassifierLAR}
\PYG{g+gp}{\PYGZgt{}\PYGZgt{}\PYGZgt{} }\PYG{n}{clasificador} \PYG{o}{=} \PYG{n}{CNNClassifierLAR}\PYG{p}{(}\PYG{p}{[}\PYG{p}{[}\PYG{l+m+mi}{5}\PYG{p}{,} \PYG{l+m+mi}{5}\PYG{p}{,} \PYG{l+m+mi}{3}\PYG{p}{,} \PYG{l+m+mi}{16}\PYG{p}{]}\PYG{p}{,} \PYG{p}{[}\PYG{l+m+mi}{3}\PYG{p}{,} \PYG{l+m+mi}{3}\PYG{p}{,} \PYG{l+m+mi}{16}\PYG{p}{,} \PYG{l+m+mi}{32}\PYG{p}{]}\PYG{p}{]}\PYG{p}{,} \PYG{l+m+mi}{2048}\PYG{p}{,} \PYG{l+m+mf}{0.01}\PYG{p}{)}
\PYG{g+gp}{\PYGZgt{}\PYGZgt{}\PYGZgt{} }\PYG{n}{clasificador}\PYG{o}{.}\PYG{n}{create\PYGZus{}graph\PYGZus{}2\PYGZus{}convo\PYGZus{}layers}\PYG{p}{(}\PYG{p}{)}
\PYG{g+gp}{\PYGZgt{}\PYGZgt{}\PYGZgt{} }\PYG{n}{clasificador}\PYG{o}{.}\PYG{n}{run\PYGZus{}graph}\PYG{p}{(}\PYG{l+m+mi}{200}\PYG{p}{,} \PYG{l+m+mi}{3000}\PYG{p}{)}
\PYG{g+go}{i; time; training loss; test loss; accuracy}
\PYG{g+go}{0;0.7688136100769043;1.5798137187957764;1.4590167999267578;0.13778409361839294}
\PYG{g+gp}{...}
\PYG{g+go}{3000;68.31876397132874;0.0036288381088525057;0.11022621393203735;0.9692234992980957}
\PYG{g+go}{Time 68.3188087940216}
\end{sphinxVerbatim}
\index{create\_graph\_1\_convo\_layer() (método de cnn\_indoor\_classifier\_model.CNNClassifierLAR)}

\begin{fulllineitems}
\phantomsection\label{\detokenize{model_desc:cnn_indoor_classifier_model.CNNClassifierLAR.create_graph_1_convo_layer}}\pysiglinewithargsret{\sphinxbfcode{\sphinxupquote{create\_graph\_1\_convo\_layer}}}{\emph{two\_fc=True}}{}
Método para contruir la gráfica de cómputo de una arquitectura
con una capa de convolución y ya sea una o dos capas completamente
conectadas.
\begin{quote}\begin{description}
\item[{Parámetros}] \leavevmode
\sphinxstyleliteralstrong{\sphinxupquote{two\_fc}} (\sphinxstyleliteralemphasis{\sphinxupquote{boolean.}}) \textendash{} Bandera para indicar si hay dos capas completamente conectadas.

\end{description}\end{quote}

\end{fulllineitems}

\index{create\_graph\_2\_convo\_layers() (método de cnn\_indoor\_classifier\_model.CNNClassifierLAR)}

\begin{fulllineitems}
\phantomsection\label{\detokenize{model_desc:cnn_indoor_classifier_model.CNNClassifierLAR.create_graph_2_convo_layers}}\pysiglinewithargsret{\sphinxbfcode{\sphinxupquote{create\_graph\_2\_convo\_layers}}}{\emph{two\_fc=True}}{}
Método para construir la gráfica de cómputo que representa a un modelo
con dos capas de convolución y a través de del argumento agregar una capa
oculta o no.
\begin{quote}\begin{description}
\item[{Parámetros}] \leavevmode
\sphinxstyleliteralstrong{\sphinxupquote{two\_fc}} (\sphinxstyleliteralemphasis{\sphinxupquote{boolean.}}) \textendash{} Una bandera que indica si se agrega una capa oculta.

\end{description}\end{quote}

\end{fulllineitems}

\index{get\_training\_and\_test\_images() (método de cnn\_indoor\_classifier\_model.CNNClassifierLAR)}

\begin{fulllineitems}
\phantomsection\label{\detokenize{model_desc:cnn_indoor_classifier_model.CNNClassifierLAR.get_training_and_test_images}}\pysiglinewithargsret{\sphinxbfcode{\sphinxupquote{get\_training\_and\_test\_images}}}{\emph{dataset\_training\_path='dataset/training\_set/*'}, \emph{dataset\_test\_path='dataset/test\_set/*'}}{}
Llena las listas con las imágenes de entrenamiento y 
de prueba y sus repectivas etiquetas.
\begin{quote}\begin{description}
\item[{Parámetros}] \leavevmode\begin{itemize}
\item {} 
\sphinxstyleliteralstrong{\sphinxupquote{dataset\_training\_path}} (\sphinxstyleliteralemphasis{\sphinxupquote{numpy.array.}}) \textendash{} La ruta relativa al directorio de trabajo donde se encuentra el conjunto de entrenamiento.

\item {} 
\sphinxstyleliteralstrong{\sphinxupquote{dataset\_test\_path}} (\sphinxstyleliteralemphasis{\sphinxupquote{numpy.array.}}) \textendash{} La ruta relativa al directorio de trabajo donde se encuentran las imágenes del conjunto de prueba.

\end{itemize}

\end{description}\end{quote}

\end{fulllineitems}

\index{next\_batch() (método de cnn\_indoor\_classifier\_model.CNNClassifierLAR)}

\begin{fulllineitems}
\phantomsection\label{\detokenize{model_desc:cnn_indoor_classifier_model.CNNClassifierLAR.next_batch}}\pysiglinewithargsret{\sphinxbfcode{\sphinxupquote{next\_batch}}}{\emph{batch\_size}}{}
Método para generar un lote del conjunto de entrenamiento
junto con sus etiquetas.
\begin{quote}\begin{description}
\item[{Parámetros}] \leavevmode
\sphinxstyleliteralstrong{\sphinxupquote{batch\_size}} (\sphinxstyleliteralemphasis{\sphinxupquote{int.}}) \textendash{} El tamaño del lote.

\item[{Devuelve}] \leavevmode
El lote del conjunto de entrenamiento y sus etiquetas.

\item[{Tipo del valor devuelto}] \leavevmode
Una dupla de listas con las imágenes de entrenamiento y sus etiquetas.

\end{description}\end{quote}

\end{fulllineitems}

\index{run\_graph() (método de cnn\_indoor\_classifier\_model.CNNClassifierLAR)}

\begin{fulllineitems}
\phantomsection\label{\detokenize{model_desc:cnn_indoor_classifier_model.CNNClassifierLAR.run_graph}}\pysiglinewithargsret{\sphinxbfcode{\sphinxupquote{run\_graph}}}{\emph{batch\_size}, \emph{epochs}}{}
Método para ejecutar la gráfica de cómputo. Realiza el entrenamiento
por lotes y evalua la red para conocer la precisión del modelo
cada 100 épocas.
\begin{quote}\begin{description}
\item[{Parámetros}] \leavevmode\begin{itemize}
\item {} 
\sphinxstyleliteralstrong{\sphinxupquote{batch\_size}} (\sphinxstyleliteralemphasis{\sphinxupquote{int.}}) \textendash{} El tamaño del lote.

\item {} 
\sphinxstyleliteralstrong{\sphinxupquote{epochs}} (\sphinxstyleliteralemphasis{\sphinxupquote{int.}}) \textendash{} El número de épocas.

\end{itemize}

\end{description}\end{quote}

\end{fulllineitems}


\end{fulllineitems}

\index{conv2d() (en el módulo cnn\_indoor\_classifier\_model)}

\begin{fulllineitems}
\phantomsection\label{\detokenize{model_desc:cnn_indoor_classifier_model.conv2d}}\pysiglinewithargsret{\sphinxcode{\sphinxupquote{cnn\_indoor\_classifier\_model.}}\sphinxbfcode{\sphinxupquote{conv2d}}}{\emph{x}, \emph{W}}{}
Realiza la convolución de un tensor que representa la entrada
y otro con los kernels. La zancada es igual a 1 y el borde de
ceros es \sphinxstylestrong{same}. Utiliza la función \sphinxcode{\sphinxupquote{conv2d}} de la API de
TensorFlow.
\begin{quote}\begin{description}
\item[{Parámetros}] \leavevmode\begin{itemize}
\item {} 
\sphinxstyleliteralstrong{\sphinxupquote{x}} (\sphinxstyleliteralemphasis{\sphinxupquote{tensor.}}) \textendash{} La entrada de la convolución.

\item {} 
\sphinxstyleliteralstrong{\sphinxupquote{W}} (\sphinxstyleliteralemphasis{\sphinxupquote{tensor.}}) \textendash{} El tensor con los kernels.

\end{itemize}

\item[{Devuelve}] \leavevmode
Un tensor con el resultado de la operación de convolución.

\end{description}\end{quote}

\end{fulllineitems}

\index{convolutional\_layer() (en el módulo cnn\_indoor\_classifier\_model)}

\begin{fulllineitems}
\phantomsection\label{\detokenize{model_desc:cnn_indoor_classifier_model.convolutional_layer}}\pysiglinewithargsret{\sphinxcode{\sphinxupquote{cnn\_indoor\_classifier\_model.}}\sphinxbfcode{\sphinxupquote{convolutional\_layer}}}{\emph{input\_x}, \emph{shape}}{}
Realiza las operaciones que se hacen en una capa
de convolución. La convolución el agrupamiento
y la aplicación de la función de activación ReLU.
\begin{quote}\begin{description}
\item[{Parámetros}] \leavevmode\begin{itemize}
\item {} 
\sphinxstyleliteralstrong{\sphinxupquote{input\_x}} (\sphinxstyleliteralemphasis{\sphinxupquote{tensor.}}) \textendash{} La entrada de la convolución.

\item {} 
\sphinxstyleliteralstrong{\sphinxupquote{shape}} \textendash{} Una lista con el tamaño de los kernels, el número de mapas de características de entrada y el número de mapas de caracterísitcas de salida.

\end{itemize}

\item[{Devuelve}] \leavevmode
Un tensor con los resultados de las operaciones de la capa de convolución.

\end{description}\end{quote}

\end{fulllineitems}

\index{get\_labels\_from\_path() (en el módulo cnn\_indoor\_classifier\_model)}

\begin{fulllineitems}
\phantomsection\label{\detokenize{model_desc:cnn_indoor_classifier_model.get_labels_from_path}}\pysiglinewithargsret{\sphinxcode{\sphinxupquote{cnn\_indoor\_classifier\_model.}}\sphinxbfcode{\sphinxupquote{get\_labels\_from\_path}}}{\emph{file\_path}}{}
Crea un arreglo de numpy con las categoría de
cada imagen según la carpeta en la que se encuentra.
Utiliza el one-hot encoding para representar
las categorías, esto es:
\begin{itemize}
\item {} 
\sphinxcode{\sphinxupquote{{[}1, 0, 0, 0{]}}}, la imagen es de la categoría \sphinxstyleemphasis{desks}.

\item {} 
\sphinxcode{\sphinxupquote{{[}0, 1, 0, 0{]}}}, la imagen es de la categoría \sphinxstyleemphasis{exit}.

\item {} 
\sphinxcode{\sphinxupquote{{[}0, 0, 1, 0{]}}}, la imagen es de la categoría \sphinxstyleemphasis{office}.

\item {} 
\sphinxcode{\sphinxupquote{{[}0, 0, 0, 1{]}}}, la imagen es de la categoría \sphinxstyleemphasis{soccer\_court}.

\end{itemize}
\begin{quote}\begin{description}
\item[{Parámetros}] \leavevmode
\sphinxstyleliteralstrong{\sphinxupquote{file\_path}} (\sphinxstyleliteralemphasis{\sphinxupquote{str.}}) \textendash{} La ruta del archivo.

\end{description}\end{quote}

\end{fulllineitems}

\index{hot\_encoding\_from\_category() (en el módulo cnn\_indoor\_classifier\_model)}

\begin{fulllineitems}
\phantomsection\label{\detokenize{model_desc:cnn_indoor_classifier_model.hot_encoding_from_category}}\pysiglinewithargsret{\sphinxcode{\sphinxupquote{cnn\_indoor\_classifier\_model.}}\sphinxbfcode{\sphinxupquote{hot\_encoding\_from\_category}}}{\emph{val}, \emph{number\_of\_categories=4}}{}
Función para crear un vector one-hot dado un número de
categorías y el índice de la clase correcta.
\begin{quote}\begin{description}
\item[{Parámetros}] \leavevmode\begin{itemize}
\item {} 
\sphinxstyleliteralstrong{\sphinxupquote{val}} (\sphinxstyleliteralemphasis{\sphinxupquote{int.}}) \textendash{} el número de la clase (índice de la categoría).

\item {} 
\sphinxstyleliteralstrong{\sphinxupquote{number\_of\_categories}} (\sphinxstyleliteralemphasis{\sphinxupquote{int.}}) \textendash{} el número de categorías 4 por defecto.

\end{itemize}

\item[{Devuelve}] \leavevmode
un arreglo de \sphinxcode{\sphinxupquote{numpy}} que es el vector one-hot.

\end{description}\end{quote}

\end{fulllineitems}

\index{init\_bias() (en el módulo cnn\_indoor\_classifier\_model)}

\begin{fulllineitems}
\phantomsection\label{\detokenize{model_desc:cnn_indoor_classifier_model.init_bias}}\pysiglinewithargsret{\sphinxcode{\sphinxupquote{cnn\_indoor\_classifier\_model.}}\sphinxbfcode{\sphinxupquote{init\_bias}}}{\emph{shape}}{}
Retorna una variable de TensorFlow con las dimensiones 
que se pasen como argumento.

La variable representa los sesgos en una red neuronal,
y estos se inicializan con un valor constante igual a 0.1.
\begin{quote}\begin{description}
\item[{Parámetros}] \leavevmode
\sphinxstyleliteralstrong{\sphinxupquote{shape}} (\sphinxstyleliteralemphasis{\sphinxupquote{list.}}) \textendash{} Las dimensiones de la variable.

\item[{Devuelve}] \leavevmode
Una variable de TensorFlow.

\end{description}\end{quote}

\end{fulllineitems}

\index{init\_weights() (en el módulo cnn\_indoor\_classifier\_model)}

\begin{fulllineitems}
\phantomsection\label{\detokenize{model_desc:cnn_indoor_classifier_model.init_weights}}\pysiglinewithargsret{\sphinxcode{\sphinxupquote{cnn\_indoor\_classifier\_model.}}\sphinxbfcode{\sphinxupquote{init\_weights}}}{\emph{shape}}{}
Retorna una variable de TensorFlow con las dimensiones 
que se pasen como argumento.

La variable representa los pesos en una red neuronal,
y estos se inicializan con una distribución normal, 
con media cero y desviación estándar igual a 0.1.
\begin{quote}\begin{description}
\item[{Parámetros}] \leavevmode
\sphinxstyleliteralstrong{\sphinxupquote{shape}} (\sphinxstyleliteralemphasis{\sphinxupquote{list.}}) \textendash{} Las dimensiones de la variable.

\item[{Devuelve}] \leavevmode
Una variable de TensorFlow.

\end{description}\end{quote}

\end{fulllineitems}

\index{list\_files\_in\_directory() (en el módulo cnn\_indoor\_classifier\_model)}

\begin{fulllineitems}
\phantomsection\label{\detokenize{model_desc:cnn_indoor_classifier_model.list_files_in_directory}}\pysiglinewithargsret{\sphinxcode{\sphinxupquote{cnn\_indoor\_classifier\_model.}}\sphinxbfcode{\sphinxupquote{list\_files\_in\_directory}}}{\emph{directory\_name}}{}
Crea una lista con las rutas de todos los archivos dentro de un
directorio.
\begin{quote}\begin{description}
\item[{Parámetros}] \leavevmode
\sphinxstyleliteralstrong{\sphinxupquote{directory\_name}} (\sphinxstyleliteralemphasis{\sphinxupquote{str.}}) \textendash{} la ruta del directorio relativa al espacio de trabajo.

\item[{Devuelve}] \leavevmode
list \textendash{} una lista de con las rutas de los archivos en el directorio dado.

\end{description}\end{quote}

\end{fulllineitems}

\index{max\_pool\_2by2() (en el módulo cnn\_indoor\_classifier\_model)}

\begin{fulllineitems}
\phantomsection\label{\detokenize{model_desc:cnn_indoor_classifier_model.max_pool_2by2}}\pysiglinewithargsret{\sphinxcode{\sphinxupquote{cnn\_indoor\_classifier\_model.}}\sphinxbfcode{\sphinxupquote{max\_pool\_2by2}}}{\emph{x}}{}
Realiza el agrupamiento máximo con vecidades de 2 por 2.
Usa el método \sphinxcode{\sphinxupquote{max\_pool}} de TF.
\begin{quote}\begin{description}
\item[{Parámetros}] \leavevmode
\sphinxstyleliteralstrong{\sphinxupquote{x}} (\sphinxstyleliteralemphasis{\sphinxupquote{tensor.}}) \textendash{} El tensor al que se le aplica el agrupamiento máximo.

\item[{Devuelve}] \leavevmode
El resultado de la operación de agrupamiento.

\end{description}\end{quote}

\end{fulllineitems}

\index{normal\_full\_layer() (en el módulo cnn\_indoor\_classifier\_model)}

\begin{fulllineitems}
\phantomsection\label{\detokenize{model_desc:cnn_indoor_classifier_model.normal_full_layer}}\pysiglinewithargsret{\sphinxcode{\sphinxupquote{cnn\_indoor\_classifier\_model.}}\sphinxbfcode{\sphinxupquote{normal\_full\_layer}}}{\emph{input\_layer}, \emph{size}}{}
Realiza las operaciones de una capa completamente 
conectada en una red neuronal, sin aplicar
la función de activación, sólo la multiplicación 
de matrices y la suma del sesgo.
\begin{quote}\begin{description}
\item[{Parámetros}] \leavevmode\begin{itemize}
\item {} 
\sphinxstyleliteralstrong{\sphinxupquote{input\_layer}} (\sphinxstyleliteralemphasis{\sphinxupquote{tensor.}}) \textendash{} La entrada de la capa.

\item {} 
\sphinxstyleliteralstrong{\sphinxupquote{size}} (\sphinxstyleliteralemphasis{\sphinxupquote{int.}}) \textendash{} el número de unidades de la capa.

\end{itemize}

\item[{Devuelve}] \leavevmode
El resultado de la multiplicación de una matriz de pesos y la entrada de la capa, más un vector de sesgos.

\end{description}\end{quote}

\end{fulllineitems}

\subsection{Resultados}

Para medir el desempeño de las distintas combinaciones de parámetros para
formar las arquitecturas antes descritas, se empleó como \textit{medida
de desempeño} la \textit{precisión}. La precisión es simplemente la proporción
de muestras para las cuales un modelo produce la salida correcta.
La ejecución del entrenamiento de cada modelo produce como salida
una tabla donde se muestra la precisión cada número de época múltiplo de $100$.
Supongamos el caso de una red con las siguientes características:

\begin{itemize}
\item \textbf{Número de épocas:} 500.
\item \textbf{Tasa de aprendizaje:} 0.0001.
\item \textbf{Tamaño del lote:} 100.
\item \textbf{Número de capas de convolución:} 1.
\item \textbf{Número de mapas de caracteríticas obtenidas de la capa de convolución:} 32.
\item \textbf{Unidades en la capa oculta:} 512.
\end{itemize}

La salida del programa al ejecutar la gráfica de cómputo que
realiza el entrenamiento de la red es la mostrada en la tabla \ref{table:output_example}.

\begin{table}[!h]
\centering
\caption{Salida de la ejecución de la red de cómputo en TensorFlow}\label{table:output_example}

\begin{tabular}{|l|l|l|l|l|}
\hline
Época                        & Tiempo transcurrido         & Error de entrenamiento & Error de prueba    & Precisión     \\ \hline
0                        & 0.1856634617 & 6.8891782761  & 6.3154983521 & 0.2571022809 \\ \hline
100                      & 1.7081215382 & 1.6550145149  & 1.512375474  & 0.354640156  \\ \hline
200                      & 3.1971487999 & 1.1340529919  & 1.1259912252 & 0.5241477489 \\ \hline
300                      & 4.6470386982 & 1.0694072247  & 0.9105868936 & 0.6197916865 \\ \hline
400                      & 6.0883014202 & 0.8224871755  & 0.7770082951 & 0.6808711886 \\ \hline
500                      & 7.5487713814 & 0.7760010362  & 0.6851239204 & 0.7452651262 \\ \hline
\end{tabular}
\end{table}

A partir de los datos generados por cada modelo se
buscó por aquellas arquitecturas con la precisión más alta.
En total se encontraron cinco estructuras que se describen en seguida.

\paragraph{Modelo 1}

Una red con dos capas convolucionales y una capa oculta.

\begin{itemize}
\item \textbf{Número de épocas:} 2000.
\item \textbf{Tasa de aprendizaje:} 0.01.
\item \textbf{Tamaño del lote:} 200.
\item \textbf{Número de capas de convolución:} 2.
\item \textbf{Dimensiones de los kernels:} $5 \times 5$ y $3 \times 3$.
\item \textbf{Número de mapas de caracteríticas obtenidas de las capa de convolución:} 16 y 32.
\item \textbf{Unidades en la capa oculta:} 1024.
\item \textbf{Precisión:} 0.9791666865.
\end{itemize}


\begin{table}[]
\centering
\caption{Modelo 1: 2 capas de convolución y una capa oculta}
\label{table:red_one_2conv2fc}
\begin{tabular}{|l|l|l|l|l|}
\hline
Época & Tiempo transcurrido & Error de entrenamiento & Error de prueba & Precisión    \\ \hline
0     & 7.3202745914        & 3.0709540844           & 6.1779737473    & 0.2504734993 \\ \hline
100   & 15.3162293434       & 0.246046558            & 0.2345223427    & 0.9384469986 \\ \hline
200   & 23.2666277885       & 0.2221536636           & 0.1732220352    & 0.9351325631 \\ \hline
300   & 31.2877190113       & 0.115122892            & 0.1277082711    & 0.9592803121 \\ \hline
400   & 39.3300793171       & 0.0641306639           & 0.1061202958    & 0.9630681872 \\ \hline
500   & 47.3877923489       & 0.0945049077           & 0.1096037999    & 0.9592803121 \\ \hline
600   & 55.337015152        & 0.04487269             & 0.0840779245    & 0.9725378752 \\ \hline
700   & 63.4053928852       & 0.0290345363           & 0.0777522177    & 0.9725378752 \\ \hline
800   & 71.4111871719       & 0.0502839647           & 0.0865348205    & 0.96875      \\ \hline
900   & 79.4459328651       & 0.0260504074           & 0.0731359869    & 0.9758522511 \\ \hline
1000  & 87.4989733696       & 0.0170939807           & 0.0684150234    & 0.9772727489 \\ \hline
1100  & 95.6583366394       & 0.0299351327           & 0.073277019     & 0.9739583135 \\ \hline
1200  & 103.6676399708      & 0.0175534561           & 0.0689379498    & 0.9749053121 \\ \hline
1300  & 111.7657170296      & 0.0112776877           & 0.0651408285    & 0.9767992496 \\ \hline
1400  & 119.8008499146      & 0.0180956665           & 0.0692276433    & 0.9777461886 \\ \hline
1500  & 127.9180393219      & 0.0126623595           & 0.06640324      & 0.9777461886 \\ \hline
1600  & 135.9379198551      & 0.0079146475           & 0.0643871278    & 0.9772727489 \\ \hline
1700  & 144.0074915886      & 0.0117598884           & 0.0699412525    & 0.9777461886 \\ \hline
1800  & 152.1184546947      & 0.0093412921           & 0.0654446557    & 0.9791666865 \\ \hline
1900  & 160.2102677822      & 0.0058462457           & 0.064642258     & 0.9791666865 \\ \hline
2000  & 168.2911362648      & 0.0085604852           & 0.0712103397    & 0.9777461886 \\ \hline
\end{tabular}
\end{table}

\paragraph{Modelo 2}


Una red con dos capas convolucionales y una capa oculta.

\begin{itemize}
\item \textbf{Número de épocas:} 2000.
\item \textbf{Tasa de aprendizaje:} 0.01.
\item \textbf{Tamaño del lote:} 300.
\item \textbf{Número de capas de convolución:} 2.
\item \textbf{Dimensiones de los kernels:} $5 \times 5$ y $3 \times 3$.
\item \textbf{Número de mapas de caracteríticas obtenidas de las capa de convolución:} 16 y 32.
\item \textbf{Unidades en la capa oculta:} 1024.
\item \textbf{Precisión:} 0.9791666865.
\end{itemize}

\begin{table}[]
\centering
\caption{Modelo 2: 2 capas de convolución y una oculta}
\label{table:2conv2fcbatch300}
\begin{tabular}{|l|l|l|l|l|}
\hline
Época & Tiempo transcurrido & Error de entrenamiento & Error de prueba & Precisión    \\ \hline
0     & 7.8421280384        & 2.8682234287           & 3.3460187912    & 0.241477266  \\ \hline
100   & 16.9836273193       & 0.3026786149           & 0.2827074528    & 0.9214015007 \\ \hline
200   & 26.0736353397       & 0.1818257272           & 0.1859790981    & 0.9441288114 \\ \hline
300   & 35.2737841606       & 0.1262233555           & 0.1477904469    & 0.9498106241 \\ \hline
400   & 44.4421823025       & 0.1001303717           & 0.1264027655    & 0.9550189376 \\ \hline
500   & 53.6251189709       & 0.0818474591           & 0.1112083346    & 0.9554924369 \\ \hline
600   & 62.7804038525       & 0.0689618215           & 0.1012903452    & 0.9625946879 \\ \hline
700   & 72.0268294811       & 0.0598993674           & 0.0959407836    & 0.9659090638 \\ \hline
800   & 81.1530988216       & 0.0549877919           & 0.0937793404    & 0.9659090638 \\ \hline
900   & 90.3930594921       & 0.0481151342           & 0.0898199677    & 0.9696969986 \\ \hline
1000  & 99.5937502384       & 0.0406286344           & 0.085127607     & 0.9734848738 \\ \hline
1100  & 108.7840344906      & 0.0342642255           & 0.08182735      & 0.9749053121 \\ \hline
1200  & 117.9713246822      & 0.0278111324           & 0.0783735365    & 0.9767992496 \\ \hline
1300  & 127.1896910667      & 0.0229252577           & 0.0762095153    & 0.9772727489 \\ \hline
1400  & 136.4318993092      & 0.0191811435           & 0.0749697983    & 0.9782196879 \\ \hline
1500  & 145.6369526386      & 0.0163648594           & 0.0740553215    & 0.9786931872 \\ \hline
1600  & 154.8179557323      & 0.0139603755           & 0.0733383074    & 0.9791666865 \\ \hline
1700  & 164.0475275517      & 0.0120949261           & 0.0730332807    & 0.9786931872 \\ \hline
1800  & 173.2340805531      & 0.0105913281           & 0.0729769841    & 0.9772727489 \\ \hline
1900  & 182.4875264168      & 0.0094167981           & 0.0730196685    & 0.9772727489 \\ \hline
2000  & 191.7189230919      & 0.0084654102           & 0.0732315555    & 0.9772727489 \\ \hline
\end{tabular}
\end{table}

\paragraph{Modelo 3}

Red con dos capas convolucionales y con una sola capa
completamente conectada, es decir, la capa de salida.

\begin{itemize}
\item \textbf{Número de épocas:} 2000.
\item \textbf{Tasa de aprendizaje:} 0.01.
\item \textbf{Tamaño del lote:} 300.
\item \textbf{Número de capas de convolución:} 2.
\item \textbf{Dimensiones de los kernels:} $7 \times 7$ y $5 \times 5$.
\item \textbf{Número de mapas de caracteríticas obtenidas de las capa de convolución:} 16 y 32.
\item \textbf{Unidades en la capa oculta:} 1024.
\item \textbf{Precisión:} 0.9715909362.
\end{itemize}

\begin{table}[]
\centering
\caption{Modelo 3: 2 capas de convolución 1 completamente conectada}
\label{table:model3}
\begin{tabular}{|l|l|l|l|l|}
\hline
Época & Tiempo transcurrido & Error de entrenamiento & Error de prueba & Precisión    \\ \hline
0     & 2.1028752327        & 5.429710865            & 7.2814497948    & 0.2642045319 \\ \hline
100   & 6.0483319759        & 0.3088294864           & 0.3315916061    & 0.9057765007 \\ \hline
200   & 9.9446578026        & 0.207638666            & 0.2255852371    & 0.9289772511 \\ \hline
300   & 13.8171753883       & 0.3040256202           & 0.2913019061    & 0.9005681872 \\ \hline
400   & 17.6553535461       & 0.1283937991           & 0.1651451886    & 0.9427083135 \\ \hline
500   & 21.5917801857       & 0.1089217067           & 0.157908693     & 0.9403409362 \\ \hline
600   & 25.4469769001       & 0.1987725496           & 0.1853734255    & 0.9346590638 \\ \hline
700   & 29.3310687542       & 0.0858137906           & 0.1235252693    & 0.9602272511 \\ \hline
800   & 33.225922823        & 0.0737339705           & 0.1292105466    & 0.9535984993 \\ \hline
900   & 37.1445102692       & 0.1441168338           & 0.1384231746    & 0.9521780014 \\ \hline
1000  & 41.0286338329       & 0.0607568994           & 0.1053966731    & 0.9692234993 \\ \hline
1100  & 44.9228360653       & 0.0545381755           & 0.1131819487    & 0.9621211886 \\ \hline
1200  & 48.7702672482       & 0.1076918915           & 0.1140853167    & 0.9616477489 \\ \hline
1300  & 52.6551091671       & 0.0456284918           & 0.0980426818    & 0.9701704383 \\ \hline
1400  & 56.4932639599       & 0.0431456864           & 0.1037823632    & 0.9673295617 \\ \hline
1500  & 60.3736059666       & 0.0825673938           & 0.1017076969    & 0.9668560624 \\ \hline
1600  & 64.2060873508       & 0.0358270295           & 0.0946439132    & 0.9701704383 \\ \hline
1700  & 68.1200900078       & 0.034217272            & 0.0975136459    & 0.9706439376 \\ \hline
1800  & 72.0071187019       & 0.0638250113           & 0.0943133309    & 0.9701704383 \\ \hline
1900  & 75.883657217        & 0.0286074877           & 0.0940102115    & 0.9711174369 \\ \hline
2000  & 79.7310190201       & 0.0272229407           & 0.0934648365    & 0.9715909362 \\ \hline
\end{tabular}
\end{table}



\paragraph{Modelo 4}

Red con una capa convolucional y con dos capas
completamente conectadas.

\begin{itemize}
\item \textbf{Número de épocas:} 2000.
\item \textbf{Tasa de aprendizaje:} 0.01.
\item \textbf{Tamaño del lote:} 200.
\item \textbf{Número de capas de convolución:} 2.
\item \textbf{Dimensiones de los kernels:} $5 \times 5$.
\item \textbf{Número de mapas de caracteríticas obtenidas de las capa de convolución:} $16$.
\item \textbf{Unidades en la capa oculta:} $512$.
\item \textbf{Precisión:} 0.9777461886.
\end{itemize}


\begin{table}[]
\centering
\caption{Tabla 4: 2 capas de convolución 1 completamente conectada.}
\label{table:model4}
\begin{tabular}{|l|l|l|l|l|}
\hline
Época & Tiempo transcurrido & Error de entrenamiento & Error de prueba & Precisión    \\ \hline
0     & 4.8190894127        & 3.4078867435           & 1.8335254192    & 0.2561553121 \\ \hline
100   & 10.5430588722       & 0.2876966596           & 0.2554202676    & 0.9190340638 \\ \hline
200   & 16.2716712952       & 0.2585280836           & 0.1728164405    & 0.9441288114 \\ \hline
300   & 21.984313488        & 0.2033729404           & 0.1497282833    & 0.9446022511 \\ \hline
400   & 27.6639053822       & 0.133350119            & 0.139832899     & 0.9464961886 \\ \hline
500   & 33.3875226974       & 0.1221942604           & 0.1160472557    & 0.9635416865 \\ \hline
600   & 39.071198225        & 0.1346343458           & 0.111930415     & 0.9663825631 \\ \hline
700   & 44.7847318649       & 0.0859281719           & 0.1102579832    & 0.9592803121 \\ \hline
800   & 50.4557569027       & 0.0763712376           & 0.0923053026    & 0.9715909362 \\ \hline
900   & 56.1923205853       & 0.0939286873           & 0.0953650922    & 0.9706439376 \\ \hline
1000  & 61.9322435856       & 0.0572218187           & 0.0942356661    & 0.9678030014 \\ \hline
1100  & 67.6439433098       & 0.0517053083           & 0.0810680985    & 0.9763257504 \\ \hline
1200  & 73.3566286564       & 0.0669792295           & 0.0883587152    & 0.9730113745 \\ \hline
1300  & 79.0891780853       & 0.0388560072           & 0.0845807865    & 0.9734848738 \\ \hline
1400  & 84.7962470055       & 0.0358598083           & 0.07602036      & 0.9772727489 \\ \hline
1500  & 90.4903838634       & 0.0488497727           & 0.0864205137    & 0.9725378752 \\ \hline
1600  & 96.1585409641       & 0.0279216766           & 0.0798797309    & 0.9763257504 \\ \hline
1700  & 101.8997781277      & 0.0248262715           & 0.0739214197    & 0.9777461886 \\ \hline
1800  & 107.6456582546      & 0.0374025591           & 0.0860014334    & 0.9734848738 \\ \hline
1900  & 113.370872736       & 0.0207491834           & 0.0776060075    & 0.9772727489 \\ \hline
2000  & 119.1330387592      & 0.0177799109           & 0.0743050575    & 0.9767992496 \\ \hline
\end{tabular}
\end{table}

\paragraph{Modelo 5}

Red con dos capas, una de convolución y una de salida.

\begin{itemize}
\item \textbf{Número de épocas:} 2000.
\item \textbf{Tasa de aprendizaje:} 0.01.
\item \textbf{Tamaño del lote:} 100.
\item \textbf{Número de capas de convolución:} 2.
\item \textbf{Dimensiones de los kernels:} $5 \times 5$.
\item \textbf{Número de mapas de caracteríticas obtenidas de las capa de convolución:} $32$.
\item \textbf{Precisión:} 0.9611742496.
\end{itemize}


\begin{table}[]
\centering
\caption{Modelo 5: 1 capa de convolución 1 capa completamente conectada}
\label{table:model5}
\begin{tabular}{|l|l|l|l|l|}
\hline
Época & Tiempo transcurrido & Error de entrenamiento & Error de prueba & Precisión    \\ \hline
0     & 1.3093357086        & 1.9210841656           & 1.9909992218    & 0.2334280312 \\ \hline
100   & 3.5064704418        & 0.4699867666           & 0.3694962263    & 0.9019886255 \\ \hline
200   & 5.718821764         & 0.2064059079           & 0.2735858858    & 0.9195075631 \\ \hline
300   & 7.9507069588        & 0.321207583            & 0.2440538406    & 0.9180871248 \\ \hline
400   & 10.1652519703       & 0.2827648222           & 0.2159627676    & 0.9261363745 \\ \hline
500   & 12.399163723        & 0.1229303852           & 0.200532645     & 0.9313446879 \\ \hline
600   & 14.6175277233       & 0.2280674428           & 0.192340821     & 0.9299242496 \\ \hline
700   & 16.8630867004       & 0.2157354206           & 0.1778713614    & 0.9308711886 \\ \hline
800   & 19.0751066208       & 0.0978225544           & 0.1696051657    & 0.9384469986 \\ \hline
900   & 21.3231885433       & 0.1785159856           & 0.1666600704    & 0.9370265007 \\ \hline
1000  & 23.5529901981       & 0.1691013724           & 0.1553606391    & 0.9427083135 \\ \hline
1100  & 25.7864727974       & 0.0828434825           & 0.1507069916    & 0.9441288114 \\ \hline
1200  & 28.0087628365       & 0.1457120925           & 0.1495355666    & 0.9450757504 \\ \hline
1300  & 30.241517067        & 0.1372746676           & 0.1403872371    & 0.9493371248 \\ \hline
1400  & 32.4557955265       & 0.0717598572           & 0.1377647668    & 0.9535984993 \\ \hline
1500  & 34.7012207508       & 0.1211939454           & 0.1370747089    & 0.9507575631 \\ \hline
1600  & 36.9285581112       & 0.1132190526           & 0.1292845905    & 0.9554924369 \\ \hline
1700  & 39.1642153263       & 0.0621341504           & 0.1280657798    & 0.9569128752 \\ \hline
1800  & 41.3848059177       & 0.1034243405           & 0.1276608855    & 0.9550189376 \\ \hline
1900  & 43.6157948971       & 0.0946448147           & 0.120853819     & 0.9592803121 \\ \hline
2000  & 45.837754488        & 0.055428572            & 0.1209859028    & 0.9611742496 \\ \hline
\end{tabular}
\end{table}

Los modelos con la mayor precisión sobre todos son el modelo uno y
dos con una proporción de valores correctamente clasficados del
0.9791666865. Se puede ver que los parámetros que comparten todas
las arquitecturas son el número de épocas y la tasa de aprendizaje,
además de que en sus kernels aparece al menos uno con una dimensión 
de $5 \times 5$.

%
%\end{document}
