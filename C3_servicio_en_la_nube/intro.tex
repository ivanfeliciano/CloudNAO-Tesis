%\documentclass[10pt,a4paper]{article}
%\usepackage[utf8]{inputenc}
%\usepackage{amsmath}
%\usepackage{amsfonts}
%\usepackage{amssymb}
%\begin{document}
%

Para completar la arquitectura CloudNAO falta implementar un modelo de aprendizaje 
profundo que se brinde como un servicio web para ser consumido por robots NAO.
A pesar de que el servicio de detección de objetos es mantenido por el LAR, y 
se adaptó para ser consumido a través de la API RESTful, no fue construido
desde cero ya que es parte de la API de detección de objetos de Tensorflow.
Es por eso que como parte final del proyecto se desarrolló un modelo para que 
resuelva la tarea de clasificar imágenes.

El problema de clasificación de imágenes es la tarea de asignarle a una imagen de
entrada una etiqueta a partir de un conjunto de categorías. Este es uno de los principales
problemas dentro del campo de la visión computacional, que a pesar de su simplicidad
tiene bastantes aplicaciones prácticas. Entre esas aplicaciones, muchas interesan al
campo de la robótica móvil, por ejemplo, para la navegación de un robot de manera
autónoma, nos gustaría que supiera en que lugar está simplemente con una fotografía
que obtenga en ese momento desde sus cámaras, así podría saber si ha llegado al 
lugar de su objetivo, o a partir de la zona donde se ubica planear una trayectoria.

Lo anterior nos inspiró en la creación de un modelo que clasificara imágenes
de algunos lugares sobre los que podría navegar el robot NAO. 
Como solución a este problema de clasificación se propuso usar 
una red neuronal convolucional, que recibiera como entrada un arreglo con los
píxeles de una imagen tomada por el robot, y la salida fuera la categoría
a la que pertenece esa imagen. 
Las clases en las que se desea clasificar las imágenes son lugares alrededor
del Laboratorio de Algoritmos para la Robótica, que se ubica en el cubículo $15$ del
Centro de Desarrollo Tecnológico de la FES Acatlán. Se eligieron las siguientes cuatro zonas:

\begin{itemize}
    \item El cubículo.
    \item La salida de emergencia.
    \item La cancha de entrenamiento de fútbol para el robot NAO.
    \item Zona de trabajo del Laboratorio.
\end{itemize}


En este capítulo se describe con detalle la confección del
conjunto de datos, además de cómo se eligió una arquitectura de la red 
convolucional que tuviera un buen desempeño a través de pruebas
de diferentes estrucuturas modificando diversos parámetros, la implementación
de las múltiples redes utilizando Python y TensorFlow, los resultados obtenidos
y finalmente cómo se integra dentro de la API de CloudNAO el modelo
de aprendizaje profundo obtenido.

%
%\end{document}