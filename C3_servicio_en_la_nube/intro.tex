%\documentclass[10pt,a4paper]{article}
%\usepackage[utf8]{inputenc}
%\usepackage{amsmath}
%\usepackage{amsfonts}
%\usepackage{amssymb}
%\begin{document}
%

Este capítulo se enfoca en el segundo 
objetivo del trabajo que es: desarrollar un 
servicio de cómputo en la nube que utilice 
algoritmos de aprendizaje automático, para 
robots NAO.

Como parte final del trabajo se desarrolló un modelo que resolviera la tarea de clasificar imágenes.
El problema de clasificación de imágenes es la tarea de asignarle a una imagen de
entrada una etiqueta a partir de un conjunto de categorías. Éste es uno de los principales
problemas dentro del campo de la visión computacional, que a pesar de su simplicidad
tiene bastantes aplicaciones prácticas. Por ejemplo, dentro de la planeación automática de movimiento, para el problema de localización de un robot
autónomo, nos gustaría que supiera en que lugar está simplemente con una fotografía
que obtenga en ese momento, así podría saber si ha llegado al lugar de su objetivo, o a partir de la zona donde se ubica planear una trayectoria.

Lo anterior inspiró a la creación de un modelo que clasificara imágenes
de algunos lugares sobre los que podría navegar el robot NAO. 
Como solución a este problema de clasificación se propuso usar 
una red neuronal convolucional, que recibiera como entrada un arreglo con los
pixeles de una imagen tomada por el robot, y la salida fuera la categoría
a la que pertenece esa imagen. 
Las clases en las que se desea clasificar las imágenes son lugares alrededor
del Laboratorio de Algoritmos para la Robótica.

En este capítulo se describe con detalle la confección del conjunto de datos, la arquitectura de la red 
convolucional, la implementación
del modelo utilizando Python y TensorFlow, los resultados obtenidos
y finalmente cómo se integra este modelo dentro de la API de CloudNAO. 

%
%\end{document}