

\section*{Flask}
\label{\detokenize{chapter_two/desc_cloudnao:flask}}

Flask es un micro framework para desarrollo web. Fue diseñado para ser un
framework extensible, sólo provee un núcleo sólido con servicios básicos,
mientras que las extensiones brindan el resto. Esto quiere decir que Flask,
te da la posibilidad de como desarrollador contar sólo con las dependencias que
necesitas.

Tiene dos dependencias principales, \sphinxstylestrong{Werkzeug} y \sphinxstylestrong{Jinja2}.
El enrutamiento, debugging y WSGI tienen procedencia de Werkzeug.
Jinga2 ofrece el soporte para el manejo de plantillas. Para nuestra
aplicación solamente hacemos uso de la primera.
\begin{quote}

\sphinxstyleemphasis{WSGI} es la Web Server Gateway Interface. Es un protocolo que describe
cómo un servido web se comunica con aplicaciones web, y cómo las aplicaciones
web se pueden encadenar para procesar una petición.
\end{quote}

Para la instalación de Flask así como de cualquier paquete de Python, se
recomienda crear un ambiente virtual con la herramienta \sphinxstylestrong{virtualenv}
Un ambiente virtual es una copia privada del intérprete de Python sobre la cual
es posible instalar paquetes de manera privada, sin afectar al intérprete
global de Python. Permite crear un ambiente de trabajo aislado para cada
proyecto, así cada aplicación tiene acceso solo a los paquetes que utiliza.

Con esto, todo lo descrito a continuación que se refiera a la instalación de
paquetes o ejecución de programas se hace dentro de un ambiente virtual.


\subsection*{Inicialización}
\label{\detokenize{chapter_two/desc_cloudnao:inicializacion}}
Todas las aplicaciones de Flask deben crear una \sphinxstyleemphasis{instancia de la aplicación}. Este
objeto maneja todas las peticiones que recibe el servidor web, usando
la especificación WSGI.
La instancia de la aplicación es un objeto de la clase Flask,
se crea de la siguiente manera:

\begin{sphinxVerbatim}[commandchars=\\\{\}]
\PYG{k+kn}{from} \PYG{n+nn}{flask} \PYG{k}{import} \PYG{n}{Flask}
\PYG{n}{app} \PYG{o}{=} \PYG{n}{Flask}\PYG{p}{(}\PYG{n+nv+vm}{\PYGZus{}\PYGZus{}name\PYGZus{}\PYGZus{}}\PYG{p}{)}
\end{sphinxVerbatim}

El único argumento requerido para el constructor de la clase Flask es el nombre
del módulo o paquete de la aplicación. Para la mayoría de los casos,
la variable de Python \sphinxcode{\_\_name\_\_} es el valor correcto. Flask usa este argumento
para determinar la ruta raíz de la aplicación, para que después pueda encontrar
los archivos de los recursos relativos a la ubicación de la aplicación.


\subsection*{Rutas}
\label{\detokenize{chapter_two/desc_cloudnao:rutas}}
Los clientes envían peticiones al servidor web, que a su vez las envía a la
instancia de la aplicación de Flask. La instancia de la aplicación necesita saber
qué código debe regresar para cada URL solicitado, así que mantiene un mapeo de
los URL a funciones de Python. La asociación entre el URL y la función que se
la maneja se llama \sphinxstyleemphasis{ruta}.

La manera más conveniente de definir una ruta en una aplicación de Flask es
a través del decorador \sphinxcode{app.route}, que registra la función decorada como
una ruta.

El siguiente ejemplo registra la función \sphinxcode{index()} como el manejador para el URL
raíz de la aplicación. Si la aplicación fue desplegada en un servidor con el
nombre del dominio \sphinxurl{http://www.lar.com}. Al acceder desde un navegador a
la dirección anterior se ejecutará la función \sphinxcode{index()} del lado del servidor.
El valor
retornado por esta función se llama \sphinxstyleemphasis{respuesta}, y es lo que el cliente recibe.
Las funciones como \sphinxcode{index()} se llaman \sphinxstyleemphasis{funciones de vista}.
En este caso el navegador mostrará \sphinxcode{Hola mundo}.

\begin{sphinxVerbatim}[commandchars=\\\{\}]
\PYG{n+nd}{@app}\PYG{o}{.}\PYG{n}{route}\PYG{p}{(}\PYG{l+s+s1}{\PYGZsq{}}\PYG{l+s+s1}{/}\PYG{l+s+s1}{\PYGZsq{}}\PYG{p}{)}
\PYG{k}{def} \PYG{n+nf}{index}\PYG{p}{(}\PYG{p}{)}\PYG{p}{:}
  \PYG{k}{return} \PYG{l+s+s2}{\PYGZdq{}}\PYG{l+s+s2}{Hola mundo}\PYG{l+s+s2}{\PYGZdq{}}
\end{sphinxVerbatim}

Flask soporta los tipos de URL con componentes dinámicos utilizando
una sintaxis especial en el decorador \sphinxcode{route}. En el siguiente ejemplo se
muestra el nombre de usuario como componente dinámico.

\begin{sphinxVerbatim}[commandchars=\\\{\}]
\PYG{n+nd}{@app}\PYG{o}{.}\PYG{n}{route}\PYG{p}{(}\PYG{l+s+s1}{\PYGZsq{}}\PYG{l+s+s1}{/usuario/\PYGZlt{}nombre\PYGZgt{}}\PYG{l+s+s1}{\PYGZsq{}}\PYG{p}{)}
\PYG{k}{def} \PYG{n+nf}{user}\PYG{p}{(}\PYG{n}{nombre}\PYG{p}{)}\PYG{p}{:}
  \PYG{k}{return} \PYG{l+s+s2}{\PYGZdq{}}\PYG{l+s+s2}{Hola, }\PYG{l+s+si}{\PYGZob{}\PYGZcb{}}\PYG{l+s+s2}{\PYGZdq{}}\PYG{o}{.}\PYG{n}{format}\PYG{p}{(}\PYG{n}{nombre}\PYG{p}{)}
\end{sphinxVerbatim}

La parte encerrada por los paréntesis angulares es el componente dinámico, así
que cualesquiera URL que coincidan con la porción estática será mapeada a esta
ruta.

El decorador \sphinxcode{app.route} cuenta con un argumento opcional \sphinxcode{methods},
el cual recibe
una lista de métodos HTTP. \sphinxcode{methods} le dice a Flask que registra a las
funciones de vista como manejadores para las peticiones de acuerdo
al tipo de método enviado. Cuando \sphinxcode{methods} no está el los parámetros de la
función, la función de vista es registrada para manejar solamente peticiones
\sphinxcode{GET}. El siguiente ejemplo se le pide a Flask que se registre a la
función de vista como manejadora de petición de tipo \sphinxcode{GET} y \sphinxcode{POST}.

\begin{sphinxVerbatim}[commandchars=\\\{\}]
\PYG{n+nd}{@app}\PYG{o}{.}\PYG{n}{route}\PYG{p}{(}\PYG{l+s+s1}{\PYGZsq{}}\PYG{l+s+s1}{/usuario/\PYGZlt{}nombre\PYGZgt{}}\PYG{l+s+s1}{\PYGZsq{}}\PYG{p}{,} \PYG{n}{methods}\PYG{o}{=}\PYG{p}{[}\PYG{l+s+s1}{\PYGZsq{}}\PYG{l+s+s1}{GET}\PYG{l+s+s1}{\PYGZsq{}}\PYG{p}{,} \PYG{l+s+s1}{\PYGZsq{}}\PYG{l+s+s1}{POST}\PYG{l+s+s1}{\PYGZsq{}}\PYG{p}{]}\PYG{p}{)}
\PYG{k}{def} \PYG{n+nf}{user}\PYG{p}{(}\PYG{n}{nombre}\PYG{p}{)}\PYG{p}{:}
  \PYG{k}{return} \PYG{l+s+s2}{\PYGZdq{}}\PYG{l+s+s2}{Hola, }\PYG{l+s+si}{\PYGZob{}\PYGZcb{}}\PYG{l+s+s2}{\PYGZdq{}}\PYG{o}{.}\PYG{n}{format}\PYG{p}{(}\PYG{n}{nombre}\PYG{p}{)}
\end{sphinxVerbatim}


\subsection*{Arranque del servidor}
\label{\detokenize{chapter_two/desc_cloudnao:arranque-del-servidor}}
La instancia de la aplicación tiene un tiene un método \sphinxcode{run} que inicia el
servidor web integrado con Flask, el cual sólo está destinado para usarse
durante el desarrollo.

\begin{sphinxVerbatim}[commandchars=\\\{\}]
\PYG{k}{if} \PYG{n+nv+vm}{\PYGZus{}\PYGZus{}name\PYGZus{}\PYGZus{}} \PYG{o}{==} \PYG{l+s+s1}{\PYGZsq{}}\PYG{l+s+s1}{\PYGZus{}\PYGZus{}main\PYGZus{}\PYGZus{}}\PYG{l+s+s1}{\PYGZsq{}}\PYG{p}{:}
  \PYG{n}{app}\PYG{o}{.}\PYG{n}{run}\PYG{p}{(}\PYG{n}{debug}\PYG{o}{=}\PYG{k+kc}{True}\PYG{p}{)}
\end{sphinxVerbatim}

Una vez que el servidor es puesto en marcha, entra en un bucle que espera por
peticiones para procesarlas. Este bucle continua hasta que la aplicación es
detenida.

Existen varios parámetros que \sphinxcode{app.run()} puede recibir para configurar el
modo de operación del servidor web. Durante el desarrollo, conviene  activar
el modo de debugging, que entre otras cosas activa el \sphinxstyleemphasis{debugger} y el \sphinxstyleemphasis{reloader}.
Esto se hace pasando el argumento \sphinxcode{debug} igual a \sphinxcode{True} .



\section*{Extensiones de Flask}
\label{\detokenize{chapter_two/desc_cloudnao:extensiones-de-flask}}
Flask está diseñado para extenderse. Funcionalidades como la autenticación y
bases de datos son funciones que el framework deja fuera intencionalmente,
dando la libertad de elegir los paquetes que se ajusten mejor a la aplicación.

%A continuación se describen las extensiones más importantes usadas en esta
%aplicación.


\subsection*{Flask-Script}
\label{\detokenize{chapter_two/desc_cloudnao:flask-script}}
Flask-Script es una extensión que añade un parseador de línea de comandos
para tu aplicación de Flask. Un ejemplo de una aplicación en la que se añade
el parseo en línea de comandos.

\begin{sphinxVerbatim}[commandchars=\\\{\}]
\PYG{k+kn}{from} \PYG{n+nn}{flask} \PYG{k}{import} \PYG{n}{Flask}
\PYG{k+kn}{from} \PYG{n+nn}{flask\PYGZus{}script} \PYG{k}{import} \PYG{n}{Manager}\PYG{p}{,} \PYG{n}{Shell}

\PYG{n}{app} \PYG{o}{=} \PYG{n}{Flask}\PYG{p}{(}\PYG{n+nv+vm}{\PYGZus{}\PYGZus{}name\PYGZus{}\PYGZus{}}\PYG{p}{)}
\PYG{n}{manager} \PYG{o}{=} \PYG{n}{Manager} \PYG{p}{(}\PYG{n}{app}\PYG{p}{)}

\PYG{k}{if} \PYG{n+nv+vm}{\PYGZus{}\PYGZus{}name\PYGZus{}\PYGZus{}} \PYG{o}{==} \PYG{l+s+s1}{\PYGZsq{}}\PYG{l+s+s1}{\PYGZus{}\PYGZus{}main\PYGZus{}\PYGZus{}}\PYG{l+s+s1}{\PYGZsq{}}\PYG{p}{:}
  \PYG{n}{manager}\PYG{o}{.}\PYG{n}{run}\PYG{p}{(}\PYG{p}{)}
\end{sphinxVerbatim}

El método de inicialización de esta extensión es común en muchas otras
extensiones: una instancia de la clase principal es inicializada pasando como
argumento en el constructor una instancia de la aplicación.

El servidor inicia a través de \sphinxcode{manager.run()}, donde la línea de comandos
es parseada.

Si corremos la aplicación anterior en una terminal, esta tiene opciones básicas
en la línea de comandos. Correr el programa anterior mostraría un mensaje como
el siguiente:

\begin{sphinxVerbatim}[commandchars=\\\{\}]
\PYGZdl{} python app.py
usage: app.py \PYG{o}{[}\PYGZhy{}h\PYG{o}{]} \PYG{o}{\PYGZob{}}shell,runserver\PYG{o}{\PYGZcb{}} ...

positional arguments:
\PYG{o}{\PYGZob{}}shell,runserver\PYG{o}{\PYGZcb{}}
  shell Runs a Python shell inside Flask application context.
  runserver Runs the Flask development server i.e. app.run\PYG{o}{(}\PYG{o}{)}
optional arguments:
  \PYGZhy{}h, \PYGZhy{}\PYGZhy{}help  show this \PYG{n+nb}{help} message and \PYG{n+nb}{exit}
\end{sphinxVerbatim}

El comando \sphinxcode{shell} es usado para iniciar un shell de Python en el contexto
de la aplicación. Sirve para tareas de mantenimiento, debugging, etc.

El comando \sphinxcode{runserver} como su nombre lo dice, inicia el servidor web. Cuenta
con varios argumentos opcionales, \sphinxcode{-{-}host, -{-}port, -{-}threaded, -{-}no-debug},
entre otros.


\subsection*{Flask-RESTful}
\label{\detokenize{chapter_two/desc_cloudnao:flask-restful}}
Flask-RESTful es una extensión para Flask que añade soporte para la rápida
construcción de una API REST. Las características básicas de esta extensión
que se necesitan conocer
son el \sphinxstyleemphasis{enrutamiento de recursos}, los \sphinxstyleemphasis{endpoints} y el \sphinxstyleemphasis{parseo de argumentos}

El componente principal que provee Flask-RESTful es la clase \sphinxcode{Resource}.
Esta brinda un fácil acceso a múltiples métodos HTTP definiendo
solamente los métodos en el recurso que se cree.

Se pueden añadir múltiples URL al objeto \sphinxcode{Api}, el \sphinxstyleemphasis{punto de entrada principal}
de la aplicación, a través de su método \sphinxcode{add\_resource()}. Cada URL será
enrutado al \sphinxcode{Recurso} que se pase como argumento.

Flask RESTful cuenta con el módulo \sphinxcode{reqparse} para la validación de datos en
la petición. Sin embargo, a pesar de ayudar con el manejo de errores, no
es lo suficientemente robusta para manejar tipos de datos más complejos dentro
de la petición.

El siguiente bloque de código muestra una API muy básica, que utiliza todas
las características antes mencionadas. Así como la estructura general de un
programa usando esta extensión.

\begin{sphinxVerbatim}[commandchars=\\\{\}]
\PYG{k+kn}{from} \PYG{n+nn}{flask} \PYG{k}{import} \PYG{n}{Flask}
\PYG{k+kn}{from} \PYG{n+nn}{flask\PYGZus{}restful} \PYG{k}{import} \PYG{n}{reqparse}\PYG{p}{,} \PYG{n}{abort}\PYG{p}{,} \PYG{n}{Api}\PYG{p}{,} \PYG{n}{Resource}

\PYG{n}{app} \PYG{o}{=} \PYG{n}{Flask}\PYG{p}{(}\PYG{n+nv+vm}{\PYGZus{}\PYGZus{}name\PYGZus{}\PYGZus{}}\PYG{p}{)}
\PYG{n}{api} \PYG{o}{=} \PYG{n}{Api}\PYG{p}{(}\PYG{n}{app}\PYG{p}{)}

\PYG{n}{TASKS} \PYG{o}{=} \PYG{p}{\PYGZob{}}
    \PYG{l+s+s1}{\PYGZsq{}}\PYG{l+s+s1}{tarea1}\PYG{l+s+s1}{\PYGZsq{}}\PYG{p}{:} \PYG{p}{\PYGZob{}}\PYG{l+s+s1}{\PYGZsq{}}\PYG{l+s+s1}{description}\PYG{l+s+s1}{\PYGZsq{}}\PYG{p}{:} \PYG{l+s+s1}{\PYGZsq{}}\PYG{l+s+s1}{Busca al chico}\PYG{l+s+s1}{\PYGZsq{}}\PYG{p}{\PYGZcb{}}\PYG{p}{,}
    \PYG{l+s+s1}{\PYGZsq{}}\PYG{l+s+s1}{tarea2}\PYG{l+s+s1}{\PYGZsq{}}\PYG{p}{:} \PYG{p}{\PYGZob{}}\PYG{l+s+s1}{\PYGZsq{}}\PYG{l+s+s1}{description}\PYG{l+s+s1}{\PYGZsq{}}\PYG{p}{:} \PYG{l+s+s1}{\PYGZsq{}}\PYG{l+s+s1}{Eres el chico}\PYG{l+s+s1}{\PYGZsq{}}\PYG{p}{\PYGZcb{}}\PYG{p}{,}
    \PYG{l+s+s1}{\PYGZsq{}}\PYG{l+s+s1}{tarea3}\PYG{l+s+s1}{\PYGZsq{}}\PYG{p}{:} \PYG{p}{\PYGZob{}}\PYG{l+s+s1}{\PYGZsq{}}\PYG{l+s+s1}{description}\PYG{l+s+s1}{\PYGZsq{}}\PYG{p}{:} \PYG{l+s+s1}{\PYGZsq{}}\PYG{l+s+s1}{Haz algo por el chico}\PYG{l+s+s1}{\PYGZsq{}}\PYG{p}{\PYGZcb{}}\PYG{p}{,}
\PYG{p}{\PYGZcb{}}


\PYG{k}{def} \PYG{n+nf}{abort\PYGZus{}if\PYGZus{}task\PYGZus{}doesnt\PYGZus{}exist}\PYG{p}{(}\PYG{n}{task\PYGZus{}id}\PYG{p}{)}\PYG{p}{:}
    \PYG{k}{if} \PYG{n}{task\PYGZus{}id} \PYG{o+ow}{not} \PYG{o+ow}{in} \PYG{n}{TASKS}\PYG{p}{:}
        \PYG{n}{abort}\PYG{p}{(}\PYG{l+m+mi}{404}\PYG{p}{,} \PYG{n}{message}\PYG{o}{=}\PYG{l+s+s2}{\PYGZdq{}}\PYG{l+s+s2}{Task }\PYG{l+s+si}{\PYGZob{}\PYGZcb{}}\PYG{l+s+s2}{ doesn}\PYG{l+s+s2}{\PYGZsq{}}\PYG{l+s+s2}{t exist}\PYG{l+s+s2}{\PYGZdq{}}\PYG{o}{.}\PYG{n}{format}\PYG{p}{(}\PYG{n}{task\PYGZus{}id}\PYG{p}{)}\PYG{p}{)}

\PYG{n}{parser} \PYG{o}{=} \PYG{n}{reqparse}\PYG{o}{.}\PYG{n}{RequestParser}\PYG{p}{(}\PYG{p}{)}
\PYG{n}{parser}\PYG{o}{.}\PYG{n}{add\PYGZus{}argument}\PYG{p}{(}\PYG{l+s+s1}{\PYGZsq{}}\PYG{l+s+s1}{task}\PYG{l+s+s1}{\PYGZsq{}}\PYG{p}{)}

\PYG{c+c1}{\PYGZsh{} muestra, crea, actualiza o elimina una tarea.}
\PYG{k}{class} \PYG{n+nc}{Task}\PYG{p}{(}\PYG{n}{Resource}\PYG{p}{)}\PYG{p}{:}
    \PYG{k}{def} \PYG{n+nf}{get}\PYG{p}{(}\PYG{n+nb+bp}{self}\PYG{p}{,} \PYG{n}{task\PYGZus{}id}\PYG{p}{)}\PYG{p}{:}
        \PYG{n}{abort\PYGZus{}if\PYGZus{}task\PYGZus{}doesnt\PYGZus{}exist}\PYG{p}{(}\PYG{n}{task\PYGZus{}id}\PYG{p}{)}
        \PYG{k}{return} \PYG{n}{TASKS}\PYG{p}{[}\PYG{n}{task\PYGZus{}id}\PYG{p}{]}

    \PYG{k}{def} \PYG{n+nf}{delete}\PYG{p}{(}\PYG{n+nb+bp}{self}\PYG{p}{,} \PYG{n}{task\PYGZus{}id}\PYG{p}{)}\PYG{p}{:}
        \PYG{n}{abort\PYGZus{}if\PYGZus{}task\PYGZus{}doesnt\PYGZus{}exist}\PYG{p}{(}\PYG{n}{task\PYGZus{}id}\PYG{p}{)}
        \PYG{k}{del} \PYG{n}{TASKS}\PYG{p}{[}\PYG{n}{task\PYGZus{}id}\PYG{p}{]}
        \PYG{k}{return} \PYG{l+s+s1}{\PYGZsq{}}\PYG{l+s+s1}{\PYGZsq{}}\PYG{p}{,} \PYG{l+m+mi}{204}

    \PYG{k}{def} \PYG{n+nf}{put}\PYG{p}{(}\PYG{n+nb+bp}{self}\PYG{p}{,} \PYG{n}{task\PYGZus{}id}\PYG{p}{)}\PYG{p}{:}
        \PYG{n}{args} \PYG{o}{=} \PYG{n}{parser}\PYG{o}{.}\PYG{n}{parse\PYGZus{}args}\PYG{p}{(}\PYG{p}{)}
        \PYG{n}{task} \PYG{o}{=} \PYG{p}{\PYGZob{}}\PYG{l+s+s1}{\PYGZsq{}}\PYG{l+s+s1}{task}\PYG{l+s+s1}{\PYGZsq{}}\PYG{p}{:} \PYG{n}{args}\PYG{p}{[}\PYG{l+s+s1}{\PYGZsq{}}\PYG{l+s+s1}{task}\PYG{l+s+s1}{\PYGZsq{}}\PYG{p}{]}\PYG{p}{\PYGZcb{}}
        \PYG{n}{TASKS}\PYG{p}{[}\PYG{n}{task\PYGZus{}id}\PYG{p}{]} \PYG{o}{=} \PYG{n}{task}
        \PYG{k}{return} \PYG{n}{task}\PYG{p}{,} \PYG{l+m+mi}{201}


\PYG{c+c1}{\PYGZsh{} muestra una lista con todas las tareas y permite crear una nueva con POST}
\PYG{k}{class} \PYG{n+nc}{TaskList}\PYG{p}{(}\PYG{n}{Resource}\PYG{p}{)}\PYG{p}{:}
    \PYG{k}{def} \PYG{n+nf}{get}\PYG{p}{(}\PYG{n+nb+bp}{self}\PYG{p}{)}\PYG{p}{:}
        \PYG{k}{return} \PYG{n}{TASKS}

    \PYG{k}{def} \PYG{n+nf}{post}\PYG{p}{(}\PYG{n+nb+bp}{self}\PYG{p}{)}\PYG{p}{:}
        \PYG{n}{args} \PYG{o}{=} \PYG{n}{parser}\PYG{o}{.}\PYG{n}{parse\PYGZus{}args}\PYG{p}{(}\PYG{p}{)}
        \PYG{n}{task\PYGZus{}id} \PYG{o}{=} \PYG{n+nb}{int}\PYG{p}{(}\PYG{n+nb}{max}\PYG{p}{(}\PYG{n}{TASKS}\PYG{o}{.}\PYG{n}{keys}\PYG{p}{(}\PYG{p}{)}\PYG{p}{)}\PYG{o}{.}\PYG{n}{lstrip}\PYG{p}{(}\PYG{l+s+s1}{\PYGZsq{}}\PYG{l+s+s1}{task}\PYG{l+s+s1}{\PYGZsq{}}\PYG{p}{)}\PYG{p}{)} \PYG{o}{+} \PYG{l+m+mi}{1}
        \PYG{n}{task\PYGZus{}id} \PYG{o}{=} \PYG{l+s+s1}{\PYGZsq{}}\PYG{l+s+s1}{tarea}\PYG{l+s+si}{\PYGZob{}\PYGZcb{}}\PYG{l+s+s1}{\PYGZsq{}}\PYG{o}{.}\PYG{n}{format}\PYG{p}{(}\PYG{n}{task\PYGZus{}id}\PYG{p}{)}
        \PYG{n}{TASKS}\PYG{p}{[}\PYG{n}{task\PYGZus{}id}\PYG{p}{]} \PYG{o}{=} \PYG{p}{\PYGZob{}}\PYG{l+s+s1}{\PYGZsq{}}\PYG{l+s+s1}{description}\PYG{l+s+s1}{\PYGZsq{}}\PYG{p}{:} \PYG{n}{args}\PYG{p}{[}\PYG{l+s+s1}{\PYGZsq{}}\PYG{l+s+s1}{task}\PYG{l+s+s1}{\PYGZsq{}}\PYG{p}{]}\PYG{p}{\PYGZcb{}}
        \PYG{k}{return} \PYG{n}{TASKS}\PYG{p}{[}\PYG{n}{task\PYGZus{}id}\PYG{p}{]}\PYG{p}{,} \PYG{l+m+mi}{201}

\PYG{c+c1}{\PYGZsh{}\PYGZsh{} enrutamiento}
\PYG{n}{api}\PYG{o}{.}\PYG{n}{add\PYGZus{}resource}\PYG{p}{(}\PYG{n}{TodoList}\PYG{p}{,} \PYG{l+s+s1}{\PYGZsq{}}\PYG{l+s+s1}{/tasks}\PYG{l+s+s1}{\PYGZsq{}}\PYG{p}{)}
\PYG{n}{api}\PYG{o}{.}\PYG{n}{add\PYGZus{}resource}\PYG{p}{(}\PYG{n}{Todo}\PYG{p}{,} \PYG{l+s+s1}{\PYGZsq{}}\PYG{l+s+s1}{/tasks/\PYGZlt{}todo\PYGZus{}id\PYGZgt{}}\PYG{l+s+s1}{\PYGZsq{}}\PYG{p}{)}


\PYG{k}{if} \PYG{n+nv+vm}{\PYGZus{}\PYGZus{}name\PYGZus{}\PYGZus{}} \PYG{o}{==} \PYG{l+s+s1}{\PYGZsq{}}\PYG{l+s+s1}{\PYGZus{}\PYGZus{}main\PYGZus{}\PYGZus{}}\PYG{l+s+s1}{\PYGZsq{}}\PYG{p}{:}
    \PYG{n}{app}\PYG{o}{.}\PYG{n}{run}\PYG{p}{(}\PYG{n}{debug}\PYG{o}{=}\PYG{k+kc}{True}\PYG{p}{)}
\end{sphinxVerbatim}


\subsection*{Flask-SQLAlchemy}
\label{\detokenize{chapter_two/desc_cloudnao:flask-sqlalchemy}}
Flask-SQLAlchemy simplifica el uso de \sphinxstyleemphasis{SQLAlchemy} dentro de una aplicación de
Flask. SQLAlchemy es un  poderoso framework para bases de datos relacionales
que soporta varios administradores de bases de datos. Ofrece un ORM de alto
nivel y acceso de bajo nivel a la funcionalidad SQL nativa de los DBMS.

En Flask-SQLAlchemy, una base de datos se especifica con su URL. Por ejemplo
usando \sphinxstyleemphasis{SQLite}, que en realidad es sólo un archivo en el disco, el URL
será: \sphinxstylestrong{sqlite:////ruta/absoluta/a/la/basededatos}.

El URL de la base de datos añadiendo la llave \sphinxcode{SQLALCHEMY\_DATABASE\_URI} al
objeto de configuración de Flask. El ejemplo siguiente muestra como inicializar
y configurar una base de datos simple en SQLite.

\begin{sphinxVerbatim}[commandchars=\\\{\}]
\PYG{k+kn}{from} \PYG{n+nn}{flask\PYGZus{}sqlalchemy} \PYG{k}{import} \PYG{n}{SQLAlchemy}
\PYG{n}{app} \PYG{o}{=} \PYG{n}{Flask}\PYG{p}{(}\PYG{n+nv+vm}{\PYGZus{}\PYGZus{}name\PYGZus{}\PYGZus{}}\PYG{p}{)}
\PYG{n}{app}\PYG{o}{.}\PYG{n}{config}\PYG{p}{[}\PYG{l+s+s1}{\PYGZsq{}}\PYG{l+s+s1}{SQLALCHEMY\PYGZus{}DATABASE\PYGZus{}URI}\PYG{l+s+s1}{\PYGZsq{}}\PYG{p}{]} \PYG{o}{=}\PYGZbs{}
  \PYG{l+s+s1}{\PYGZsq{}}\PYG{l+s+s1}{sqlite:///data.sqlite}\PYG{l+s+s1}{\PYGZsq{}}
\PYG{n}{db} \PYG{o}{=} \PYG{n}{SQLAlchemy}\PYG{p}{(}\PYG{n}{app}\PYG{p}{)}
\end{sphinxVerbatim}

El objeto \sphinxcode{db} instanciado de la clase \sphinxcode{SQLAlchemy} representa a la base
de datos y brinda acceso a toda la funcionalidad de Flask-SQLAlchemy.


%A continuación describo de manera muy general algunas características importantes
%que se necesitan conocer sobre esta extensión para entender mejor la estructura
%de la API.
%\subparagraph{\sphinxstylestrong{Definición de un modelo}}
%\label{\detokenize{chapter_two/desc_cloudnao:definicion-de-un-modelo}}
El término \sphinxstyleemphasis{modelo} es usado para referirse a entidades persistentes dentro de la
aplicación. En el contexto de un ORM, un modelo es una clase de Python con
atributos que son iguales a las columnas de la tabla de una base de datos.

La instancia de una base de datos en Flask-SQLAlchemy tiene una clase base
para los modelos así como un conjunto de clases y funciones auxiliares que
ayudan a definir su estructura. Para el siguiente ejemplo definiremos una
base de datos muy simple con una tabla de \sphinxcode{Usuarios} y otra de \sphinxcode{Recursos},
la relación se define de uno a muchos, cada usuario tiene varios recursos.
La \hyperref[\detokenize{chapter_two/desc_cloudnao:relationship-example}]{Figura \ref{\detokenize{chapter_two/desc_cloudnao:relationship-example}}}, muestra lo descrito.

\begin{figure}[htbp]
\centering
\capstart

\noindent\sphinxincludegraphics{{relationship_example}.png}
\caption{Modelo relacional de la tabla usuarios y la de recursos.}\label{\detokenize{chapter_two/desc_cloudnao:relationship-example}}\end{figure}

El fragmento de código siguiente definen los modelos \sphinxcode{Usuario} y \sphinxcode{Recurso}.

\begin{sphinxVerbatim}[commandchars=\\\{\}]
\PYG{k}{class} \PYG{n+nc}{Usuario}\PYG{p}{(}\PYG{n}{db}\PYG{o}{.}\PYG{n}{Model}\PYG{p}{)}\PYG{p}{:}
  \PYG{n}{\PYGZus{}\PYGZus{}tablename\PYGZus{}\PYGZus{}} \PYG{o}{=} \PYG{l+s+s1}{\PYGZsq{}}\PYG{l+s+s1}{users}\PYG{l+s+s1}{\PYGZsq{}}
  \PYG{n+nb}{id} \PYG{o}{=} \PYG{n}{db}\PYG{o}{.}\PYG{n}{Column}\PYG{p}{(}\PYG{n}{db}\PYG{o}{.}\PYG{n}{Integer}\PYG{p}{,} \PYG{n}{primary\PYGZus{}key}\PYG{o}{=}\PYG{k+kc}{True}\PYG{p}{)}
  \PYG{n}{nombre} \PYG{o}{=} \PYG{n}{db}\PYG{o}{.}\PYG{n}{Column}\PYG{p}{(}\PYG{n}{db}\PYG{o}{.}\PYG{n}{String}\PYG{p}{(}\PYG{l+m+mi}{64}\PYG{p}{)}\PYG{p}{,} \PYG{n}{unique}\PYG{o}{=}\PYG{k+kc}{True}\PYG{p}{,} \PYG{n}{index}\PYG{o}{=}\PYG{k+kc}{True}\PYG{p}{)}
  \PYG{n}{password} \PYG{o}{=} \PYG{n}{db}\PYG{o}{.}\PYG{n}{Column}\PYG{p}{(}\PYG{n}{db}\PYG{o}{.}\PYG{n}{String}\PYG{p}{(}\PYG{l+m+mi}{128}\PYG{p}{)}\PYG{p}{)}
\PYG{k}{class} \PYG{n+nc}{Recurso}\PYG{p}{(}\PYG{n}{db}\PYG{o}{.}\PYG{n}{Model}\PYG{p}{)}\PYG{p}{:}
  \PYG{n}{\PYGZus{}\PYGZus{}tablename\PYGZus{}\PYGZus{}} \PYG{o}{=} \PYG{l+s+s1}{\PYGZsq{}}\PYG{l+s+s1}{resources}\PYG{l+s+s1}{\PYGZsq{}}
  \PYG{n+nb}{id} \PYG{o}{=} \PYG{n}{db}\PYG{o}{.}\PYG{n}{Column}\PYG{p}{(}\PYG{n}{db}\PYG{o}{.}\PYG{n}{Integer}\PYG{p}{,} \PYG{n}{primary\PYGZus{}key}\PYG{o}{=}\PYG{k+kc}{True}\PYG{p}{)}
  \PYG{n}{data} \PYG{o}{=} \PYG{n}{db}\PYG{o}{.}\PYG{n}{Column}\PYG{p}{(}\PYG{n}{db}\PYG{o}{.}\PYG{n}{Text}\PYG{p}{)}
\end{sphinxVerbatim}

El atributo \sphinxcode{\_\_tablename\_\_}  define el nombre de la tabla en la base de datos.
Flask-SQLAlchemy asigna un nombre por defecto si se omite \sphinxcode{\_\_tablename\_\_}.
Las variables restantes son los atributos del modelo, definidas como instancias
de la clase \sphinxcode{db.Column}.

El primer argumento en el constructor de \sphinxcode{db.Column} es el tipo de dato
para la columna de la base de datos. \sphinxcode{Integer, Float, String, Text}, son
sólo algunos de todas las opciones disponibles.
Los argumentos restantes son opciones de configuración para el atributo. Entre
las más comunes están \sphinxcode{primary\_key, unique, index, default} y  \sphinxcode{nullable}.


%\subparagraph{\sphinxstylestrong{Relaciones}}
%\label{\detokenize{chapter_two/desc_cloudnao:relaciones}}
Las bases de datos relaciones establecen conexiones entre tuplas de diferentes
tablas, con el uso de relaciones. En el diagrama relacional de la \hyperref[\detokenize{chapter_two/desc_cloudnao:relationship-example}]{Figura \ref{\detokenize{chapter_two/desc_cloudnao:relationship-example}}}
se establece una relación de \sphinxstyleemphasis{uno a muchos} de usuarios a recursos.

Esta relación uno a muchos se representa en nuestras clases modelo añadiendo
un atributo más, en la clase \sphinxcode{Recurso} agregamos como
llave foránea el id del usuario al que le
pertenece \sphinxcode{id\_usuario = db.Column(db.Integer, db.ForeignKey('usuari\\
os.id'))}.
Para el modelo de \sphinxcode{Usuario} agregamos el atributo
\sphinxcode{recursos = db.relationship('Recurso', backref='usuario')}. El primer argumento
indica que modelo está del otro lado de la relación, \sphinxcode{backref} establece la
relación en la dirección opuesta.

Hay otros tipos de relaciones. La relación \sphinxstyleemphasis{uno a uno}  puede ser expresada
como la relación anterior pero agregando la bandera \sphinxcode{uselist} a
\sphinxcode{db.relationship()}, que indica  un atributo escalar en vez de una colección
del lado de los muchos en la relación.


\subsubsection{Operaciones sobre la base de datos}
\label{\detokenize{chapter_two/desc_cloudnao:operaciones-sobre-la-base-de-datos}}

\begin{itemize}
\item
Crear tablas: \sphinxcode{db.create\_all()}.
Crea la base de datos basándose en las clases modelos.
\item
Borrar tablas: \sphinxcode{db.drop\_all()}.
Destruye los datos en la base de datos.
\item
Añadir tuplas: \sphinxcode{db.session.add(Objeto){}`}.
Los cambios  se manejan a través de una \sphinxcode{session} en la base de datos. Se pasa como parámetro un objeto del modelo.
\item
Escribir cambios: \sphinxcode{db.session.commit()}.
Para guardar los objetos en la base de datos, las sesiones necesitan ser \sphinxstyleemphasis{confirmadas}.
\item
Modificar tupas: \sphinxcode{Objeto.atributo\_a\_actualizar = nuevo\_valor} \sphinxcode{db.session.add(Objeto)}.
El método \sphinxcode{add} también se puede usar para actualizar los modelos.
\item
Eliminar tuplas: \sphinxcode{db.session.delete(Objeto)}.
El método \sphinxcode{delete} recibe como parámetro al objeto que desea eliminarse.
\item
Consultas: \sphinxcode{Modelo.query.all()} \sphinxcode{Modelo.query.filter\_by(Criterio)}.
La consulta más básica retorna todo el contenido de la tabla correspondiente \sphinxcode{Modelo.query.all()}.Se pueden usar \sphinxstyleemphasis{filtros} para buscar elementos más específicos en la base de datos.
\end{itemize}

\section*{Requests}
\label{\detokenize{chapter_two/desc_cloudnao:requests}}
Requests permite hacer peticiones HTTP de una manera muy sencilla. Un petición
de tipo GET se convierte en algo simple como lo siguiente:

\begin{sphinxVerbatim}[commandchars=\\\{\}]
\PYG{g+gp}{\PYGZgt{}\PYGZgt{}\PYGZgt{} }\PYG{k+kn}{import} \PYG{n+nn}{requests}
\PYG{g+gp}{\PYGZgt{}\PYGZgt{}\PYGZgt{} }\PYG{n}{r} \PYG{o}{=} \PYG{n}{requests}\PYG{o}{.}\PYG{n}{get}\PYG{p}{(}\PYG{l+s+s1}{\PYGZsq{}}\PYG{l+s+s1}{https://swapi.co/api/planets/1/}\PYG{l+s+s1}{\PYGZsq{}}\PYG{p}{)}
\PYG{g+gp}{\PYGZgt{}\PYGZgt{}\PYGZgt{} }\PYG{n}{r}\PYG{o}{.}\PYG{n}{status\PYGZus{}code}
\PYG{g+go}{200}
\PYG{g+gp}{\PYGZgt{}\PYGZgt{}\PYGZgt{} }\PYG{n}{r}\PYG{o}{.}\PYG{n}{headers}\PYG{p}{[}\PYG{l+s+s1}{\PYGZsq{}}\PYG{l+s+s1}{content\PYGZhy{}type}\PYG{l+s+s1}{\PYGZsq{}}\PYG{p}{]}
\PYG{g+go}{\PYGZsq{}application/json\PYGZsq{}}
\PYG{g+gp}{\PYGZgt{}\PYGZgt{}\PYGZgt{} }\PYG{n}{r}\PYG{o}{.}\PYG{n}{text}
\PYG{g+go}{u\PYGZsq{}\PYGZob{}\PYGZca{}name\PYGZca{}:\PYGZca{}Tatooine\PYGZca{},\PYGZca{}rotation\PYGZus{}period\PYGZca{}:\PYGZca{}23\PYGZca{},\PYGZca{}orbital\PYGZus{}period\PYGZca{}:\PYGZca{}304\PYGZca{},\PYGZca{}diameter\PYGZca{}:\PYGZca{}10465\PYGZca{},\PYGZca{}climate\PYGZca{}:\PYGZca{}arid\PYGZca{}...\PYGZcb{}}
\PYG{g+gp}{\PYGZgt{}\PYGZgt{}\PYGZgt{} }\PYG{n}{r}\PYG{o}{.}\PYG{n}{json}\PYG{p}{(}\PYG{p}{)}
\PYG{g+go}{\PYGZob{}u\PYGZsq{}diameter\PYGZsq{}: u\PYGZsq{}10465\PYGZsq{}, u\PYGZsq{}climate\PYGZsq{}: u\PYGZsq{}arid\PYGZsq{}, u\PYGZsq{}surface\PYGZus{}water\PYGZsq{}: u\PYGZsq{}1\PYGZsq{}, u\PYGZsq{}name\PYGZsq{}: u\PYGZsq{}Tatooine\PYGZsq{}, ...\PYGZcb{}}
\end{sphinxVerbatim}

Hay varios métodos para hacer una petición, sin embargo, nos enfocaremos en uno.
El método \sphinxcode{request} de la clase \sphinxcode{requests}, construye, envía una petición
y tiene  como valor de retorno un objeto \sphinxcode{Response}.
A continuación de los parámetros más comunes del método \sphinxcode{requests.request(method, url, **kwargs)}
\begin{itemize}
\item {} 
\texttt{method}, (obligatorio) el método de la petición, una cadena con valor \sphinxcode{post}, \sphinxcode{get}, etc.

\item {} 
\texttt{url}, (obligatorio) la URL a la que se hace la petición.

\item {} 
\texttt{params}, (opcional) diccionario o bytes enviados en la URL.

\item {} 
\texttt{data}, (opcional)  diccionario o lista de tuplas \sphinxcode{{[}(llave, valor){]}}, bytes u objeto similar a un archivo.

\item {} 
\texttt{json}, (opcional) json a enviar en el cuerpo de la petición.

\item {} 
\texttt{headers}, (opcional) un diccionario con los headers HTTP que se enviarán.

\end{itemize}

Dependiendo del contenido de la respuesta por parte del servidor, se elige cómo
procesar la respuesta. Si es solo texto, se puede usar \sphinxcode{Response.text},
que automáticamente decodificara el contenido enviado por el servidor,
\sphinxcode{Response.content}, si queremos acceder al cuerpo del mensaje como bytes.
También cuenta con un decodificador de JSON, \sphinxcode{Response.json()}.

Podemos checar el código de estado de la respuesta con
\sphinxcode{Response.status\_code}. Y también los headers, utilizando \sphinxcode{Response.headers}.